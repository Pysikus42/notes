% Created 2016-12-07 Mi 19:09
\documentclass[a4paper]{scrartcl}
\usepackage[utf8]{inputenc}
\usepackage[T1]{fontenc}
\usepackage{fixltx2e}
\usepackage{graphicx}
\usepackage{longtable}
\usepackage{float}
\usepackage{wrapfig}
\usepackage{rotating}
\usepackage[normalem]{ulem}
\usepackage{amsmath}
\usepackage{textcomp}
\usepackage{marvosym}
\usepackage{wasysym}
\usepackage{amssymb}
\usepackage{hyperref}
\tolerance=1000
\usepackage{siunitx}
\usepackage{fontspec}
\sisetup{load-configurations = abbrevations}
\newcommand{\estimates}{\overset{\scriptscriptstyle\wedge}{=}}
\usepackage{mathtools}
\DeclarePairedDelimiter\abs{\lvert}{\rvert}%
\DeclarePairedDelimiter\norm{\lVert}{\rVert}%
\DeclareMathOperator{\Exists}{\exists}
\DeclareMathOperator{\Forall}{\forall}
\DeclareMathOperator{\cha}{char}
\DeclareMathOperator{\Abb}{Abb}
\DeclareMathOperator{\Lin}{Lin}
\DeclareMathOperator{\Span}{span}
\def\colvec#1{\left(\vcenter{\halign{\hfil$##$\hfil\cr \colvecA#1;;}}\right)}
\def\colvecA#1;{\if;#1;\else #1\cr \expandafter \colvecA \fi}
\usepackage{stmaryrd}
\usepackage{amsthm}
\theoremstyle{definition}
\newtheorem{defn}{Definition}
\theoremstyle{plain}
\newtheorem{thm}{Satz}
\theoremstyle{plain}
\newtheorem{lemma}{Lemma}
\theoremstyle{remark}
\newtheorem{remark}{Bemerkung}
\theoremstyle{remark}
\newtheorem{note}{Anmerkung}
\theoremstyle{remark}
\newtheorem{conc}{Folgerung}
\theoremstyle{remark}
\newtheorem{ex}{Beispiel}
\usepackage{etoolbox}
\patchcmd{\thmhead}{(#3)}{#3}{}{}
\renewcommand*{\proofname}{Beweis}
\usepackage{wasysym}
\usepackage{xparse}% http://ctan.org/pkg/xparse
\NewDocumentCommand{\overarrow}{O{=} O{\uparrow} m}{%
\overset{\makebox[0pt]{\begin{tabular}{@{}c@{}}#3\\[0pt]\ensuremath{#2}\end{tabular}}}{#1}
}
\NewDocumentCommand{\underarrow}{O{=} O{\downarrow} m}{%
\underset{\makebox[0pt]{\begin{tabular}{@{}c@{}}\ensuremath{#2}\\[0pt]#3\end{tabular}}}{#1}
}
\newcommand{\I}{\ensuremath{\imath}}%
\usepackage{stackengine}
\author{Robin Heinemann}
\date{\today}
\title{Lineare Algebra (Vogel)}
\hypersetup{
  pdfkeywords={},
  pdfsubject={},
  pdfcreator={Emacs 25.1.1 (Org mode 8.2.10)}}
\begin{document}

\maketitle
\tableofcontents


\section{Einleitung}
\label{sec-1}
Übungsblätter/Lösungen:
jew. Donnerstag / folgender Donnerstag
Abgabe Donnerstag 9:30
50\% der Übungsblätter
\subsection{Plenarübung}
\label{sec-1-1}
Aufgeteilt
\subsection{Moodle}
\label{sec-1-2}
Passwort: vektorraumhomomorphismus
\subsection{Klausur}
\label{sec-1-3}
24.02.2017
\section{Grundlagen}
\label{sec-2}
\subsection{Naive Aussagenlogik}
\label{sec-2-1}
naive Logik: wir vewenden die sprachliche Vorstellung ($\neq$ mathematische Logik: eigne Vorlesung)
Eine Aussage ist ein festehender Satz, dem genau einer der Wahrheitswerte "wahr" oder "falsch" zugeordnet werden kann.
Aus einfachen Aussagen kann man durch logische Verknüpfungen kompliziertere Aussagen bilden.
Angabe der Wahrheitswertes der zusammengesetzten Aussage erfolgt duch Wahrheitstafeln (liefern den Wahrheitswert der zusammengesetzten Aussage, aus dem Wahrheitswert der einzelnen Aussagen).
Im folgenden seien $A$ und $B$ Aussagen.
\begin{itemize}
\item Negation (NICHT-Verknüpfung)
\begin{itemize}
\item Symbol: \$$\neg{}$
\item Wahrheitstafel:
\begin{center}
\begin{tabular}{ll}
$A$ & $\neg A$\\
\hline
w & f\\
f & w\\
\end{tabular}
\end{center}
\item Beispiel: $A$: 7 ist eine Primzahl (w)
$\neg A$: 7 ist keine Primzahl (f)
\end{itemize}

\item Konjunktion (UND-Verknüpfung)
\begin{itemize}
\item Symbol $\wedge$
\item Wahrheitstafel:
\begin{center}
\begin{tabular}{lll}
$A$ & $B$ & $A\wedge B$\\
\hline
w & w & w\\
w & f & f\\
f & w & f\\
f & f & f\\
\end{tabular}
\end{center}
\end{itemize}

\item Disjunktion (ODER-Verknüpfung)
\begin{itemize}
\item Symbol: $\vee$
\item Wahrheitstafel:
\begin{center}
\begin{tabular}{lll}
$A$ & $B$ & $A\vee B$\\
\hline
w & w & w\\
w & f & w\\
f & w & w\\
f & f & f\\
\end{tabular}
\end{center}
\item exklusives oder: $(A\vee B) \wedge (\neg(A\wedge B))$
\end{itemize}
\item Beispiel $A$: 7 ist eine Primzahl (w), $B$: 5 ist gerade (f)
\begin{itemize}
\item $A\wedge B$ 7 ist eine Primzahl und 5 ist gerade (f)
\item $A\vee B$ 7 ist eine Primzahl oder 5 ist gerade (w)
\end{itemize}

\item Implikation (WENN-DANN-Verknüpfung)
\begin{itemize}
\item Symbol: $\implies$
\item Wahrheitstafel:
\begin{center}
\begin{tabular}{lll}
$A$ & $B$ & $A\implies B$\\
\hline
w & w & w\\
w & f & f\\
f & w & w\\
f & f & w\\
\end{tabular}
\end{center}
\item Sprechweise: $A$ impliziert $B$, aus $A$ folgt $B$, $A$ ist eine hinreichende Bedingung für $B$ (ist $A\implies B$ wahr, dann folgt aus $A$ wahr, $B$ ist wahr), $B$ ist eine notwendige Bedingung für $A$ (ist $A\implies B$ wahr, dann kann $A$ nur dann wahr sein, wenn Aussage $B$ wahr ist)
\item Beispiel Es seinen $m,n\in\mathbb{N}$
\begin{itemize}
\item $A$: m ist gerade
\item $B$: $mn$ ist gerade
\item Dann gilt $\Forall m,n \in\mathbb{N}~A\implies B~\text{wahr}$ \\
         Fallunterscheidung:
\begin{itemize}
\item $m$ gerade, $n$ gerade, dann ist $A$ wahr, $B$ wahr, d.h. $A\implies B$ wahr
\item $m$ gerade, $n$ ungerade, dann ist $A$ wahr, $B$ wahr, d.h. $A\implies B$ wahr
\item $m$ ungerade, $n$ gerade, dann ist $A$ falsch, $B$ wahr, d.h. $A\implies B$ wahr
\item $m$ ungerade, $n$ ungerade, dann ist $A$ falsch, $B$ falsh, d.h. $A\implies B$ wahr
\end{itemize}
\end{itemize}
\end{itemize}
\item Äquivalenz (GENAU-DANN-WENN-Verknüpfung)
\begin{itemize}
\item Symbol $\iff$
\item Wahrheitstafel:
\begin{center}
\begin{tabular}{lll}
$A$ & $B$ & $A\iff$ B\\
\hline
w & w & w\\
w & f & f\\
f & w & f\\
f & f & w\\
\end{tabular}
\end{center}
\item Sprechweise: $A$ gilt genau dann, wenn $B$ gilt, $A$ ist hinreichend und notwendig für $B$ \\
       Die Aussagen $A\iff B$ und $(A\implies B)\wedge (B\implies A)$ sind gleichbedeutend:
\begin{center}
\begin{tabular}{llllll}
$A$ & $B$ & $A\iff B$ & $A\implies B$ & $B\implies A$ & $(A\implies B)\wedge (B\implies A)$\\
\hline
w & w & w & w & w & w\\
w & f & f & f & w & f\\
f & w & f & w & f & f\\
f & w & f & w & f & f\\
f & f & w & w & w & w\\
\end{tabular}
\end{center}
\item Beispiel: Es sei $n$ eine ganze Zahl \\
       $A:~n-2>1$ \\
       $B:~n>3$ \\
       $\Forall n\in\mathbb{N}~\text{gilt}~A\iff B$
       $C:~n>0$ \\
       $D:~n^2>0$ \\
       Für $n=-1$ ist die Äquivalenz $C\iff$ falsch ($C$ falsch, $D$ wahr) \\
       Für alle ganzen Zahlen $n$ gilt zumindest die Implikation $C\implies D$
\end{itemize}
\end{itemize}
\subsection{Beweis}
\label{sec-2-2}
Mathematische Sätze, Bemerkungen, Folgerungen, etc. sind meistens in Form wahrer Implikationen formuliert
\subsubsection{beweisen}
\label{sec-2-2-1}
Begründen warum diese Implikation wahr ist
\subsubsection{Beweismethoden for diese Implikation $A\implies B$}
\label{sec-2-2-2}
\begin{itemize}
\item direkter Beweis ($A\implies B$)
\item Beweis durch Kontraposition ($\neq B \implies \neg A$)
\item Widerspruchbeweis ($\neg (A\wedge \neg B)$)
\end{itemize}
Diese sind äquivalent zueinander
\begin{center}
\begin{tabular}{lllllll}
$A$ & $B$ & $\neg A$ & $\neg B$ & $A\implies B$ & $\neg B \implies \neg A$ & $\neg (A \wedge \neg B)$\\
\hline
w & w & f & f & w & w & w\\
w & f & f & w & f & f & f\\
f & w & w & f & w & w & w\\
f & f & w & w & w & w & w\\
\end{tabular}
\end{center}
\paragraph{Beispiel}
\label{sec-2-2-2-1}
$m,n$ natürliche Zahlen \\
     \[A:~m^2 < n^2\]
\[B:~m < n\]
Wir wollen zeigen, dass $A\implies B$ für alle natürlichen Zahlen $m,n$ wahr ist
\begin{itemize}
\item direkter Beweis: \\
       \[A:~m^2 < n^2 \implies 0 < n^2 - m^2 \implies 0 < (n-m)\underbrace{(n+m)}_{>0} \implies 0 < n-m \implies m<n\]
\item Beweis durch Kontraposition: \\
       \[\neg B:~m \geq n \implies m^2\geq n m \wedge m n \geq n^2 \implies m^2 \geq n^2 \implies \neg A\]
\item Beweis durch Widerspruch: \\
       \[A\wedge \neg B \implies m^2 < n^2 \wedge n\leq m \implies m^2 < n^2 \wedge m n \leq m^2 \wedge n^2 \leq m n \implies m n \leq m^2 < n^2 \leq m n\]
       Wiederspruch
\end{itemize}
\subsection{Existenz- und Allquantor}
\label{sec-2-3}
\subsubsection{Existenzquantor}
\label{sec-2-3-1}
\$A(x) Aussage, die von Variable x abhängt \\
    $\exists x:~A(x)$ ist gleichbedeutend mit "Es existiert ein x, für das $A(x)$ wahr ist" (hierbei ist "existiert ein x" im Sinne von "existiert mindestens ein x" zu verstehen) \\
    Beispiel:
\[\exists n\in\mathbb{N}:~n>5\quad\text{(w)}\]
$\exists !x:~A(x)$ ist gleichbedeutend mit "Es existiert genau ein x, für dass $A(x)$ wahr ist"
\subsubsection{Allquantor}
\label{sec-2-3-2}
$\Forall x:~A(x)$ ist gleichbedeutend mit "Für alle x ist A(x) wahr"
Beispiel:
\[\Forall n\in\mathbb{N}: 4n~\text{ist gerade}\]
\subsubsection{Negation von Existenz- und Allquantor}
\label{sec-2-3-3}
\[\neg(\exists x:~A(x)) \iff \Forall x:~\neg A(x)\]
\[\neg(\Forall x:~A(x)) \iff \exists x:~\neg A(x)\]
\subsubsection{Spezielle Beweistechniken für Existenz und Allaussagen}
\label{sec-2-3-4}
\begin{itemize}
\item Angabe eines Beispiel, um zu zeigen, dass deine Existenzaussage wahr ist. \\
      Beispiel:
\[\exists n\in\mathbb{N}:~n>5 \text{ist wahr, denn für $n = 7$ ist die Aussage $n > 5$ wahr}\]
\item Angabe eines Gegenbeispiel, um zu zeigen, dass eine Allausage falsch ist. \\
      Beispiel:
\[\Forall n\in\mathbb{N}:~n\leq 5 \text{ist flasch, dann für $n=7$ ist die Aussage $n\leq 5$ falsch}\]
\end{itemize}
\subsection{Naive Mengenlehre}
\label{sec-2-4}
Mengenbegriff nach Cantor: \\
   Eine Menge ist eine Zusammenfassung von bestimmten, wohlunterschiedenen Objekten userer Anschauung oder useres Denkens (die Elemente genannt werden) zu einem Ganzen

\subsubsection{Schreibweise}
\label{sec-2-4-1}
\begin{itemize}
\item $x\in M$, falls $x$ ein Element von $M$ ist
\item $x\not\in M$, falls $x$ kein Element von $M$ ist
\item $M=N$, falls $M$ und $N$ die gleichen Elemente besitzen, $M\subseteq N \wedge N\subseteq M$
\end{itemize}

\subsubsection{Angabe von Mengen}
\label{sec-2-4-2}
\begin{itemize}
\item Reihenfolge ist unrelevant (\$\{1,2,3\}=\{1,3,2\})
\item Elemente sind wohlunterschieden $\{1,2,2\} = \{1,2\}$
\item Auflisten der Elemente $M=\{a,b,c,\ldots\}$
\item Beschreibung der Elemente durch Eigenschaften: $M=\{x\mid E(x)\}$ \\
     (Elemente x, für die E(x) wahr)
\begin{itemize}
\item Beispiel:
\[\{2,4,6,8\} = \{x\mid x\in\mathbb{N}, x~\text{gerade}, 1 < x < 10\}\]
\end{itemize}
\end{itemize}
\subsubsection{leere Menge}
\label{sec-2-4-3}
Die leere Menge $\emptyset$ enthält keine Elemente
\paragraph{Beispiel}
\label{sec-2-4-3-1}
\[\{x\mid x\in\mathbb{N}, x < -5\} = \emptyset\]

\subsubsection{Zahlenbereiche}
\label{sec-2-4-4}
Menge der natürlichen Zahlen:
\[\mathbb{N} := \{1,2,3,\ldots\}\]

Menge der natürlichen Zahlen mit Null:
\[\mathbb{N}_0 := \{0, 1,2,3,\ldots\}\]

Menge der Ganzen Zahlen:
\[\mathbb{Z} := \{0,1,-1,2,-2\}\]

Menge der rationalen Zahlen:
\[\mathbb{Q} := \{\frac{m}{n} \mid m\in\mathbb{Z}, n\in\mathbb{N}\}\]

Menge der reellen Zahlen: $\mathbb{R}$

\subsubsection{Teilmenge}
\label{sec-2-4-5}
$A,B$ seien Mengen. \\
    $A$ heißt Teilmenge von $B~(A\subseteq B) \xLeftrightarrow{\text{Def.}} \Forall x\in A: x\in B$
$A$ heißt echte Teilmenge von $B~(A\subset B) \xLeftrightarrow{\text{Def.}} A\subseteq B \wedge A\neq B$
\paragraph{Anmerkung}
\label{sec-2-4-5-1}
Offenbar gilt für Mengen $A,B$:
\[A=B \iff A\subseteq B \wedge B\subseteq A\]
$\emptyset$ ist Teilmenge jeder Menge

\paragraph{Beipspiel}
\label{sec-2-4-5-2}
\[\mathbb{N}\subset\mathbb{N}_0\subset\mathbb{Z}\subset\mathbb{Q}\]

\subsubsection{Durschnitt}
\label{sec-2-4-6}
\[A \cap B := \{x\mid x\in A \wedge x\in B\}\]
\paragraph{Beispiel}
\label{sec-2-4-6-1}
\[A=\{2,3,5,7\}, B=\{3,4,6,7\}, A\cap B = \{3,7\}\]

\subsubsection{Vereinigung}
\label{sec-2-4-7}
\[A\cup B := \{x\mid x\in A \vee x\in B\}\]
\paragraph{Beispiel}
\label{sec-2-4-7-1}
\[A=\{2,3,5,7\}, B=\{3,4,6,7\}, A\cup B = \{2,3,4,5,6,7\}\]

\subsubsection{Differenz}
\label{sec-2-4-8}
\[A\setminus B := \{x\mid x\in A \wedge x\not\in B\}\]
Im Fall $B\subseteq A$ nennt man $A\setminus B$ auch das Komplement von $B$ in $A$ und schreibt $\mathcal{c}_A(B) = A\setminus B$
\paragraph{Beispiel}
\label{sec-2-4-8-1}
\[A=\{2,3,5,7\}, B=\{3,4,6,7\}, A\setminus B = \{2,5\}\]
\subsubsection{Bemerkung zu Vereinigung und Durschnitt}
\label{sec-2-4-9}
$A,B$ seien zwei Mengen. Dann gilt \[A\cap (B\cup C) = (A\cap B) \cup (A\cap C)\]
\paragraph{Beweis}
\label{sec-2-4-9-1}
\[A\cap(B\cup C) \subseteq (A\cap B) \cup (A\cap C)\]
\[A\cap(B\cup C) \supseteq (A\cap B) \cup (A\cap C)\]
"$\subseteq$" Sei $x\in A \cap (B\cup C)$. Dann ist $x\in A \wedge x\in B\cup C$
\begin{itemize}
\item 1. Fall: $x\in A \wedge x\in B$
       \[\implies x\in A\cap B \implies x \in (A\cap B) \cup (A\cap C)\]
\item 2. Fall $x\in A \wedge x\in C$
       \[\implies x\in A\cap C \implies x\in (A\cap B)\cup(A\cap C)\]
\end{itemize}
Damit ist "$\subseteq$" gezeigt.
"$\supseteq$" Sei \$x$\in$ (A$\cap$ B) $\cup$ (A$\cap$ C)
\[\implies x\in A\cap B \vee x\in A\cap C \\ \implies (x\in A \wedge x\in B) \vee (x\in A \wedge x\in C) \\ \implies x\in A \wedge (x\in B\vee x\in C) \\ \implies x\in A \wedge x\in B\cup C \\ \implies x\in A\cap (B\cup C)\]
Damit ist "$\supseteq$" gezeigt.
\subsubsection{Bemerkung zu Äquivalenz von Mengen}
\label{sec-2-4-10}
Seien $A,B$ Mengen, dann sind äquivalent:
\begin{enumerate}
\item $A\cup B = B$
\item $A\subseteq B$
\end{enumerate}
\paragraph{Beweis}
\label{sec-2-4-10-1}
Wir zeigen 1) $\implies$ 2) und 2) $\implies$ 1.
\[1) \implies 2):~\text{Es gelte}~A\cup B = B,~\text{zu zeigen ist}~A\subseteq B \\ \text{Sei}~x\in A \implies x\in A \wedge x \in B \implies x\in A\cup B = B\]
\[2) \implies 1):~\text{Es gelte}~A\subseteq B\text{, zu zeigen ist}~A\cup B = B \]
"$\subseteq$": Sei $x\in A\cup B \implies x\in A \vee x\in B \xRightarrow{A\subseteq B} x\in B$
"$\supseteq$": $B\subseteq A\cup B$ klar
\subsubsection{Kartesisches Produkt}
\label{sec-2-4-11}
Seien $A,B$ Mengen
\[A\times B := \{(a,b)\mid a\in A, b\in B\}\]
heipt das kartesische Produkt von $A$ und $B$. Hierbei ist $(a,b) = (a',b') \xLeftrightarrow{\text{Def}} a = a' \wedge b = b'$ a = a' $\wedge$ b = b'\$

\paragraph{Beispiel}
\label{sec-2-4-11-1}
\begin{itemize}
\item \[\{1,2\}\times \{1,3,4\} = \{(1,1),(1,3),(1,4),(2,1),(2,3),(2,4)\}\]
\item \[\mathbb{R}\times\mathbb{R}=\{(x,y)|mid x,y \in \mathbb{R}\} = \mathbb{R}^2\]
\end{itemize}
\subsubsection{Potenzmenge}
\label{sec-2-4-12}
$A$ sei eine Menge
\[\mathcal{P} (A) := \{M\mid M\subseteq A\}\]
heißt die Potenzmenge von $A$
\paragraph{Beispiel}
\label{sec-2-4-12-1}
\[\mathcal{P} (\{1,2,3\}) =  \{\emptyset, \{1\}, \{2\},\{3\},\{1,2\},\{1,3\},\{2,3\}\{1,2,3,4\}\}\]
\subsubsection{Kardinalität}
\label{sec-2-4-13}
$M$ sei eine Menge. Wir setzen
\[\abs{M} := \begin{cases} n & \text{falls $M$ eine endliche Menge ist und $n$ Elemente enthält} \\ \infty & \text{falls $M$ nicht endlich ist} \end{cases}\]
$\abs{M}$ heißt Kardinalität von A
\paragraph{Beispiel}
\label{sec-2-4-13-1}
\begin{itemize}
\item $\abs{\{7,11,16\}} = 3$
\item $\abs{\mathbb{N}} = \infty$
\end{itemize}
\subsubsection{Bemerkung zu natürlichen Zahlen}
\label{sec-2-4-14}
Für die natürlichen Zahlen gilt das Induktionsaxiom
Ist $M\subseteq N$ eine Teilmenge, für die gilt:
\[1\in M \wedge \Forall n\in M : n\in M \implies n+1 \in M\]
dann ist $M = \mathbb{N}$
\subsubsection{Prinzip der vollständigen Induktion}
\label{sec-2-4-15}
Für jedes $n\in \mathbb{N}$ sei eine Aussage $A(n)$ gegeben. Die Aussagen $A(N)$ gelten für alle $n\in\mathbb{N}$, wenn man folgendes zeigen kann: \\
\begin{itemize}
\item (IA) $A(1)$ ist wahr
\item (IS) Für jedes $n\in\mathbb{N}$ gilt: $A(n) \implies A(n+1)$
\end{itemize}
Der Schritt (IA) heißt Induktionsanfang, die Implikation $A(n) \implies A(n+1)$ heißt Induktionsschritt
\paragraph{Beweis}
\label{sec-2-4-15-1}
Setze $M := \{n\in \mathbb{N}\mid A(n)~\text{ist wahr}\}$
Wegen (IA) ist $1\in M$, wegen (IS) gilt: $n\in M \implies n+1 \in M$ \\
     Nach Induktionsaxiom folgt $M = \mathbb{N}$, das heißt $A(n)$ ist wahr für alle $n\in \mathbb{N}$
\paragraph{Beispiel}
\label{sec-2-4-15-2}
Für $n\in\mathbb{N}$ sei $A(n)$ die Aussage: $1+\ldots + n = \frac{n(n+1)}{2}$
Wir zeigen: $A(n)$ ist wahr für alle $n\in \mathbb{N}$, und zwar durch vollständige Induktion
\begin{itemize}
\item (IA) $A(1)$ ist wahr, denn $1 = \frac{1(1+1)}{2}$
\item (IS) zu zeigen: $A(n) \implies A(n+1)$ \\
       Es gelte $A(n)$, das heißt $1+\ldots+n = \frac{n(n+1)}{2}$ ist wahr \[\implies 1 + \ldots + n + (n + 1) = \frac{n(n+1)}{2} + (n+1) =  \frac{n(n+1) + 2(n+1)}{2} = \frac{(n+1)(n+2)}{2} \square\]
\end{itemize}
\subsection{Relationen}
\label{sec-2-5}
\subsubsection{Definiton}
\label{sec-2-5-1}
Eine Relation auf $M$ ist eine Teilmenge $R\subseteq M\times M$
Wir schreiben $a\sim b \xLeftrightarrow{\text{Def}} (a,b) \in R$ ("a steht in Relation zu b")

\begin{itemize}
\item anschaulich: eine Relation auf $M$ stellt eine "Beziehung" zwischen den Elementen von $M$ her.
\item Für $a,b \in M$ gilt entweder $a\sim b$ oder $a\not\sim b$, denn: entweder ist $(a,b) \in R$ oder $(a,b)\not\in R$
\end{itemize}
\paragraph{Anmerkung}
\label{sec-2-5-1-1}
Aufgrund der obigen Notation spricht man in der Regel von Relation "\$$\sim$" auf $M$ als von der Relation $R \subseteq M\times M$
\paragraph{Beispiel}
\label{sec-2-5-1-2}
\$M = \{1,2,3\}. Durch $R = \{(1,1), (1,2), (3,3) \subseteq M\times M\}$ ist eine Relation auf $M$ gegeben. Es gilt dann: $1\sim 1, 1\sim 2, 3\sim 3$ (aber zum Beispiel: $1\not\sim 3, 2\not\sim 1, 2\not\sim 2$)

\subsubsection{Eigenschaften von Relationen}
\label{sec-2-5-2}
$M$ Menge, $\sim$ Relation auf $M$ \\
    $\sim$ heißt:
\begin{itemize}
\item reflexiv $\xLeftrightarrow{\text{Def}}$ für alle $a\in M$ gilt $a\sim a$
\item symmetrisch $\xLeftrightarrow{\text{Def}}$ für alle $a,b\in M$ gilt: $a\sim b \implies b\sim a$
\item antisymmetrisch $\xLeftrightarrow{\text{Def}}$ für alle $a,b \in M$ gilt: $a\sim b \wedge b\sim a \implies a = b$
\item transitiv $\xLeftrightarrow{\text{Def}}$ für alle $a,b,c\in M$ gilt: $a\sim b \wedge b\sim v \implies a\sim c$
\item total $\xLeftrightarrow{\text{Def}}$ für alle $a,b\in M$ gilt: $a\sim b \vee b\sim a$
\end{itemize}
\paragraph{Beispiel}
\label{sec-2-5-2-1}
Sei $M$ die Menge der Studierenden in der LA1-Vorlesung
\begin{enumerate}
\item Für $a,b \in M$ sei $a\sim b \xLeftrightarrow{\text{Def}}$ $a$ hat den selben Vornamen wie $b$ \\
        $\sim$ reflexiv, symmetrisch, nicht antisymmetrisch, transitiv, nicht total
\item Für $a,b \in M$ sei $a\sim b \xLeftrightarrow{\text{Def}}$ Martrikelnummer von $a$ ist kleiner gleich als die Martrikelnummer von $b$ \\
        $\sim$ ist reflexiv, nicht symmetrisch, antisymmetrisch, transitiv, total
\item Für $a,b \in M$ sei $a\sim b \xLeftrightarrow{\text{Def}}$ $a$ sitzt auf dem Platz recht von $b$ \\
        $\sim$ ist nicht reflexiv, nicht symmetrisch, nicht antisymmetrisch, nicht transitiv, nicht total
\end{enumerate}
\subsubsection{Halbordnung / Totalordung}
\label{sec-2-5-3}
$\sim$ heißt
\begin{itemize}
\item Halbordnung auf $M\xLeftrightarrow{\text{Def}}~\sim$ ist reflexiv, antisymmetrisch und transitiv
\item Totalordung auf $M\xLeftrightarrow{\text{Def}}~\sim$ ist eine Halbordnung und $\sim$ ist total
\end{itemize}
In diesen Fällen sagt man auch: Das Tupel $(M,\sim)$ ist eine halbgeordnete, beziehungsweise totalgeordnete Menge.
\paragraph{Beispiel}
\label{sec-2-5-3-1}
\begin{enumerate}
\item $\leq$ auf $\mathbb{N}$ ist eine Totalordung
\item Sei $M = \mathcal{P}(\{1,2,3\})$. $\subseteq$ ist auf $M$ eine Halbordung, aber keine Totalordung (es ist zum Beispiel weder $\{1\} \subseteq \{3\}$ noch $\{3\}\subseteq \{\}$)
\end{enumerate}
\paragraph{Anmerkung}
\label{sec-2-5-3-2}
Wegen der Analogie zur $\leq$ auf $\mathbb{N}$ bezeichnen wir Halbordnungen in der Regel mit $\leq$
\subsubsection{Größtes / kleinstes Element}
\label{sec-2-5-4}
$(M, \leq)$ halbgeordnete Menge, $a\in M$ \\
    $a$ heißt ein
\begin{itemize}
\item größtes Element von $M\xLeftrightarrow{\text{Def}}$ Für alle $x\in M$ gilt $x\leq a$
\item kleinstes Element von $M\xLeftrightarrow{\text{Def}}$ Für alle $x\in M$ gilt $a\leq x$
\end{itemize}
\paragraph{Bemerkung}
\label{sec-2-5-4-1}
$(M,\leq)$ halbgeordnete Menge \\
     Dann gilt: Existiert in $M$ ein größtes (beziehungsweise kleinstes) Element, so ist dieses eindeutig bestimmt
\subparagraph{Beweis}
\label{sec-2-5-4-1-1}
Es seien $a,b\in M$ größte Elemente von $M$ \\
      $\implies x\leq a$ für alle $x\in M$, also auch $b\leq a$ \\
      Außerdem: $x \leq b$ für alle $x\in M$, also auch $a\leq b$ \\
      $\xRightarrow{\text{Antisymmetrie}} a = b$ \\
      Analog für kleinstes Element
\subparagraph{Anmerkung}
\label{sec-2-5-4-1-2}
Dies sagt nichts darüber aus, ob ein größtes (beziehungsweise kleinstes) Element in $M$ überhaupt existiert.
\paragraph{Beispiel}
\label{sec-2-5-4-2}
\begin{enumerate}
\item In $(\mathbb{N},\leq)$ ist 1 das kleinste Element, ein größtes Element gibt es nicht
\item $(\{\{1\},\{2\},\{3\},\{1,2\},\{1,3\},\{2,3\}\}, \subseteq)$ ist eine halbgeordnete Menge ohne kleinstes beziehungsweise größtes Element
\end{enumerate}
\subsubsection{maximales / minimales Element}
\label{sec-2-5-5}
$(M,\leq)$ halbgeordnete Menge, $a\in M$ \\
    $a$ heißt ein
\begin{itemize}
\item maximales Element von $M \xLeftrightarrow{\text{Def}}$ für alle $x\in M$ gilt: $a\leq x \implies a = x$
\item minmales Element von $M \xLeftrightarrow{\text{Def}}$ für alle $x\in M$ gilt: $x\leq a \implies a = x$
\end{itemize}
\paragraph{Beispiel}
\label{sec-2-5-5-1}
In $(\{\{1\},\{2\},\{3\},\{1,2\},\{1,3\},\{2,3\}\}, \subseteq)$ sind $\{1,2\},\{1,3\},\{2,3\}$ maximale Elemente und $\{1\},\{2\},\{3\}$ sind minimale Elemente.
\paragraph{Bemerkung}
\label{sec-2-5-5-2}
$(M,\leq)$ halbgeordnete Menge, $a\in M$ \\
     Dann gilt: Ist $a$ ein größtes (beziehungsweise kleinstes) Element von $M$, dann ist $a$ ein maximales (beziehungsweise minimales) Element von $M$.
\subparagraph{Beweis}
\label{sec-2-5-5-2-1}
Sei $a$ ein größtes Element von $M$. \\
      zu zeigen ist: Für alle $x\in M$ gilt $a\leq x \implies a = x$
Sei $x\in M$ mit $a\leq x$. Da $a$ größtes Element von $M$ ist, gilt auch $x\leq a$ \\
      $\xLeftrightarrow{\text{Antisymmetrie}} a = x$ \\
      Analog für kleinstes Element.
\subsubsection{Äquivalenzrelation}
\label{sec-2-5-6}
$M$ Menge, $\sim$ auf $M$ \\
    $\sim$ heißt Äquivalenzrelation $\xLeftrightarrow{\text{Def}}~\sim$ ist reflexiv, symmetrisch und transitiv.
In dem Fll sagen wir für $a\sim b$ auch $a$ ist äquivalent zu $b$. Für $a\in M$ heißt $[a]:=\{b\in M \mid b\sim a\}$ heißt die Äquivalentklasse von $a$.
Elemente aus $[a]$ nennt man Vertreter oder Repräsentanten von $a$
\paragraph{Beispiel}
\label{sec-2-5-6-1}
$M$ Menge aller Bürgerinnen und Bürger Deutschlands. \\
     Wir definieren für $a,b\in M$ $a\sim b \xLeftrightarrow{\text{Def}} a$ und $b$ sind im selben Jahr geboren. \\
     $\sim$ ist ein Äquivalenzrelation. \\
     Jerôme Boateng wurde 1988 geboren. \\
     $[\text{Jerôme Boateng}] = \{b\in M\mid b~\text{ist im selben Jahr geboren wie Jerôme Boateng}\} = \{b\in M\mid b~\text{wurde 1988 geboren}\}$
Weitere Vertreter von $[\text{Jerôme Boateng}]$ sind zum Beispiel Mesut Özil, Mats Hummels.
Es ist $[\text{Jerôme Boateng}] = [\text{Mesut Özil}] = [\text{Mats Hummels}]$.
Man sieht in diesem Beispiel: Die Menge $M$ zerfällt komplett in verschiedene Äquivalentzklassen:
\begin{itemize}
\item Jeder Bürger / jede Bürgerinn Detuschalnds ist in genau einer Äquivalenzklasse enthalten
\item Jede zwei Äquivalentklasse sind entweder gleich oder disjunkt (haben leeren Durchschnitt)
\end{itemize}
\paragraph{Bemerkung}
\label{sec-2-5-6-2}
$M$ Menge, $\sim$ Äquivalenzrelation auf $M$ \\
     Dann gilt:
\begin{enumerate}
\item Jedes Element von $M$ liegt in genau einer Äquivalenzklasse
\item Je zwei Äquivalenzklassen sind entweder gleich oder disjunkt
\end{enumerate}
Man sagt auch: Die Äquivalenzklassen bezüglich "$\sim$" bilden eine \textbf{Partition} von $M$.
\subparagraph{Beweis}
\label{sec-2-5-6-2-1}
\begin{enumerate}
\item Sei $a\in M$ \\
         zu zeigen: Es gibt genau eine Äquivalenzklassen, in der $a$ liegt
\begin{enumerate}
\item Es gibt eine Äquivalenzklasse, in der $a$ liegt, denn \$a$\in$ [a], denn $a\sim a$
\item Ist \$a$\in$[b] und a$\in$[c], dann ist [b]=[c] (d.h. $a$ liegt in höchstens einer Äquivalenzklasse) \\
            denn: Seien $b,c\in M$ mit $a\in[b]$ und $a\in[c]$
            $\implies a\sim b$ und $a\sim c \xRightarrow{\text{Symmetrie}} b\sim a$ und $a\sim c \xRightarrow{\text{Transitivität}} b\sim c$
            Behautptung $[b] =[c]$
            denn: "$\subseteq$" Sei $x\in [b] \implies x\sim b \xRightarrow{Transitivität}^{b\sim c} x\sim c \implies x\in [c]$
            denn: "$\supseteq$" Sei $x\in [c] \implies x\sim c \xRightarrow{Transitivität}^{c\sim b} x\sim b \implies x\in [b]$
\end{enumerate}
\item Sind $b,c\in M$ mit $[b] \cap [c] \neq \emptyset$, dann existiert ein \$a$\in$ [b]$\cap$ [c], und es folgt wie in 2.: \\
         $[b] = [c]$
         Für $b,c\in M$ gilt also entweder $[b]\cap[c] =\emptyset$ oder $[b] = [c]\hfill\square$
\end{enumerate}
\paragraph{Faktormenge}
\label{sec-2-5-6-3}
$M$ Menge, $\sim$ Äquivalenzrelation auf $M$
$M/\sim := \{[a]|a\in M\}$ (Menge der Äquivalenzklassen) heißt die Faktormenge (Quotientenmenge) von $M$ nach $\sim$
\subparagraph{Beispiel}
\label{sec-2-5-6-3-1}
\[M= \{1,2,3,-1,-2,-3\}\]
Für $a,b,c \in M$ setzen wir $a\sim b \xLeftrightarrow{\text{Def.}} \abs{x} = \abs{b}$
Das ist eine Äquivalenzrelation auf $M$
Es ist $[1] = \{1,-1\},[2]=\{2,-2\},[3]=\{3,-3\}$
Somit: $M/sim := \{[1],[2],[3]\} = \{\{1,-1\},\{2,-2\},\{3,-3\}\}$
\subparagraph{Anmerkung}
\label{sec-2-5-6-3-2}
Der Übergang zur Äquivalenzklassen soll (für eine jeweils gegebene Relation) irrelevante Informationen abstreifen.
\subsection{Abbildungen}
\label{sec-2-6}
\textbf{naive Definition}: \\
   Eine Abbildung $f$ von $M$ nach $N$ ist eine Vorschrift, die jedem $n\in M$ genau ein Element aus $N$ zuordnet, dieses wird mit $f(n)$ bezeichnet.
\textbf{Notation}: \\
   \[f:M\to N,m\mapsto f(m)\]

Zwei Abbildungen $f,g:M\to N$ sind gleich, wenn gilt $\Forall n\in M:f(n) = g(n)$
$M$ heißt die Definitionsmenge von $f$, $N$ heißt die Zielmenge von $f$
\subsubsection{Definition}
\label{sec-2-6-1}
Eine Abbildung $f$ von $M$ nach $N$ ist ein Tupel $(M,N,G_f)$, wobei $G_f$ eine Teilmenge von $M\times N$ mit der Eigenschaft ist, dass für jedes Element $m\in M$ genau ein Element $n\in N$ mit $(m,n) \in G_f$ existiert.
(für dieses Element $n$ schreiben wir auch $f(m)$). $G_f$ heißt der Graph von $f$.
\subsubsection{Beispiel}
\label{sec-2-6-2}
\begin{enumerate}
\item $f:\mathbb{R}\to\mathbb{R}, x\mapsto x^2$
\item $f:\mathbb{R}\to\mathbb{R}^2,x\mapsto (x,x+1)$
\item $M$ Menge, $id_M: M\to M,m\mapsto m$ heißt Identität (identische Abbildung) auf $M$
\item $I$,$M$ Mengen: Eine über $I$ indizierte Familie von Elementen von $M$ ist eine Abbildung: \\
       $m:I\to M,i\mapsto m(i) =: m_i$. Wir schreiben für die Familie auch kurz $(m_i)_{i\in I}$. $I$ heißt Indexmenge der Familie.
\item Spezialfall von 4.: $I = \mathbb{N},M = \mathbb{R}:~((m_i)_{i\in\mathbb{N}})$ nennt man auch Folge reeler Zahlen.
\end{enumerate}
\subsubsection{Anmerkung über den Begriff der Familie}
\label{sec-2-6-3}
Über den Begriff der Familie lassen sich diverse Konstruktionen aus der naiven Mengenlehre verallgemeinern.
Ist $(M_i)_{i\in I}$ eine Familie von Mengen, dann ist:
\[\cup_{i\in I} M_i:=\{x\mid\exists i\in I: x\in M_i\}\]
\[\cap_{i\in I}M_i := \{x\mid\Forall i\in I: x\in M_i\}\]
\[\prod_{i\in I}M_i := \{(x_i)_{i\in I}\mid \Forall i\in I: x_i \in M\}\]
\subsubsection{Bild}
\label{sec-2-6-4}
$m,N$ Mengen, $f:M\to n$ Abbildung. \\
    Sind $m\in M,n\in N$ mit $n = f(m)$ dann nennen wir $n$ ein \textbf{Bild} von $m$ unter $f$ und wir nennen $m$ ein \textbf{Urbild} von $n$ unter $f$.
\paragraph{Anmerkung}
\label{sec-2-6-4-1}
In obiger Situation ist das Bild von $m$ unter $f$ eindeutig bestimmt (nach der Definition einer Abbildung)
Urbilder sind im allgemeinen nicht eindeutig bestimmt, und im Allgemeinen besitzt nicht jedes Element aus $N$ ein Urbild.
\paragraph{Beispiel}
\label{sec-2-6-4-2}
$f:\mathbb{R}\to\mathbb{R},x\mapsto x^2$, dann ist $4=f(2) = f(-2)$, das heißt $2$ und $-2$ sind Urbilder von $4$, das Element $-5$ hat kein Urbild unter $f$, denn es existiert kein $x\in\mathbb{R}$ mit $x^2 = -5$
\paragraph{Definition}
\label{sec-2-6-4-3}
$M, N$ Mengen, $f:M\to N$ Abbildung, $A\subseteq M, B\subseteq N$ \\
     $f(A) := \{f(a)\mid a\in A\} \subseteq N$ heißt das Bild von $A$ unter $f$. \\
     $f^-1(B) := \{m\in M\mid f(m) \in B\} \subseteq M$ heißt das Urbild von $B$ unter $f$
\subparagraph{Beispiel}
\label{sec-2-6-4-3-1}
\[f:\mathbb{R}\to\mathbb{R},x\mapsto x^2\]
\[f(\{1,2,3\}) = \{1,4,9\}\]
\[f^-1(\{4,-5\}) = \{2,-2\}\]
\[f^-1(\{4\}) = \{2,-2\}\]
\[f^-1(\{-5\}) = \emptyset\]
\[f(\mathbb{R}) = {x^2\mid x\in \mathbb{R}} = \{x\in\mathbb{R}\mid x\geq 0\} =:\mathbb{R}_{\geq 0}\]
\subsubsection{Restriktion}
\label{sec-2-6-5}
$M,N$ Mengen, $f:M\to N$ Abbildung, $A\subseteq M$
\[f\mid_A:A\to N, m\mapsto f(m)\]
heißt die Restriktion von $f$ auf $A$.
Ist $B\subseteq N$ mit $f(A) \subseteq B$, dann setzen wir
\[f\mid_A^B: A\to B,m\mapsto f(m)\]
Ist $f(M) \subseteq B$ dann setzen wir:
\[f\mid^B := f\mid_M^B,M\to B, m\mapsto f(m)\]
\subsubsection{Komposition}
\label{sec-2-6-6}
$L,M,N$ Mengen, $f:L\to M,g:M\to N$ Abbildung \\
    \[g\circ f: L\to N, x\mapsto(g\circ f)(x):=g(f(x))\]
heißt die Komposition (Hintereinanderausführung) von $f$ und $g$
\paragraph{Beispiel}
\label{sec-2-6-6-1}
\[f:\mathbb{R}\to\mathbb{R},x\mapsto x^2, g:\mathbb{R}\to\mathbb{R}:x\mapsto x + 1\]
\[\implies g\circ f:\mathbb{R}\to\mathbb{R},x\mapsto g(f(x)) = g(x^2) = x^2 + 1\]
\paragraph{Assoziativität}
\label{sec-2-6-6-2}
$L,M,N,P$ Mengen, $f:L\to M, g:M\to N,h:n\to p$ \\
     Dann gilt
\[h\circ (g\circ f) = (h\circ g)\circ f\]
das heißt die Verknüpfung von Abbildungen ist assoziativ.
\subparagraph{Beweis}
\label{sec-2-6-6-2-1}
Für $x\in L ist$ \\
      \[(h\circ (g\circ f)) = h((g\circ f)(x)) = h(g(f(x))) = (h\circ g)(f(x)) = ((h\circ g)\circ f)(x)\hfill\square\]
\subsubsection{Eigenschaften von Abbildungen}
\label{sec-2-6-7}
$M,N$ Mengen, $f:M\to N$ Abbildung
\paragraph{Injektivität}
\label{sec-2-6-7-1}
$f$ heißt injektiv: \[\xLeftrightarrow{\text{Def}} \Forall m_1,m_2\in M: f(m_1) = f(m_2) \implies m_1 = m_2 \iff \Forall m_1,m_2\in M: m_1\neq m_2 \implies f(m_1)\neq f(m_2)\]
\paragraph{Surjektivität}
\label{sec-2-6-7-2}
$f$ heißt sujektiv:
\[\xLeftrightarrow{\text{Def}} \Forall n\in M :\exists m\in M: f(m) = n \iff f(M) = N\]
\paragraph{Bijektivität}
\label{sec-2-6-7-3}
$f$ heißt bijektiv: $\xLeftrightarrow{\text{Def}}$ $f$ ist injektiv und surjektiv
\paragraph{Beispiel}
\label{sec-2-6-7-4}
\begin{enumerate}
\item $f:\mathbb{R}\to\mathbb{R},x\mapsto x^2$ ist:
\begin{itemize}
\item nicht injektiv, denn $f(2) = f(-2)$, aber $2\neq -2$
\item nicht surjektiv, denn es existier kein $m\in\mathbb{R}$ mit $f(m) = -1$
\item nicht bijektiv
\end{itemize}
\item $f:\mathbb{R}_{\geq 0} \to \mathbb{R}, x\mapsto x^2$ ist:
\begin{itemize}
\item injektiv, denn für $m_1,m_2 \in\mathbb{R}_{\geq 0}$ gilt: $f(m_1) = f(m_2) \implies m_1^2 = m_2^2 \xRightarrow{m_1,m_2 > 0} m_1 = m_2$
\item nicht surjektiv, denn es existier kein $m\in\mathbb{R}_{\geq 0}$ mit $f(m) = -1$
\item nicht bijektiv
\end{itemize}
\item $f:\mathbb{R}_{\geq 0} \to \mathbb{R}_{\geq 0}, x\mapsto x^2$ ist:
\begin{itemize}
\item injektiv, denn für $m_1,m_2 \in\mathbb{R}_{\geq 0}$ gilt: $f(m_1) = f(m_2) \implies m_1^2 = m_2^2 \xRightarrow{m_1,m_2 > 0} m_1 = m_2$
\item surjektiv, denn für $m\in\mathbb{R}_{\geq 0}$ ist $f(\sqrt{m}) = (\sqrt{m})^2 = m$
\item bijektiv
\end{itemize}
\end{enumerate}
\paragraph{Bemerkung 4.12}
\label{sec-2-6-7-5}
$M,N$ Mengen, $f:M\to N, g:n\to M$ mit $g\circ f = id_M$
Dann ist $f$ injektiv und $g$ surjektiv.
\subparagraph{Beweis}
\label{sec-2-6-7-5-1}
\begin{enumerate}
\item $f$ ist injektiv, denn: \\
         Seien $m_1, m_2 \in M$ mit $f(m_1) = f(m_2) \implies g(f(m_1)) = g(f(m_2)) \implies (g\circ f)(m_1) = (g\circ f)(m_2) \implies id_m(m_1) = id_M(m_2)\implies m_1 = m_2$
\item $g$ ist surjektiv, denn: \\
         Sei $m\in M$ Dann ist $m=id_M(m) = (g\circ f)(m) = g(f(m))$
\end{enumerate}
\paragraph{Bemerkung}
\label{sec-2-6-7-6}
Sei $f:M\to N$, $N,M$ Mengen
Dann sind äquivalent:
\begin{enumerate}
\item $f$ ist bijektiv
\item Zu jedem $n\in N$ gibt es genau ein $m\in M$ mit $f(m) = n$
\item Es gibt genau eine Abbildung $g:N\to M$ mit $g\circ f = id_M$ und $f\circ g = id_N$
\end{enumerate}
In diesem Fall bezeichnen wir die Abbildung $g:N\to M$ aus 3. mit $f^{-1}$ und nennen $f^{-1}$ die Umkehrabbildung von $f$. Sie ist gegeben durch
\[f^{-1} : N\to M, n\mapsto~\text{Das eindeutig bestimmte Element $m\in M$ mit $f(m) = n$}\]
\subparagraph{Beweis}
\label{sec-2-6-7-6-1}
Statt 1. $\iff$ 2. und 2. $\iff$ 3. zeigen 1. $\implies$ 2. $\implies$ 3. $\implies$ 1.
\begin{itemize}
\item 1. $\implies$ 2. Sei $f$ bijektiv \\
        zz: Ist $n\in N$, dann existiert genau ein $m\in M$ mit $f(m) = n$ \\
\begin{itemize}
\item Existenz folgt aus Surjektivität von $f$
\item Eindeutigkeit: Seien \$m$_{\text{1}}$,m$_{\text{2}}$ $\in$ M  mit \(f(m_1) = n, f(m_2) = n \implies f(m_1) = f(m_2) \xRightarrow{f injektiv} m_1 = m_2\)
\end{itemize}
\item 2. $\implies$ 3. Zu jedem $n\in M$ existiere genau ein $m\in M$ mit $f(m) = n$ \\
        zz: Ex existert genau eine Abbildung $g:N\to M$ mit $f\circ f = id_M$ und $f\circ g = id_N$
\begin{itemize}
\item Existenz: Wir definieren \(g:N\to M, n\mapsto~\text{das nach 2. eindeutig \\
		  bestimmte Element $m\in M$ mit $f(m) = n$}\) \\
Dann gilt für $m\in M$: \[(g\circ f)(m) = f(f(m)) = m,~text{das heißt}~ g\circ f = id_M\]
und für $n\in N$ ist $(f\circ g)(n) = f(g(n)) = n$ also $f\circ g = id_N$
\item Eindeutigkeit: Es seinen $g_1,g_2:N\to M$ mit $g_i \circ f = id_M, f\circ g_i = id_N$ für $i = 1,2$ \\
          \[\implies g_1 = g_1 \circ id_N = g_1 \circ (f\circ g_2) = (g_1 \circ f) \circ g_2 = id_M \circ g_2 = g_2\]
\end{itemize}
\item 3. $\implies$ 1. Wegen 3. existier $g:N\to M$ mit $g\circ f = id_M,f\circ g = id_N$ \\
        \[\xRightarrow{[[Bemerkung 4.12]]} f~\text{injektiv}~,f~\text{surjektiv}~\implies f~\text{bijektiv}\implies~\text{1.}\]
\end{itemize}
\subparagraph{Anmerkung}
\label{sec-2-6-7-6-2}
\begin{itemize}
\item Bitte stets aufpassen, ob mit $f^{-1}$ die Unmkerhabbildung (falls existent) oder das Bilden der Urbildmenge gemeint ist.
\item Im Beweis von 3. $\implies$ 1. haben wir die Eindeutigkeit von $g$ garnicht verwendet, das heißt wir haben sogar gezeigt: \\
        $f$ bijektiv $\iff$ 3.' Es existiert eine Abbildung $g:N\to M$ mit $f\circ g = id_N$ und $f\circ f = id_M$ Soch eine Abbildung $g$ ist in diesem Fall automatisch bestimmt.
\end{itemize}
\subparagraph{Beispiel}
\label{sec-2-6-7-6-3}
Im Beispiel vorher haben wir gesehen $f:\mathbb{R}_{\geq 0} \to \mathbb{R}_{\geq 0}, x\mapsto x^2$ ist bijektiv.
Die Umkehrabbildung ist gegeben durch $f^{-1}:\mathbb{R}_{\geq 0} \to \mathbb{R}_{\geq 0}, x\mapsto \sqrt{x}$
\paragraph{Bemerkung}
\label{sec-2-6-7-7}
$M,N$ Mengen, $f:M\to N$ Dann gilt:
\begin{enumerate}
\item $f$ injektiv $\iff$ Es existiert $g:N\to M$ mit $g\circ f = id_M$ \\
        \textbf{Beweis:}
\begin{itemize}
\item "$\impliedby$" folgt aus \ref{sec-2-6-7-5}
\item "$\implies$" Sei $f$ injektiv. Sein $x$ ein beliebiges Element aus $M$
          Wir definieren \[g:N\to M,n\mapsto \begin{cases} x & n\not\in f(M) \\ \text{das eindeutig bestimmte Element $m\in M$ mit $f(m) = n$} & n\in f(M) \end{cases}\]
          Für alle $m\in M$ ist dann $(g\circ f)(m) = g(f(m)) = m$ das geißt $g\circ f = id_M$
\end{itemize}
\item $f$ surjektiv $\iff$ Es existiert $g:N\to M$ mit $f\circ g = id_N$ \\
        \textbf{Beweis:}
\begin{itemize}
\item "$\impliedby$" folgt aus \ref{sec-2-6-7-5}
\item "$\implies$" Sei $f$ surjektiv. Für jedes Element $n\in N$ wählen wir ein Element $\tilde n\in f^{-1}(\{n\}) \neq \emptyset$ und sehen
$g:N\to M,n\mapsto \tilde n$. Dann ist $(f\circ g)(n) = f(g(n)) = n$ für alle $n\in N$ und das heißt $f\circ g = id_N \hfill\square$
\end{itemize}
\end{enumerate}
\subparagraph{Anmerkung}
\label{sec-2-6-7-7-1}
Das wir stets einen Auswahlprozess wie im Beweis von 2. "$\implies$" vornehmen können ist ein Axiom der Mengenlehre (erkennen wir als gültig an, ist jedoch nicht beweisbar), das \textbf{Auswahlaxiom}: \\
      Ist $I$ eine Indexmenge und $(A_i)_{i\in I}$ eine Familie von nichtleeren Mengen, dann gibt es eine Abbildung $\gamma:I\to \bigcup_{i\in I} A_i$ mit $\gamma (i) \in A_i$ für alle $i\in I$ (im obigen Beweis ist $I = N,A_n = f^{-1}(\{n\})$ für $n\in N$)
\paragraph{Bemerkung 4.16}
\label{sec-2-6-7-8}
$L,M,N$ Mengen, $f:L\to M, g:M\to N$ \\
     Dann gilt: $g,f$ beide injektiv (beziehungsweise surjektiv oder bijektiv) $\implies g\circ f$ injektiv (beziehungsweise sujektiv oder bijektiv)
\paragraph{Definition 4.17}
\label{sec-2-6-7-9}
\paragraph{Bemerkung 4.19}
\label{sec-2-6-7-10}
$M,N$ endliche Mengen mit $\abs{M} = \abs{N},f:M\to N$ Dann sind äquivalent:
\begin{enumerate}
\item $f$ ist injektiv
\item $f$ ist surjektiv
\item $f$ ist bijektiv
\end{enumerate}
\subparagraph{Beweis}
\label{sec-2-6-7-10-1}
\begin{itemize}
\item 1. $\implies$ 2. Sei $f$ injektiv $\implies$ $\abs{f(M)} = \abs{M} = \abs{N}$ wegen $f(M) \subseteq N$ folgt $f(M) = N \implies f$ surjektiv
\item 2. $\implies$ 3. Sei $f$ sujektiv, das heißt $f(M) = N$ \\
        Annahme: $f$ ist nicht bijektiv $\implies f$ nicht injektiv $\implies \exists m_1,m_2\in M: m_1\neq m_2 \wedge f(M_1) = f(m_2) \implies \abs{f(M)} < \abs{M} = \abs{N}$ Wiederspruch zu $f(M) = N$
\item 3. $\implies$ 1. trivial
\end{itemize}
\section{Gruppen, Ringe, Körper}
\label{sec-3}
\subsection{Gruppe}
\label{sec-3-1}
\subsubsection{Verknüpfung}
\label{sec-3-1-1}
$M$ Menge, Eine Verknüpfung (inverse Verknüpfung) auf $M$ ist ein Abbildung \[*:M\times M \to M\]
Anstelle von $*(a,b)$ schreiben wir $a * b$
\paragraph{Beispiel}
\label{sec-3-1-1-1}
\begin{itemize}
\item $+: \mathbb{R} \times \mathbb{R} \to \mathbb{R},(a,b) \mapsto a + b$
\item $\cdot: \mathbb{R} \times \mathbb{R} \to \mathbb{R},(a,b) \mapsto a\cdot b$
\end{itemize}
sind Verknüpfungen
\subsubsection{Monoid}
\label{sec-3-1-2}
Ein Monoid ist ein Tupel $(M,*)$, bestehend aus einer Menge $M$ und einer Verküpfung \\
    $*:M\times M \to M$, welche folgende Bedingungen genügt:
\begin{itemize}
\item (M1) Die Verküpfung ist assoziativ, das heißt \[\Forall a,b,c\in M:(a*b)*c = a*(b*c) \]
\item (M2) Ex existiert ein neutrales Element $e$ in $M$, das heißt \[\exists e\in M:\Forall a\in M e*a = a = a*e\]
\end{itemize}
\paragraph{Beispiel}
\label{sec-3-1-2-1}
\begin{itemize}
\item $(\mathbb{N}_0,+), (\mathbb{Z},+)$ sind Monoide (neutrales Element: $0$)
\item $(\mathbb{N},+)$ ist kein Monoid (ex existiert kein neutrales Element)
\item $(\mathbb{N},\cdot),(\mathbb{Z},\cdot)$ sind Monoide (neutrales Element: $1$)
\end{itemize}
\paragraph{Bemerkung}
\label{sec-3-1-2-2}
$(M,*)$ Monoid. Dann gibt es in $M$ genau ein neutrales Element.
\subparagraph{Beweis}
\label{sec-3-1-2-2-1}
\begin{itemize}
\item Existenz: Es existert ein neutrales Element: folgt aus Definition eines Monoids
\item Eindeutigkeit: Seien $e,\tilde e \in M$ neutrale Element \[\implies e = e * \tilde e = \tilde e\]
\end{itemize}
\subsubsection{Inverses}
\label{sec-3-1-3}
$(M,*)$ Monoid mit neutralem Element $e$, $a\in M$
Ein Element $b\in M$ heit Inverses zu $a \xLeftrightarrow{\text{Def}} a * b = e = b * a$
\paragraph{Beispiel}
\label{sec-3-1-3-1}
\begin{itemize}
\item In $(\mathbb{Z},+)$ ist $-2$ ein Inverses zu $2$ denn $2 + (-2) = 0 = (-2) + 2$
\item In $(\mathbb{N}_0,+)$ existiert kein Inverses zu $2$, denn es existiert kein $n\in \mathbb{N}_0$ mit $n + n = 0 = n + 2$
\item \label{invex} In $(\mathbb{Z},\cdot)$ existiert kein Inverses zu $2$, denn es existiert kein $n\in\mathbb{Z}$ mit $2\cdot n = 1 = n \cdot 2$
\end{itemize}
\paragraph{Bemerkung}
\label{sec-3-1-3-2}
\label{beminv}
$(M,*)$ Monid, $a\in M$ Dann gilt: besitzt $a$ ein Inverses, dann ist dieses eindeutig bestimmt.
\subparagraph{Beweis}
\label{sec-3-1-3-2-1}
Seinen $b,\tilde b$ Inversen zu $a$, sein $e\in M$ das neutrale Element
\[\implies b = e * b = (\tilde b * a) * b = \tilde b * (a * b) = \tilde b\]
\subsubsection{Gruppe}
\label{sec-3-1-4}
Eine Gruppe ist ein Tupel $(G,*)$, bestehen aus einer Menge $G$ und einer Verknüpfung $*:G\times G \to G$, sodass gilt:
\begin{itemize}
\item (G1) $(G,*)$ ist ein Monoid
\item (G2) Jedes Element aus $G$ besitzt ein Inverses
\end{itemize}
In diesem Fall schreiben wir $a'$ für das nach \ref{beminv} eindeutig bestimmte Inverse eines Elements $a\in G$
\paragraph{Beispiel}
\label{sec-3-1-4-1}
\begin{itemize}
\item $(\mathbb{Z},+)$ ist eine Gruppe, denn $(\mathbb{Z},+)$ ist ein Monoid und für $a\in\mathbb{Z}$ ist $-a$ das inverse Element: $a + (-a) = 0 = (-a) + a$
\item $(\mathbb{Z},\cdot)$ ist keine Gruppe, denn das Element $2\in\mathbb{Z}$ hat kein Inverses (vergleiche \ref{invex}).
\item $(\mathbb{Q}\setminus \{0\},\cdot)$ ist eine Gruppe denn es ist ein Monoid mit neutralem Element $1$ und für jedes Element $a\in\mathbb{Q}\setminus\{0\}$ existiert ein $b\in \mathbb{Q}\setminus \{0\}$ mit $a\cdot b = 1 = b\cdot a$, nämlich $b = \frac{1}{a}$
\end{itemize}
\paragraph{Bemerkung 5.11}
\label{sec-3-1-4-2}
$(G,*)$ Gruppe mit neutralem Element $e,a,b,c \in G$. Dann gilt
\begin{enumerate}
\item (Kürzungsregel) \[a*b = a*c \implies b = c\] \[a*c = b * c \implies a = b\]
\item $a*b = e \implies b = a'$
\item $(a')' = a$
\item (Regel von Hemd und Jacke) $(a*b)' = b' * a'$
\end{enumerate}
\subparagraph{Beweis}
\label{sec-3-1-4-2-1}
\begin{enumerate}
\item Sei $a * b = a * c \implies a'*(a*b) = a'*(a*c) \implies (a'*a)*b=(a'*a)*c \implies e*b = e*c \implies b = c$
\item aus 1. $a*b = c = a*a' \implies b = a'$
\item Es ist $a*a' = e = a' * a$, das heißt $a$ ist Inverses zu $a' \implies (a')' = a$
\item Es ist $(a*b)*(b'*a') = a*(b*b')*a' = a*a' = e \implies b' * a' \xRightarrow{\text{2.}} (a *b)'$
\end{enumerate}
\subsubsection{Abelsche Gruppe}
\label{sec-3-1-5}
$(M,*)$ Monoid / Gruppe heißt kommutativ (abelsch)
\[\xLeftrightarrow{\text{Def}} \Forall a,b\in M: a*b = b*a\]
\paragraph{Beispiel}
\label{sec-3-1-5-1}
Alle bisher betrachteten Beispiele von Monoiden beziehungsweise Gruppen sind abelsch
\paragraph{Bemerkung 5.14}
\label{sec-3-1-5-2}
$M$ Menge, Wir setzten $S(M):= \{f:M\to M | f~\text{bijektiv}\}$
Dann ist $(S(M),\circ)$ eine Gruppen, die \textbf{symmetrische} Gruppe auf $M$
\subparagraph{Beweis}
\label{sec-3-1-5-2-1}
\begin{enumerate}
\item "\$\^{}" ist wohl definiert, das heißt für $f,g\in S(M)$ ist $f\circ g \in S(M)$ folgt aus \ref{sec-2-6-7-10}
\item "\$\^{}" ist assoziativ $f\circ(g\circ h) = (f\circ g) \circ h \Forall f,g\in S(M)$ nach \texttt{4.9}
\item $id_M$ ist neutral: $id_M \in S(M)$ und $id_M\circ f = f = f\circ id_M \Forall f\in S(M)$
\item Existenz von Inversen: $f\in S(M) \implies f$ bijektiv $\implies$ Es existiert Umkehrabbildung $f^{-1}\in S(M)$ zu $f$
         für diese gilt: $f\circ f^{-1} = id_M = f^{-1}\circ f$ das heißt $f^{-1}$ ist immer zu $f$ bezüglich "\$\^{}"
\end{enumerate}
\subsubsection{Permutationen}
\label{sec-3-1-6}
$n\in\mathbb{N}$
\[S_n := S(\{1,\ldots,n\}) = \{\pi \{1,\ldots,n\} \to \{1,\ldots,n\} \mid \pi~\text{ist bijektiv}\}\]
$(S_n,\circ)$ heißt die symmetrische Gruppe auf $n$ Ziffern, Elemente aus $S_n$ heißen Permutationen.
Wir schreiben Permutationen $\pi \in S_n$ in der Form:
\begin{equation}
\pi =
\begin{pmatrix}
1 & 2 & \ldots & n \\
\pi(1) & \pi(2) & \ldots & \pi(n)
\end{pmatrix}
\end{equation}
\paragraph{Beispiel}
\label{sec-3-1-6-1}
In $S_3$ ist
\begin{equation}
\begin{pmatrix}
1 & 2 & 3 \\
1 & 3 & 2 \\
\end{pmatrix}
\circ
\begin{pmatrix}
1 & 2 & 3 \\
3 & 2 & 1 \\
\end{pmatrix}
=
\begin{pmatrix}
1 & 2 & 3 \\
2 & 3 & 1 \\
\end{pmatrix}
\end{equation}

\begin{equation}
\begin{pmatrix}
1 & 2 & 3 \\
3 & 2 & 1 \\
\end{pmatrix}
\circ
\begin{pmatrix}
1 & 2 & 3 \\
1 & 3 & 2 \\
\end{pmatrix}
=
\begin{pmatrix}
1 & 2 & 3 \\
3 & 1 & 2 \\
\end{pmatrix}
\end{equation}
das heißt $(S_3,\circ)$ ist nicht abelsch.
\subsubsection{Restklassen}
\label{sec-3-1-7}
\paragraph{Motivation}
\label{sec-3-1-7-1}
Im täglischen Leben verwendet man zur Bestimmung von Uhrzeiten das Rechnen "modulo $24$", zum Beispiel 22Uhr + 7h = 5Uhr. Wir wollen dies mathematisch präzisieren und verallgemeinern
\paragraph{Bemerkung 5.17}
\label{sec-3-1-7-2}
$n\in\mathbb{N}$. Dann ist durch \[a \sim b \xLeftrightarrow{\text{Def}} \Exists q\in\mathbb{Z}:a - b = q n\]
eine Äquivalenzrelatioin auf $\mathbb{Z}$ gegeben.
Anstelle von $a\sim b$ schreiben wir auch $a\equiv b(\mod n)$ ("$n$ ist kongruent $b$ modulo $n$")
Die Äquivalenzklasse von $a\in \mathbb{Z}$ ist durch
\[\bar a := \{b\in\mathbb{Z}\mid b\equiv a(\mod n)\} = a + n\mathbb{Z} := \{a + n q \mid q\in \mathbb{Z}\}\]
gegeben und heißt die Restklasse von $a$ modulo $n$.
Die Menge aller Restklassen modulo $n$ wird $\frac{\mathbb{Z}}{n\mathbb{Z}}$ bezeichnet ("$\mathbb{Z}$ modulo $n\mathbb{Z}$")
Es ist: \[\frac{\mathbb{Z}}{n\mathbb{Z}} = \{\bar 0, \bar 1, \ldots, \overline{n - 1}\}\]
und die Restklassen $\bar 0, \ldots, \overline{n - 1}$ sind paarweise verschieden
\subparagraph{Beweis}
\label{sec-3-1-7-2-1}
\begin{enumerate}
\item "$\equiv$" ist eine Äquivalenzrelation, denn:
\begin{itemize}
\item "$\equiv$" ist reflexiv: Fpr $a\in\mathbb{Z}$ ist $a\equiv a(\mod n)$ denn $a - a = 0 = 0 n$
\item "$\equiv$" ist symmetrisch: Seien $a,b\in\mathbb{Z}$ mit $a\equiv b(\mod n) \exists q\in\mathbb{Z}:a - b = q n$ \\
           $\implies b - a = (-q) n \implies b \equiv a(\mod n)$
\item "$\equiv$" ist transitiv: Seien $a,b,c\in\mathbb{Z}$ mit $a\equiv b(\mod n), b\equiv c(\mod n)$
\begin{itemize}
\item $\implies \exists q_1,q_2 \in\mathbb{Z}$ mit $a - b = q_1 n, b - c = q_2 n$
\item $\implies a - c = (a - b) + (b - c) = q_1 n + q_2 n = (q_1 + 1_2) n \implies a \equiv c(\mod n)$
\end{itemize}
\end{itemize}
\item Die Äquivalenzklasse von $n\in\mathbb{Z}$ ist gegeben durch
\[\{b\in\mathbb{Z} \mid b = a(\mod n)\}\]
\[= \{b\in\mathbb{Z} \mid \exists q\in\mathbb{Z}:b - a = qn\}\]
\[= \{b\in\mathbb{Z} \mid \exists q\in\mathbb{Z}:b = a + q n\}\]
\[= a + n\mathbb{Z} \]
\item \[\frac{\mathbb{Z}}{n\mathbb{Z}} = \{\bar 0, \bar 1, \ldots, \overline{n - 1}\}\]
         denn:
\begin{itemize}
\item Ist $a\in\mathbb{Z}$ beliebig, dann liefert Division mit Rest durch $n$: \\
           Es gibt $q,r\in\mathbb{Z}$ mit $a = q n + r,0\leq r < n$
           \[\implies a - r = q n \implies q \equiv r(\mod n) \implies \bar a = \bar r\]
           Das heißt: Jede Restklasse ist von der Form $\bar r$ mit $r\in \{0,\ldots,n - 1\}$ \\
\item Die Restklassen $\bar 0, \bar 1, \ldots, \overline{n - 1}$ sind paarweise verschieden denn: \\
           Seien $a,b\in\{0,\ldots,n - 1\}$ mit $\bar a = \bar b \implies a \equiv b(\mod n) \implies \exists q\in \mathbb{Z}: a - b = q n \implies \abs{a - b} = \abs{q} n$.
\begin{itemize}
\item Wäre $q\neq 0$, dann $\abs{q} \geq 1$ wegen $q\in\mathbb{Z} \implies \abs{a - b} \geq n$ \textbf{Wiederspruch} zu $a,b\in\{0,\ldots,n - 1\}$ \\
             Also: $q = 0$ das heißt $a = b$
\end{itemize}
\end{itemize}
\end{enumerate}
\paragraph{Beispiel}
\label{sec-3-1-7-3}
$n = 3: a\equiv b (\mod 3) \iff \exists q\in\mathbb{Z}: a - b = 3 q$ \\
     zum Beispiel: $11 \equiv 5(\mod 3)$, denn $11 - 5 = 6 = 2 \cdot 3$ \\
     zum Beispiel: $7 \not\equiv 2(\mod 3)$, denn $7 - 2 = 5$ und es gibt kein $q\in\mathbb{Z}$ mit $5 = 3 q$
\[\bar 0 = \{a\in\mathbb{Z}\mid a \equiv 0 (\mod 3)\} = \{a\in\mathbb{Z} \mid  \exists q\in\mathbb{Z}: a = 3q\} = 3\mathbb{Z} = \{\ldots,-6,-3,0,3,6,\ldots\}\]
\[\bar 1 = \{a\in\mathbb{Z}\mid a \equiv 1 (\mod 3)\} = \{a\in\mathbb{Z} \mid  \exists q\in\mathbb{Z}: a - 1 = 3q\} = 1 + 3\mathbb{Z} = \{\ldots,-5,-2,1,4,7,\ldots\}\]
\[\bar 2 = \{a\in\mathbb{Z}\mid a \equiv 2 (\mod 3)\} = \{a\in\mathbb{Z} \mid  \exists q\in\mathbb{Z}: a - 2 = 3q\} = 2 + 3\mathbb{Z} = \{\ldots,-4,-1,2,5,8,\ldots\}\]
\[\bar 3 = \{a\in\mathbb{Z}\mid a \equiv 3 (\mod 3)\} = \{a\in\mathbb{Z} \mid  \exists q\in\mathbb{Z}: a - 3 = 3q\} = \{a\in\mathbb{Z}\mid \exists q\in\mathbb{Z}:a=3(q + 1)\}3\mathbb{Z} = \bar 0\]
\[\bar 4 = \bar 1,\bar 5 = \bar 2,\overline{-1} = \bar 2\]
\paragraph{Bemerkung 5.18}
\label{sec-3-1-7-4}
$n\in\mathbb{N}$ wir definieren eine Verküpfung (Addition) auf $\frac{\mathbb{Z}}{n\mathbb{Z}}$ wie folgt: \\
     Für $\bar a,\bar b \in\frac{\mathbb{Z}}{n\mathbb{Z}}$ setzen wir $\bar a + \bar b = \overline{a + b}$
Dann gilt $(\frac{\mathbb{Z}}{n\mathbb{Z}},+)$ ist eine abelsche Gruppe
\subparagraph{Beweis}
\label{sec-3-1-7-4-1}
\begin{enumerate}
\item Die Verknüpfung ist wohldefiniert: \\ Problem: Die Addition verweendet Vertreter von Restklassen. Es ist zum Beispiel in $\frac{\mathbb{Z}}{n\mathbb{Z}}: \bar 3 + \bar 4 = \overline{3 + 4} = \bar 7 = \bar 2$, aber man könnte auch Rechnen:
$\bar 3 + \bar 4 = \bar 8 + \bar 9 = \overline{8 + 9} = \overline{17} = \bar 2$ \\
         Wir müssen nachweisen, dass die Wahl der Vertreter keinen Einfluss auf das Ergebnis hat, das heißt die Verknüfung ist "vertreter unahbhängig": \\
         Seien $a_1,a_2 ,b_1,b_2 \in\mathbb{Z},\overline{a_1} = \overline{a_2},\overline{b_1} = \overline{b_2}$
\begin{align}
&\implies a_1 \equiv a_2(\mod n), b_1 \equiv b_2(\mod n) \\
&\implies\exists q_1,q_2\in\mathbb{Z}: a_1 - a_2 = q_1 n, b_1 - b_2 = q_2 n n, b_1 - b_2 = q_2 n \\
&\implies (a_1 + b_1) - (a_2 + b_2) = (a_1 - a_2)+ (b_1 - b_2) = q_1 n + q_2 n = (q_1 + q_2) n \\
&\implies a_1 + b_ 1 \equiv a_2 + b_2 (\mod n) \\
&\implies \overline{a_1 + b_1} = \overline{a_2 + b_2}
\end{align}
\item $(\frac{\mathbb{Z}}{n\mathbb{Z}})$ ist eine abelsche Gruppe:
\begin{itemize}
\item Assoziativgesetz: Für alle $a,b,c\in\mathbb{Z}$ ist
\[(\bar a + \bar b) + \bar c = \overline{a + b} + \bar c = \overline{(a + b) + c} = \overline{a + (b + c)} = \bar a + \overline{b + c} = \bar a + (\bar b + \bar c)\]
\item $\bar 0$ ist neutrales Element, denn $\Forall a\in\mathbb{Z}:\bar 0 + \bar a = \overline{0 + a} = \bar a = \bar a + \bar 0$
\item Für $a\in\mathbb{Z}$ inst $\overline{-a}$ das inverse Element zu $\bar a$, denn $\bar a + \overline{-a} = \overline{a + (- a)} = \bar 0 = \overline{-a} + \bar a$
\item Kommutativgesetz: $\Forall a,b\in\mathbb{Z}:\bar a + \bar b = \overline{a + b} = \overline{b + a} = \bar b + \bar a$
\end{itemize}
\end{enumerate}
\subparagraph{Beispiel}
\label{sec-3-1-7-4-2}
Wir tragen die Ergebnisse der Verknüpfung "$+$" in einer Verknüpfungstafel zusamme:
n = 3
\begin{center}
\begin{tabular}{llll}
$+$ & $\bar 0$ & $\bar 1$ & $\bar 2$\\
\hline
$\bar 0$ & $\bar 0$ & $\bar 1$ & $\bar 2$\\
$\bar 1$ & $\bar 1$ & $\bar 2$ & $\bar 0$\\
$\bar 2$ & $\bar 2$ & $\bar 0$ & $\bar 1$\\
\end{tabular}
\end{center}

n = 4
\begin{center}
\begin{tabular}{lllll}
$+$ & $\bar 0$ & $\bar 1$ & $\bar 2$ & $\bar 3$\\
\hline
$\bar 0$ & $\bar 0$ & $\bar 1$ & $\bar 2$ & $\bar 3$\\
$\bar 1$ & $\bar 1$ & $\bar 2$ & $\bar 3$ & $\bar 0$\\
$\bar 2$ & $\bar 2$ & $\bar 3$ & $\bar 0$ & $\bar 1$\\
$\bar 3$ & $\bar 3$ & $\bar 0$ & $\bar 1$ & $\bar 2$\\
\end{tabular}
\end{center}
\subsubsection{Gruppenhomomorphismus}
\label{sec-3-1-8}
$(G,+),(H,\oast), \varphi : G \to H$ Abbildung \\
    $\varphi$ heißt ein Gruppenhomomorphismus $\xLeftrightarrow{\text{Def}} \Forall a,b,c\in G: \varphi(a*b) = \varphi(a) \oast \varphi(b)$ \\
    $\varphi$ heißt ein Gruppenisomorphismus $\xLeftrightarrow{\text{Def}} \varphi$ ist bijektiver Gruppenhomomorphismus
\paragraph{Beispiel}
\label{sec-3-1-8-1}
\begin{enumerate}
\item $\varphi:\mathbb{Z} \to \mathbb{Z}, a\mapsto 2 a$ ist Gruppenhomomorphismus von $(\mathbb{Z},+)$ nach $(\mathbb{Z},+)$ denn:
\[\varphi( a+ b) = 2(a + b) = 2 a + 2 b = \varphi(a) + \varphi(b) \Forall a,b\in\mathbb{Z}\]
$\varphi$ ist aber kein Gruppenisomorphismus, denn $\varphi$ ist nicht surjektiv $(1\not\in \varphi = \varphi{\mathbb{Z}})$
\item $n\in\mathbb{N}$. Dann gilt $\varphi:\mathbb{Z}\to\frac{\mathbb{Z}}{n\mathbb{Z}},a\mapsto\bar a$ ist ein Gruppenhomomorphismus von $(\mathbb{Z},+)$ nach $(\frac{\mathbb{Z}}{n\mathbb{Z}}, +)$, denn
\[\Forall a,b\in\mathbb{Z}:\varphi(a+b) = \overline{a + b} = \bar a + \bar b =\varphi(a) + \varphi(b)\]
$\varphi$ ist kein Gruppenisomorphismus, denn $\varphi$ ist nicht injektiv ($\varphi(0) = \bar 0 = \bar n = \varphi(n)$, aber $0\neq n$)
\item $\varphi:\mathbb{Z}\to\mathbb{Z},a\mapsto a + 1$ ist kein Gruppenhomomorphismus von $(\mathbb{Z},+)$ nach $(\mathbb{Z},+)$, denn
\[\varphi(2 + 6) = \varphi(8) = 9,~\text{aber}~\varphi(2)+\varphi(6) = 3 + 7 = 10\]
\item $\exp:\mathbb{R}\to\mathbb{R}_{\geq 0}, x\mapsto \exp{x} = e^x$ ist ein Gruppenisomorphismus von $(\mathbb{R},+)$ nach $(\mathbb{R}_{\geq 0},\cdot)$, denn:
\begin{itemize}
\item \[\exp(a + b) = \exp(a)\exp(b) \Forall a,b\in\mathbb{R}\]
\item $exp$ ist bijektiv (vgl. Ana1 - Vorlesung)
\end{itemize}
\end{enumerate}
\paragraph{Bemerkung 5.23}
\label{sec-3-1-8-2}
$(G,*),(H,\oast)$ Gruppen mit neutralen Elementen $e_G$ beziehungsweise $e_H,\varphi:G\to H$ Gruppenhomomorphismus. Dann gilt
\begin{enumerate}
\item $\varphi(e_G) = e_H$
\item $\Forall a\in G:\varphi(a') = \varphi(a)'$ (Hierbei ist $'$ das Inverse)
\item Ist $\varphi$ Gruppenisomorphismus, dann gilt $\varphi^{-1}:H\to G$ ebenfalls Gruppenisomorphismus
\end{enumerate}
$(G,*),(H,\oast)$ heißen isomorph $\xLeftrightarrow{\text{Def}}$ Ex existert ein Gruppenisomorphismus $\phi:G\to H$ Wir schreiben dann $(G,*) \cong (H,\oast)$
\subparagraph{Beweis}
\label{sec-3-1-8-2-1}
\begin{enumerate}
\item Es $e_H\oast \varphi(e_G) = \varphi(e_G) = \varphi(e_G*e_G) = \varphi(e_G) \oast(e_G) \implies e_H = \varphi(e_G)$
\item Sei $a\in G$ Dann ist $e_H = \varphi(e_G) = \varphi(a*a') = \varphi(a)\oast(a') \implies \varphi(a') = \varphi(a)'$
\item $\varphi^{-1}$ ist bijektiv, noch zu zeigen: $\varphi^{-1}$ ist ein Gruppehomomorphismus, das heißt
\[\varphi^{-1}(c\oast d) = \varphi^{-1}(c)*\varphi^{-1}(d) \Forall c,d\in H\]
Seien $c,d\in H$ Weil $\varphi$ bijektiv: $\exists a,b\in G:\varphi(a) = c,\varphi(b) =d$
\[\implies \varphi^{-1}(c\oast d) = \varphi^{-1}(\varphi(a)*\varphi(b)) = \varphi^{-1}(\varphi(a*b)) = a*b = \varphi^{-1}(c)*\varphi^{-1}(d)\hfill\square\]
\end{enumerate}
\subsection{Ring}
\label{sec-3-2}
Ein Ring ist ein Tupel $(R,+,\cdot)$, bestehend aus einer Menge R und 2 Verknüpfungen:
\begin{itemize}
\item $+:R\times R \to R,(a,b)\mapsto a + b$ \hfill genannt Addition
\item $\cdot:R\times R\to R, (a,b)\mapsto a\cdot b$ \hfill genannt Multiplikation
\end{itemize}
welche den folgenden Bedingungen genügen
\begin{itemize}
\item (R1) $(R,+)$ ist eine abelsche Gruppe
\item (R2) $(R,\cdot)$ ist ein Monoid
\item (R3) Es gelten die Distributivgeseze, das heisßt
\[\Forall a,b,c\in R: a\cdot(a + b) = a\cdot b + a\cdot c, (a+b)\cdot c = a\cdot c + b\cdot c\]
\end{itemize}
Ein Ring heißt \textbf{kommutativ} $\xLeftrightarrow{\text{Def}}$ die Multiplikation ist kommutativ, das heißt $\Forall a,b\in R: a\cdot b = b\cdot a$
\subsubsection{Anmerkung}
\label{sec-3-2-1}
\begin{itemize}
\item ohne Klammerung gilt die Konvention "$\cdot$" vor "$+$", "$\cdot$" wird häufig weggelassen
\item das neutrale Element bezüglich "$+$" bezeichnen wir mit $0_R$ (Nullelement), das neutrale Element bezüglich "$\cdot$" mit $1_R$ (Einselement).
Das zu $a\in R$ bezüglich "$+$" inverse Element bezeichnen wir mit $-a$,
für $a + (-b)$ schreben wir $a - b$. Existiert zu $a\in R$ ein Inverses bezüglich "$\cdot$", so bezeichnen wir diesses mit $a^{-1}$
\item Wir schreiben häufig verkürzend "$R$ Ring" statt "$(R,+,\cdot)$ Ring"
\item In der Literatur wird gelegentlich die Forderung der Existenz eines neutralen Elements bezüglich "$\cdot$" weggelassen, "unser" Ringbegriff
entspricht dort dem Begriff "Ring mit Eins"
\end{itemize}
\subsubsection{Beispiel}
\label{sec-3-2-2}
\begin{enumerate}
\item $(\mathbb{Z},+,\cdot)$ ist ein kommutativer Ring
\item Nullring $(\{0\},+,\cdot)$ mit $0 + 0 = 0, 0\cdot 0 = 0$ ist ein kummutativer Ring \\
       (hier ist Nullelement = Einselement = 0). Wir bezeichnen den Nullring kurz mit $0$.
\end{enumerate}
\subsubsection{Bemerkung 6.3}
\label{sec-3-2-3}
$R$ Ring. Dann gilt:
\begin{enumerate}
\item $0_R\cdot a = 0_R = a\cdot 0_R\Forall a\in R$
\item $a\cdot (-a) = - a b = (-a) \cdot b \Forall a,b\in R$
\item Ist $R\neq 0$, dann ist $1_R\neq 0_R$
\end{enumerate}
\paragraph{Beweis}
\label{sec-3-2-3-1}
\begin{enumerate}
\item $0_R + 0_R\cdot a = 0_R\cdot a = (0_R + 0_R)\cdot a = 0_R\cdot a + 0_R\cdot \xRightarrow{\text{"kürzen s. [[Bemerkung 5.11]]"}} 0_R = 0_R \cdot a$, $a\cdot 0_R = 0_R$ analog
\item $0_R = 0_R\cdot b = (a + (-a))\cdot b = a\cdot b + (-a) \cdot b \implies{\text{[[Bemerkung 5.11]]}} - a b = (-a)\cdot b$, $a\cdot(-b) 0 -a b$ analog
\item Beweis durch Kontraposition: Sei $1_R = 0_R$
        \[\implies \Forall a\in R: a = a\cdot 1_R = a\cdot 0_R = 0_R\]
        das heißt $R = 0\hfill\square$
\end{enumerate}
\subsubsection{Bemerkung 6.4}
\label{sec-3-2-4}
$n\in\mathbb{N}$ Für $\bar a, \bar b \in\frac{\mathbb{Z}}{n\mathbb{Z}}$ setzen wir $\bar a + \bar b := \overline{a + b}, \bar a\cdot \bar b := \overline{ab}$, dann ist $(\frac{\mathbb{Z}}{n\mathbb{Z}},+,\cdot)$ ein kommutativer Ring.

Wenn wir ab jetzt vom Ring $\frac{\mathbb{Z}}{n\mathbb{Z}}$ sprechen, dann meinen wir $(\frac{\mathbb{Z}}{n\mathbb{Z}},+,\cdot)$ mit den obigen Verknüpfungen
\paragraph{Beweis}
\label{sec-3-2-4-1}
\begin{enumerate}
\item Multiplikaiton ist wohldefiniert (das heißt "vertreterunabhängig", vergleiche \ref{sec-3-1-7-4}) \\
        Sei $a_1,a_2,b_1,b_2 \in\mathbb{Z}$ mit $\overline{a_1} = \overline{a_2},\overline{b_2} = \overline{b_2}$
\begin{align}
&\implies a_1 \equiv a_2 (\mod n), b_1\equiv b_2 (\mod n) \\
&\implies \exists q_1,q_2\in\mathbb{Z}:a_1 - a_2 = q_1 n, b_1 - b_2 = q_2 n \\
&\implies a_1 b_2 - a_2 b_2 = a_1(b_1 - b_2) + b_2 (a_1 - a_2) = a_q q_2 n + b_2 q_1 n = (a_1 q_2 + b_2 q_1) n \\
&\implies a_1 b_1 \equiv a_2 b_2 (\mod n) \\
&\implies \overline{a_1 b_1} = \overline{a_2 b_2}
\end{align}
\item Multiplikation ist assoziativ, Für $a,b,c\in\mathbb{Z}$ ist
\[\bar a\cdot (\bar b\cdot \bar c) = \bar a \cdot \overline{a\cdot c} = \overline{a\cdot(b\cdot c)} = \overline{(a\cdot b)\cdot c} = \overline{a\cdot b} \cdot \bar c = (\bar a\cdot \bar b) \cdot \bar c\]
\item Existenz eines Enselements: $\Forall a\in\mathbb{Z}:\bar 1 \cdot \bar a = \overline{1\cdot a} = \bar a = \bar a\cdot \bar 1$
\item Multiplikation ist kommutativ:
\[\Forall a,b\in\mathbb{Z}:\bar a\cdot \bar b = \overline{a\cdot b} = \overline{b\cdot a} = \bar b \cdot \bar a\]
\item $(\frac{\mathbb{Z}}{n\mathbb{Z}},+)$ ist abelsche Gruppe nach \ref{sec-3-1-7-4}
\item Distributivgesetz:
\begin{align}
\bar a\cdot (\bar b + \bar c) &= \bar a \cdot \overline{b + c} \\
&= \overline{a\cdot (b + c)} \\
&= \overline{a\cdot b + a\cdot c} \\
&= \overline{a\cdot b} + \overline{a\cdot c}
&= \bar a\cdot \bar b + \bar a \cdot\bar c
\end{align}
$(\bar a + \bar b)\cdot \bar c = \bar a\cdot \bar c + \bar b \cdot \bar c$ folgt wegen Kommutativität der Multiplikation
\end{enumerate}
\paragraph{Beispiel 6.5}
\label{sec-3-2-4-2}
Verknüpfungstafeln für $\frac{\mathbb{Z}}{n\mathbb{Z}}$
n = 3:
\begin{center}
\begin{tabular}{llll}
$+$ & $\bar 0$ & $\bar 1$ & $\bar 2$\\
\hline
$\bar 0$ & $\bar 0$ & $\bar 1$ & $\bar 2$\\
$\bar 1$ & $\bar 1$ & $\bar 2$ & $\bar 0$\\
$\bar 2$ & $\bar 2$ & $\bar 0$ & $\bar 1$\\
\end{tabular}
\end{center}

\begin{center}
\begin{tabular}{llll}
$\cdot$ & $\bar 0$ & $\bar 1$ & $\bar 2$\\
\hline
$\bar 0$ & $\bar 0$ & $\bar 0$ & $\bar 0$\\
$\bar 1$ & $\bar 0$ & $\bar 1$ & $\bar 2$\\
$\bar 2$ & $\bar 0$ & $\bar 2$ & $\bar 1$\\
\end{tabular}
\end{center}
n = 4:
\begin{center}
\begin{tabular}{lllll}
$+$ & $\bar 0$ & $\bar 1$ & $\bar 2$ & $\bar 3$\\
\hline
$\bar 0$ & $\bar 0$ & $\bar 1$ & $\bar 2$ & $\bar 3$\\
$\bar 1$ & $\bar 1$ & $\bar 2$ & $\bar 3$ & $\bar 0$\\
$\bar 2$ & $\bar 2$ & $\bar 3$ & $\bar 0$ & $\bar 1$\\
$\bar 3$ & $\bar 3$ & $\bar 0$ & $\bar 1$ & $\bar 2$\\
\end{tabular}
\end{center}

\begin{center}
\begin{tabular}{lllll}
$\cdot$ & $\bar 0$ & $\bar 1$ & $\bar 2$ & $\bar 3$\\
\hline
$\bar 0$ & $\bar 0$ & $\bar 0$ & $\bar 0$ & $\bar 0$\\
$\bar 1$ & $\bar 0$ & $\bar 1$ & $\bar 2$ & $\bar 3$\\
$\bar 2$ & $\bar 0$ & $\bar 2$ & $\bar 0$ & $\bar 2$\\
$\bar 3$ & $\bar 0$ & $\bar 3$ & $\bar 2$ & $\bar 1$\\
\end{tabular}
\end{center}

In $\frac{\mathbb{Z}}{n\mathbb{Z}}$ ist $\bar 2 \cdot \bar 2 = \bar 0$, aber $\bar 2\neq \bar 0$.
\subsubsection{Integritätsbereich}
\label{sec-3-2-5}
\label{Definition-6.6}
ist ein kommuativer Ring $(R,+,\cdot)$ mit $R\neq 0$, in dem gilt:
\[\Forall a,b\in R: a\cdot b = 0_R \implies a = 0_R\vee b = 0_R\]
beziehungsweise äquivalent dazu:
\[a\neq 0_R \wedge b\neq 0_R \implies a\cdot b \neq 0_R\]
\paragraph{Beispiel 6.7}
\label{sec-3-2-5-1}
\begin{itemize}
\item $\frac{\mathbb{Z}}{3\mathbb{Z}}$ ist ein Integritätsbereich, $\frac{\mathbb{Z}}{4\mathbb{Z}}$ ist kein Integritätsbereich, denn $\bar 2\cdot \bar 2 = \bar 0$, aber $\bar 2 \neq \bar 0$
\end{itemize}
\paragraph{Bemerkung 6.8}
\label{sec-3-2-5-2}
$n\in\mathbb{N}$ Dann sind äquivalent
\begin{enumerate}
\item \label{6.8.1} $\frac{\mathbb{Z}}{n\mathbb{Z}}$ ist ein Integritätsbereich
\item \label{6.8.2} $n$ ist eine Primzahl
\end{enumerate}
\subparagraph{Beweis}
\label{sec-3-2-5-2-1}
\ref{6.8.1} $\implies [[6.8.2]]$ zeigen wir durch Kontraposition, das heißt \$$\neg{}$\$\ref{6.8.2} $\implies$ $\neg{}$\$ \ref{6.8.1} \\
      Sei $n\in\mathbb{N}$ keine Primzahl. Falls $n = 1$ dann ist $\frac{\mathbb{Z}}{n\mathbb{Z}} = \{\bar 0\}$ (Nullring), das heißt $\frac{\mathbb{Z}}{n\mathbb{Z}}$ ist kein Integritätsbereich. Seim im Folgenden $n > 1$ und keine Primzahl.
\begin{align}
&\implies \exists a,b\in\mathbb{N}:1\ <a,b<n \wedge n = a\cdot b \\
&\implies \bar 0 = \bar n = \overline{a b} = \bar a \cdot \bar b
\end{align}
und es ist $bar a,\bar b\neq \bar 0 \implies \frac{\mathbb{Z}}{n\mathbb{Z}}$ kein Integrationsbereich. \\
      \ref{6.8.2} $\implies$ \ref{6.8.1}: Sein $n$ eine Primzahl $\implies n > 1$, insbesondere $\frac{\mathbb{Z}}{n\mathbb{Z}} \neq 0$. Seien $\bar a, \bar b \in \frac{\mathbb{Z}}{n\mathbb{Z}}$ mit $\bar a\cdot \bar b = \bar 0$
\[\implies \exists q\in\mathbb{Z}:a b = q n\]
Da $n$ Primzahl, kommt $n$ n der Primfaktorzerlengung von $a b$ als Primfaktor vor \\
      $\implies n$ kommt in der Primfaktorzerlegung von $a$ oder $b$ als Primfaktor vor \[\implies n\mid a \vee n\mid b \implies \bar a = \bar 0 \vee \bar b = \bar 0\]
\subsection{Körper}
\label{sec-3-3}
\label{Definition-6.9}
Ein Körper ist ein kommutativer Ring $(K,+,\cdot)$, in dem gilt $K\neq 0$ und jedes Element $a\in K, a\neq 0$ besitzt ein Inverses in $K$ bezüglich "$\cdot$", das heißt: $\exists b\in K:a\cdot b = 1_K$. Wir setzen $K^\ast := K\setminus\{0\}$
\subsubsection{Beispiel}
\label{sec-3-3-1}
\begin{enumerate}
\item $(\mathbb{R},+,\cdot),(\mathbb{Q},+,\cdot)$ sind Körper (mit den üblichen $+,\cdot$)
\item $\frac{\mathbb{Z}}{3\mathbb{Z}}$ ist ein Körper (betrachte Verknüpfungstafel)
\item $\frac{\mathbb{Z}}{4\mathbb{Z}}$ ist ein kein Körper: Das Element $\bar 2$ besitzt kein Inverses bezüglich "$\cdot$"
\end{enumerate}
\subsubsection{Bemerkung 6.11}
\label{sec-3-3-2}
\label{remark:integ_neutral}
$K$ Körper, Dann gilt:
\begin{enumerate}
\item $0_K \neq 1_K$
\item \label{6.11.2} $K$ ist ein Integritätsbereich
\item $(K^\ast,\cdot)$ ist eine abelsche Gruppe mit neutralem Element $1_K$
\end{enumerate}
\paragraph{Beweis}
\label{sec-3-3-2-1}
\begin{enumerate}
\item folgt aus \ref{sec-3-2-3}
\item $K\neq 0$ nach Definition. Seien $a,b\in K$ mit $a b = 0_K$. Falls $a\neq 0_K$ dann
\[b = 1_K \cdot b = (a^{-1} a)\cdot b = a^{-a}(a b) = a^{-1}\cdot 0_K = 0_K\]
Insebesondere gilt: $a = 0\vee b = 0$
\item $K^\ast\times K^\ast \to K^\ast$ ist wohldefiniert nach \ref{6.11.2} (aus $a,b\in K^\ast$ folgt $a b\in K^\ast$) \\
        Da $(K,\cdot)$ abelscher Monoid mit neutralem Element $1_K$ ist auch $(K^\ast,\cdot)$ abelscher Monid mit neutralem Element $1_K$.
Nach \ref{Definition-6.9} besitzt jedes Element $a\in K^\ast$ ein Inverses $b\in K$ mit $a b = 1_K$ Wegen $0_K \neq 1_K$ ist $b\neq 0_K$ (sonst $a b = a\cdot 0_K = 0_K \neq 1_K$), das heißt $b\in K^\ast\hfill\square$
\end{enumerate}
\subsubsection{Bemerkung 6.12}
\label{sec-3-3-3}
$R$ Integritätsbereich, der nur endlich viele Elemente hat. Dann ist $R$ ein Körper.
\paragraph{Beweis}
\label{sec-3-3-3-1}
$R$ Integritätsbereich $\implies R\neq 0$ \\
     Noch zu zeigen: $a\in R\setminus\{0_R\} \implies \exists b\in R: a b = 1_R$
Sei $a\in R\setminus\{0_R\}$. Wir betrachten die Abbildung $\varphi_a: R\to R,x\mapsto a x$
\begin{enumerate}
\item Behauptung: $\varphi_a$ ist injektiv, denn:
\begin{align}
\intertext{Seien $x,y\in R$ mit}
\varphi_a(x) =\varphi_a(y) \implies a x = a y \implies a x + (-(a y)) = 0_R \\
\intertext{Mit [[Bemerkung 6.3]] folgt:}
\implies a x + a(-a) = -R \implies a(x - y) = 0_R  \\
\intertext{Aus $R$ Integrationsbereich und $a\neq 0$ folgt:}
x - y = 0 \implies x = y
\end{align}
\item Da $R$ endlich ist und $\varphi_a$ inejktiv ist, ist $\varphi_a$ nach \ref{sec-2-6-7-10} surjektiv
\[\implies  \exists b\in R: \varphi_a(b) = 1_R \implies a b = 1_R\]
\end{enumerate}
\subsubsection{Folgerung 6.13}
\label{sec-3-3-4}
$n\in\mathbb{N}$ Dann sind äquivalent
\begin{enumerate}
\item \label{6.13.1} $\frac{\mathbb{Z}}{n\mathbb{Z}}$ ist ein Körper
\item \label{6.13.2} $n$ ist eine Primzahl
\end{enumerate}
\paragraph{Beweis}
\label{sec-3-3-4-1}
\ref{6.13.1} $\implies$ \ref{6.13.2} durch Kontraposition: $\neq$ \ref{6.13.2} $\implies \neg$ \ref{6.13.1} \\
     Sei $n$ keine Primzahl $\implies \frac{\mathbb{Z}}{n\mathbb{Z}}$ kein Integritätsbereich $\implies \frac{\mathbb{Z}}{n\mathbb{Z}}$ kein Körper \\
     \ref{6.13.2} $\implies$ \ref{6.13.1} Sei $n$ eine Primzahl $\implies \frac{\mathbb{Z}}{n\mathbb{Z}}$ Integritätsbereich, der nur endlich viele Elemente hat $\implies \frac{\mathbb{Z}}{n\mathbb{Z}}$ Körper
\paragraph{Notation}
\label{sec-3-3-4-2}
$p$ Primzahl. Man nennt $\mathbb{F}_P := \frac{\mathbb{Z}}{p\mathbb{Z}}$ auch den endlichen Körper mit $p$ Elemente
\subsubsection{Definition 6.14}
\label{sec-3-3-5}
$R$ Ring
\[\cha(R):=\begin{cases} 0 & \sum_{k = 1}^n 1_R \neq 0 \Forall n\in\mathbb{N} \\ \min\{n\in\mathbb{N}\mid \sum_{k = 1}^n 1_R = 0_R\} & \text{sonst}\end{cases}\]
heißt die Charakteristik von $R$
\begin{ex} \mbox{}
\begin{enumerate}
\item $\cha(\mathbb{Z}) = \cha(\mathbb{Q}) = \cha(\mathbb{R}) = 0$
\item $\cha(\frac{\mathbb{Z}}{n\mathbb{Z}}) = n$, denn $\sum_{k = 1}^n \bar 1 = \bar n = \bar 0$ und $\sum_{k = 1}^m \bar 1 = \bar m \neq \bar 0$ für $m\in\{1,\ldots,n - 1\}$
\end{enumerate}
\end{ex}
\begin{remark}
\label{remark:cha_prim}
$R$ Integritätsbereich. Dann ist $\cha(R) = 0$ oder $\cha(R)$ ist eine Primzahl
\end{remark}
\begin{proof}
Beweis durch Widerspruch. Annahme: $\cha(R) \neq 0$ und $\cha(R)$ ist keine Primzahl. \\
Da $R$ Integritätsbereich ist ist $1_R \neq 0_R$ also $\cha(R) \neq 1$
\begin{align*}
&\implies \exists a,b\in\mathbb{N}, 1 < a,b < \cha(R): \cha(R) = a b \\
&\implies 0_R = \sum_{k = 1}^{\cha(R)} 1_R = \sum_{k = 1}^a 1_R \cdot \sum_{k = 1}^b 1_R \\
&\xRightarrow{R~\text{Integritätsbereich}} \sum_{k = 1}^a 1_R = 0_R \vee \sum_{k = 1}^b = 0_R \\
&\implies \cha(R) \leq a \vee \cha(R)\leq b \lightning~\text{zu}~ a,b < \cha(R) \tag*{\qedhere}
\end{align*}
\end{proof}
\begin{remark}
$K$ Körper, dann ist $\cha(K) = 0$ oder $\cha(K)$ ist Primzahl.
\end{remark}
\begin{proof}
Folgt aus \ref{remark:cha_prim} und \ref{remark:integ_neutral}
\end{proof}
\begin{ex}
$p$ Primzahl, dann ist $\cha(\mathbb{F_p}) = p$
\end{ex}
\section{Polynome}
\label{sec-4}
\begin{defn}[7.1 Polynome]
$K$ Körper, ein Polynom in der Varablen $t$ über $K$ ist ein Ausdruch der Form
\[f = \sum_{k = 0}^n a_k t^k\]
mit $n\in\mathbb{N}_0$ (das heißt insbesondere nur endliche Summanden), $a_0,\ldots,a_n \in K$ (fehlende $a = 0$, ebenso setzen wir $a_{k > n} = 0)$. Die $a_k$ heißen die Koeffizienten von $f$
\[\deg(f) := \begin{cases}-\infty & f = 0 \\  \max\{k\in\mathbb{N}_0 \mid a_k \neq 0\} & f\neq 0\end{cases}\]
heißt Grad von $f$. für $f\neq 0$ heißt $l(f) := a_{\deg(f)}$ heißt der Leitkoeffizient von $f$, $l(0) := 0$. $f$ heißt normiert $\xLeftrightarrow{\text{Def}} l(f) = 1$
Hierbei sind zei Polynome $f = \sum_{k = 0}^n a_k t^k,g = \sum_{k =0}^m b_m t^k$ gleich ($f = g$) $\xLeftrightarrow{\text{Def}} \deg(f) = \deg(g) =: r$ und $a_r = b_r,\ldots,a_1 = b_1, a_0 = b_0$
\end{defn}
\begin{remark}
Man kann das auch präzise machen (Algebra 1, WS15/16, Blatt 5, Aufgabe 3)
\end{remark}
\begin{ex}[7.2] \mbox{}
\begin{enumerate}
\item $f = \frac{3}{4}x^2 - 7 x + \frac{1}{2} \in \mathbb{Q}[x] \implies \deg(f) = 2, l(f) = \frac{3}{4}, f$ ist nicht normiert
\item $f = x^5 - \frac{1}{3} x + \frac{2}{5} \in\mathbb{Q}[x] \implies \deg(f) = 5, l(f) = 1, f$ ist normiert
\end{enumerate}
\end{ex}
\begin{remark}[7.3]
$K$ Körper, $f,g \in K[t], f = \sum_{k = 0}^n a_k t^k, g = \sum_{k = 0}^m b_k t^k$. Wir setzen $r:= \max\{m,n\}$ und definieren
\begin{align*}
f + g &= (a_r + b_r)t^r + \ldots + (a_1 + b_1)t + (a_0 + b_0) \\
f \cdot g &= c_{n + m} t^{n + m} + \ldots + c_1 t + c_0, c_k := \sum_{\substack{i,j \in\mathbb{N}_0 \\ i + j = k}} a_i b_j
\end{align*}
Mittels der Verknüpfung $+,\cdot$ wir die Menge aller Polynome über $K$ in der Variablen $t (=: K[t])$ zu einem kommutativen Ring, dem Polynomring über $K$ in der Variablen $t$
\end{remark}
\begin{proof}
Man rechnet die Ringaxiome nach
\end{proof}
\begin{remark}[7.4]
\label{remark:74}
$K$ Körper, $f,g\in K[t]$, Dann gilt:
\begin{enumerate}
\item $\deg(f + g) \leq \max\{\deg(f),\deg(g)\}$
\item $\deg(f g) = \deg(f) + \deg(g)$
\end{enumerate}
(Hierbei setzt man Formel für $n\in\mathbb{N}_0: -\infty < n, n + (-\infty) = -\infty = (-\infty) + n, (-\infty) + -(\infty) = -\infty)$
\end{remark}
\begin{proof}
Falls $f = 0$ oder $g = 0$, dann sind 1. und 2. klar. Im Folgenden seien $f,g\neq 0$, etwa $f = \sum_{k = 0}^n a_k t^k, g = \sum_{k =0}^m b_k t^k$ mit $a_n, b_m \neq 0$ (insbesondere $\deg(f) = n, \deg(g) = m$)
\begin{enumerate}
\item Wir setzen $k:= \max\{m,n\}$
\begin{align*}
&\implies f + g = (a_k + b_k)t^k + \ldots + (a_1 + b_1)t + (a_0 + b_0) \\
&\implies \deg(f + g) \leq k \tag{\text{beachte: Ex könnte $a_k + b_k = 0$ sein}}
\end{align*}
\item Es sei $f g = a_n b_m t^{n + m} + \ldots + a_0 b_0$ und es ist $a_n b_m \neq 0$ da $K$ als Körper ein Integritätsbereich ist $\implies \deg(f g) = n + m$
\end{enumerate}
\end{proof}
\begin{conc}[7.5]
$K$ Körper, dann ist $K[t]$ ein Integritätsbereich
\end{conc}
\begin{proof}
$K[t] \neq 0$ klar (zum Beispiel $t\in\mathbb{t}$)
Seien $f,g\in K[t], f,g\neq 0 \implies \deg(f),\deg(g) \geq 0 \implies \deg(f g) = \deg(f) + \deg(g) \geq 0 \implies f g \neq 0$
\end{proof}
\begin{remark}
$K[t]$ ist kein Körper: Das Polynom $t\in K[t]$ besitzt kein Inverses bezüglich "$\cdot$", denn: \\
  Wäre $f\in K[t]$ invers zu $t$, dann wäre $f t = 0 \implies \deg(1) = 0 \deg(f t) = \deg(f) + \deg(t) = \deg (f) + 1 \implies \deg(f) = -1 \lightning$
\end{remark}

\begin{thm}[7.6 Polynomdivision]
\label{thm:poly_div}
$K$ Körper, $f,g\in K[t], g\neq 0$ \\
  Dann existieren eindeutig bestimmte Polynome $q,r \in K[t]$, mit $f = q g + r$ und $\deg(r) < \deg(g)$
\end{thm}

\begin{ex}[7.7]
$f = 3 t^3 + 5 t + 1, g = t^2 + 1 \in\mathbb{Q}[t]$
\[(3 t^3 + 5 t + 1) : (t ^2 + 1) = 3 t\]
Also $3 t^3 + 5t + 1 = 3^t (t^2 + 1) + 2 t + 1, q = 3 t, r = 2 t + 1$
\end{ex}

\begin{proof}
\begin{enumerate}
\item Existenz: \\
     Falls $f = 0$, setzen wir $q := 0, r:= 0$ fertig. \\
     Im Folgenden sei $f\neq 0$, das Polynom $g$ sei fixiert. Wir zeigen die Existenz von $q,r$ per Induktion nach $\deg(f) \in\mathbb{N}_0$ \\
\begin{itemize}
\item Induktionsanfang: (etwas unkonventionell, geht aber auch): $\deg(f) \in \{0,\ldots,\deg(g) - 1\}$ (das heißt $\deg(f) < \deg(g)$) \\
       Setze $q:= 0, r:= g$, dann ist $f = q g + r, \deg(r) = \deg(f) < \deg(g)$.
\item Induktionsschritt: Es sei $\deg(f) \geq \deg(g)$ und die Behauptung sei für alle Polynome aus $K[t]$ von Grad < $\deg(f)$ schon gezeigt. \\
       Wir sezen $n:= \deg(f), m:=\deg(g)$ und schreiben:
\[f = l(f)t^n + ~\text{Terme kleineren Grades}\]
\[g = l(g) t^m + ~\text{Terme kleineren Grades}\]
Es ist $f - \frac{l(f)}{l(g)}t^{n - m}g =$
\[l(f) t^n + ~\text{Terme kleineren Grades} - \underbrace{\frac{l(f)}{l(g)} t^{n - m}l(g) t^m}_{l(f)t^n} + ~\text{Terme kleineren Grades}\]
\[\implies \deg(f - \frac{l(f)}{l(g)}t^{n - m}) < n\]
Nach Induktionsannahme gilt: Es existert $q_1,r_1 \in K[t]$ mit
\[f - \frac{l(f)}{l(g)}t^{n -m}g = q_1 g + r_1, ~\text{mit}~\deg{r_1} < \deg(g)\]
\[\rightarrow f = (q_1 + \frac{l(f)}{l(g)} t^{n - 1})g + r_1\]
Setze $q:= q_1 + \frac{l(f)}{l(g)}t^{n - m}, r:= r_1$, dann ist $f = q g + r$ und $\deg(r) < \deg(g)$
\end{itemize}
\item Eindeutigkeit: Seien $q_1,q_2,r_1,r_2\in K[T]$ mit $f = q_1 g + r_1 + q_2 g + r_2$ und $\deg(r_1) < \deg(g), \deg(r_2) < \deg(g)$
\begin{align*}
&\implies (q_1 - q_2) g = r_2 - r_1 \\
&\implies \deg(g_1 - q_2) + \deg(g) = \deg(r_1 - r_2)
\end{align*}
Falls $q_1 \neq q_2$, dann sind beide Seiten der Gleichung in $\mathbb{N}_0$ und es wäre
\[\deg(g) \leq \deg(r_2 - r_1)\]
Nach \ref{remark:74} ist $\deg(r_2 - r_1) \leq \max\{\deg(r_2),\deg(-r_1)\} < \deg(g) \lightning$
Also $q_1 = q_2$, somit $r_2 - r_1 = \underbrace{q_1 - q_2}_{= 0} g = 0$, also $r_1 = r_2$
\end{enumerate}
\end{proof}
\begin{defn}[7.8, Nullstelle]
$\in K[t], f = a_n t^n + \ldots + a_1 t + a_0, \lambda \in K$ \\
  Wir setzen $f(\lambda) := a_n \lambda^n + \ldots + a_1\lambda + a_0 \in K$
$\lambda$ heißt Nullstelle von $f \xLeftrightarrow{Def.} f(\lambda) = 0$
\end{defn}
\begin{remark}[7.9]
\label{remark:79}
$K$ Körper, $f\in K[t], \lambda \in K$ Nullstelle von $f$. Dann gibt es in $K[t]$ ein eindeutig bestimmtes Polynom $q$ mit $f = (t - \lambda)q$.
Es ist $\deg(q) = \deg(f) - 1$
\end{remark}
\begin{proof}
\textbf{Existenz:} Nach \ref{thm:poly_div} existiert $q,r\in K[t]$ mit $f = (t - \lambda)q + r$ und $\deg(r) < \underbrace{\deg(t - \lambda)}_{= 1}$ \\
  $\implies r$ ist konstantes Polynom und es gilt
\[0 = f(\lambda) = \underbrace{(\lambda - \lambda)}_{= 0} q + r(\lambda) = r(\lambda) \implies r = 0 \implies f = (t - \lambda) q\]
\textbf{Eindeutigkeit:} aus Eindeutigkeit aus \ref{thm:poly_div}
\end{proof}
\begin{conc}
$K$ Körper, $f\in K[t], f\neq 0, n:= \deg(f)$ \\
  Dann besitzt $f$ in $K$ höchstens $n$ Nullstellen.
\end{conc}
\begin{proof}
per Induktion nach $n$
\textbf{Induktionsanfang:} $n = 0$ Ein konstantes Polynom $\neq 0$ besitzt keine Nullstellen. \\
  \textbf{Induktionsschritt:} Es sein $n > 0$ und die Aussage sei für alle Polynome von Grad $< n$ schon gezeigt.
Falls $f$ keine Nullstelle besitzt, dann fertig.
Im Folgenden besitzte $f$ eine Nullstelle sei $\lambda \in K$ eine solche, daraus folgt mit \ref{label:79}
\[\exists q\in K[t]: f(t - \lambda)q, \deg{q} = n - 1\]
Ist $\varepsilon \neq \lambda$ eine weitere Nullstelle von $f$ dann ist
\[0 = f(\varepsilon) = \underbrace{\varepsilon - \lambda}_{\neq 0}q(\varepsilon)\]
Da $K$ als Körper ein Integritätsbereich ist folt: $q(\varepsilon) = 0$, das heißt $\varepsilon$ ist Nullstelle von $q$.
Nach Induktionsvorraussetzung hat $q$ höchstens $n-1$ Nullstellen $\implies f$ hat höchstens $n$ Nullstellen
\end{proof}
\begin{defn}[7.11]
\label{dfn:zero}
$K$ Körper, $f\in K[t], f\neq 0,\lambda \in K$
\[\mu (f,\lambda) := \max\{e\in\mathbb{N}_0 \mid \Exists g\in K[t]: f = (t - \lambda)^e g\}\]
heißt die Vielfachheit der Nullstelle $\lambda$ von $f$.
\end{defn}
\begin{remark}
\begin{itemize}
\item Es ist $\mu(f,\lambda) = 0 \iff f(\lambda) \neq 0 \lambda$ keine Nullstelle von $f$ (denn: $f(\lambda) = 0 \iff \Exists q\in K[t]: f=(t - \lambda)q \iff \mu(f,\lambda) \neq 0$)
\item Die Vielfachheit von $\lambda$ gibt an, wie oft der Linearfaktor $t - \lambda$ in $f$ vorkommt
\item Sind $\lambda_1,\ldots,\lambda_m \in K$ sämtliche verschiedene Nullstellen von $f$ und es ist $e_i := \mu(f,\lambda_i),i = 1,\ldots,m$ dann existerit ein Polynom $g\in K[t]$ mit
\[f = (t - \lambda_1)^{e_1}\cdot\ldots\cdot(t-\lambda_m)^{e_m}g\]
und den Eigenschaften, dass $g$ in $K$ kein Nullstelle besitzt und, dass $\deg(g) = \deg(f) - (e_1 + \ldots + e_m)$.
\item "bester Fall:" $\deg(g) = 0$ ("$f$ zerfällt in Linearfaktoren"):
Dann existert $a\in K\setminus\{0\}, \lambda_1,\ldots, \lambda_m \in k$ paarweise verschieden, $e_1,\ldots,e_n \in \mathbb{N}$  mit
\[f = a(t - \lambda_1)^{e_1}\cdot\ldots\cdot(t - \lambda_m)^{e_m}, e_1 + \ldots + e_m = \deg(f)\]
Alternative Darstellung:
\[f = a(t - \tilde\lambda_1) \cdot \ldots\cdot(t - \tilde\lambda_n), n = \deg(f), \tilde\lambda_1,\ldots\tilde\lambda_n~\text{nicht notwendig verschieden}\]
\end{itemize}
\end{remark}
\begin{thm}[7.12 Fundamentalsatz der Algebra]
Jedes Polynom $f \in \mathbb{C}[t]$ mit $\deg(f) \geq 1$ besitzt eine Nullstelle.
\end{thm}
\begin{proof}
Zum Beispiel in Vorlesung Funktionentheorie 1, Algebra 1
\end{proof}
\begin{conc}[7.13]
$f\in\mathbb{C}[t], f\neq 0$
Dann zerfällt $f$ in Linearfaktoren.
\end{conc}
\begin{proof}
\textbf{Induktionsanfang:} $n = 0 \implies f$ ist konstantes Polynom, fertig \\
  \textbf{Induktionsscritt:} Sei $n\geq 1$ und die Assage sei für alle Polynome vom Grad $< n$ bereits bewiesen. Nach Fundamentalsatz der Algebra existiert eine Nullstelle $\lambda$ von $f$
\[\xRightarrow{\ref{remark:79}} \Exists g\in \mathbb{C}[t]: f = (t - \lambda)g, \deg(g) = n - 1 \]
Nach Induktionsannahme $\Exists a\in\mathbb{C},\lambda_1,\ldots,\lambda_{n - 1} \in \mathbb{C}$ (nicht notwendig verschieden)
\[g = a(t - \lambda_1)\cdot\ldots\cdot(t - \lambda_{n-1})\]
Setze $\lambda_n := x \implies f = g(t - \lambda_n) = a(t - \lambda_1)\cdot\ldots\cdot(t - \lambda_{n - 1})(t - \lambda_n)$
\end{proof}

\begin{defn}[7.14]
$K$ Körper, $f\in K[t]$
$f$ induziert eine Abbildung $\tilde f: K\to K,\lambda\to f(\lambda)$, $\tilde f$ heißt die Polynomfunktion zum Polynom $f$
\end{defn}
\begin{ex}[7.15]
\label{ex:715}
Es ist wichtig zwischen dem Polynom $f\in K[t]$ und der dazugehörigen Polynomfunktion $\tilde f: K\to K$ zu unterscheiden
Sei $f = t^2 + t \in \mathbb{F}_2 [t]$. Dann ist $f(\bar 0) = \bar 0^2 + \bar 0 = \bar 0, f(\bar 1) = \bar 1^2 + \bar 1 = \bar 0$
das heißt $\tilde f: \mathbb{F}_2 \to \mathbb{F}_2$ ist die Nullabbildung, aber $f$ ist nicht das Nullpolynom
\end{ex}
\begin{remark}[7.16]
\label{remark:716}
$K$ Körper mit unendlich vielen Elementen. \\
  Dann ist die Abbildung $\tilde : K[t]\to$ Abb$(K,K):=\{g:K\to K ~\text{Abbildung}\}, f\mapsto \tilde f$ injektiv, das heißt:
Ist $K$ unendlich und sind $f_1,f_2 \in K[t]$, dann gilt $f_1 = f_2 \iff \tilde f_1 = \tilde f_2$
\end{remark}
\begin{proof}
Es seien $f_1,f_2 \in K[t]$ mit $\tilde f_1 = \tilde f_2$ wir setzen $g:= f_1 - f_2$ \\
  $\implies$ Für alle $a\in K$ ist $g(a) = (f_1 - f_2)(a) = f_1(a) - f_2(a) = \tilde f_1(a) - \tilde f_2(a) = 0$
$\xRightarrow{K~\text{unendlich}}$ g hat unendlich viele Nullstellen, mit \ref{dfn:zero} folgt: $g = 0\implies f_1 = f_2$
\end{proof}
\begin{remark}
\begin{itemize}
\item Lässt man \ref{remark:716} die Vorraussetzung $K$ hat unendlich viele Elemente weg, wird die Aussage falsch, siehe Beispiel \ref{ex:715}
\item Mit dem Wissen von \ref{ex:715} und \ref{remark:716} im Hintergrund bezeichnet man die vom Polynom $f$ induzierte Polynomfunktion mit $f$ anstelle von $\tilde f$
\end{itemize}
\end{remark}
\section{Vektorräume}
\label{sec-5}
In diesem Kapitel sei $K$ stets ein Körper
\begin{defn}[8.1]
Ein K-Vektorraum ist ein Tripel $(V,+,\cdot)$ bestehend aus
\begin{itemize}
\item einer Menge $V$
\item einer Verknüpfung $+:V\times V \to V, (v,w)\mapsto v + w$ \hfill (Addition)
\item und einer äußeren Verknüpfung $\cdot : K\times V \to V, (\lambda,v) \mapsto \lambda \cdot v$ \hfill (skalare Multiplikaiton)
\end{itemize}
Welche folgende Bedingungen genügen:
\begin{enumerate}
\item (V1) $(V,+)$ ist eine abelsche Gruppe
\item (V2) Die skalare Multiplikaiton ist in folgender Weise mit der anderen Verknüpfung (auf $V$ und $K$) verträglich:
\begin{align*}
\intertext{$\Forall \lambda,\mu \in K, v,w\in V$}
(\lambda + \mu) \cdot v  &= \lambda \cdot v + \mu \cdot v \\
\lambda \cdot (v + w) = \lambda \cdot v + \lambda \cdot w \\
\lambda \cdot (\mu \cdot w) = (\lambda \cdot \mu) \cdot v \\
1\cdot v = v
\end{align*}
\end{enumerate}
\end{defn}
\begin{remark}
\begin{itemize}
\item Es ist wichtig, zwischen Addition "$+$" und skalarer Multiplikation "$\cdot$" auf $V$ und Addition und Multiplikation in $K$ zu unterschieden:
In der Gleichung $(\lambda + \mu) \cdot v = \lambda \cdot v + \mu\cdot v$ sind die Verknüpfungen wie folgt zu verstehen:
\[(\lambda \qquad\underarrow[+]{Addition in $K$}\qquad \mu) \qquad \overarrow[\cdot]{skalare Multiplikation} \qquad v = \lambda \qquad \underarrow[\cdot]{skal. Mult.} \qquad v \overarrow[+]{Addition in $V$} \mu\qquad\underarrow[\cdot]{skal. Mult.} \qquad v\]
\item Das neutrale Element bezüglich "$+$" auf $V$ bezeichnen wir mit $0_v$ (Nullvektor), das inverse zu $v\in V$ bezüglich "$+$" mit $-v$. Das Zeichen "$\cdot$" fpr die skalare Multiplikation lassen wir ab jetzt meistens weg und schreiben $\lambda v$ statt $\lambda\cdot v$ ($\Forall \lambda\in K, v\in V$)
\item Wir schreiben meist verkürzend "V K-Vektorraum" (beziehungsweise: "V K-VR") anstelle von $(V,+,\cdot)$ K-Vektorraum.
\end{itemize}
\end{remark}
\begin{ex}[8.2]
\mbox{}
\begin{enumerate}
\item \[K^n := \{(x_1,\ldots,x_n) \mid x_1,\ldots,x_n \in K\}\]
mit
\[(x_1,\ldots,x_n) + (y_1,\ldots,y_n) := (x_1 + y_1,\ldots,x_n + y_n)\]
\[\lambda\cdot(x_1,\ldots,x_1) := (\lambda x_1,\ldots,\lambda x_n)\]
\[\lambda \in K, (x_1,\ldots,x_n),(y_1,\ldots,y_n)\in K^n\]
ist ein K-Vektorraum, der sogenannte Standardvektorraum über $K$
Die Axiome rechnet man nach, exemplarisch: sind $\lambda,\mu\in K, (x_1,\ldots,x_n)\in K^n$, dann ist
\[(\lambda + \mu) \cdot (x-1,\ldots,x_n) = ((\lambda + \mu)x_1,\ldots,(\lambda + \mu)x_n)\]
Mit dem Distributivgesetz in $K$ folgt:
\[(\lambda x_1,\ldots,\lambda x_n) + (\mu x_1,\ldots,\mu x_n) = \lambda(x_1,\ldots,x_n) + \mu(x_1,\ldots,x_n)\]
Der Nullvektor ist gegeben duch $0_{K^n} = (0,\ldots,0)$, für $x =(x_1,\ldots,x_n)$ ist $-x = (-x_1,\ldots,-x_n)$
\item $\mathbb{C}$ ist ein $\mathbb{R}$-VK bezüglich
\begin{itemize}
\item $+$ = übliche Addition auf $\mathbb{C}$
\item skalare Multiplikation $\cdot:\mathbb{R}\times\mathbb{C} \to \mathbb{C}, \lambda (a + \I b) := \lambda a + \I \lambda b$
\end{itemize}
\item $K[t]$ Polynomring über $K$ in der Variablen $t$ wird zum $K$-VR durch
\begin{itemize}
\item $+$ = Addition von Polynomen
\item akalare Multiplikation, \$$\cdot$: K\texttimes{} K[t] $\to$ K[t]: $\lambda$ $\cdot$ $\sum$$_{\text{k = 0}}^{\text{n}}$ a$_{\text{k}}$ t$^{\text{k}}$ := $\sum$$_{\text{k = 0}}^{\text{n}}$ $\lambda$ a$_{\text{k}}$ t$^{\text{k}}$
\end{itemize}
\item $M$ Menge, $\Abb(M,K):= \{f: M \to K~\text{Abbildung}\}$ wird zum $K$-Vektorraum durch die folgende Verknüpfungen:
\begin{itemize}
\item Addition: Zu $f,g\in \Abb(M,K)$ wird $f + g: M \to K$ definiert über
\[(f + g)(x) := f(x) + g(x), x\in M\]
\item skalare Multiplikation: Zu $\lambda \in K, f\in \Abb(M,K)$ wird $\lambda f: M \to K$ definiert über
\[(\lambda f)(x) := \lambda f(x), x\in M\]
\end{itemize}
("punktweise Addition und skalare Multiplikation")
\end{enumerate}
\end{ex}
\begin{remark}[8.3]
$V~K$-VR. Dann gilt:
\begin{enumerate}
\item $0_K \cdot v = 0_V \Forall v\in V$
\item $\lambda \cdot 0_V = 0_V \Forall \lambda\in K$
\item $\lambda v = 0_V \implies \lambda = 0_K \vee v = 0_V$
\item $(-1_K)\cdot v = -v \Forall v\in V$
\end{enumerate}
\end{remark}
\begin{proof}
\begin{enumerate}
\item Sei $v \in V$
     \[\implies 0_V + 0_K \cdot v = 0_K \cdot v = (0_K + 0_K)\cdot v = 0_K\cdot v + 0_K \cdot v \implies 0_K \cdot v = 0_V\]
\item Sei $\lambda \in K$
     \[\implies \lambda \cdot 0_V + 0_V = \lambda \cdot 0_V = \lambda \cdot (0_V + 0_V) = \lambda\cdot 0_V + \lambda \cdot 0_V \implies \lambda \cdot 0_V = 0_V\]
\item Seien $\lambda \in K, v\in V, \lambda\cdot v = 0$. Falls $\lambda \neq 0_K:$
     \[v = 1_K \cdot v = (\lambda^{-1} \lambda)\cdot v = \lambda^{-1}\cdot \underbrace{(\lambda v)}_{= 0_V} = \lambda^{-1} \cdot 0_V = 0_V\]
\item Für $v\in V$ ist
\[v + (-1_K) \cdot v = 1_K \cdot v + (-1_K)\cdot v = (1_K + (-1_K))\cdot v = 0_K \cdot v = 0_V \implies (-1_K)\cdot v = -v\]
\end{enumerate}
\end{proof}
\begin{defn}[8.4]
$V~K$-VR, $U\subseteq V$ \\
  $U$ heißt Untervektorraum ($K$-Untervektorraum), kurz UVR von $V$ $\xLeftrightarrow{\text{Def}}$ Die folgenden Bedingungen sind erfüllt
\begin{itemize}
\item (U1) $U\neq \emptyset$
\item (U2) $v,w \in U \implies v + w \in U$ \hfill (das heißt $U$ ist abgeschlossen bezüglich Addition)
\item (U3) $v\in U, \lambda \in K \implies \lambda v \in U$ \hfill (das heißt $U$ ist abgeschlossen bezüglich skalarer Multiplikation)
\end{itemize}
\end{defn}
\begin{remark}[8.5]
$V~K$-VR, $U\subseteq V$ \\
  Dann sind äquivalent
\begin{enumerate}
\item $U$ ist ein UVR von $V$
\item Addition und skalare Multiplikation auf $V$ induzieren durch Einschränkung auf $U$ Verknüpfung $+: U\times U\to U, \cdot:K\times U \to U$, und bezüglich dieser Verknüpfung ist $U$ ein $K$-VR
\end{enumerate}
\end{remark}
\begin{proof}
(1.) $\implies$ (2.): Sei $U$ ein UVR von $V$ \\
\begin{enumerate}
\item Die Verknüpfung $+:U\times U\to U, \cdot :K\times U \to U$ sind wohldefiniert wegen (U2), (U3)
\item (V1) gilt (das heißt $(U,+)$ ist eine abelsche Gruppe), denn:
\begin{itemize}
\item Asooziativgesetz, Kommutativgesetz bezüglich "$+$" gelten, weil sie schon in $V$ gelten.
\item $0_V$ ist neutral bezüglich "$+$" und lieght in $U$, denn wegen $U \neq \emptyset$ existiert ein $u\in U$, wegen (U3) ist ann auch $\underbrace{0_K \cdot u}_{= 0_V} \in U$, also $0_V \in U$
\item $-u$ ist invers zu $u$ und liegt in $U$, denn: Mit $u \in U$ ist nach (U3) auch $(-1_K)\cdot u$ in $U$
\end{itemize}
\item (V2) gilt, da es schon in $V$ gilt
\end{enumerate}
(2.) $\implies$ (1.) Es gelte (2.) \\
\begin{itemize}
\item (U1): $U\neq \emptyset$, denn $U$ ist abelsche Gruppe bezüglich der eingeschränkten Addition
\item (U2),(U3): folgt direkt aus der Wohldefiniertheit der Verknüpfung $+: U\times U \to Z, \cdot : K\times U \to U$
\end{itemize}
\end{proof}
\begin{remark}
\begin{itemize}
\item der Beweis von (1.) $\implies$ (2.) hat gezeigt: Ist $U \subseteq V$ ein UVR, dann ist $0_V \in U$
\item Ab jetzt lassen wir bei $0_V$ beziehungsweise $0_K$ meist die Indizes $V$ beziehungsweise $K$ weg und schreiben für beide kurz $0$.
\end{itemize}
\end{remark}
\begin{ex}[8.6]
\mbox{}
\begin{enumerate}
\item $K = \mathbb{R}, V = \mathbb{R}^2$ \\
     Es sei $U = \{(x_1, x_2) \in \mathbb{R}^2 \mid x_1 - 2 x_2 = 0\}$
\begin{itemize}
\item (U1): $(0, 0) \in U$ also $U\neq \emptyset$
\item (U2): Es seien $(x_1, x_2) \in U, (y_1, y_2) \in U \implies x_1 - 2 x_2 = 0, y_1 - 2 y_2 = 0$
       \[\implies (x_1 + y_1) - 2(x_2 + y_2) = 0 \implies (x_1, x_2) + (y_1, y_2) = (x_1 + y_1, x_2 + y_2) \in U\]
\item (U3): Sei $(x_1, x_2) \in U, \lambda \in \mathbb{R} \implies x_1 - 2 x_2 = 0 \implies \lambda x_1 - 2\lambda x_2 = 0$
       \[(\lambda x_1, \lambda x_2) = \lambda (x_1, x_2) \in U\]
\end{itemize}
Also: $U$ ist ein UVR von $V = \mathbb{R}^2$
\item $K = \mathbb{R}, V = \mathbb{R}$ \\
     Es sei $U = \{(x_1, x_2) \in \mathbb{R}^2 \mid x_1 - 2 x_2 = 1\}$ \\
     Es ist $(0,0) ( = 0_V) \not\in U \implies U$ ist kein UVR von $V = \mathbb{R}^2$
\item $K = \mathbb{R}, V = \mathbb{R^2}$ \\
     Es sei $U = \{(x_1, x_2) \in \mathbb{R}^2 \mid x_1 \geq 0 \wedge x_2 \geq 0\}$ \\
     $U$ ist kein UVR von $V$, denn: $(5,2) \in U$, aber $(-1)\cdot (5,2) = (-5, -2) \not\in U$
\item $V = K[t]$ \\
     Es sei $U = \{f \in K[t] \mid \deg f \leq 2\} = \{f\in K[t] \mid \Exists a_0, a_1, a_2 \in K: f = a_2 t^2 + a_1 t + a_0\}$
\begin{itemize}
\item (U1): $0 \in U$, also $U \neq \emptyset$
\item (U2): Es seien $f,g \in U \implies \deg (f) \leq 2, \deg (g) \leq 2 \implies \deg(f + g) \leq 2 \implies f + g \in U$
\item (U3): Es sei $f \in U, \lambda \in K \implies \deg(f) \leq 2 \implies \deg(\lambda f) \leq 2 \implies \lambda f \in U$
\end{itemize}
Also $U$ ist ein UVR von $V$
\item $V$ K-VR. Dann sind $\{0\}, V$ UVR von $V$ ("triviale UVR"), $\{0\}$ heißt Nullvektorraum (Nullraum)
\end{enumerate}
\end{ex}
\begin{remark}[8.7]
$V$ K-VR, $I$ Indexmenge, $(U_i)_{i\in I}$ Familie von UVR von $V$ (das heißt für jedes $i\in I$ ist ein UVR $U_i$ von $V$ gegeben)
Dann gilt:
\[U := \bigcap_{i \in I} U_i\]
ist ien UVR von $V$. Mit anderen Worten: der Durchschnitt von UVRen von $V$ ist wieder ein UVR von $V$
\end{remark}
\begin{proof}
\begin{enumerate}
\item (U1): $U \neq \emptyset$, denn $0 \in U_i \Forall i\in I$, also $0 \in U$
\item (U2): Seien $v,w \in U \implies \Forall i \in I: v\in U_i, w \in U_i \implies \Forall i\in I: v + w \in U_i \implies v + w \in U$
\item (U3): Sei $v \in U, \lambda \in K \implies \Forall i\in I: v\in U_i \implies \Forall i\in I: \lambda v \in U_i \implies \lambda v \in \bigcap_{i\in I} U_i = U$
\end{enumerate}
\end{proof}
\begin{ex}[8.8]
Die Vereinigung von UVR ist im Allgemeinen kein UVR, zum Beispiel $K = \mathbb{R}, V =\mathbb{R}^2$
\begin{itemize}
\item $U_1 = \{(x_1, x_2) \in \mathbb{R}^2 \mid x_1 = x_2\}$
\item $U_2 = \{(x_1, x_2) \in \mathbb{R}^2 \mid 2 x_1 = x_2\}$
\end{itemize}
Aber: $U_1 \cup U_2$ ist kein UVR von $\mathbb{R}^2$, denn
\[(1,1) \in U_1 \subseteq U_1 \cup U_2, (2, 4) \in U_2 \subseteq U_1 \cup U_2\]
aber: $(1, 1) + (2, 4) = (3, 5) \not\in U_1 \cup U_2$
\end{ex}
\begin{defn}[8.9]
$V$ K-VR, $(v_1, \ldots, v_r)$ endliche Familie von Vektoren aus $V$
\[\Lin((v_1, \ldots, v_r)) := \{\alpha_1 v_1 + \ldots + \alpha_r v_r \mid \alpha_1, \ldots, \alpha_r \in K\}\]
heißt die Lineare Hülle (das Erzeugnis) der Familie $v_1, \ldots, v_r$ \\
  $v \in V$ heißt Linearkombination von $v_1, \ldots, v_r$
\[\xLeftrightarrow{\text{Def}} v\in \Lin((v_1, \ldots, v_r)) \iff \Exists \alpha_1, \ldots \alpha_r \in K: v = \alpha_1 v_1 + \ldots + \alpha_r v_r\]
\end{defn}
\begin{remark}
Andere Notation für $\Lin: \Span(\ldots), <\ldots>$
\end{remark}
\begin{ex}[8.10]
\mbox{}
\begin{enumerate}
\item $V = \mathbb{R}^3, K = \mathbb{R}$
\begin{itemize}
\item $v \in V, v \neq 0 \implies \Lin((v)) = \{\alpha v \mid \alpha \mathbb{R}\}$ = Gerade durch $0$ und $v$
\item \[v,w \in V, v \neq 0 \implies \Lin((v,w)) = \{\alpha_1 v + \alpha_2 w \mid \alpha_1, \alpha_2 \in \mathbb{R}\} = \begin{cases} ~\text{Gerade durch $0$} & w \in \Lin((v)) \\ ~\text{Ebene durch $0,v,w$} & w\not\in \Lin((v)) \end{cases}\]
\end{itemize}
\item $V = K^n$ als K-VR \\
     \[e_i := (0, \ldots, 0, \underarrow[1]{i-te Stelle}, 0, \ldots, 0)\]
\begin{align*}
\Lin((e_1, \ldots, e_n)) &= \{\alpha_1 e_1 + \ldots \alpha_n e_n \mid \alpha_1, \ldots, \alpha_n \in K \} \\
&= \{(\alpha_1, 0, \ldots, 0) + (0, \alpha_2, 0, \ldots, 0) + \ldots + (0, \ldots, 0, \alpha_n) \mid \alpha_1, \ldots, \alpha_n \in K\} \\
&= \{(\alpha_1, \ldots, \alpha_n) \mid \alpha_1, \ldots, \alpha_n \in K\} \\
&= K^n
\end{align*}
\end{enumerate}
\end{ex}
\begin{defn}[8.11]
$V$ K-VR, $(v_i)_{i\in I}$ Familie von Vektoren aus $V$
\[\Lin((v_i)_{i \in I}) := \{\sum_{i\in I} \alpha_i v_i \mid \alpha_i \in K \Forall i\in I, \alpha_i = 0 ~\text{für fast alle $i\in I$}\}\]
heißt die lineare Hülle (das Erzeugnis) der Familie $(v_i)_{i \in I}$. Heibei bedeutet "$\alpha_i = 0$ für fast alle $i\in I$": Es gibt nur endlich viele $i\in I$ mit $\alpha_i \neq 0$,
das heißt die auftretenden Summen sind endliche Summen. Falls $I = \emptyset$ setzen wir $\Lin((v_i)_{i\in\emptyset}) := \{0\}$
\end{defn}
\begin{remark}
Ein Element $v \in V$ ist genau dann in $\Lin((v_i)_{i\in I})$ enthalten, wenn es eine endliche Teilmenge $\{i_1, \ldots, i_r\} \subseteq I$ und Elemente $\alpha_{i_1}, \ldots, \alpha_{i_r} \in K$ gibt mit
\[v = \alpha_{i_1} v_{i_1} + \ldots + \alpha_{i_r} v_{i_r}\]
Insbesondere ist mit $\Lin((v_i)_{i\in I}) = \bigcup_{J\subseteq I} \Lin((v_i)_{i\in J})$
\end{remark}
\begin{ex}[8.12]
$V = K[t]$ als K-VR \\
  Es ist \[\Lin((t^n)_{n\in\mathbb{N}_0}) = \{\sum_{i\in\mathbb{N}_0} \alpha_i t^i \mid \alpha_i \in K, \alpha_i = 0~\text{für fast alle $i\in \mathbb{N}_0$}\} = K[t]\]
\end{ex}
\begin{remark}[8.13]
$V$ K-VR, $(v_i)_{i\in I}$ Familie von Vektoren aus $V$. Dann gilt:
\begin{enumerate}
\item $\Lin((v_i)_{i\in I})$ ist ein UVR von $V$
\item Ist $U\subseteq V$ ein UVR mit $v_i \in U\Forall i\in I$, ann ist $\Lin((v_i)_{i\in I}) \subseteq U$
     das heißt $\Lin((v_i)_{i\in I})$ ist das bezüglich "$\subseteq$" kleinste Element der Menge derjenigen UVR von $V$ die alle $v_i, i\in I$ enthalten
\item \[\Lin((v_i)_{i\in I}) = \bigcap_{\mathclap{\text{$U$ UVR von $V$ mit $v_i \in U \Forall i\in I$}}} U\]
\end{enumerate}
\end{remark}
\begin{proof}
Falls $I = \emptyset$, dann $\Lin((v_i)_{i\in I}) = \{0\}$, dann 1. klar, und jeder UVR $U$ von $V$ enthält alle $v_i, i\in \emptyset$, und enthält $\{0\} \implies$ 2. Außerdem
\[\bigcap_{\text{$U$ UVR von $V$ mit $v_i \in U \Forall i\in I$}} U = \bigcap_{\text{$U$ UVR von $V$}} U = \{0\} = \Lin((v_i)_{i\in\emptyset})\]
es folgt 3. \\

Im Folgenden sie $I \neq \emptyset$. Wir setzen $W:= \Lin((v_i)_{i\in I})$
\begin{enumerate}
\item \begin{itemize}
\item (U1): Sei $i \in I$ (Existenz wegen $I \neq \emptyset$). Dann ist $0\cdot v_i = 0\in W$, insbesondere $W\neq\emptyset$
\item (U2): Es seinen $v,w \in W$
\begin{align*}
\implies \Exists r\in\mathbb{N}, \{i_1, \ldots, i_r\} \subseteq I, \alpha_{i_1}, \ldots, \alpha_{i_r} \in K,~\text{mit}~v = \alpha_{i_1} v_{i_1} + \ldots + \alpha_{i_r} v_{i_r} \\
\intertext{sowie}
s\in\mathbb{N}, \{j_1, \ldots, j_r\} \subseteq I, \beta_{j_1}, \ldots, \beta_{j_r} \in K,~\text{mit}~w = \beta_{j_1} v_{j_1} + \ldots + \beta_{j_r} v_{j_r} \\
\implies v + w = \alpha_{i_1} v_{i_1} + \ldots + \alpha_{i_r} v_{i_r} + \beta_{j_1} v_{j_1} + \ldots + \beta_{j_s} v_{j_s} \in W
\end{align*}
\item (U3): Für $\lambda \in K, v\in W$ wie bei (U2) ist
\[\lambda v = \lambda \alpha_{i_1} v_{i_1} + \ldots + \lambda \alpha_{i_r} v_{i_r} \in W\]
\end{itemize}
\item Sei $U\subseteq V$ UVR mit $v_i \in U \Forall i\in I$ \\
     $\implies$ Jedes Element der Form $\displaystyle \sum_{i \in I} \alpha_i v_i$ mit $\alpha_i \in K \Forall i\in I, \alpha_i = 0$ für fast alle $i\in I$, liegt
in $U$. $\implies \Lin((v_i)_{i\in I}) = W \subseteq U$.
\item zu zeigen: \[\Lin((v_i)_{i \in I}) = \bigcap_{\text{$U$ UVR von $V$ mit $v_i \in U \Forall i\in I$}} U\]
"$\subseteq$" Wegen 2. liegt $\Lin((v_i)_{i\in I})$ in jedem UVR $U$ von $V$, der alle $v_i, i\in I$ enthält
\[\implies \Lin((v_i)_{i\in I} = \bigcap_{\text{$U$ UVR von $V$ mit $v_i \in U \Forall i\in I$}} U\]
"$\supseteq$" Nach 1. ist $W = \Lin((v_i)_{i\in I}$ ist ein UVR von $V$ mit $v_i \in W \Forall i\in I$, das heißt $W$ ist einer der UVR, über die der obige Durchschnitt gebildet wird
\[\implies \bigcap_{\text{$U$ UVR von $V$ mit $v_i \in U \Forall i\in I$}} U \subseteq W = \Lin((v_i)_{i\in I})\]
\end{enumerate}
\end{proof}
\textbf{Notation}:
Ist $M\subseteq V$, dann setzen wir $\Lin(M) := \Lin((m)_{m \in M})$ (= kleinster UVR von $V$, der alle Elemente aus $M$ enthält)
Vorteil der Definition von $\Lin(\ldots)$ für Familien von Vektoren: Bei Familien ist es sinnvoll zu sagen, dass ein Vektor mehrfach vorkommt (im Gegensatz zu Mengen), darüber hinaus haben die Vektoren der Familie $(v_i)_{i \in I}$ im wichtigen Spezailfall $I = \{1,\ldots,n\}$,
Familie $(v_1, \ldots, v_n)$ eine natürliche Reigenfolge. Diese geht verloren, wenn man die Menge $\{v_1, \ldots, v_n\}$ betrachtet (zum Beispiel in $\mathbb{R}^2:\{e_1, e_2\} = \{e_2, e_1\}$, aber $(e_1, e_2) \neq (e_2, e_1)$)
\begin{defn}[8.14]
$V$ K-VR, $(v_1, \ldots, v_r)$ endliche Familie von Vektoren aus $V$, $(v_1, \ldots, v_r)$ \textbf{linear unabhängig}
\[\xLeftrightarrow{\text{Def}} \lambda_1, \ldots, \lambda_r \in K, \lambda_1 v_1 + \ldots + \lambda_r v_r = 0 \implies \lambda_1 = \ldots = \lambda_r = 0\]
Mit anderen Worten: Der Nullvektor lässt sich nur auf triviale Weise aus der Familie $(v_1, \ldots, v_r)$ linear kombinieren. \\
  $(v_i)_{i\in I}$ heißt \textbf{linear abhängig} $\xLeftrightarrow{\text{Def}} (v_1, \ldots, v_r)$ ist nicht linear unabhängig
\[\iff \Exists \lambda_1, \ldots, \lambda_r \in K: (\lambda_1, \ldots, \lambda_r) \neq (0, \ldots, 0) \wedge \lambda_1 v_1 + \ldots + \lambda_r v_r = 0\]
$(v_i)_{i\in I}$ Familie von Vektoren aus $V$ \\
  $(v_i)_{i\in I}$ heißt linear unabhängig $\xLeftrightarrow{\text{Def}}$ Jede endliche Teilfamilie von $(v_i)_{i\in I}$ ist linear unabhängig, das heißt für jede endliche Teilmenge $J\subseteq I$ ist $(v_i)_{i\in I}$ linear unabhängig. \\
  $(v_i)_{i\in I}$ heißt lenar abhängig $\xLeftrightarrow{\text{Def}}$ $(v_i)_{i\in I}$ ist nicht linear unabhängig \\
  $\iff \Exists$ eine endliche Teilfamilie $(v_i)_{i\in J}$ von $(v_i)_{i \in I}$, die linear abhängig ist \\
  $\iff$ Es gibt eine endliche Teilmenge $J = \{i_1, \ldots, i_r\} \subseteq I, \lambda_{i_1},\ldots,\lambda_{i_r} \in K$ mit
\[(\lambda_{i_1}, \ldots, \lambda_{i_r}) \neq (0, \ldots, 0) \wedge \lambda_{i_1} v_{i_1} + \ldots + \lambda_{i_r} v_{i_r} = 0\]
$M \subseteq V$ heißt linear (un-)ubhängig $\iff (m)_{m\in M}$ ist linear (un-)abhängig.
\end{defn}
\begin{remark}
\begin{itemize}
\item Man sagt häufig statt "$(v_1, \ldots, v_r)$ ist linear (un-)abhängig" auch "die Vektoren $v_1, \ldots, v_r$" sind linear (un-)abhängig."
\item Konvention: $()$ ist linear unabhängig.
\end{itemize}
\end{remark}
\begin{ex}[8.15]
\mbox{}
\begin{enumerate}
\item $V = K^n$ als K-VR \\
     Die Familie $(e_1, \ldots, e_n)$ (vgl 8.10) ist linear unabhängig, denn:
Sind $\lambda_1, \ldots, \lambda_n \in K$ mit $\lambda_1 e_1 + \ldots + \lambda_n e_n = 0$, dann ist
\[\underbrace{\underbrace{\lambda_1 (1, 0, \ldots, 0)}_{=(\lambda_1, 0, \ldots, 0)} + \underbrace{\lambda_2 (0, 1, 0, \ldots, 0)}_{= (0, \lambda_2, 0, \ldots, 0)} + \underbrace{\ldots + \lambda_n (0, \ldots, 0, 1)}_{=(0,\ldots,0,\lambda_n)}}_{= (\lambda_1, \ldots, \lambda_n)} = 0\]
$\implies \lambda_1 = \lambda_2 = \ldots = \lambda_n = 0$
\item $K = \mathbb{R}, V = \mathbb{R}^2$ \\
     Die Familie $((1, -1), (0,2), (1,2))$ ist linearabhängig, denn:
\[2\cdot(1,-1) + 3\cdot(0,2) - 2\cdot(1,2) = 0\]
es gibt also eine nichttiviale Linearkombination der Null aus den Vektoren dieser Familie.
\item $V = K[t]$ als K-VR \\
     Die Familie $(t^n)_{n\in \mathbb{N}_0}$ ist linear unabhängig, denn: \\
     Sei $J = \{n_1, \ldots, n_r\} \subseteq \mathbb{N}_0$ eine endliche Teilmenge von $\mathbb{N}_0$, und sind $\lambda_{n_1}, \ldots, \lambda_{n_r} \in K$
dann folgt aus
\[\lambda_{n_1}t^{n_1} + \ldots + \lambda_{n_r} t^{n_r} = 0\]
sofort: $\lambda_{n_1} = \ldots = \lambda_{n_r} = 0$ (vergleiche Definition von "$=$" von Polynomen)
Also: Jede endliche Teilfamilie von $(t^n)_{n\in\mathbb{N}_0}$ ist linear unabhängig, also ist $(t^n)_{n\in\mathbb{N}_0}$ linear unabhängig.
\end{enumerate}
\end{ex}
\begin{remark}[8.16]
$V$ K-VR, $(v_i)_{i\in I}$ Familie von Vektoren aus $V$ \\
  Dann sind äquivalent:
\begin{enumerate}
\item $(v_i)_{i \in I}$ ist linear unabhängig
\item Jeder Vektor $v \in \Lin((v_i)_{i\in I})$ lässt sich in eindeutiger Weise aus Vektoren deren Familie $(v_i)_{i\in I}$ linear kombinieren.
\end{enumerate}
\end{remark}
\begin{proof}
\begin{enumerate}
\item $\implies$ 2.: Sei $(v_i)_{i\in I}$ linear unabhängig, $v\in \Lin((v_i)_{i\in I}) \implies \Exists$ eine Familie $(\lambda_i)_{i\in I}$ von Elementen aus $K$ mit
$\lambda_i = 0$ für fast alle $i\in I$, sodass
\[v = \sum_{i=I} \lambda_i v_i\]
($\implies$ Existenz einer Linearkombination) \\
     Es sei nun $(\mu_i)_{i\in I}$ eine weitere Familie von Elementen aus $K$ mit $\mu_i = 0$ für fast alle $i \in I$ sodass
\[v = \sum_{i = I} \lambda_i v_i = \sum_{i\in I} \mu_i v_i\]
Wir setzen $J:= \{i\in I \mid \lambda_i \neq 0\} \cup \{i \in I \mid \mu_i \neq 0\}$. Nach Konstruktion ist $J$ endlich, und es ist
\[\underbrace{\sum_{i\in J} (\lambda_i - \mu_i) v_i}_{=\sum_{i\in I} (\lambda_i - \mu_i) v_i} = 0\]
Da $(v_i)_{i\in I}$ linear unabhängig, ist die endliche Teilfamilie $(v_i)_{i\in J}$ linear unabhängig $\implies \lambda_i - \mu_i = 0 \Forall i\in J \implies \lambda_i = \mu_i \Forall i\in J$
für $i\in J\setminus J$ ist ohnehin $\lambda_i = \mu_i = 0$
\[\implies \lambda_i = \mu_i \Forall i\in I\]
\item $\implies$ 1.: Wir setzen voraus, dass sich jeder Vektor $v\in\Lin((v_i)_{i\in I})$ eindeutig aus Vektoren der Familie $(v_i)_{i\in I}$ linear kombinieren lässt. \\
     zu zeigen: $(v_i)_{i\in I}$ ist linear unabhängig, das heißt jede endliche Teilfamilie $(v_i)_{i\in I}$ ist linear unabhängig denn:
Sei $J \subseteq I$ endlich, und sei $(\lambda_i)_{i\in J}$ eine Familie von Elemente aus $K$ mit
\[\sum_{i\in J}\lambda_i v_i = 0\]
Da auch
\[\sum_{i\in J} 0 \cdot \v_i = 0\]
ist, folgt aus der vorrausgesetzen Eindeutigkeit der Linearkombination, dass $\lambda_i = 0 \Forall i\in J \implies (v_i)_{i\in J}$ ist linear unabhängig
\end{enumerate}
\end{proof}
\begin{remark}[8.17]
Sei $V$ K-Vektorraum, Dann gilt:
\begin{enumerate}
\item Ist $v\in V$, dann gilt $(v)$ linear unabhängig $\iff v \neq 0$
\item Gehört der Nullvektor zu einer Familie, dann ist sie linear abhängig
\item Kommt der gleiche Vektor in einer Familie mehrfach vor so ist sie linear abhängig
\item Ist $r\geq 2$, so gilt: Die Familie $(v_1, \ldots, v_r)$ von Vektoren aus $V$ ist linear abhängig $\iff \Exists i \in \{1, \ldots, r\}: v_i$ Linearkombination von $v_1, \ldots, v_{i - 1}, v_{i + 1}, \ldots, v_r$
\end{enumerate}
\end{remark}
\begin{proof}
\begin{enumerate}
\item "$\implies$" (Durch Kontraposition): Sei $v = 0$. Dann ist $1 v = 0$, das heißt $(v)$ ist linear abhängig
"$\impliedby$" (Durch Kontraposition) Sei $(v)$ linear abhängig $\implies \Exists \lambda \in K \setminus \{0\}: \lambda v = 0 \implies v = 0$
\item Wegen $1 \cdot 0_v = 0_v$ existiert in diesem Fall eine nicht triviale Linearkombination der Null.
\item Sei $(v_i)_{i\in I}$ eine Familie, sodass $i_1, i_2 \in I$ existiert mit $i_1 \neq i_2$ und $v_{i_1} = v_{i_2} \implies 1 \cdot v_{i_1} + (-1) \cdot v_{i_2} = 0 \implies (v_i)_{i\in I}$ linear abhängig
\item Sei $r\geq 2, (v_1, \ldots, v_r)$ Familie von Vektoren aus $V$ \\
     "$\implies$" Sei $v_1, \ldots, v_r$ linear abhängig $\implies \Exists \lambda_1, \ldots, \lambda_r \in K: (\lambda_1, \ldots, \lambda_r) \neq (0, \ldots, 0) \wedge \lambda_1 v_1 + \ldots \lambda_r v_r = 0$
     Insbesondere existiert ein $i\in \{1, \ldots ,r\}$, mit $\lambda_i \neq 0$
     \[\implies v_i = -\frac{\lambda_1}{\lambda_i} v_1 - \ldots - \frac{\lambda_{i - 1}}{\lambda_i} v_{i - 1} - \frac{\lambda_{i + 1}}{\lambda_i} v_{i + 1} - \ldots - \frac{\lambda_{r}}{\lambda_i} v_{r}\]
\begin{align*}
\intertext{"$\impliedby$" Sei $i\in\{1,\ldots, r\}$, so dass}
v_i = \lambda_1 v_1 + \ldots + \lambda_{i - 1} v_{i - 1} + \lambda_{i + 1} v_{i + 1} + \ldots + \lambda_{r} v_{r} \\
\intertext{mit geeigneten $\lambda_j \in K$:}
\implies \lambda_1 v_1 + \ldots + \lambda_{i - 1} v_{i - 1} + (-1) v_i + \lambda{i + 1} v_{i + 1} + \ldots + \lambda_r v_r = 0 \\
\implies (v_1, \ldots, v_r) ~\text{linear abhängig} \\
\end{align*}
\end{enumerate}
\end{proof}
\section{Basis und Dimension}
\label{sec-6}
In diesem Abschnitt sei $V$ stets ein K-VR
\begin{defn}[9.1]
$(v_i)_{i\in I}$ Familie von Vektoren aus $V$. $(v_i)_{i\i I}$ heißt \textbf{Erzeugendessystem} (ES) von $V \xLeftrightarrow{Def} V = \Lin((v_i)_{i\in I})$ \\
  $V$ heißt \textbf{endlich erzeugt} $\xLeftrightarrow{\text{Def}} V$ besitzt ein endliches Erzeugendessystem (das heißt es existiert eine endliche Familie $(v_1, \ldots, v_n)$ von Vektoren aus $V$ mit $V = \Lin((v_1, \ldots, v_n))$) \\
  $(v_i)_{i\in I}$ heißt \textbf{Basis} von $V \xLeftrightarrow{\text{Def}} (v_i)_{i\in I}$ ist linear unabhängiges Erzeugendessystem von $V$ \\
  Ist $\mathcal{B} = (v_1, \ldots, v_n)$ eine endliche Basis von $V$, dann heißt $n$ die \textbf{Länge} von $\mathcal{B}$
+end$_{\text{defn}}$
\begin{ex}[9.2]
\begin{enumerate}
\item Die Familie $(e_1, \ldots, e_n)$ ist eine Basis des K$_{\text{VR}}$ $K^n$, da $\Lin((e_1, \ldots, e_n)) = K^n$ (vergleiche 8.10.2) und somit $(e_1,\ldots, e_n)$ Erzeugendessystem des $K^n$, und $(e_1, \ldots, e_n)$ linear unabhängig nach 8.15.1.
Die Länge der Basis ist $(e_1, \ldots, e_n)$ ist $n$. $(e_1, \ldots, e_n)$ heißt die kanonische Basis oder Standardbasis der $K^n$.
\item Die Familie $(t^n)_{n\in\mathbb{N}_0}$ ist eine Basis der K-VR $K[t]$, denn: $\Lin((t^n)_{n\in\mathbb{N}_0}) = K[t]$ nach 8.12, $(t^n)_{n\in\mathbb{N}_0}$ ist linear unabhängig nach 8.15.3
\item $((1, -1), (0,2), (1,2))$ ist ein Erzeugendessystem des $\mathbb{R}$-VR $\mathbb{R}^2$, denn für jedes $(x_1, x_2) \in \mathbb{R}^2$ ist $(x_1, x_2) = x_1(1, -1) + \frac{x_1 + x_2}{e}(0,2)\in \Lin((1,-1, (0,2), (1,2))$,
$((1, -1), (0,2), (1,2))$ ist jedoch keine Basis des $\mathbb{R}$, da linear abhängig nach 8.15.2
\item Die leere Familie $()$ ist eine Basis des Nullraums $\{0\}$: vergleiche 8.11 und Annahme nach 8.14
\end{enumerate}
\end{ex}
\begin{note}
Jeder Vektorraum $V$ besitzt ein ES, denn es ist $V = \Lin((v)_{v \in V})$ \\
  Spannende Frage: Besitzt jeder VR eine Basis? \\
  Wir werden das zunächst für den Fall endlich erzeugter Vektorräqume untersuchen
\end{note}
\begin{thm}[9.3]
$V \neq \{0\}, \mathcal{B} = (v_1, \ldots, v_n)$ endliche Familie von Vektoren aus $V$, dann sind äquivalent:
\begin{enumerate}
\item $\mathcal{B}$ ist eine Basis von $V$, das heißt eine linear unabhängiges ES von $V$
\item $\mathcal{B}$ ist ein unverkürzbares ES von $V$, das heißt $\mathcal{B}$ ist ein ES und für jedes $r\in\{1,\ldots,n\}$ ist $(v_1, \ldots, v_{r - 1}, v_{r + 1}, \ldots, v_n)$ kein ES von $V$ mehr.
\item Zu jedem $v\in V$ gibt es eindeutig bestimmte $\lambda_1, \ldots, \lambda_n \in K$ mit
\[v = \lambda_1 v_1 + \ldots + \lambda_v v_n\]
\item $\mathcal{B}$ ist unverlängerbar linear unabhängig, das heißt $\mathcal{B}$ ist linear unabhängig und für jedes $v\in V$ ist die Familie $(v_1, \ldots, v_n, v)$ linear abhängig
\end{enumerate}
\end{thm}
\begin{proof}
Wir zeigen 1. $\implies$ 2. $\implies$ 3. $\implies$ 4. $\implies$ 1. \\
\begin{enumerate}
\item $\implies$ 2.: Sei $\mathcal{B} = (v_1, \ldots, v_n)$ eine Basis von $V \implies \mathcal{B}$ ist ES von $V$ \\
     Annahme: $\mathcal{B}$ ist verkürzbar, das heißt es gibt ein $r\in\{r, \ldots, n\}$, sodass $(v_1, \ldots, v_{r - 1}, v_{r + 1}, \ldots, v_n)$ immernoch ein ES von $V$
$\implies$ $v_r \in \Lin((v_1, \ldots, v_{r - 1}, v_{r + 1}, \ldots, v_n))$, das heißt
\[\Exists \lambda_1, \ldots, \lambda_{r - 1}, \lambda_{r + 1}, \ldots, \lambda_n\in K: v_r = \lambda_1 v_1 + \ldots + \lambda_{r - 1} v_{r - 1} + (-1)v_r + \lambda_{r + 1} v_{r + 1} + \ldots + \lambda_n v_n\]
$\implies$ $\mathcal{B}$ linear abhängig $\lightning$ zu $\mathcal{B}$ Basis von $V$
\item $\implies$ 1.: Sei $\mathcal{B} = (v_1, \ldots, v_n)$ ein unverkürzbares ES von $V$ $\implies$ Für jedes $v\in V$ existiert $\lambda_1, \ldots, \lambda_n \in K$ mit $v = \lambda_1 v_1 + \ldots + \lambda_n v_n$
     Annahme: Es gibt $v\in V, \lambda_1, \ldots, \lambda_v v_n = \mu_1 v_1 + \ldots \mu_n v_n$
\begin{align*}
\implies (\lambda_i - \mu_i) v_i = (\mu_1 - \lambda_1)v_i + \ldots + (\mu_{i - 1} - \lambda_{i - 1})v_{i - 1} + (\mu_{i + 1} - \lambda_{i + 1}) v_{i + 1} + \ldots + (\mu_n - \lambda_n) v_n \\
\implies v_1 = \frac{\mu_{1} - \lambda_{1}}{\lambda_{i} - \mu_{i}} v_1 + \ldots + \frac{\mu_{i - 1} - \lambda_{i - 1}}{\lambda_{i} - \mu_{i}} v_{i - 1} + \frac{\mu_{i + 1} - \lambda_{i + 1}}{\lambda_{i} - \mu_{i}} v_{i + 1} + \ldots + \frac{\mu_{n} - \lambda_{n}}{\lambda_{i} - \mu_{i}} v_n \\
\end{align*}
$\implies$ Jeder UVR von $v_i$ der $v_1, \ldots, v_{i - 1}, v_{i + 1}, \ldots, v_n$  enthält, enthält auch $v_i$ \\
     $\implies$ $\Lin((v_1, \ldots, v_{i - 1}, v_{i + 1}, \ldots, v_n)) = \Lin((v_1, \ldots, v_n)) = v$ \\
     $\implies$ $\mathcal{B}$ ist verkürzbar $\lightning$
\item $\implies$ 4.: Wir setzte 3. vorraus, das heißt für jedes $v\in V$ existieren eindeutig bestimmt $\lambda_1, \ldots, \lambda_n \in K$ mit $v = \lambda_1 v_1 + \ldots + \lambda_n v_n$. \\
     zu zeigen: $\mathcal{B}$ ist unverlängerbar linear unabhängig \\
     denn Insbesondere existiert für jedes $v\in \Lin((c_1, \ldots, v_n))$ eindeutig bestimmt $\lambda_1, \ldots, \lambda_n \in K$ mit $v = \lambda_1 v_1 + \ldots + \lambda_n v_n \implies (v_1,\ldots, v_n)$ linear unabhängig.
Ist $v\in V$, dann existiert $\lambda_1, \ldots, \lambda_n \in K$ mit $v = \lambda_1 v_1 + \ldots + \lambda_n v_n \implies (v_1, \ldots, v_n)$ linear abhängig. Somit: $\mathcal{B}$ unverlängerbar linear unabhängig
\item $\implies$ 1. Sei $\mathcal{B}$ unverlängerbar linear unabhängig
zu zeigen: $\mathcal{B}$ ist Basis von $V$, das heißt es bleibt noch zu zeigen, dass $\mathcal{B}$ ein ES von $V$ ist \\
     Sei $v\in V \implies (v_1, \ldots, v_n, v)$ linear abhängig $\implies \Exists \lambda_1, \ldots, \lambda_n \in K, (\lambda_1, \ldots, \lambda_n) \neq (0, \ldots, 0): \lambda_1 v_1 + \ldots + \lambda_n v_n + \lambda v = 0$
Es ist $\lambda \neq 0$, da sonst $(v_1, \ldots, v_n)$ linear abhängig
\[\implies v = -\frac{\lambda_1}{\lambda} v_1 - \ldots - -\frac{\lambda_n}{\lambda} v_n \in \Lin((v_1, \ldots, v_n))\]
$\implies$ $\mathcal{B}$ ist ES von $V$
\end{enumerate}
\end{proof}
\begin{conc}[9.4 Basiswahlsatz]
Besitzt $V$ ein endliches ES $(v_1, \ldots, v_n)$, dan kann man aus diesem eine Basis auswählen, das heißt es gibt eine Teilmenge $\{i_1, \ldots, i_r\} \subseteq \{1, \ldots, n\}$, sodass $(v_{i_0}, \ldots, v_{i_r})$ ein Basis von $V$ ist. Insbesondere bestitz jeder endlich erzeuge Vektorraum eine Basis
\end{conc}
\begin{proof}
Entferne aus dem ES $(v_1, \ldots, v_n)$ nacheinander solange Elemente, bis die resultierende Familie ein unverkürzbares ES und somit nach 8.3 eine Basis von $V$ ist.
\end{proof}
\begin{conc}[9.5]
Jeder endlich erzeugte K-VR besitzt eine Basis von endlicher Länge. \\
  Fragen:
\begin{itemize}
\item Ist jede Basis von $V$ von endlicher Länge?
\item Sind je zwei endliche Basen von $V$ gleicher Länge
\end{itemize}
\end{conc}
\begin{thm}[9.6 Austauschlemma]
$V$ endlich erzeugter K-VR, $\mathcal{B} = (v_1, \ldots, v_r)$ von $V,\lambda_1, \ldots,\lambda_r \in K,w = \lambda_1 v_1 + \ldots + \lambda_r v_r$. Dann gilt:
Ist $k\in \{1,\ldots, r\}$ mit $\lambda_k \neq 0$, dann ist
\[\mathcal{B}' = (v_1, \ldots, v_{k - 1}, w, v_{k + 1}, \ldots, v_r)\]
ebenfalls eine Basis von $V$, das heißt man kann $v_k$ gegen $w$ austuaschen
\end{thm}
\begin{proof}
Wir können ohne Einschränkung ("ohne Beschränkung der Allgemeinheit") annehmen, dass $k = 1$ ist (können wir duch Umsortieren erreichen). Gegeben ist $w = \lambda_1 v_1 + \ldots + \lambda_r v_r \in V$, mit $\lambda_1 \neq 0$
zu zeigen ist, dass $\mathcal{B}' = (w, v_2, \ldots, v_r)$ ein Basis von $V$ ist.
\begin{enumerate}
\item $\mathcal{B}'$ ist ein ES von $V$ \\
     Sei $v\in V \implies \Exists \mu_1, \ldots, \mu_r \in K: v = \mu_1 v_1 + \ldots + \mu_r v_r$ \\
     Aus $w = \lambda_1 v_1 + \ldots + \lambda_r v_r$ folgt wegen $\lambda_1 \neq 0$:
\begin{align*}
v_1 = \frac{1}{\lambda_1} w - \frac{\lambda_2}{\lambda_1} v_2 - \ldots - \frac{\lambda_r}{\lambda_1} v \\
v =\frac{\mu_1}{\lambda_1} w + (\mu_2 - \mu_1 \frac{\lambda_2}{\lambda_1}) v_2 + \ldots + (\mu_r - \mu_1 \frac{\lambda_r}{\lambda_1}) v_r \in \Lin ((w, v_2, \ldots, v_r))
\end{align*}
\item $\mathcal{B}'$ ist linear unabhängig, denn: \\
\begin{align*}
\intertext{Sei $\mu, \mu_2, \ldots, \mu_r \in K$ mit $\mu w + \mu_2 v_2 + \ldots + \mu_r v_r = 0$}
\implies \mu(\lambda_1 v_1 + \ldots + \lambda_r v_r) + \mu_2 v_2 + \ldots + \mu_r v_r = 0 \\
\implies \mu \lambda_1 v_1 + (\mu \lambda_2 + \mu_2) v_2 + \ldots + (\mu \lambda_r + \mu_r) v_r = 0 \\
\implies \mu \lambda_1 = 0 \implies \mu = 0 \implies \mu_2 v_2 + \ldots + \mu_r v_r = 0 \implies \mu_2 = \ldots = \mu_r = 0
\end{align*}
\end{enumerate}
\end{proof}
\begin{thm}[Austauschsatz]
$V$ endlich erzeugter K-VR, $(w_1, \ldots, w_n)$ linear unabhängige Familie in $V$. Dann gilt
\begin{enumerate}
\item Ist $\mathcal{B} = (v_1, \ldots, v_r)$ eine Basis von $V$, dann ist $r \geq n$
\item Es giibt Indizes $i_1, \ldots, i_n \in \{1, \ldots, r\}$ derart, dass man aus der Basis $\mathcal{B} = (v_1, \ldots v_r)$ von $V$ nach Austausch von $v_{i_1}$ gegen $w_1$, $v_{i_2}$ gegen $w_2$, $\ldots$, $v_{i_n}$ gegen $w_n$ wieder eine Basis von $V$ erhält.
Nummeriert man $\mathcal{B}$ so um, dass $i_1 = 1, i_2 = 2, \ldots, i_n = n$, bedeutet dies, dass $\mathcal{B}^\ast :=(w_1, \ldots, w_n, w_{n + 1}, \ldots, v_r)$ eine Basis von $V$ ist.
\end{enumerate}
\end{thm}
\begin{proof}
Wir zeigen 1. und 2. zusammen per Induktion nach $n$ \\
  Induktionsanfang: $n = 0$: $(w_1, \ldots, w_n)$ leere Familie \\
  Induktionsschritt: Sei $n \geq 1$, und die Aussage sei für $n - 1$ schon bewiesen. Wegen $(w_1, \ldots, w_a)$ linear unabhängige Familie ist auch $(w_1, \ldots, w_{n - 1})$ linear unabhängig $\implies n - 1 \leq r$ und nach Umnummerieren von $\mathcal{B}$ ist auch
\[\tilde{\mathcal{B}} := (v_1, \ldots, w_{n - 1}, v_n, \ldots, v_r)\]
eine Basis von $V$.

Falls $n - 1 = r$, dann wäre $\tilde{\mathcal{B}} = (w_1, \ldots, w_{n - 1})$ eine Basis von V $\lightning$ (zu 9.3, denn auch $(w_1, \ldots, w_n)$ linear unabhängig).
Also $n - 1 < r$, das heißt $n \leq r$

Da $\tilde{\mathcal{B}}$ Basis von $V, \Exists \lambda_1, \ldots, \lambda_r \in K: w_n = \lambda_1 w_1 + \ldots, \lambda_{n - 1} w_{n - 1} \lambda_n w_n + \ldots + \lambda_r$ \\
  Falls $\lambda_n = \ldots = \lambda_r = 0$, dann $(w_1, \ldots, w_n)$ linear abhängig $\lightning$ \\
  Also existiert ein $k \in \{n, \ldots, r\}$ mit $\lambda_k \neq 0$ Nach umnummerieren von $v_n, \ldots, v_r$ kann man $\lambda_n \neq 0$ erreichen
\[\implies \mathcal{B}^\ast := (w_1, \ldots, w_{n - 1}, w _n, v_{n + 1}, \ldots, v_r)\] ist eine Basis von $V$ (tausche $v_n$ gegen $w_n$)
\end{proof}
\begin{conc}[9.8]
Es gilt:
\begin{enumerate}
\item Ist $V$ endlich erzeutgt, dann ist jede Basis von $V$ von endlicher Länge und je zwei Basen haben dieselbe Länge
\item Ist $V$ nicht endlich erzeugt, dann existiert für $V$ keine Basis von endlicher Länge
\end{enumerate}
\end{conc}
\begin{proof}
\begin{enumerate}
\item \begin{itemize}
\item $V$ endlich erzeugt $\implies$ ex existiert eine endliche Basis $(v_1, \ldots, v_r)$ von $V$, sei $(w_i)_{i\in I}$ beliebige Basis von $V$. Falls $I$ unendlich, dann existiert $i_1, \ldots, i_{r + 1} \in I$, dosass $(w_{i_1} + \ldots + w_{i_{r + 1}})$
        linear unabhängig $\implies$ $r + 1 \leq r \lightning$
\item Sind $(v_1, \ldots, v_r), (w_1, \ldots, w_k)$ endliche Basen von $V$, dann folgt nach Austauschsatz $k \leq r$, sowie $r\leq k \implies r = k$
\end{itemize}
\item Besitzt $V$ eine Basis endlicher Länge, dann ist diese auch ein endliches ES, das heißt $V$ endlich erzeugt $\lightning$
\end{enumerate}
\end{proof}
\#+begin$_{\text{defn}}$ latex
\[\dim_k V := \begin{cases} r & ~\text{falls $V$ endlich erzeugt, $r$ Länge jeder Basis von $V$} \\ \infty & ~\text{falls $V$ nicht endlich erzeugt}\end{cases}\]
heißt die Dimension von $V$ über $K$. Ist $\dim_k V\in\mathbb{N}_0$, dann heißt $V$ endlichdimensional über $K$.
\end{defn}
\begin{note}
\begin{itemize}
\item Der Dimensionsbegriff ist wohldefiniert nach 9.8
\end{itemize}
\end{note}
\begin{ex}[9.10]
\begin{enumerate}
\item $V = K^n$ Die Standardbasis $(e_1, \ldots, e_n)$ von $K^n$ hat Länge von $n$, das heißt $\dim_k K^n = n$. Insbesondere hat jede Basis von $K^n$ die Länge $n$
\item In $K[t]$ ist die Familie $(t^n)_{n\in\mathbb{N}_0}$ eine Basis unendlicher Länge. Wäre $K[t]$ endlichdimensional über $K$, dann wäre jede Basis von $K[t]$ als K-VR von endlicher Länge. Also $\dim_k K[t] = \infty$
\item $\dim_\mathbb{C} \mathbb{C} = 1$, aber $\dim_\mathbb{R} \mathbb{C} = 2$, (denn: $(1,\I)$ inst eine Basis von $\mathbb{C}$ also $\mathbb{R}$-VR)
\end{enumerate}
\end{ex}
\begin{note}
\begin{itemize}
\item Ist klar, welcher Körper gemeint ist schreibt man kurz $\dim V$ statt $\dim_K V$.
\item Offenbar gilt $V$ endlich erzeugt $\iff$ $V$ endlichdimensional
\end{itemize}
\end{note}
\begin{conc}[9.11]
$V$ endlichdimensionaler K-VR, $U\subseteq V$ UVR von $V$ Dann gilt:
\begin{enumerate}
\item $U$ ist endlichdimensional
\item $\dim_k U \leq \dim_K V$
\item Es ist $U = V \iff \dim_k U = \dim_k V$
\end{enumerate}
\end{conc}
\begin{proof}
\begin{enumerate}
\item Annahme: $U$ ist nicht endlichdimensional \\
     Beweis: per Induktion nach $N$ \\
     Induktionsanfang: $n = 1$: Da $n\neq \{0\}$ wegen $\dim_k U = \infty$ existiert $u_1 \in U \setminus \{0\}, (u_1)$ ist linear unabhängig
Induktionsschritt: Sei $n > 1$, die Assage sei für $n - 1$ bewiesen. $\implies$ ex existiert linear unabhängige Familie $(u_1, \ldots, u_{n - 1})$.
Falls $(u_1, \ldots, u_{n - 1}, u)$ linear abhängig für alle $u \in U$, dann wäre $(u_1, \ldots u_{n - 1})$ unverlängerbar linear abhängig und somit nach 9.3 eine Basis von $U~\lightning$ zu $U$ nicht endlichdimensional.
Also existiert ein $u_1 \in U$ mit $(u_1, \ldots, u_n)$ linear unabhängig $\implies$ Behauptung
Wir setzen $r:= \sum_K V$, dann existiert nach 1. eine lineare Familie $(u_1, \ldots, u_{r + 1})$ in $U$. Die Familie $(u_1, \ldots, u_{r + 1})$ ist auch eine linear unabhängige Famnile in $V \implies r + 1 \leq r ~ \lightning$ Das heißt $U$ ist endlich
\item Annahme: $n := \dim_k U > \dim V$ \\
     Sei $(u_1, \ldots, u_n)$ Basis von $U$, das heißt insbesondere ist die Familie $(u_1, \ldots, u_n)$ eine linear unabhängige Familie in $V$ $\implies$ $n \leq \sum_k V ~\lightning$
\item "$\implies$" trivial \\
     "$\impliedby$" Annahme: $U \subsetneq V$ \\
     Sei $(u_1, \ldots, u_r)$ Basis von $U$. Wegen $U \subsetneq V$ ist $(u_1, \ldots, u_r)$ keine Basis von $V$ $\implies$ $\Exists v\in V: (u_1, \ldots, u_r, v)$ linear unabhängig.
$\implies$ es existiert $v \in V$, sodass $(u_1, \ldots, u_r, v)$ linear unabhänig $\implies$ $r + 1 \leq \dim V = \dim U = r \lightning$
\end{enumerate}
\end{proof}
\begin{thm}[9.12 Besisergänzungssatz]
$V$ endlichdimensionaler K-VR, $(u_1, \ldots, u_n)$ linear unabhängige Famile von $V$ \\
  Dass existiert $u_{n + 1}, \ldots, u_r \in V, r = \sum V$, sodass $\mathcal{B} = (u_1, \ldots, u_n, u_{n + 1}, \ldots, u_r)$ eine Basis von $V$ ist (das heißt $(u_1, \ldots, u_n)$ kann zu einer Basis erhänzt werden)
\end{thm}
\begin{proof}
Sei $(v_1, \ldots, v_r)$ eine Basis von $V$. Aus Austauschsatz folgt: Nach umnummerierung von $v_1, \ldots, v_r$ ist $(u_1, \ldots, u_n,  v_{n + 1}, \ldots, v_r)$ eine Basis von $B$ Setze $u_{n + 1} := v_{n + 1}, \ldots, u_r := v_r$
\end{proof}
\begin{thm}[9.13 Zorsches Lemma]
Jede induktiv geordnete nichtleere Menge $(M, \leq)$ besitzt ein maximales Element.
Hierbei heißt eine halbgeordnete Menge $(m, \leq)$ \textbf{induktiv geordnet} $\xLeftrightarrow{\text{Def}}$ Jede Teilmenge $T\subseteq M$, für die $(T, \leq)$ totalgeordnet ist,
besitzt eine obere Schanke in $(M, \leq)$, das heißt $\Exists S\in M: t\leq S \Forall t\in T$
\end{thm}
\begin{note}
Das Zornsche Lemma ist äquivalent zum Auswahlaxiom
\end{note}
\begin{defn}[9.14]
$(u_j)_{j\in J}$ linear unabhängige Familie in $V$. Dann kann $(u_j)_{j \in J}$ zu einer
Basis von $V$ ergänzt werden, das heißt $\Exists I: J\subseteq I, (v_i)_{i\in I}: v_j = u_j \Forall j\in J$, sodass $(v_i)_{i\in I}$  eine Basis von $V$ ist.
Insbesondere besitzt jeder K-VR eine Basis.
\end{defn}
\begin{proof}
\begin{enumerate}
\item Wir setzen $A:= \{uj \mid j \in J\}, M:= \{X \subseteq V \mid A\subseteq X \wedge X ~\text{ist linear unabhängig}\}$
\begin{itemize}
\item $(M, \subseteq)$ ist eine halbgeordnete Menge
\item $(M \neq \emptyset)$, denn $A\in M$
\item $(m, \subseteq)$ ist induktiv geordnet, denn:
Sei $T\subseteq M$, sodass $(T,\subseteq)$ totalgeordnet ist. \\
       zu zeigen: $T$ besitzt eine obere Schranke in $M$. \\
       Wir setzen $\displaystyle S:= \bigcup_{X\in T} X$, dann ist $X\subseteq S \Forall X\in T$.
Noch zu zeigen: $S\in M$, das heißt $A\subseteq S$ und $S$ ist linear unabhängig
\begin{itemize}
\item $A\subseteq S$ klar, denn $A\subseteq X\Forall X\in T$
\item $S$ ist linear unabhängig, das heißt jede endliche Teilfamilie von $(s)_{s\in S}$ inst linear unabhägig: \\
         Seien $s_1, \ldots, s_n \in S$ paarweise verschieden $\implies \Exists X_1, \ldots X_n \in T: s_i \in X_i, i = 1, \ldots, n$ \\
         Da $(T, \subseteq)$ totalgeordnet ist existiert ein $i\in \{1, \ldots, n\}$ mit $X_j \subseteq X_i$ für alle $j\in \{1,\ldots,n\} \implies s_1, \ldots, s_n \subseteq X_i \xRightarrow{X_i \in M} (x_1, \ldots, s_n)$ linear unabhängig.
\end{itemize}
\end{itemize}
\item Nach 1. können wir das Zornsche Lemma auf $(M,\subseteq)$ anwenden $\implies \Exists \max B \in M$ \\
     Behauptung: $(b)_{b\in B}$ ist eine Basis von $V$ mit $A\subseteq B$, denn: Da $(b)_{b\in B}$ linear unabhängig wegen $V\in M$, gilt zu Zeigen, dass $\Lin(B) = V)$ \\
     "$\subseteq$" klar \\
     "$\supseteq$"
\begin{align*}
\intertext{Sei $v \in V$. Falls $v\in B$, dann $v\in\Lin(B)$, falls $v = 0$, dann $v\in \Lin(B)$, im Folgenden sei $v\not\in B, v\neq 0$}
\implies A \subseteq B \subsetneq B\cup \{v\} \\
\intertext{Da $B$ Maximum von $(M,\subseteq)$ ist, ist $B \cup \{v\} \not \in M$, das heißt $B\cup \{v\}$ ist linear abhängig}
\implies \Exists n\in\mathbb{N}_0, \lambda_1, \ldots, \lambda_n \in K, \lambda \in K, (\lambda, \lambda_1, \ldots, \lambda_n) \neq (0, \ldots, 0), b_1, \ldots, b_n \in B: \\
\lambda v + \lambda_1 b_1 + \ldots + \lambda_n b_n = 0 \\
\intertext{Falls $n = 0$}
\lambda v = 0 \xRightarrow{v \neq 0} \lambda = 0 \lightning \\
\intertext{Also $n \geq 1$, Falls $\lambda = 0$}
(b_1, \ldots, b_n) ~\text{linear abhängig} \lightning \\
\intertext{Also $\lambda \neq 0$, somit:}
v = -\frac{\lambda_1}{\lambda} v_1 - \ldots - \frac{\lambda_n}{\lambda} b_n \in\Lin((b_1, \ldots, b_n)) \subseteq \Lin(B)
\end{align*}
\end{enumerate}
\end{proof}
\begin{note}
Der Satz "Jeder VR hat eine Basis" ist äquivalent zum Auswahlaxiom.
\end{note}
\section{{\bfseries\sffamily DONE} Matrizen}
\label{sec-7}
In diesem Abschnitt seinen $m,n,r\in\mathbb{N}$ \\
  Frage: Gegeben sei ein UVR $U = \Lin((v_1, \ldots, v_m)) \subseteq K^n$ Wie bestimmt man effizient die Basis von $U$?
\begin{defn}[10.1]
Eine $m\times n$-Matrix mit Einträgen aus $K$ ist eine Familie $(a_{11}, \ldots, a_{1n}, a_{21}, \ldots, a_{2n}, \ldots, a_{m1}, \ldots, a_{mn})$
von $m n$ Elementen aus $K$, die wir in der Form
\[(a_{ij})_{\substack{1\leq i \leq m \\ 1 \leq j \leq n}} = \begin{pmatrix} a_{11} & \ldots & a_{1n} \\ \vdots & & \vdots \\ a_{m 1} & \ldots & a_{mn} \end{pmatrix} \tag{kurz $(a_{ij})$ wenn $m,n$ klar sind}\]
schreiben. Die Menge aller $m\times n$ Matrizen mit Einträgen aus $K$ bezeichnen wir mit $M(m\times n, K)$.
Für $A = (a_{ij})$ wie oben heißen
\[a_i := (1_{i1} \ldots a_{in}), i = 1,\ldots, m\]
Die Zeilen der Matrix $A$. Im Folgenden fassen wir die Zeilen von $A$ als Elemente von $K^n$ auf: $a_i = (a_{i1}, \ldots, a_{in})$
\[\begin{pmatrix} a_{1j} \\ \vdots \\ a_{mj}\end{pmatrix} \in M(m\times 1, K), j = 1, \ldots, n\]
heißen die Spalten der Matrix $A$
\end{defn}
\begin{remark}[10.2]
Es gilt:
\begin{enumerate}
\item $M(m\times n, K)$ ist bezüglich der Verknüpfungen
\begin{align*}
+ &: M(m\times n, K) \times M(m\times n, K) \to M(m\times n, K), (a_{ij}) + (b_ij):= (a_{ij} + b_{ij}) \\
\cdot &:K\times M(m\times n, K) \to M(m\times n, K), \lambda \cdot (a_{ij}):=(\lambda a_{ij})
\end{align*}
ein K$_{\text{VR}}$. Es ist $\dim_k M(m\times n, K) = m\cdot n$
\item Durch
\[\cdot : M(m\times n, K) \times M(n\times r, K) \to M(m\times r, K)\]
\[(a_{ij})_{\substack{1\leq i \leq m \\ 1\leq j\leq n}} \cdot (b_{jk})_{\substack{1\leq j \leq n \\ 1\leq k \leq r}} := (c_{ik})_{\substack{1\leq i \leq m \\ 1\leq k \leq r}}\]
mit
\[c_{ik} := \sum_{j = 1}^{n}a_{ij} b_{jk}\]
ist die Multiplikation von Matrizen erklärt. Visualisierung:
\begin{equation}
\begin{pmatrix}
\\
a_{i1} & a_{i2} & \ldots & a_{in} \\
\\
\end{pmatrix}
\cdot
\begin{pmatrix}
& b_{1k} & \\
& \vdots & \\
& b_{nk} & \\
\end{pmatrix}
=
\begin{pmatrix}
&  & \\
& c_{ij} & \\
&  & \\
\end{pmatrix}
\end{equation}
Für diese gilt: Sind $A_1, A_2 \in M(m\times n, K), B_1, B_2 \in M(n\times r, K), C\in M(r\times s, K), \lambda \in K$, dann ist
\begin{itemize}
\item $A(B_1 + B_2) = A\cdot B_1 + A \cdot B_2, (A_1 + A_2) B = A_1 B + A_2 B$
\item $A(\lambda B) = (\lambda A) B = \lambda (A B)$
\item $A(B C) = (AB) C$
\item $E_m \cdot A = A \cdot E_n = A$
\end{itemize}
Hierbei ist für $l \in \mathbb{N}$
\[E_l := \begin{pmatrix} 1 & & 0 \\ & \ddots & \\ 0 & & 1 \end{pmatrix} \in M(l\times l, K)\]
die $l\times l$- Einheitsmatrix über $K$
\item $M(n\times n, K)$ ist bezüglich
\[+,\cdot : M(n\times n, K) \times M(n\times n, K) \to M(n\times n, K)\]
("$+$" siehe 1., "$\cdot$" siehe 2.) \\
     ein Ring. (Einselement: $E_n$). Für $n > 1$ ist dieser Ring \textbf{nicht} kommutativ
\end{enumerate}
\end{remark}
\begin{proof}
Nachrechnen \\
  Zu $\dim M(m\times n, K) = m\cdot n$: Eine Basis von $M(m\times n, K)$ ist durch die Familie der Matrizen $E_{ij}, i = 1, \ldots, m, j = 1,\ldots, n$ gegeben, wobei $E_{ij}$ dijenige $m\times n$-Matrix mit Einträgen in $K$ bezeichen, die in der i-ten Zeile, j-ten Spalte eine Eins
und sonst nur Nullen stehen hat. \\
  zu 3.: $M(n\times n, K)$ ist nicht kommutativ für $n > 1$:
\begin{align*}
&\sbox0{$\begin{matrix} 1 & 0 \\ 0 & 0 \end{matrix}$}%
\sbox1{$\begin{matrix} 0 & 1 \\ 0 & 0 \end{matrix}$}%
\left(
\begin{array}{c|c}
\usebox{0}&\makebox[\wd0]{$0$} \\
\hline
\vphantom{\usebox{0}}\makebox[\wd0]{$0$}&\makebox[\wd0]{$0$}
\end{array}
\right)
\cdot
\left(
\begin{array}{c|c}
\usebox{1}&\makebox[\wd1]{$0$} \\
\hline
\vphantom{\usebox{1}}\makebox[\wd1]{$0$}&\makebox[\wd1]{$0$}
\end{array}
\right)
=
\left(
\begin{array}{c|c}
\usebox{1}&\makebox[\wd1]{$0$} \\
\hline
\vphantom{\usebox{1}}\makebox[\wd1]{$0$}&\makebox[\wd1]{$0$}
\end{array}
\right) \\
\neq&
\sbox0{$\begin{matrix} 1 & 0 \\ 0 & 0 \end{matrix}$}%
\sbox1{$\begin{matrix} 0 & 1 \\ 0 & 0 \end{matrix}$}%
\sbox2{$\begin{matrix} 0 & 0 \\ 0 & 0 \end{matrix}$}%
\left(
\begin{array}{c|c}
\usebox{1}&\makebox[\wd1]{$0$} \\
\hline
\vphantom{\usebox{1}}\makebox[\wd1]{$0$}&\makebox[\wd1]{$0$}
\end{array}
\right)
\cdot
\left(
\begin{array}{c|c}
\usebox{0}&\makebox[\wd0]{$0$} \\
\hline
\vphantom{\usebox{0}}\makebox[\wd0]{$0$}&\makebox[\wd0]{$0$}
\end{array}
\right)
=
\left(
\begin{array}{c|c}
\usebox{2}&\makebox[\wd1]{$0$} \\
\hline
\vphantom{\usebox{1}}\makebox[\wd1]{$0$}&\makebox[\wd1]{$0$}
\end{array}
\right) \\
\end{align*}
\end{proof}
\begin{defn}[10.3]
$A$ heißt invertierbar $\xLeftrightarrow{\text{Def}} \Exists B\in M(n\times n, K): AB = BA = E_n$
\end{defn}
\begin{remark}[10.4]
Es gilt:
\[GL(n, K) := \{A\in M(n\times n, K) \mid A ~\text{ist invertierbar}\}\]
ist bezüglich der Matrizenmultiplikation eine Gruppe, die sogenannte allgemiene lineare Gruppe.
Das neutrale Element ist $E_n$, das zu $A\in GL(n,K)$ inverse Element bezeichnen wir mit $A^{-1}$.
\end{remark}
\begin{proof}
Wohldefiniertheit von "$\cdot$" auf $GL(n, K)$ \\
  zu zeigen ist: $A_1, A_2 \in GL(n, K) \implies A_1 A_2 \in GL(n,K)$ \\
  Dies gilt, denn:
\begin{align*}
A_1, A_2 \in GL(n, K) \implies \Exists B_1, B_2 \in M(n\times n, K): A_1 B_1 = B_1 A_1 = E_n, A_2 B_2 = B_2 A_2 = E_N \\
\implies (A_1 A_2) \cdot (B_2 B_1) = A_1(A_2 B_2) B_1 = A_1 E_n B_1 = A_1 B_1 = E_n \\
(B_2 B_1) \cdot (A_1 A_2) = B_2 (B_1 A_1) A_2 = B_2 E_n A_2 = B_2 A_2 = E_n
\end{align*}
das heißt $A_1 A_2 \in GL(n,K)$
\begin{itemize}
\item Assoziativiät: Vererbt sich von $M(n\times n, K)$
\item neutrales Element: $E_n$
\item Existenz von Inversen: Sei $A\in GL(n,K) \implies \Exists B\in M(n\times n, K): AB = BA = E_n$ also ist $B\in GL(n,K)$ und $B$ ist invers zu $A$ bezüglich "$\cdot$".
\end{itemize}
\end{proof}
% Emacs 25.1.1 (Org mode 8.2.10)
\end{document}