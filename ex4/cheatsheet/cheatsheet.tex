\documentclass[9pt, landscape,a4paper]{extarticle}
\usepackage{tikz}
\usepackage{multicol}
\usepackage[top=1cm,bottom=1cm,left=1cm,right=1cm]{geometry}
\usepackage{braket}
\usepackage{microtype}
\usepackage{amsfonts}
\usepackage{amssymb}
\usepackage{mathtools}
\usepackage{siunitx}
\usepackage{mhchem}
%\usepackage{mathspec}
\usepackage[ngerman]{babel}
\usepackage{nath}
%\usepackage{unicode-math}
\usepackage{stmaryrd}
\usepackage{stackengine}
%\usepackage{polyglossia}
\usepackage{newunicodechar}
\newunicodechar{α}{\alpha}
\newunicodechar{Α}{\alpha}
\newunicodechar{β}{\beta}
\newunicodechar{Β}{\beta}
\newunicodechar{γ}{\gamma}
\newunicodechar{δ}{\delta}
\newunicodechar{ε}{\varepsilon}
\newunicodechar{Ε}{\varepsilon}
\newunicodechar{ζ}{\zeta}
\newunicodechar{η}{\eta}
\newunicodechar{θ}{\theta}
\newunicodechar{ι}{\iota}
\newunicodechar{κ}{\kappa}
\newunicodechar{λ}{\lambda}
\newunicodechar{μ}{\mu}
\newunicodechar{Μ}{M}
\newunicodechar{ν}{\nu}
\newunicodechar{Ν}{N}
\newunicodechar{ξ}{\xi}
\newunicodechar{ο}{\omicron}
\newunicodechar{π}{\pi}
\newunicodechar{Π}{\Pi}
\newunicodechar{ρ}{\rho}
\newunicodechar{Ρ}{P}
\newunicodechar{σ}{\sigma}
\newunicodechar{τ}{\tau}
\newunicodechar{Τ}{T}
\newunicodechar{υ}{\upsilon}
\newunicodechar{φ}{\varphi}
\newunicodechar{χ}{\chi}
\newunicodechar{ψ}{\psi}
\newunicodechar{ω}{\omega}
\newunicodechar{∀}{\forall}
\newunicodechar{×}{\times}
\newunicodechar{Γ}{\Gamma}
\newunicodechar{Δ}{\Delta}
\newunicodechar{∃}{\exists}
\newunicodechar{ℤ}{\mathbb{Z}}
\newunicodechar{∧}{\wedge}
\newunicodechar{Θ}{\Theta}
\newunicodechar{⇒}{\implies}
\newunicodechar{∩}{\cap}
\newunicodechar{Λ}{\Lambda}
\newunicodechar{∫}{\int}
\newunicodechar{ℕ}{\mathbb{N}}
\newunicodechar{Ξ}{\Xi}
\newunicodechar{∇}{\nabla}
\newunicodechar{Π}{\Pi}
\newunicodechar{ℝ}{\mathbb{R}}
\newunicodechar{Σ}{\Sigma}
\newunicodechar{⇔}{\iff}
\newunicodechar{Υ}{\Upsilon}
\newunicodechar{Φ}{\Phi}
\newunicodechar{ℂ}{\mathbb{C}}
\newunicodechar{Ψ}{\Psi}
\newunicodechar{Ω}{\Omega}
\newunicodechar{ϑ}{\vartheta}
\newunicodechar{∞}{\infty}
\newunicodechar{∈}{\in}
\newunicodechar{⊂}{\subset}
\newunicodechar{ϰ}{\varkappa}
\newunicodechar{ϕ}{\phi}
\newunicodechar{∨}{\vee}
\newunicodechar{∮}{\oint}
\newunicodechar{↦}{\mapsto}
\newunicodechar{ℚ}{\mathbb{Q}}
\newunicodechar{⊆}{\subseteq}
\newunicodechar{⊊}{\subsetneq}
\newunicodechar{∪}{\cup}
\newunicodechar{·}{\cdot}
%\setdefaultlanguage[spelling=new, babelshorthands=true]{german}
\makeatletter
\let\mathop\o@mathop
\makeatother

%\definecolor{myblue}{cmyk}{1,.72,0,.38}
\definecolor{myblue}{cmyk}{1,1,1, 1}
\def\firstcircle{(0,0) circle (1.5cm)}
\def\secondcircle{(0:2cm) circle (1.5cm)}

\everymath\expandafter{\the\everymath \color{myblue}}
\everydisplay\expandafter{\the\everydisplay \color{myblue}}

\renewcommand{\baselinestretch}{.8}
\pagestyle{empty}
\setlength{\mathindent}{0pt}

\makeatletter
\renewcommand{\section}{\@startsection{section}{1}{0mm}%
                                {.2ex}%
                                {.2ex}%x
                                {\sffamily\small\bfseries}}
\renewcommand{\subsection}{\@startsection{subsection}{1}{0mm}%
                                {.2ex}%
                                {.2ex}%x
                                {\sffamily\bfseries}}
\renewcommand{\subsubsection}{\@startsection{subsubsection}{1}{0mm}%
                                {.2ex}%
                                {.2ex}%x
                                {\sffamily\small\bfseries}}



\def\multi@column@out{%
   \ifnum\outputpenalty <-\@M
   \speci@ls \else
   \ifvoid\colbreak@box\else
     \mult@info\@ne{Re-adding forced
               break(s) for splitting}%
     \setbox\@cclv\vbox{%
        \unvbox\colbreak@box
        \penalty-\@Mv\unvbox\@cclv}%
   \fi
   \splittopskip\topskip
   \splitmaxdepth\maxdepth
   \dimen@\@colroom
   \divide\skip\footins\col@number
   \ifvoid\footins \else
      \leave@mult@footins
   \fi
   \let\ifshr@kingsaved\ifshr@king
   \ifvbox \@kludgeins
     \advance \dimen@ -\ht\@kludgeins
     \ifdim \wd\@kludgeins>\z@
        \shr@nkingtrue
     \fi
   \fi
   \process@cols\mult@gfirstbox{%
%%%%% START CHANGE
\ifnum\count@=\numexpr\mult@rightbox+2\relax
          \setbox\count@\vsplit\@cclv to \dimexpr \dimen@-1cm\relax
% \setbox\count@\vbox to \dimen@{\vbox to 1cm{\header}\unvbox\count@\vss}%
\else
      \setbox\count@\vsplit\@cclv to \dimen@
\fi
%%%%% END CHANGE
            \set@keptmarks
            \setbox\count@
                 \vbox to\dimen@
                  {\unvbox\count@
                   \remove@discardable@items
                   \ifshr@nking\vfill\fi}%
           }%
   \setbox\mult@rightbox
       \vsplit\@cclv to\dimen@
   \set@keptmarks
   \setbox\mult@rightbox\vbox to\dimen@
          {\unvbox\mult@rightbox
           \remove@discardable@items
           \ifshr@nking\vfill\fi}%
   \let\ifshr@king\ifshr@kingsaved
   \ifvoid\@cclv \else
       \unvbox\@cclv
       \ifnum\outputpenalty=\@M
       \else
          \penalty\outputpenalty
       \fi
       \ifvoid\footins\else
         \PackageWarning{multicol}%
          {I moved some lines to
           the next page.\MessageBreak
           Footnotes on page
           \thepage\space might be wrong}%
       \fi
       \ifnum \c@tracingmulticols>\thr@@
                    \hrule\allowbreak \fi
   \fi
   \ifx\@empty\kept@firstmark
      \let\firstmark\kept@topmark
      \let\botmark\kept@topmark
   \else
      \let\firstmark\kept@firstmark
      \let\botmark\kept@botmark
   \fi
   \let\topmark\kept@topmark
   \mult@info\tw@
        {Use kept top mark:\MessageBreak
          \meaning\kept@topmark
         \MessageBreak
         Use kept first mark:\MessageBreak
          \meaning\kept@firstmark
        \MessageBreak
         Use kept bot mark:\MessageBreak
          \meaning\kept@botmark
        \MessageBreak
         Produce first mark:\MessageBreak
          \meaning\firstmark
        \MessageBreak
        Produce bot mark:\MessageBreak
          \meaning\botmark
         \@gobbletwo}%
   \setbox\@cclv\vbox{\unvbox\partial@page
                      \page@sofar}%
   \@makecol\@outputpage
     \global\let\kept@topmark\botmark
     \global\let\kept@firstmark\@empty
     \global\let\kept@botmark\@empty
     \mult@info\tw@
        {(Re)Init top mark:\MessageBreak
         \meaning\kept@topmark
         \@gobbletwo}%
   \global\@colroom\@colht
   \global \@mparbottom \z@
   \process@deferreds
   \@whilesw\if@fcolmade\fi{\@outputpage
      \global\@colroom\@colht
      \process@deferreds}%
   \mult@info\@ne
     {Colroom:\MessageBreak
      \the\@colht\space
              after float space removed
              = \the\@colroom \@gobble}%
    \set@mult@vsize \global
  \fi}

\setlength{\parindent}{0pt}
\newcommand\ubar[1]{\stackunder[1.2pt]{\(#1\)}{\rule{1.25ex}{.08ex}}}

\expandafter\def\expandafter\normalsize\expandafter{%
    \normalsize
    \setlength\abovedisplayskip{-100pt}
    \setlength\belowdisplayskip{-100pt}
    \setlength\abovedisplayshortskip{-100pt}
    \setlength\belowdisplayshortskip{-100pt}
%    \setlength\beloweqnsskip{-100pt}
%    \setlength\displaybaselineskip{-100pt}
%    \setlength\displaylineskip{-100pt}
}

\renewcommand\v[1]{\vec{#1}}
\renewcommand\d{\mathrm{d}}
\renewcommand{\vec}[1]{\mathbf{#1}}
\newcommand*\abs[1]{\lvert#1\rvert}
\newcommand*\Laplace{\mathop{}\!\mathbin\triangle}
\newcommand\VAR{\mathrm{VAR}}
\newcommand{\dd}[2]{\frac{\d #1}{\d #2}}
\newcommand{\pp}[2]{\frac{\partial #1}{\partial #2}}
\newcommand{\const}{\ensuremath{\text{ const.}}}%
%\newcommand*{\estimates}{\overset{\scriptscriptstyle\wedge}{=}}%

\raggedbottom
\begin{document}
\small
\begin{multicols*}{3}
\raggedcolumns
Konstanten: \\
$α = 1/137$, $\hbar c \approx \SI{197}{\electronvolt\nano\meter}$, $\SI{1}{\barn} = \SI{e24}{\centi\meter\squared}$, $N_A = \SI{6e23}{\per\mol}$ \\
$\hbar = \SI{6.58212e-16}{\electronvolt\second}$, $e = \SI{1.602e-19}{\coulomb}, c = \SI{2.998e8}{\meter\per\second}$

Wasserstoff: $E_n = -Ry Z^2 / n^2$ \\
Bohr'scher Atomradius: $a_0 = \frac{4π ε_0 \hbar^2}{m_e e^2}$ \\
Ionisations-Energie: $E_n = \frac{μ Z^2 e^2}{8ε_0^2 h^2 n^2}$ \\
Mehrelektronensystem: Wechselwirkung zwischen $e^{-}$ + Kopplung, Abschirmung. \\
Auswahlregeln: $ΔL = \pm 1, Δ m_L = 0, \pm 1, ΔS = 0, ΔJ = 0, \pm 1, J = 0 \not \to J = 0$ \\
Helium: $S = 0$: Para-Helium, $S = 1$: Ortho-Helium \\
Spektroskopische Notation: $\ce{^{2S+1} L_J}$, $L$: $s = 0, p = 1, d = 2, f = 3, \dots$ \\
Näherungsverfahren: Einteilchenlösung + effektives Potential (iterativ) \\
\textbf{Drehimpulskopplung}:
$L-S$-Kopplung: falls Kopplungsenergie zwischen Bahndrehimpulsen und zwischen Spin groß ($Z$ < 10). \\
$\abs{l_1 - l_2} \leq L \leq l_1 + l_2$, $\abs{s_1 - s_2} \leq S \leq s_1 + s_2$, $\abs{S - L} \leq J \leq S + L$ \\
$c \v L \v S = c(J(J + 1) - L(L + 1) - S(S + 1))$ \\
$JJ$-Kopplung: $j_i = l_i + s_i, J = Σ j_i$ ($Z > 50$) \\
Reihenfolge der Energiezustände: \\
$1s, 2s, 2p, 3s, 3p, [4s, 3d], 4p, [5s, 4d], 5p, [6s, 5d 4f], 6p, [7s, 6d, 5f]$ \\
Schale $n$: $K = 1$, $L = 2$, $M = 3$, $\dots$ \\
\textbf{Hundsche Regeln}: \\
1. Nach	energetischer Reihenfolge alle möglichen Unterschalen komplett befüllen ($L = 0, S = 0, J = 0$) \\
2. In offener Schale Spin maximieren (symmetrische Spinwellenfunktion $\to$ maximaler Abstand $\to$ minimale Coulombenergie) \\
3. $L$ maximieren \\
4. offene Schale weniger als halb gefüllt? $J = \abs{L - S}$, mehr als halb gefüllt: $J = L + S$ \\

Pauli: Fermionen sind ununterscheidbar $\to$ keine gleichen Sets von Quantenzahlen. \\
$\to$ Gesamtwellenfunktion antisymmetrisch, Spin $1$: symmetrisch, Spin $0$: antisymmetrisch \\
Symmetrie der Ortswellenfunktion: $(-1)^L$ \\

Zerfall angeregter Zustände: $N(t) = N_0 \exp(-t/τ)$ \\
\textbf{Linienbreite}: $Δω = 1/τ, ΔE τ \geq \hbar$ \\
$P(ω) = P_0 \frac{γ/(2π)}{(ω - ω_0)^2 + (γ/2)^2}, γ = 1/τ_i + 1/τ_k$ \\
Dipolstrahlung: $ΔE = E_k - E_i = (Z - Σ)^2 Ry h c (1/n^2_k - 1/n^2_i)$	\\
\textbf{Röntgenstrahlung}: \\
Streuung + Absorption: $I(d) = I_0 e^{-μd}$ \\
Absorptionskante: Energie um gerade Elektronen aun $K, L, M$ Schale zu ionisieren. \\

$\ce{^{A}_Z X_N}$: $A$ Massenzahl, $Z$ Kernladungszahl, $N$ Neutronenzahl \\
% Kernradius: $R_K^3 \sim A, R_k \approx \SI{1.22}{\femto\meter} A^{1/3}$ \\
Wechselwirkungen: Starke $\sim 1$, elektromagnetische $\sim 10^{-2}$, schache $\sim 10^{-7}$, Gravitation $\sim 10^{-39}$. \\
Stärke der Wechselwirkung $⇔$ Dominanz bei Zerfällen. \\

Energieimpulsbeziehung: $E^2 = p^2 c^2 + m^2 c^4, E = γ m c^2, γ = \sqrt{1 - (\frac{v}{c})^2}^{-1}$ \\
$pβc = 2 E_{\text{kin}}$, $βγ = p/m \sim p$, $E / (m c^2) = γ$, $p c / E = β^{-1}$ \\
4er-Impuls: $p = (E / c, \v p)$, invariante Masse: $p^2$ \\
Energie eines Teilchens, dass an einem Teilchen mit Masse $M$ unter Winkel $θ$ streut: \\
$\displayed{E' = E \frac{1}{1 + \frac{E}{M c^2}(1 - \cos θ)}}$ \\

Feynman-Diagram: innere Linien $⇔$ virtuelle Teilchen, Verletzung der Energie-Impulsbeziehung, äußere Linien $⇔$ reelle	Teilchen, Einhaltung der Energie-Impulsbeziehung. \\
raumartig: $q^2 > 0$, zeitartig: $q^2 < 0$ \\
Propagator Elektromagnetischer Wechselwirkung: $A \sim e^2 / q^2 \sim α/q^2$ \\
Vakuumpolarisation $⇒$ Running $α$ \\

\textbf{Wirkungquerschnitt} \\
Teilchenfluss $ϕ = \dot N_i / A = n_i v_i$ \\
$\dot N_i$: Teilchenrate, $n_i$: Teilchendichte, $v_i$: Teilchengeschwindigkeit \\
$\d \dot N_s \sim ϕ N_t \d Q$ \\
Proportionalitätskostante: differentieller Wirkungsquerschnitt: \\
$σ_{diff} = \dd{σ}{Ω} = \frac{\d \dot N_s}{ϕ N_t \d Ω}$ \\
Totaler Wirkungsquerschnitt: $σ_{tot} = \dot N_s / (ϕ N_t)$ \\
$n_t = ρ N_A / M_{mol}$, mittlere freie Weglänge $λ = 1/(n_t σ)$ \\
$ϕ(x) = ϕ_0 e^{-x/λ}$ \\
Luminosität $\dot N_s = Lσ, ϕ N_t = L$, Fixed-target: $L= \dot N_i n_t d$, Speicherring: $L = f N_1 N_2 / A$ \\
$N_1, N_2$ = Zahl der Teilchen pro Strahl, $f$ Umlauffrequenz, $A$ = Strahlquerschnitt. \\
Linienbreite $Γ = \hbar / τ = ΔE$ \\
Breit-Wigner Form:
$\displayed{\tilde P(E) = \frac{1}{2π} \frac{Γ}{(E - mc^2)^2 + Γ^4 / 4}}$ \\
Lebensdauer $ω = 1/τ$, $ω = \frac{\dot N_s}{N_i N_t} = v_i σ / V$ \\

Fermi's goldene Regel: $ω_{fi} = \frac{2π}{\hbar} \abs{A_{fi}}^2 ρ(E_f)$ \\
\textbf{Wirkungsquerschnitt} $σ = ω_{fi} / v_i$ \\
klassisch: Teilchenzustand beschrieben durch Ort und Impuls \\
QM: Unschärfe $⇒$ Phasenraumvolumen, $L d p_x \approx h$ \\
1-Teilchen-Zustandsdichte:
$\displayed{\dd{N_1^{3D}}{E_1} = \frac{V}{(2π\hbar)^3} \frac{E_1 p_1}{c^2}∫ \d Ω_1}$ \\
2-Teilchen-Zustandsdichte:
$\displayed{\dd{N_2^{3D}}{E_1} = \frac{V}{(2π\hbar)^3} \frac{E_1  E_2 p_1}{c^2 (E_1 + E_2)}∫ \d Ω_1}$ \\
n-Teilchen-Zustandsdichte:
$\displayed{\dd{N_n^{3D}}{E_1} = \frac{V^{n - 1}}{(2π\hbar)^{3(n - 1)}} \dd{}{E} ∫ \d^3 p_2 \dots \d^3 p_{n - 1}}$ \\
2-Teilchenreaktion $a + b \to 1 + 2$ \\
$\displayed{ω_{fi} = ∫ \frac{2π}{\hbar} \abs{A_{fi}}^2 \frac{1}{(2π\hbar)^3} \frac{1}{c^2} \frac{E_1 E_2}{E_1 + E_2} p_1 \d Ω_1}$ \\
$\displayed{⇒ \d σ = \frac{1}{(2π)^2} \frac{1}{\hbar^4 c^4} \frac{p_1^2 c^2}{β_i β_f} \abs{A_{fi}}^2 \d Ω_1}$ \\
$Γ = \hbar / τ, \d Γ = \hbar \d ω_{fi}$ \\

$f\bar f$ Anihilation: $A_{fi} \sim e_i e_f / q^2 \sim 4π α Q_f \hbar^3 c / q^2$ \\
Spin Anteil: Photon hat Spin $1 ⇒$ nur $\ket{\uparrow \uparrow} \sim A_1 (1 + \cos θ_f) / 2$ oder $\ket{\downarrow\downarrow} \sim A_2 (1 - \cos θ_f)$ erlaubt. \\
Vertauschung von $f \bar f$ zu $\bar f f ⇒ A_3 = A_2, A_4 = A_1$ \\
$\abs{A_{fi}}^2 = \sum \abs{A_i}^2 / 4  = (1 + \cos^2 θ_f) \frac{(4 π α)^2}{4 E_{\text{CMS}^2}} (\hbar c)^3$ \\
$\displayed{\dd{σ}{Ω} = \frac{α^2}{4 E_{CMS}^2}(1 + \cos^2 θ_f) (\hbar c)^2}$ \\
$\displayed{σ_{tot} = \frac{4πα^2}{3} Q_f^2 \frac{1}{E_{CMS}^2} (\hbar c)^2}$ % = Q_f^2 \frac{\SI{87}{\nano\barn}}{E_{CMS}^2[\SI{}{\giga\electronvolt}]}}$ \\
\textbf{Ionisation}: Bethe-Bloch-Formel: \\
$\displayed{-\dd{E}{x} = (ρ N_A \frac{Z}{A}) \frac{4π^2 z^2 e^4}{m_e c^2 β^2}(\ln \frac{2m_e c^2 β^2 γ^2}{I} - β^2)}$ \\
$I \approx Z \SI{10}{\electronvolt}$ (mittleres Ionisationspotential der Elektronen) \\
$\displayed{-\frac{1}{ρ} \dd{E}{x} = K \frac{Z}{A} z^2 \frac{1}{β^2}(\ln \frac{2 m_e c^2 β^2 γ^2}{I} - β^2)}$, $K = \SI{0.307}{\mega\electronvolt\per\gram\per\centi\meter\cubed}$ \\
In der Nähe des	Minimums ($βγ \approx 3, \dots, 4$): $\displayed{-\frac{1}{ρ}\dd{E}{x} \approx \SI{2}{\mega\electronvolt\per\centi\meter\cubed\per\gram}}$ \\
kleines $βγ: -\dd{E}{x} \sim 1/β^2$ \\

$\displayed{r_e^2 = (\frac{e^2}{4πε_0} \frac{1}{m_e c^2})^2}$ \\
\textbf{Bremsstrahlung}: $\displayed{-(\dd{E}{x}) = ρ 4α N_A \frac{Z^2}{A^2} r_e^2 \ln(\frac{193}{Z^{1/3}}) E = E / X_0}$ \\
$1 / X_0 \sim 1/m_e^2 \sim Z^2 \sim ρ N_A / A$, $E(x) = E_0 e^{-X / X_0}$ \\
mit Strahlungslänge $X_0$ \\

\textbf{Cherenkov-Strahlung} bei Medium mit Brechungsindex $n$: nur kleiner Energieverlust, Öffnungswinkel: $\cos θ_c = 1/(β n)$ \\

\textbf{Photonenabsorption}: $I(x) = I_0 \exp(-μ x)$, Absorptionskoeffizient $μ$ \\
$μ$ = Photoeffekt $μ_{ph}$ + Comptoneffekt $μ_c$ + Paarbildung $μ_{paar}$ \\
(existiert auch inverser Comptoneffekt) \\

\textbf{Paarbildung}: $E_γ > \SI{2.04}{\mega\electronvolt}$ \\
Wirkungsquerschnitt: $σ_{paar} = \frac{7}{9} \frac{A}{N_A} \frac{1}{X_0}$ \\

Elektromagnetischer Schauer: im Mittel nach $X_0$ Bremsstrahlung / Paarbildung $⇒$ Halbierung der Energie, Verdopplung der Teilchen. Hört auf, wenn Ionisation einsetzt.
Maximale Eindringtiefe: \\
$\displayed{X_{max} = \frac{\ln(E_0 / E_c)}{\ln 2} X_0, E_c \approx \SI{2}{\mega\electronvolt}}$

\textbf{Hadronen in Materie}: starke Wechselwirkung, Wechselwirkunglänge für Hadronen: $λ_{WW} = \frac{1}{n σ_{WW}} = \frac{A}{N_A ρ} \frac{1}{σ_{WW}}$ \\
Es gilt $λ_{WW} > X_0$ \\

\textbf{Detektorsysteme}: \\
- Rekonstruktion der Trajektorien von geladenen Teilchen (Spure) (Spurdetektoren) \\
- Impulsmessung geladerner Teilchen (Spektrometer) \\
- Energiemessung elektromagnetischer und hadronischer Schauer (E-Kalorimeter, H-Kalorimeter) \\
- Teilchenidentifikation \\

Spurdetektoren: \\
- Gasdetektoren: Ionisation von Gas, Elektronen werden an Anode nachgewiesen \\
- Halbleiterdtektoren: Teilchen erzeugen Elektron-Lochpaare ($\sim$ Energieverlust), die nachgewiesen werden \\

Impulsmessung: \\
- Ablenkung innerhalb eines Magnetfelds: $p = q B R$ (transversal zum Magnefeld) ($p[\si{\giga\electronvolt\per\clight}] = 0.3 B [\si{\tesla}] R [\si{\meter}]$) \\

Kalorimeter \\
- elektromagnetische Schauer $\to$ großes $Z$, homogene Kalorimeter (Absorber = Nachweismedium), Sampling-Kalorimeter (Absorber \& Nachweismedium verschieden) \\
- hadronische Schauer $\to$ kleineres $Z$, nur Sampling-Kalorimeter \\

Teilchenidentifikation: \\
- Energieverlust bei bekanntem Impuls \\
- Flugzeitmessung bei bekanntem Impuls \\
- Messung des Cherenkov-Winkels bei bekanntem Impuls \\

Sonderfall: Myonen: einziges ionisierendes Teilchen, das einen dicken Absorber durchquert. \\

- kontinuierliche Symmetrie Transformation $\to$ additive Erhaltungsgrößen \\
- diskrete Symmetrie Transformationen $\to$ multiplikative Erhaltungsgrößen \\

Observable $\hat O$ ist Erhaltungsgröße, wenn $[\hat O, \hat H] = 0$ \\


Kopplung von Spins: $j_1, m_1 + j_2, m_2 \to j, m$, $m = m_1 + m_2$, $\abs{j_1 - j_2} \leq j \leq \abs{j_1 + j_2}$ \\

Translation: Impulsoperator $\hat h$ ist Generator, $\hat p$ ist Erhaltungsgröße \\
Rotation: Drehimpuls, $\hat L$ Erhaltungsgröße \\
globale Phase: Ladung \\
Ladungsgrößen: Elektrische Ladung, Leptonenzahl $L = L_e + L_μ + L_τ$ (auch einzeln, außer Neutrino-Mixing), \\
Neutrinos: Dirac-Teilchen $\leftrightarrow$ Majorana-Teilchen (Teilchen = eigenes Anti-Teilchen) $\to$ Doppelbeta-Zerfall \\
Baryonenzahlerhaltung: Baryonenzahl: $\tilde B = (n_q - n_{\bar q})/3$. \\
starker Isospin: Erhalten in starker Wechselwirkung, $\ket{p} = \ket{1/2, 1/2}, \ket{n} = \ket{1/2, -1/2}$ \\
$\ket{π^+} = \ket{1, 1}, \ket{π^{-}} = \ket{1, -1}, \ket{π^0} = \ket{1, 0}$, $\ket{η} = \ket{0, 0}$ \\
$\frac{Q}{e} = I_3 + \frac{\tilde B}{2} + \frac{S}{2} = I_3 + \frac{Y}{2}$, $S$ Strangeness, $Y = \tilde B + S$ Hyperladung \\
Auch hier: Wellenfunktion muss antisymmetrisch sein! \\
Deuteron: $d = \ket{pn}$ mit $J = 1, l = 0$, $l = 0 ⇒$ Raumwellenfunktion ist symmetrisch \\
$J = 1 ⇒$ Spin ist symmetrisch $⇒$ Isospin muss antisymmetrisch sein. \\
$\ket{d} = \ket{I = 0, I_3 = 0} = (\ket{p}\ket{n} - \ket{n}\ket{p})/\sqrt{2}$ \\
Atomkern: $I_3 = (Z - N)/2$, $Z$: Anzahl Protonen, $N$: Anzahl Neutronen, $\abs{Z - N} / 2 \leq I \leq \abs{Z + N} / 2$	\\
Strangeness erhalten in starker Wechselwirkung, $S = n_{\bar s} - n_{s}$ \\
Quarkflauvorzahlen ($U, D, S, C, B$) anstatt Isospin heutzutage, auch in starker Wechselwirkung erhalten. \\

\textbf{Diskrete Transformationen}: $\hat U_D^2 = 1$ \\
Parität: $\v x \to - \v x, \v p \to - \v p$ \\
Kernzustände + Elementarteilchen sind Eigenzustände des	Paritätsoperators. \\
$J^P$: $J ⇔$ Spin, $P ⇔$ Parität, Parität erhalten in elektromagnetischer und starker Wechselwirkung. \\
Ladungskonjugation: $C \ket{\text{Teilchen}} = η_c \ket{\text{Antiteilchen}}$ \\
C-Parität ist erhalten in starker und e.m. Wechselwirkung, $C\ket{γ} = - \ket{γ}, C\ket{π^0} = +1 \ket{π^0}$ \\
Zeitumkehr $T$ (nicht unitär, keine Erhaltungsgröße, aber anti-unitär), $t \to -t, \v x \to \v x$, elektromagnetische + starker Wechselwirkung erhalten $T$ \\

Spin entgegen Impuls $⇔$ linkshändig (LH) \\
Spin parallel zu Impuls $⇔$ rechtshändig (RH) \\
$W^{\pm}$-Bosonen koppeln nur zu linkshändige Teilchen und rechtshändige Antiteilchen. \\

C, P-Verletzung: Beta-Zerfall hat unter eingefrorenem Spin für $\uparrow$ und $\downarrow$ underschiedliche Winkelverteilungen. \\
CP, T-Verletzung: $K^0 \to π^+ π^-$ (Quarkmischung) \\
CPT Theorem: CPT-Invarianz (Eigenschaft lorentzinvarianter, lokaler und kausaler Feldtheoreme) \\

leichtes, spinloses, geladenes Teilchen $z e$ an schwerem spinlosen Streuzentrum $Z e$: \\
$\displayed{A_{fi} = \frac{z e Z e}{q^2} \frac{(\hbar c)^3}{c^2}}$ (keine Spins) $\to$ Rutherford: $\displayed{\dd{σ}{Ω} \frac{z^2 Z^2 α^2}{16 E_{\text{kin}}^2 \sin^4 θ/2} (\hbar c)^2}$ \\
Auflösungsvermögen von Streuuexperimenten: $λ = \hbar / \abs{\v q}$ \\
Einfluss des Spin des leichten Teilchens: Helizität ist Erhaltungsgröße (im Limit hochrelativistischer Teilchen, $β \to 1$) ($H = \v s · \v p / \hbar$), unterdrückt Rückwärtsstreuung. \\
$\displayed{(\dd{σ}{Ω})_{\text{Mott}} = (\dd{σ}{W})_{\text{Rutherford}}(1 - β^2 \sin^2 \frac{θ}{2})}$ \\
Born'sche Näherung: Ein- und Auslaufende Teilchen werden durch Wellenfunktion des freien Teilchens beschrieben. \\
$\to A_{fi} \sim ∫ V(r) \exp(i \v q · \v r / \hbar)$ \\
$A_{fi}$ ist Fouriertransformation von $V(r)$ im Impulsraum \\
Ausgedehnte Ladungsverteilung: $V(\v r) = e/(4π) ∫ ρ(\v r) / \abs{\v r - \v r'} \d \v r'$ \\
$\to A_{fi} = e^2 \hbar^2 F(\v q) / \v q^2$ \\
mit \textbf{Formfaktor}: \\
$F(\v q) = ∫ f(\v r) e^{i/\hbar \v q \v r} \d \v r, ρ(\v r) = e f(\v r), f(\v r) \to f(r) \Rightarrow F(\v q) \to F(\v q^2)$ \\
$\displayed{(\dd{σ}{Ω}) = (\dd{σ}{W})_{\text{Punktförmig}} \abs{F(\v q^2)}^2}$ \\
Messung von Formfaktor: praktische nicht möglich: Phase wird benötigt, extrem großer $\v q^2$ Bereich benötigt, stattdessen: Modellansätze + Anpassung an die Daten. \\

\textbf{Fermi-Verteilung}: $\displayed{ρ(r) = \frac{ρ_0}{1 + e^{(r - c)/a}}}$ \\
$c$ = Halbdichte Radius $\approx 1.07 A^{1/3} \si{\femto\meter}$, $a \approx \SI{0.545}{\femto\meter}$, $A$ groß: Mittlerer quadratischer Radius: $\sqrt{\braket{r^2}} \approx \SI{0.94}{\femto\meter} A^{1/3}$ \\
Proton + Neutron haben Spin $1/2$ $⇒$ magnetisches Moment: $\v μ_N = g_N μ_N \v s / \hbar \approx \SI{3.15e-14}{\mega\electronvolt\per\tesla}$, $m_s = \pm 1/2$ \\
$μ_p \approx \pm 2.79 μ_N$, $μ_n \approx \mp 1.91 μ_N$ \\
Elektronstreuung an punktförmigen Spin 1/2 Protonen: \\
$\displayed{(\dd{σ}{Ω})_{\text{Dirac}} = (\dd{σ}{Ω})_{\text{Rutherford}}(\cos^2 \frac{θ}{2} + \frac{θ^2}{2 M_p^2 c^2} \sin^2 \frac{θ}{2})}, Q^2 = -q^2$ \\
Rückstoß: Energie des Elektron: $\displayed{E \to E', (\dd{σ}{Ω})_{\text{Rückstoß}} = (\dd{σ}{Ω}) \frac{E'}{E}}$ \\
Elektron-Proton-Streung: 2 Formfaktoren $G_E(Q^2), G_M(Q^2)$: elektrische- und magnetische Verteilung: Rosenbluth-Formel: \\
$\displayed{\dd{σ}{Ω} = (\dd{σ}{Ω})_{\text{Mott}}(G_E^2(Q^2 + 2 τ G_M^2(Q^2) \tan^2 \frac{Q}{2}))}$ \\
$τ = Q^2 / (4 M^2_p c^2)$ \\
Formfaktoren skalieren: $G_E^P(Q^2) = G_M^P(Q^2) / μ_p = G_M^n(Q^2) / μ_n$ \\
% $\displayed{G_E(Q^2) = (1 + \frac{Q^2}{(\SI{0.71}{\giga\electronvolt})^2})^{-2} \Leftrightarrow f(r) = f_0 e^{-ar}, \sqrt{\braket{r^2}}\approx \SI{0.88}{\femto\meter}}$ \\

Höher Energieübertrag: Anregung innerer Freiheitsgrade des Protons,	Nukleon-Resonanzen, noch größerer Energieübetrag ($ν = E - E' > M$) Kontinuum, tief inelastisch. \\
$x = Q^2 / (2 M ν)$, $0 < x < 1$ ($x = 1 ⇔$ elastisch), $x$ Scaling-Variable \\
Strukturfunktionen: $W_1(Q^2, ν), W_2(Q^2, ν)$ \\
$\displayed{\to \frac{\d^2 σ}{\d Ω' \d E'} = (\dd{σ}{Ω})_{\text{Mott}}\frac{1}{ν}(F_2(Q^2, x) + F_1(Q^2, x) \frac{Q^2}{x M^2 c^2} \tan^2 \frac{θ}{2})}$ \\
$ν W_2 = F_2, M c W_1 = F_1$, für $Q^2 \gg M^2 c^2: F_{1,2}(Q^2, x) \approx F_{1,2}(x)$ $\to$ Bjorken-Scaling: Strukturfunktionen sind unabhängig von Skalentransformationen, bei denen $x$ unverändert bleibt. \\

Parton-Modell:
Nukleon besteht aus punktförmigen, quasifreien Partonen $q_j$ mit Spin $1/2$, jedes	Parton trägt einen Anteil $z$ an Protonimppuls.
Elektronen streuen elastische an Partonen, dies ist nur möglich bei	Impulsanteil $z = Q^2 / (2mν)$, gestreutes Parton verlässt Proton und bildet Hadronen.
$f_{q_i}(x):$ Wahrscheinlichkeit Parton $q_i$ mit $x ∈ [x, x + \d x]$ zu finden. \\
$\to F_2(x) = x \sum_{q_i} e^2_{q_i} f_{q_i}(x), \to 2x F_1(x) = x \sum_{q_i} e^2_{q_i} f_{q_i}(x)$ \\
$2 x F_1(x) = F_2(x)$. Gluonenaustausch $⇒$ nicht nur 3 Quarks, sondern Wechselwirkung + Seequarks + Antiquarks \\
Seequarks tragen sehr kleines $x$, Valenzquarks-Verteilung hat maximum bei $x \approx \SI{0.15}{} \dots \SI{0.2}{}$, Impulsanteil der geladenen Quarks $\approx \SI{50}{\percent}$ \\

\textbf{Starke Wechselwirkung}: Austausch von Farbladung $r, g, b, \bar r, \bar g, \bar b$ \\
$\to$ Gluonen tragen Farbladunge: $9$ Gluonen $r \bar g, r \bar b, g \bar b, g \bar r, b \bar r, b \bar g, (r \bar r - b \bar b)/\sqrt{2}, (r \bar r + g \bar g - 2 b \bar b/\sqrt{3}), (r \bar r + g \bar g + b \bar b) / \sqrt{3}$, nur die farbigen (ersten $8$) relevant. \\

Wirkungsquerschnitt $e^{+} e^{-} \to$ Hadronen: $σ(e^{ +} e^{-} \to μ^{ +} μ^{-}) \sum_{q_i} Q^2_{q_i} N_f$ \\
$q_i$: kinematisch mögliche Flavour. $N_f = 3$ ($r,g,b$) \\
3-Jet-Ereignisse: Quark strahlt reelle Gluonen ab, die auch hadronisieren. \\
Quark-Antiquarkpotential: $V_{q \bar q}(r) = - \frac{4}{3}(\hbar c) \frac{α_s(r)}{r} + k r$ (großer Abstand: Feldlinien verlaufen "schlauchartig") $\to$ Confinment. \\
$α_s(Q^2) \to ∞: α_s(Q^2) \to 0$ $\to$ Quarks sind bei großen $Q^2$-Werten quasi-frei. Für $Q^2 \to 0: α_s \to ∞$, pertubative Behandlung nicht mehr ausreichend $\to$ Gittertheorien.
Hadronisierung: Farbschlauch wird gespannt $\to$ ab gewisser Spannung: Bildung neuer $q \bar q$-Paar in Mitte des Schlauchs. \\

\textbf{Schwache Wechselwirkung}: \\
Propagator: $(q^2 - M_W^2 c^2)^{-1} \to$ für $q^2 \ll M_W^2 c^2$ unabhängig von $q^2$ \\
$W^{\pm}$-Bosonen koppeln nur an (RH) LH (Anti)teilchen, $Z$-Bosonen koppeln an	LH und RH aber mit unterschiedlicher Stärke.
$W$-Bosonen transportieren Ladung und schwachen Isospin.
Für Leptonen "wechseln" $W$-Bosonen zwischen Neutrino und Teilchen, Kopplung $\sim g_W$.
Quarks werden in Generation gemischt, aber immer Wechsel der Ladung. Beschrieben durch Matrix $V_{CKM}$, Kopplungsstärke $g_W' = (V_{CKM})_{ij} g_W$, $\abs{A_{fi}}^2 \sim g_W^{\prime 2} \abs{V_{q q'}}^2$ \\
3-Teilchen-Phasenraumdichte: \\
$\displayed{ρ_3(E_f) = \frac{V^2}{(2π\hbar)^6} \dd{}{E_0} ∫ p e^2 \d p_e \d Ω_e p_ν^2 \d p_ν \d Ω_ν}$. \\
$E_ν = p_ν c ⇒$ $\displayed{\d ρ_3 = \frac{\d Ω_e Ω_ν}{(2π\hbar)^6 c} p_e^2 p_ν^2 \d p_e}$ \\
$p_ν^2 = (E_0 - E_e)^2 / c^2$ (Rückstoß $\approx 0$) \\
Coulomb-Potential des $β$-Teilchen und Kerns: Coulomb-Korrektur $F(\pm, Z, E_e)$.
$f(Z, E_0) τ = \frac{2π^3}{\abs{A_{fi}}^2} \frac{\hbar^8}{m_e^5 c^4}$
kleine $Z$, große Energien $\to τ \sim E_0^{-5}$

Auswahlregel für Kern-$β$-Zerfall: $l = 0$ von Neutrino + Elektron $⇒ ΔP = 0$ (Parität), $l = 1$ ("verboten") $ΔP = 1$ \\
- Fermi-Übergang: Spin von des Quarks bleibt erhalten. Spinänderung des Kerns: $ΔJ = 0$ \\
- Gamov-Feller-Übergang: Spin des Quarks und des Nukleons werden geflippt $\to$ Spinänderung des Kerns $ΔJ = 0, \pm 1$ \\

"Super erlaubte" $β$-Zerfälle: Ursprungskern und entstehender Kern gehören zum gleichen starken Isospin Multiplett.

$B(A, Z) = a_v A - a_0 A^{2/3} - a_c Z^2 A^{-1/3} - a_{sym} (N - Z)^2 / A + δ A^{-1/2}$ \\
$a_v A$: Volumenbeitrag, $a_0 A^{2/3}: $Oberflächeneffekt, $-a_c Z^2 A^{-1/3}$: Coulomb-Abstoßung, Asymmetrie-Term: $-a_{sym} (N - Z)^2 / A$, Paarungs-Term $δ A^{-1/2}$, $δ \pm \SI{11.2}{\mega\electronvolt}$ $Z, N$ gerade / ungerade, 0 für $Z$ oder $N$ gerade
\end{multicols*}
\end{document}
