% Created 2016-12-16 Fr 11:41
\documentclass[a4paper]{scrartcl}
\usepackage[utf8]{inputenc}
\usepackage[T1]{fontenc}
\usepackage{fixltx2e}
\usepackage{graphicx}
\usepackage{longtable}
\usepackage{float}
\usepackage{wrapfig}
\usepackage{rotating}
\usepackage[normalem]{ulem}
\usepackage{amsmath}
\usepackage{textcomp}
\usepackage{marvosym}
\usepackage{wasysym}
\usepackage{amssymb}
\usepackage{hyperref}
\tolerance=1000
\usepackage{siunitx}%
\usepackage{fontspec}%
\sisetup{load-configurations = abbrevations}%
\newcommand{\estimates}{\overset{\scriptscriptstyle\wedge}{=}}%
\usepackage{mathtools}%
\DeclarePairedDelimiter\abs{\lvert}{\rvert}%
\DeclarePairedDelimiter\norm{\lVert}{\rVert}%
\DeclareMathOperator{\Exists}{\exists}%
\DeclareMathOperator{\Forall}{\forall}%
\DeclareMathOperator{\sgn}{sgn}
\def\colvec#1{\left(\vcenter{\halign{\hfil$##$\hfil\cr \colvecA#1;;}}\right)}
\def\colvecA#1;{\if;#1;\else #1\cr \expandafter \colvecA \fi}
\usepackage{xparse}% http://ctan.org/pkg/xparse
\NewDocumentCommand{\overarrow}{O{=} O{\uparrow} m}{%
\overset{\makebox[0pt]{\begin{tabular}{@{}c@{}}#3\\[0pt]\ensuremath{#2}\end{tabular}}}{#1}
}
\NewDocumentCommand{\underarrow}{O{=} O{\downarrow} m}{%
\underset{\makebox[0pt]{\begin{tabular}{@{}c@{}}\ensuremath{#2}\\[0pt]#3\end{tabular}}}{#1}
}
\newcommand{\ubar}[1]{\text{\b{$#1$}}}
\makeatletter
\newcommand{\pushright}[1]{\ifmeasuring@#1\else\omit\hfill$\displaystyle#1$\fi\ignorespaces}
\newcommand{\pushleft}[1]{\ifmeasuring@#1\else\omit$\displaystyle#1$\hfill\fi\ignorespaces}
\makeatother
\newcommand{\I}{\ensuremath{\imath}}%
\newcommand{\eps}{\ensuremath{\varepsilon}}%
\usepackage{amsthm}
\theoremstyle{definition}
\newtheorem{defn}{Definition}
\theoremstyle{plain}
\newtheorem{thm}{Satz}
\theoremstyle{plain}
\newtheorem{lemma}{Lemma}
\theoremstyle{remark}
\newtheorem{remark}{Bemerkung}
\theoremstyle{remark}
\newtheorem{ex}{Beispiel}
\theoremstyle{remark}
\newtheorem{conc}{Folgerung}
\theoremstyle{remark}
\newtheorem{korollar}{Korollar}
\theoremstyle{remark}
\newtheorem{note}{Anmerkung}
\usepackage{etoolbox}
\patchcmd{\thmhead}{(#3)}{#3}{}{}
\renewcommand*{\proofname}{Beweis}
\usepackage{wasysym}
\newcommand{\euler}{e}
\makeatletter
\def\resetMathstrut@{%
\setbox\z@\hbox{%
\mathchardef\@tempa\mathcode`\[\relax
\def\@tempb##1"##2##3{\the\textfont"##3\char"}%
\expandafter\@tempb\meaning\@tempa \relax
}%
\ht\Mathstrutbox@\ht\z@ \dp\Mathstrutbox@\dp\z@}
\makeatother
\catcode`(=\active
\catcode`)=\active
\newcommand{(}{\ifmmode\left\string(\else\string(\fi}
\newcommand{)}{\ifmmode\right\string)\else\string)\fi}
\usepackage{xcolor}
\newcommand\RED{\color{red}}
\author{Robin Heinemann}
\date{\today}
\title{Analysis I (Marciniak-Czochra)}
\hypersetup{
  pdfkeywords={},
  pdfsubject={},
  pdfcreator={Emacs 25.1.1 (Org mode 8.2.10)}}
\begin{document}

\maketitle
\tableofcontents


\section{Einleitung}
\label{sec-1}
Webseite www.biostruct.uni-heidelberg.de/Analysis1.php
Klausurzulassung: 50\%
Klausur 18.2.2017 9-12Uhr
\section{Mengen und Zahlen}
\label{sec-2}
\subsection{Logische Regeln und Zeichen}
\label{sec-2-1}
\subsubsection{Quantoren}
\label{sec-2-1-1}
\begin{center}
\begin{tabular}{ll}
$\Forall x$ & für alle $x$\\
$\exists x$ & es gibt (mindestens) ein $x$\\
$\exists! x$ & es gibt genau ein $x$\\
\end{tabular}
\end{center}
\subsubsection{Hinreichend und Notwendig}
\label{sec-2-1-2}
\begin{itemize}
\item $A\implies B$: wenn $A$ gilt, gilt auch $B$, $A$ ist \textbf{hinreichend} für $B$, daraus folgt: $B$ ist \textbf{notwendig} für $A$, Ungültigkeit von $B$ impliziert die Ungültigkeit von $A$ ($\neg B \implies \neg A$)
\item $A \iff B$: $A$ gilt, genau dann, wenn $B$ gilt
\end{itemize}
\subsubsection{Beweistypen}
\label{sec-2-1-3}
\paragraph{Direkter Schluss}
\label{sec-2-1-3-1}
$A\implies B$
\subparagraph{Beispiel}
\label{sec-2-1-3-1-1}
$m$ gerade Zahl $\implies$ $m^2$ gerade Zahl
\begin{enumerate}
\item Beweis
\label{sec-2-1-3-1-1-1}
$m$ gerade $\implies \exists n\in\mathbb{N}~\text{sodass}~m = 2n \implies m^2 = 4n^2 = 2k,~\text{wobei}~k=2n^2\in\mathbb{N} \square$
\end{enumerate}
\paragraph{Beweis der Transponierten (der Kontraposition)}
\label{sec-2-1-3-2}
Zum Beweis $A\implies B$ zeigt man $\neg B\implies \neg A~(A\implies B)\iff (\neg B) \implies (\neg A)$
\subparagraph{Beispiel}
\label{sec-2-1-3-2-1}
Sei $m\in\mathbb{N}$, dann gilt $m^2~\text{gerade}~\implies m~\text{gerade}$
\begin{enumerate}
\item Beweis
\label{sec-2-1-3-2-1-1}
Wir zeigen: $m$ ist ungerade $\implies m^2$ ungerade
\[\exists n\in\mathbb{N}:~m=2n+1\implies m^2 = (2n+1)^2 = 2k+1, k=2n^2 + 2n\in\mathbb{N}\implies m^2~\text{ungerade} \square\]
\end{enumerate}
\paragraph{Indirekter Schluss ( Beweis durch Widerspruch)}
\label{sec-2-1-3-3}
Man nimmt an, dass $A\implies B$ nicht gilt, das heißt $A \wedge \neg B$ und zeigt, dass dann für eine Aussage $C$ gelten muss $C\implies \neg C$, also ein Widerspruch
\subparagraph{Beispiel}
\label{sec-2-1-3-3-1}
$\not\exists q\in\mathbb{Q}: a^2 = 2$
\begin{enumerate}
\item Beweis
\label{sec-2-1-3-3-1-1}
Wir nehmen an, dass $\exists a\in\mathbb{Q}: a^2=2$ Dann folgt:
$\exists b,c\in\mathbb{Z}$ teilerfremd (ohne Einschränkung, denn sonst kürzen soweit wie möglich) mit $a=\frac{b}{c}$
Falls \[a^2=2\implies (\frac{b}{c})^2=2=\frac{b^2}{c^2}=2 \implies b^2 = 2c^2 \implies b^2~\text{gerade}~\implies b~\text{ist gerade (schon gezeigt)}\] \[\implies\exists d\in\mathbb{N}~\text{sodass}~b=2d\implies b^2=4d^2\]
Außerdem $b^2=2c^2\implies 2c^2=4d^2\implies c^2=2d^2\implies c$ ist auch gerade. Also müssen $b$ und $c$ beide gerade sein, also nicht teilerfremd, damit haben wir einen Widerspruch hergeleitet $\square$
\end{enumerate}
\subsubsection{Summenzeichen und Produktzeichen}
\label{sec-2-1-4}
\paragraph{Summenzeichen}
\label{sec-2-1-4-1}
Wir definieren für $m > 0$ \[\sum_{k=m}^m a_k := a_m + \ldots + a_n\] falls $n\geq m$
\[\sum_{k=m}^n a_k := 0\] falls $n < m$ (sogenannten leere Summe)
\paragraph{Produktzeichen}
\label{sec-2-1-4-2}
\[\prod_{k=m}^n a_k := \begin{cases} a_m \cdot ... \cdot a_n & \text{falls}~n\geq m\\ 1 & \text{falls}~n<m~\text{(sog. leeres Produkt)}\end{cases}\]
\subsection{Mengen}
\label{sec-2-2}
\subsubsection{Definition}
\label{sec-2-2-1}
(Georg Cantor 1885) Unter einer \uline{Menge} verstehen wir jede Zusammenfassung $M$ von bestimmten wohl unterschiedenen Objekten (welche die Elemente von $M$ genannt werden), zu einem Ganzen
$M$ dadurch ist charakterisiert, dass von jedem vorliegendem Objekt $x$ feststeht, ab gilt
\begin{itemize}
\item $x\in M$ (x Element von M)
\item x \textlnot{}$\in$ M (x kein Element von M)
\end{itemize}
\[M = \{x_1, x_2, \ldots, x_n\}\]
\[M=\{x\mid A(x)\} \rightarrow~\text{eine Menge}~M\text{für die}~x\in\ M \iff A(x)\]
\subsubsection{Mengenrelationen}
\label{sec-2-2-2}
\begin{itemize}
\item Mengeninklusion $A\subseteq M$ ($A$ ist eine Teilmenge von $M$)
\[\Forall x: (x\in A \implies x\in M)\] ,zum Beispiel $\mathbb{N} \subseteq \mathbb{Z}$
\item \[A = B \iff \Forall x: (x\in A \iff x\in B)\]
\item \[A \subset M~\text{(strikte Teilmenge)}~\iff A\subset M \wedge A \neq M\]
\item \[\emptyset:~\text{leere Menge}~\not\exists x: x\in\emptyset\]. Wir setzen fest, dass $\emptyset$ eine Teilmenge jeder Menge ist. Zum Beispiel \[\{x\in\mathbb{R}: x^2 + 1 = 0\}\]
\item Durchschnitt \[A\cap B := \{x\mid x\in A \wedge x\in B\}\]
\item Vereinigung \[A\cup B := \{x \mid x\in A \vee x\in B\}\]
\item Differenz (auch Komplement von $B$ in $A$) \[A\setminus B := \{x\mid x\in A \wedge x\not\in B\} := C_a B~\text{(auch $B^c$)}\]
\end{itemize}
\subsubsection{Potenzmenge}
\label{sec-2-2-3}
Potenzmenge $A$
\[\mathcal{P}(A) := \{B\mid B\subseteq A\}\]
Alle Teilmengen von $A$
\paragraph{Beispiel}
\label{sec-2-2-3-1}
\[\mathcal{P}(\{1,2\}) = \{\{1\}, \{2\}, \{1,2\}, \emptyset\}\]
\subsubsection{Familien von Mengen}
\label{sec-2-2-4}
Sei $I$ eine Indexmenge, $I \subseteq \mathbb{N}, (A_i)_{i\in I}$ eine Familie von Mengen $A$
\paragraph{Durchschnitt von $A$}
\label{sec-2-2-4-1}
\[\cap_{i\in I} = \{x\mid \Forall_{i\in I} x\in A_i\}\]
\paragraph{Vereinigung}
\label{sec-2-2-4-2}
\[\cup_{i\in I} = \{x\mid\exists i\in I: x\in A_i\}\]
\subsubsection{Rechenregeln}
\label{sec-2-2-5}
$A,B,C,D$ seien Mengen
\begin{itemize}
\item $\emptyset \subseteq A$
\item $A\subseteq A$ \hfill Reflexivität
\item $A\subseteq B, B\subseteq C \implies A\subseteq C$ \hfill Transitivität
\item $A\cap B = B\cap A$ \\ $A\cup B = B\cup A$ \hfill Kommutativität
\item $(A\cap B)\cap C = A\cap (B\cap C)$ \\ $(A\cup B) \cup C = A\cup (B\cup C)$ \hfill Assoziativität
\item $A\cap (B\cup C) =(A\cap B) \cup (A\cap C)$ \\ $A\cup (B\cap C) =(A\cup B) \cap (A\cup C)$
\item Eigenschaften der Komplementbildung: \\
      Seien $A,B \subseteq D (C_D A: = D\setminus A)$, dann gilt \[C_D (C_D A) = A\] \[C_D(A\cap B) = C_D A \cup C_D B\] \[C_D(A\cup B) = C_D A \cap C_D B\]
\begin{itemize}
\item Beweis:
\[x\in C_D(A\cap B) \iff x\in D \wedge (x\not\in (A\cap B)) \iff x\in D \wedge (x\not\in A \vee x\not\in B)\] \[\iff (x\in D\wedge x\not\in A) \vee (x\in D \wedge x\not\in B)\] \[\iff (x\in D\setminus A) \vee (x\in D\setminus B) \iff x\in D\setminus(A\cup B)~\square\]
\item Bemerkung: Komplement kann man auch mit $A^c$ bezeichnen
\end{itemize}
\end{itemize}
\subsubsection{geordneter Tupel}
\label{sec-2-2-6}
Sei $x_1, x_2, \ldots, x_n$ (nicht notwendig verschiedene) Objekte. Ein geordneter n-Tupel \[(x_1,x_2,\ldots,x_n) = (y_1,\ldots,y_n) \iff x_1 = y_1, \ldots, x_n = y_n\]
Beachte:
\[\{x_1, \ldots, x_n\} = \{y_i,\ldots,y_n\}\not\implies x_1 = y_1, \ldots, x_n = y_n\]
\subsubsection{Kartesisches Produkt}
\label{sec-2-2-7}
Seien \[A_1\times A_2\times \ldots \times A_n = \{(x_1,x_2,\ldots,x_n)\mid x_j \in A_j j\in\mathbb{N}, j \leq n\}\]
\paragraph{Beispiel}
\label{sec-2-2-7-1}
\begin{itemize}
\item \[\mathbb{Z}^2 = \mathbb{Z}\times \mathbb{Z}\]
\item $R^n$ n-dimensionaler Raum von reellen Zahlen
\end{itemize}
\subsubsection{Äquivalenzrelation}
\label{sec-2-2-8}
Eine Äquivalenzrelation auf eine Menge $A$ ist eine Beziehung zwischen ihren Elementen (Bezeichnung: $a \sim b$), sodass
\begin{itemize}
\item Für jede zwei $a,b\in A$ gilt entweder $a\sim b \vee a\not\sim b$
\item $a\sim a$ \hfill Reflexivität
\item $a\sim b \implies b\sim a$ \hfill Symmetrie
\item $a \sim b, b \sim c \implies a \sim c$ \hfill Transitivität
\end{itemize}
Mit Hilfe einer Äquivalenzrelation lassen sich die Elemente einer Menge in so genannte Äquivalenzklassen einordnen: $[a]:\{b\in A\mid b\sim a\}$
\subsection{Relationen und Abbildungen}
\label{sec-2-3}
\subsubsection{Relationen}
\label{sec-2-3-1}
Unter einer \textbf{Relation} verstehen wir eine Teilmenge $R\subseteq X\times Y$ wobei $X, Y$ Mengen sind. Für $x\in X$ definieren wir, das \textbf{Bild} von $x$ unter $R$
\[R(X) := \{y\in Y \mid (x,y) \in R\}\]
und *Definitionsbereiche von $R$ (bezüglich $X$)
\[D(R):= \{x\in X\mid R(x)\neq\emptyset\}\]
\subsubsection{Graph der Abbildung}
\label{sec-2-3-2}
$R\subseteq X\times Y$ heißt Graph der Abbildung (Funktion) \[f:X\rightarrow Y \iff D(R) = X, \Forall x\in X: R(x) = \{f(x)\}\]
also enthält $R(x)$ genau ein Element. \\
    $X$ heißt Definitionsbereich von $f$ \\
    $Y$ heißt Werte- oder Bildbereich von $f$ (Bild) \\
    $x\in X$ heißt Argument \\
    $f(x)\in Y$ heißt Wert von $f$ an der Stelle x
\paragraph{Beispiel}
\label{sec-2-3-2-1}
$f: \mathbb{R}\rightarrow\mathbb{R}, x\rightarrow x^2$ dann ist der Graph von $f = \{(x,y)\in\mathbb{R}^2, y=x^2\}$
\subparagraph{Bemerkung}
\label{sec-2-3-2-1-1}
\[M^{*}(x) = \{(x,y)\in\mathbb{R}^2;x=y^2\} = \{(x,y)\in\mathbb{R}^2: x \geq 0, y=\sqrt{x} \vee y = -\sqrt{x}\}\]
Ist kein Graph einer Funktion $\mathbb{R}\rightarrow\mathbb{R}$, denn $M^{ *}(x) = \{\sqrt{x},-\sqrt{x}, x\geq 0\}$
$f$ heißt
\begin{itemize}
\item surjektiv, wenn gilt $f(X) = Y$
\item injektiv, $\Forall x_1,x_2\in X: f(x_1) = f(x_2) \implies x_1 = x_2$
\item bijektiv, wenn $f$ surjektiv und injektiv ist
\end{itemize}
\subsubsection{Umkehrabbildung}
\label{sec-2-3-3}
Sei die Abbildung $f: X\rightarrow Y$ bijektiv. Dann definieren wir die Umkehrabbildung $f^{-1}:Y\rightarrow X$ durch $y\rightarrow x\in X$, eindeutig bestimmt durch $y = f(x)$
\paragraph{Bemerkung}
\label{sec-2-3-3-1}
\[(x,y) \in~\text{Graph }f\iff (y,x)\in~\text{Graph }f^{-1}\]
\subsubsection{Komposition}
\label{sec-2-3-4}
Seien $f:X\rightarrow Y, g:Y\rightarrow Z$ Abbildungen. Die Komposition von $g$ und $f$ \[g\circ f: X\rightarrow Z~\text{ist durch}~x\rightarrow g(f(x))~\text{definiert}\]
\subsubsection{Identitäts Abbildung}
\label{sec-2-3-5}
Für jede Menge $X$ definieren wir die identische Abbildung \[I_d(A) = I_A: A\rightarrow A,~\text{durch}~x\rightarrow x\]
\paragraph{Beispiel}
\label{sec-2-3-5-1}
\begin{itemize}
\item \[\{(x,y)\in\mathbb{R}^2\mid x^2 + y^2 = 1\} = S^1\] \[S^{n-1} := \{(x_1 \ldots x_n) \in \mathbb{R}^n; \sum_{i = 1}^n x_i^2 = 1\}\] $(n - 1)$ dimensionale sphere in $\mathbb{R}^n$
\item Seien $X,Y$ Mengen, $M\subseteq X\times Y, f:M\rightarrow X$ \\ $f$ heißt Projektion, $f$ surjektiv \[f(M) = \{x\mid \exists y \in Y : (x,y) \in M\} = X\]
\end{itemize}
\subsubsection{Homomorphe Abbildungen}
\label{sec-2-3-6}
Existieren auf Mengen $X$ und $Y$ mit gewissen Operationen $\oplus_x$ bzw. $\oplus_y$ (zum Beispiel Addition, Ordnungsrelation), so heißt die Abbildung $f:X\to Y$ homomorph (strukturerhaltend), wenn gilt $\Forall x_1,x_2 \in X f(x_1\oplus_x x_2) = f(x_1)\oplus_y f(x_2)$
Eine bijektive Homomorphie heißt Isomorphisumus, beziehungsweise $X\approx Y$ (äquivalent, isomorph)
\subsection{Natürliche Zahlen}
\label{sec-2-4}
$\mathbb{N} = \{1,2,3,\ldots\},~\mathbb{N}_0 := \mathbb{N}\cup \{0\}$
\subsubsection{Peanosche Axiomensystem der natürlichen Zahlen}
\label{sec-2-4-1}
\begin{enumerate}
\item Die Zahl $1$ ist eine natürliche Zahl $1\in\mathbb{N}$
\item Zu jeder natürlichen Zahl $n$, gibt es genau einen "Nachfolger" $n' (=: n+1)$
\item Die Zahl 1 ist kein Nachfolger einer natürlichen Zahl
\item $n' = m' \implies n = m$
\item Enthält eine Teilmenge  $M \subseteq \mathbb{N}$ die Zahl $1$ und von jedem $n\in m$ auch den Nachfolger $n'$ ist $M = \mathbb{N}$
\end{enumerate}
Bemerkung: \\
    Mit Hilfe der Axiome lassen sich auf $\mathbb{N}$ Addition ($+$), Multiplikation ($\cdot$) und Ordnung ($\leq$) einführen.
Wir definieren: \\
    $1' = 2, 2' = 3, \ldots$
$n + 1 := m'$
$n + m' := (n+m)';~n\cdot m' := n m + n$
Man kann zeigen, dass jede Menge, welche die Peano Axiome erfüllt isomorph bezüglich Multiplikation und Addition zu $\mathbb{N}$ ist
Wir definieren $n < m \iff \exists x\in \mathbb{N}: x + m = m$
\subsubsection{Vollständige Induktion}
\label{sec-2-4-2}
\paragraph{Induktionsprinzip}
\label{sec-2-4-2-1}
Es seien die folgende Schritte vollzogen:
\begin{enumerate}
\item Induktionsverankerung (Induktionsanfang): Die Aussage $A(1)$ gilt
\item Induktionsschluss: Ist für ein $n\in\mathbb{N}~A(n)$ gültig, so folgt auch die Gültigkeit von $A(n+1)$
\end{enumerate}
Dann sind alle Aussagen $A(n),n\in\mathbb{N}$ gültig.
\paragraph{Beweis:}
\label{sec-2-4-2-2}
Wir definieren die Teilmenge $M\subseteq\mathbb{N},~M:=\{n\in \mathbb{N}\mid A(N)~\text{ist gültig}\}$
Die Induktionsverankerung besagt, dass $1\in M$ und die Induktionsannahme $n\in M\implies n + 1 \in M$. Folglich ist nach dem 5. Axiom von Peano $M = \mathbb{N}\hfill\square$
\paragraph{Beispiel 1}
\label{sec-2-4-2-3}
Zu Beweisen: \[\Forall n\in\mathbb{N} \sum_{i = 1}^n i^2 = \frac{n(n+1)(2n+1)}{6}\]
\subparagraph{Beweis}
\label{sec-2-4-2-3-1}
\begin{enumerate}
\item Induktionsverankerung: $1^2 = \frac{1}{6}\cdot 1\cdot 2\cdot 3$
\item Annahme: $A(n)$ gültig für $n\in\mathbb{N}: \sum_{i = 1}^n i^2 = \frac{n(n+1)(2n+1)}{6}$ \\
         Zu zeigen $A(n + 1): 1^2 + \ldots + (n+1)^2 = \frac{1}{6} (n+1)(n+2)(2n+3)$
         \[1^2 + \ldots + n^2 + (n+1)^2 = \frac{1}{2} n(n+1)(2n+1) + (n+1)^2 = (n+1)(\frac{1}{3}n^2 + \frac{1}{6}n + n + 1)\]
         \[= \frac{1}{6}(n+1)(2n^2+7n+6) = \frac{1}{6}(n+1)(2n+3)(n+2)\hfill\square\]
\end{enumerate}
\paragraph{Beispiel 2}
\label{sec-2-4-2-4}
Definition von Potenzen
\[x^0 := 1\]
\[\Forall n\in\mathbb{N} x^n := x^{n - 1}x\]
(iterative (rekursive) Definition) \\
     Auf $\mathbb{N}$ sind diese elementaren Operationen erklärt:
\begin{itemize}
\item Addition $a+b$
\item Multiplikation $a\cdot b$
\item (unter gewissen Voraussetzungen):
\begin{itemize}
\item Subtraktion $a-b$
\item Division $\frac{a}{b}$
\end{itemize}
\end{itemize}
$\mathbb{N}$ ist bezüglich "$-$" oder "$/$" nicht vollständig, das heißt $n+x = m$ ist nicht lösbar in $\mathbb{N}$
Erweiterungen:
\begin{itemize}
\item Ganze Zahlen $\mathbb{Z}:=\{0; \pm, n\in\mathbb{N}\}$ \\
       Negative Zahl $(-n)$ ist definiert durch $n+(-n) = 0$
\item Rationale Zahlen $\mathbb{Q}~(b x = y)$
\end{itemize}
Man sagt, dass $(\mathbb{Q},+,\cdot)$ einen Körper bildet.
\subsubsection{Definition Körper}
\label{sec-2-4-3}
$\mathbb{K}$ sei eine Menge auf der Addition und Multiplikation sei. $\mathbb{K}$ heißt ein Körper, wenn die folgende Axiome erfüllt sind:
\begin{itemize}
\item Addition: $(\mathbb{K}, +)$ ist eine kommutative Gruppe, das heißt $\Forall a,b,c\in \mathbb{K}$:
\begin{enumerate}
\item \((a+b)+c = a+(b+c)\) \hfill Assoziativität
\item \(a+b = b+a\) \hfill Kommutativität
\item \(\exists! 0\in\mathbb{K}:a+0 = a\)\hfill Existenz des Nullelement
\item \(\exists x\in\mathbb{K}: a+x = 0\)\hfill Existenz des Negativen
\end{enumerate}
\item Multiplikation: $(\mathbb{K}\setminus\{0\},\cdot)$ ist eine kommutative Gruppe, das heißt $\Forall a,b,c\in\mathbb{K}$
\begin{enumerate}
\item \((a\cdot b)\cdot c = a\cdot(b\cdot c)\)\hfill Assoziativität
\item \(a\cdot b = b\cdot a\)\hfill Kommutativität
\item \(\exists!1\in\mathbb{K}:a\cdot 1 = a\)\hfill Existenz des Einselement
\item Für \(a\neq 0, \exists! y\in\mathbb{K}:a\cdot y = 1\)\hfill Inverse
\end{enumerate}
\item Verträglichkeit
\begin{enumerate}
\item \(a\cdot (b + c) = (a\cdot b)+ (a\cdot c)\)\hfill Distributivität
\end{enumerate}
\end{itemize}
\paragraph{Satz}
\label{sec-2-4-3-1}
$(\mathbb{Q},+,\cdot)$ ist ein Körper. Definieren auf $\mathbb{Q}$ eine Ordnung "$\leq$" durch \[x\leq y \iff\exists m\in\mathbb{N}_0, n\in\mathbb{N}:y - x = \frac{m}{n}\]
dann ist auch diese Ordnung mit der Addition und Multiplikation in $\mathbb{Q}$ in folgendem Sinne verträglich (Axiom M0):
\begin{itemize}
\item \(a\leq b\implies a+c \leq b + c\)
\item \(0\leq a\wedge 0\leq b \implies 0\leq a\cdot b\)
\end{itemize}
\paragraph{Bemerkung}
\label{sec-2-4-3-2}
\[\{a\in\mathbb{Q}: a = \frac{r}{s},r\in\mathbb{N}_0,s\in\mathbb{N}\} =: \mathbb{Q}_+ (\mathbb{Q}_{\geq 0})\]
\subsection{Abzählbarkeit}
\label{sec-2-5}
\subsubsection{Abzählbarkeit von Mengen}
\label{sec-2-5-1}
Sei $A$ eine Menge\\
\begin{itemize}
\item $A$ heißt endlich mit $\abs{A} = n$ Elementen ist äquivalent zu
\[\abs{A} = \begin{cases} A = \emptyset & n = 0 \\ \exists f:A\to \{1,\ldots,n\} &f~\text{bijektiv},n < \infty\end{cases}\]
\item $A$ heißt abzählbar unendlich genau dann wenn \[\exists f: A\to \mathbb{N}~\text{bijektiv}\]
\item $A$ heißt über abzählbar genau dann wenn: $A$ ist weder endlich oder abzählbar unendlich
\end{itemize}
\paragraph{Beispiel}
\label{sec-2-5-1-1}
$\mathbb{Z}$ ist abzählbar unendlich
\subparagraph{Beweis}
\label{sec-2-5-1-1-1}
Die Abbildung $f:\mathbb{Z}\to\mathbb{N}$
\[z\mapsto \begin{cases} 2z & z\geq 0\\ -2z - 1 & x < 0\end{cases}\]
\begin{itemize}
\item Surjektivität: zu zeigen $f(\mathbb{Z}) = \mathbb{N}$ \\
        Offenbar $f(\mathbb{Z}) \subseteq \mathbb{N}$. Wir zeigen $\mathbb{N} \subseteq f(\mathbb{Z})$. Sei $n\in\mathbb{N}$, finde $z\in\mathbb{Z}$ mit $f(z) = n$.
Man unterscheide:
\begin{itemize}
\item n gerade $\rightarrow$ Wähle $z=\frac{n}{2}$
\item n ungerade $\rightarrow$ $z=-\frac{n + 1}{2}$
\end{itemize}
\item Injektivität: Sei $z_1,z_2 \in\mathbb{Z}$ und $f(z_1) = f(z_2)$ \\
        ohne Beschränkung der Allgemeinheit $z_1 \leq z_2$. Entweder $z_1,z_2 \geq 0$ oder $z_1,z_2 < 0$, denn sonst währe $f(z_1)$ ungerade und $f(z_1)$ gerade \textbf{Widerspruch}.
Falls
\begin{itemize}
\item $z_1,z-2 \geq 0 \implies 2z_1 = f(z_1) = f(z_2) = 2z_2 \implies z_1 = z_2$
\item $z_1,z-2 < 0 \implies -2z_1 -1 = f(z_1) = f(z_2) = -2z_2-1 \implies z_1 = z_2 \hfill\square$
\end{itemize}
\end{itemize}
\paragraph{Beispiel}
\label{sec-2-5-1-2}
\begin{itemize}
\item $\mathbb{N}^2 = \mathbb{N}\times\mathbb{N}$ abzählbar unendlich
\item $\mathbb{Q}$ abzählbar unendlich
\item $\mathbb{R}$ über abzählbar
\end{itemize}
\paragraph{Abzählbarkeit von $\mathbb{N}\times\mathbb{N}$}
\label{sec-2-5-1-3}
\[(1,1) \to (1,2) \to (2,1) \to (2,2) \to (1,3) \to (2,3) \to (3,2) \to (3,1)\]
\paragraph{Korollar 1.30}
\label{sec-2-5-1-4}
$M_1,M_2,\ldots,M_n$ abzählbar $\implies M_1 \times \ldots \times M_n$ abzählbar.
\subparagraph{Beweis}
\label{sec-2-5-1-4-1}
Durch vollständige Induktion $M_1\times(M_2\times\ldots \times M_n)\approx \mathbb{N}\times\mathbb{N}\approx\mathbb{N}$
\paragraph{Satz}
\label{sec-2-5-1-5}
Die Menge aller Folgen $f:\mathbb{N}\to\{0,1\}$ ist über abzählbar. (Zum Beispiel: $1,0,0,0,\ldots, \underarrow[1]{\text{k-te Stelle}},\ldots,0,\ldots$)
\subparagraph{Beweis}
\label{sec-2-5-1-5-1}
$M$ ist unendlich, denn die Folgen $f_k:0,,\ldots,0,1,0,\ldots$ sind paarweise verschieden. Angenommen $M$ wäre abzählbar. Sei $f_1,f_2, \ldots$ eine Abzählung mit $f_k = ({z_{kn}}_{n\in \mathbb{N}})$.
\[\begin{matrix}1 & 0 & 0 & \ldots \\ 0 & 1 & \ldots \\ 0 & 0 & 0 & \ldots \\ 1 & 1 & 1 & 1 & \ldots \end{matrix}\]
$f:0 0 1 0$ Man setze $f=(z_n)_{n\in\mathbb{N}}$ mit \[z_n := \begin{cases} 1 & z_{nn} = 0 \\ 0 & z_{nn} = 1\end{cases}\]
Dann $f\in M$, aber $f\neq f_k \Forall k\in\mathbb{N}$. Also ist $M$ nicht abzählbar. ("Cantorsches Diagonalverfahren").
\subsection{Ordnung}
\label{sec-2-6}
\subsubsection{Definition}
\label{sec-2-6-1}
Sei $A$ eine Menge. Relation $R\subseteq A\times A$ heißt Teilordnung (Halbordnung) auf $A$, wenn $\Forall y,x,z\in A$ gilt:
\begin{enumerate}
\item $x\leq x$ \hfill (Reflexivität)
\item $x\leq y \wedge y\leq x \implies x = y$ \hfill (Symmetrie)
\item $x\leq y \wedge y\leq z \implies x\leq z$ \hfill (Transitivität)
\end{enumerate}
Wenn außerdem noch $\Forall x,y\in A$ gilt:
\begin{enumerate}
\setcounter{enumi}{3}
\item $x\leq y \vee y\leq x$ \hfill (Vergleichbarkeit je zweier Elemente)
\end{enumerate}
so heißt $R$ (totale) Ordnung auf $A$. $(A,\leq)$ heißt teilweise beziehungsweise (total) geordnete Menge.
\paragraph{Beispiel}
\label{sec-2-6-1-1}
\begin{enumerate}
\item $(\mathbb{Q},\leq)$ mit der üblichen Ordnung ist eine total geordnete Menge
\item Wir definieren auf der Potenzmenge $\mathcal{P}(A)$ einer Menge $A$ eine Teilordnung "$\leq$": \[B\leq C \iff B \subseteq C\Forall B,C\in \mathcal{P}(A)\] \\
        \textbf{Beweis}: 1. - 3. sind trivial, 4. geht nicht (keine Totalordnung). Wähle $B,C\in \mathcal{P}(a), B,C\neq \emptyset, B\cap C = \emptyset$. Dann gilt weder $B\subseteq C$ noch $C\subseteq B\hfill\square$
\item Sei $F:=\{f\mid f:A\to\mathbb{R}\}$ für eine Menge $A\subseteq \mathbb{R}$. Wir definieren $f\leq g \iff \Forall x\in A: f(x) \leq g(x)$ \\
        (1.) - (3.) trivial, 4. gilt nicht. Falls $A$ mehr als ein Element hat, gibt es eine Funktion, die nicht miteinander verglichen werden können.
\end{enumerate}
\subsection{Maximum und Minimum einer Menge}
\label{sec-2-7}
\subsubsection{Definition}
\label{sec-2-7-1}
Sei $(A,\leq)$ eine teilweise geordnete Menge, $a\in A$ \\
    Maximum:
\[a = \max A \iff \Forall x\in A: x\leq a\]
Minimum:
\[a = \max A \iff \Forall x\in A: a\leq x\]
\subsubsection{Bemerkung}
\label{sec-2-7-2}
Durch die Aussagen ist $a$ eindeutig bestimmt, denn seien:
\[a_1,a_2\in A:\Forall x\in A \begin{cases}x\leq a_1 \\ x\leq a_2 \end{cases} \implies \begin{cases} a_2 \leq a_1 \\ a_1 \leq a_2 \end{cases} \xRightarrow{\text{Symmetrie}} a_1 = a_2 \]
\subsection{Schranken}
\label{sec-2-8}
Sei $(A,\leq)$ eine (total geordnete) Menge, $B\subseteq A$
\begin{enumerate}
\item $S\in A$ heißt obere Schranke zu $B \iff \Forall x\in B: x\leq S$ \\
      $S\in A$ heißt untere Schranke zu $B \iff \Forall x\in B: S\leq x$
\item $\bar{S}(B):= \{S\in A \mid S~\text{S ist untere Schranke zu}~B\}$ \\
      $\ubar{S}(B):= \{S\in A \mid S~\text{S ist obere Schranke zu}~B\}$
\item Existiert $g:=\min \ubar{S}(B)$ beziehungsweise $g:=\max \bar{S}$ so sagen wir: \\
      $g = \sup B$ (kleinste obere Schranke, \uline{Supremum}, obere "Grenze" von $B$ in $A$)
$g = \inf B$ (größte obere Schranke, \uline{Infimum}, untere "Grenze" von $B$ in $A$)
\end{enumerate}
\subsubsection{Bemerkung}
\label{sec-2-8-1}
\begin{enumerate}
\item Existiert $\max B = \bar{b}$, so folgt $\sup B = \bar{b}$, denn $\bar{b} \in \ubar{S}(B)$ nach Definition.
\[s\in \ubar{S}(B) \implies \bar{b} \leq s,~\text{da}~\bar{b}\in B\]
Ebenso gilt: $\exists\min B = \ubar{b} \implies \inf B = \ubar{b}$
\end{enumerate}
\subsubsection{Beispiel}
\label{sec-2-8-2}
\begin{enumerate}
\item $B = \{\frac{1}{n}\mid n\in\mathbb{N}\}, A = \mathbb{R},~(1, \frac{1}{2},\ldots)$
\begin{itemize}
\item Es gilt $1\in B, \Forall n\in\mathbb{N}$ gilt $\frac{1}{n} \leq 1$, daher folgt $\max B = \sup B = 1$
\item Sei $s\leq 0$, dann gilt $\Forall n\in\mathbb{N}: s\leq \frac{1}{n}$, also $s\in \bar{S}(B)$ \\
         Sei $s > 0 \implies s > \frac{1}{n} \iff n > \frac{1}{s}$, also $s\not\in\bar{S}(B)$ \\
         Es folgt $\bar{S}(B) = \{x\in\mathbb{R}\mid s\leq 0\}$ insbesondere $0\in\bar{S}(B)$ \\
         Ferner gilt $\Forall s\in \bar{S}(B):s\leq 0 \implies \ubar{0} = \max \bar{S}(B) = \inf B$
\end{itemize}
\item $A = \mathbb{Q}, B = \{x\in\mathbb{Q} : 0 \leq x \wedge x^2 \leq 2\}$. Es gilt $0 = \min B = \inf B$, aber $\sup B$ existiert nicht in $\mathbb{Q}$
\end{enumerate}
\subsection{Reelle Zahlen}
\label{sec-2-9}
$x^2 = 2$ hat keine Lösungen in $\mathbb{Q}$. Allerdings können wir $\sqrt{2}$ "beliebig gut" durch $y\in \mathbb{Q}$ approximieren, das heißt $\Forall \eps > 0\exists y\in\mathbb{Q}:2 - \eps \leq y^2 \leq 2 + \eps$
Das motiviert die folgende Vorstellung:
\begin{enumerate}
\item $\mathbb{Q}$ ist "unvollständig"
\item $\mathbb{Q}$ ist "dicht" in $\mathbb{R}$
\end{enumerate}
\subsubsection{Vollständigkeitsaxiom (Archimedes)}
\label{sec-2-9-1}
Jede nach oben (unten) beschränkte Teilmenge hat ein Supremum oder Infimum.
\subsubsection{Axiomatischer Standpunkt}
\label{sec-2-9-2}
Es gibt eine Menge $\mathbb{R}$ (genannt Menge der reellen Zahlen) mit Addition, Multiplikation, Ordnung, die die Definition eines Körper und das Vollständigkeitsaxiom erfüllt und $(\mathbb{R},+,\cdot)$ mit "$\leq$" eine Ordnung bildet.
\subsubsection{Bemerkung}
\label{sec-2-9-3}
\begin{enumerate}
\item Bis auf Isomorphie gibt es höchstens ein solches $\mathbb{R}$, das heißt $\tilde{\mathbb{R}}$ ein weiteres System der reellen Zahlen ist, dann $\exists$ bijektive Abbildung $f:\mathbb{R}\to\tilde{\mathbb{R}}$ die bezüglich Addition, Multiplikation, Ordnung eine Homomorphie ist.
\[\Forall x,y\in \mathbb{R}:\]
\[f(x+y) = f(x) + f(y)\]
\[f(x y) = f(x) f(y)\]
\[x\leq y \implies f(x) \leq f(y)\]
\item $\mathbb{N}$ (und damit auch $\mathbb{Z},\mathbb{Q}$) lassen sich durch injektive Homomorphismus $g:\mathbb{N}\to\mathbb{R}$ in $\mathbb{R}$ einbetten
\[g(\tilde{0}_{\in\mathbb{N}}) = 0_{\in\mathbb{R}}\]
\[g(\tilde{n}_{\in\mathbb{N}} + 1) = g(n_{\in\mathbb{R}}) + 1\]
\[g(1_{\in\mathbb{N}}) = 1_{\in\mathbb{R}}\]
\end{enumerate}
\subsubsection{Konstruktiver Standpunkt}
\label{sec-2-9-4}
Wir können $\mathbb{R}$ ausgehend von $\mathbb{Q}$ konstruieren.
\paragraph{Methode der Abschnitte}
\label{sec-2-9-4-1}
Jede reelle Zahl wird charakterisiert durch ein "rechts offenes, unbeschränktes Intervall", dessen "rechte Grenze" die Zahl erstellt.
\[\mathbb{R}:=\{A\subseteq \mathbb{Q}\begin{cases}A\neq\emptyset \\ x\in A, y\leq x\implies y\in A \\ \Forall x\in A\exists y\in A, x<y\end{cases}\]
\paragraph{Methode der Cauchy-Folgen}
\label{sec-2-9-4-2}
Jede reelle Zahl wird charakterisiert als "Grenzwert" eine Klasse äquivalenter "Cauchy Folgen" aus $\mathbb{Q}$ (später)
\subsubsection{Definition 1.37}
\label{sec-2-9-5}
\begin{itemize}
\item \[x\in \mathbb{R}~\text{heißt}~\begin{cases}\text{positiv} & 0 < x \\ \text{nicht negativ} & 0\leq x \\ \text{negativ} & x < 0 \\ \text{nicht positiv} & x\geq 0 \end{cases}\]
\item Die Betragsfunktion $\abs \cdot:\mathbb{R} \to \mathbb{R}$ wird definiert durch $\abs{x} = \max \{x,-x\} = \begin{cases} x & x \geq 0 \\ -x & x < 0\end{cases}$
\item Die Vorzeichen- oder Signumfunktion \[\sgn:\mathbb{R}\to\mathbb{R},\sgn{x} = \begin{cases}\frac{x}{\abs{x}} & x \neq 0 \\ 0 & x = 0\end{cases} = \begin{cases} 1 & x > 0 \\ -1 & x < 0 \\ 0 & x = 0\end{cases}\]
\end{itemize}
\subsubsection{Satz 1.38}
\label{sec-2-9-6}
\begin{enumerate}
\item $\abs{x y} = \abs{x} \abs{y}$
\item $\abs{x + y} \leq \abs{x} + \abs{y}$ \\
       \textbf{Beweis:} \\
\begin{align}
\abs{x + y}^2 &= (x+y)^2 = x^2 + 2x y + y^2 = \abs{x}^2 + 2xy + \abs{y}^2 \\
&\leq \abs{x}^2 + 2\abs{x y} + \abs{y}^2 = \abs{x}^2 + 2\abs{x}\abs{y} + \abs{y^2} \\
&= (\abs{x} + \abs{y})^2 \implies \abs{x + y} \leq \abs{\abs{x} + \abs{y}} = \abs{x} + \abs{y} \tag*{$\square$}
\end{align}
\item $\abs{x + y} = \abs{x} + \abs{y} \iff x y \geq 0$
\end{enumerate}
\subsubsection{Satz 1.39}
\label{sec-2-9-7}
\begin{enumerate}
\item $\abs{\abs{x} - \abs{y}} \leq \abs{x - y}$ \\
       \textbf{Beweis:} \\
\begin{align}
\abs{x} &= \abs{x - y + y} \leq \abs{x - y} + \abs{y} \implies \abs{x} - \abs{y} \leq \abs{x - y} \\
\abs{y} &= \abs{y - x + x} \leq \abs{y - x} + \abs{x} \implies \abs{y} - \abs{x} \leq \abs{x - y} \\
\abs{\abs{x} - \abs{y}} &= \max \{\abs{x} - \abs{y},\abs{y} - \abs{x}\} \leq \abs{x - y} \tag*{$\square$}
\end{align}
\item \[\abs{x - y} \leq \eps \iff \begin{cases} x - \eps \leq y \leq x + \eps \\ y - \eps \leq x \leq y + \eps \end{cases}\]
       \textbf{Beweis:} \\
\begin{align}
\abs{x - y} = \max\{x - y , y - x\} \leq \eps \iff \begin{cases} x - y \leq \eps \\ y - x \leq \eps\end{cases} \iff \begin{cases} x \leq y + \eps \\ y - x \leq \eps \end{cases} \iff y - \eps \leq x \leq y + \eps
\end{align}
Vertausche $x$ und $y$ $\implies$ $x - \eps \leq x + \eps \hfill \square$
\end{enumerate}
\subsubsection{Definition 1.40}
\label{sec-2-9-8}
Sei $a,b\in\mathbb{R},a\leq b$
\begin{itemize}
\item $[a,b]:=\{x\in\mathbb{R}: a\leq x \leq b\}$ \hfill abgeschlossenes Intervall
\item $(a,b):= \{x\in\mathbb{R}: a < x < b\} = ]a,b[$ \hfill offenes Intervall
\item $\left[a,b) := \{x\in\mathbb{R}:a\leq x < b\}$ \hfill rechts-halboffenes Intervall
\item $(a,b\right]:=\{x\in\mathbb{R}:a<x\leq b\}$ \hfill links-halboffenes Intervall
\item $\eps > 0, I_\eps (x) := (x -\eps,x + \eps) = \{y\in\mathbb{R}:\abs{x - y} < \eps = B_\eps (x) (\text{Kugel})\}$
\end{itemize}
\subsubsection{Lemma 1.41}
\label{sec-2-9-9}
Es gilt $y\in I_\eps (x) \implies \exists \delta > 0: I_\delta (y) \subseteq I_\eps (x)$
\paragraph{Beweis}
\label{sec-2-9-9-1}
Sei $y\in I_\eps (x) \implies \abs{x - y}  < \eps \iff \eps - \abs{x - y} > 0$
Wähle $0 < \delta < \eps - \abs{x - y}$. Es ist nun zu zeigen $I_\delta (y) \subseteq I_\eps (x)$, das heißt
$z\in I_\delta(y) \implies z\in I_\eps(x)$. Es gilt
\begin{align}
&z\in I_\delta (y) \implies \abs{z - y}  <\delta \\
\implies &\abs{z - x} = \abs{z - y + y - x} \leq \abs{z - y} + \abs{y - x} \leq \delta + \abs{x - y} < \eps \\
\implies &z\in I_\eps (x) \tag*{$\square$}
\end{align}
\subsubsection{Definition 1.42}
\label{sec-2-9-10}
$A,B$ seien geordnete Mengen, $f:A\to B$ heißt:
\begin{itemize}
\item monoton $\begin{cases} \text{wachsend} & x \leq y \implies f(x) \leq f(y)  \\ \text{fallend} & x \leq y \implies  f(x) \leq f(y) \end{cases}$
\item streng monoton $\begin{cases} \text{wachsend} & x < y \implies f(x) < f(y) \\ \text{fallend} & x < y \implies f(x) > f(y) \end{cases}$
\end{itemize}
\paragraph{Beispiel 1.43}
\label{sec-2-9-10-1}
$\mathbb{R}_+\setminus \{0\} \to \mathbb{R}_+\setminus\{0\}, x\mapsto x^n$ ist streng monoton wachsend $\Forall n\in\mathbb{N}$
\subparagraph{Beweis}
\label{sec-2-9-10-1-1}
Induktion + Axiom M0 $\hfill\square$
\subsubsection{Lemma 1.44}
\label{sec-2-9-11}
Sei $M,N \subseteq \mathbb{R}, f:M\to N$ streng monoton und bijektiv. Dann ist $f^{-1}$ streng monoton.
\paragraph{Beweis}
\label{sec-2-9-11-1}
Wir betrachten den Fall $f$ streng monoton wachsend. Seien $y_1,y_2 \in N, y_1 < y_2,x_1 = f^{-1}(y_1), x_2 = f^{-1}(y_2)$. \\
     Behauptung $x_1 < x_2$ (sonst wäre $x_1 \geq x_2$). \\
     Falls $x_1 > x_2 \xRightarrow{\text{streng monoton}} f(x_2) > f(x_2)$ \textbf{Widerspruch} zu $y_1 < y_2$ \\
     Falls $x_1 = x_2 \implies y_1 = y_2$ \textbf{Widerspruch} zur Annahme $y_1 < y_2 \hfill \square$
\subsubsection{Definition 1.45 Produktzeichen}
\label{sec-2-9-12}
Für $a\in\mathbb{R},n\in\mathbb{N}$ definieren wir $a^n := \prod_{j=1}^n a$ und für $a\in\mathbb{R}\setminus\{0\},n\in\mathbb{N}$ $a^{-n} := \frac{1}{a^n}$.
\subsubsection{Satz 1.46}
\label{sec-2-9-13}
Es gilt $\Forall a,b\in\mathbb{R}$ (beziehungsweise $\mathbb{R}\setminus \{0\}$),$n,m\in\mathbb{N}_0$ (beziehungsweise $\mathbb{Z}$)
\begin{enumerate}
\item $a^n a^m = a^{n+m}$
\item $(a^n)^m = a^{n m}$
\item $(ab)^m = a^m b^m$
\end{enumerate}
\paragraph{Beweis}
\label{sec-2-9-13-1}
Zunächst f+r $n,m\in\mathbb{N}_0$ durch Induktion nach $n$, dann für $n,m\in\mathbb{Z}$ (mit Hilfe der Definition von $a^{-n}$)
\subsubsection{Definition 1.47}
\label{sec-2-9-14}
Sei $n,k\in\mathbb{N}_0$ \[\binom{n}{k}:=\prod_{j=1}^k \frac{n -j + 1}{j}\]
\subsubsection{Lemma 1.48}
\label{sec-2-9-15}
Sei $k,n\in\mathbb{N}_0$
\begin{enumerate}
\item $\binom{n}{k} = 0$ für $k > n$ \\
       $\binom{n}{k} = \frac{n!}{k!(n -k)!} = \binom{n}{n - k}$ für $k\leq n$
\item $\binom{n}{k} = \binom{n - 1}{k - 1} + \binom{n - 1}{k}$ für $1 \leq k \leq n$
\end{enumerate}
\subsubsection{Satz 1.49}
\label{sec-2-9-16}
$\Forall n\in\mathbb{N}_0,\Forall x,y\in\mathbb{R}$ gilt
\[(x + y)^n = \sum_{j = 0}^n \binom{n}{j}x^{n - j}y^j\]
\paragraph{Beweis}
\label{sec-2-9-16-1}
Induktion:
\begin{itemize}
\item Induktionsanfang: $n = 0,(x+y)^0 = 1,\binom{0}{j}x^0y^0 = 1$ nach Definition
\item Induktionsschritt $n \to n + 1:$
\begin{align*}
\left\string(x + y\right\string)^{n + 1} &= (x +y)(x+y)^n \\
\shortintertext{mit der Induktionsvoraussetzung}
&=(x + y)\sum_{j = 0}^n \binom{n}{j}x^{n - j} y^j \\
&= \sum_{j=0}^n \binom{n}{j}x^{n - j + 1}y^j + \sum_{j = 0}^n \binom{n}{j} x^{n - j} y^{j + 1} \\
&= \binom{n}{0}x^{n + 1} + \sum_{j = 1}^n \binom{n}{j}x^{n + 1 - j}y^j + \underbrace{\sum_{i = 1}^n \binom{n}{i - 1} x^{n - i + 1} y^i}_{\text{Substitution $i:= j + 1$}} + \binom{n}{n}y^{n + 1} \\
&= x^{n + 1} + \sum_{j = 1}^n \underbrace{(\binom{n}{j} + \binom{n}{j - 1})}_{\binom{n + 1}{j}\text{nach Lemma 1.48}} x^{n + 1 - j} y^j + y^{n + 1} \\
&= \sum{j = 0}^{n + 1} \binom{n + 1}{j}x^{n + 1 - j}y^j\tag*{$\square$}
\end{align*}
\end{itemize}
\subsubsection{Folgerung 1.50}
\label{sec-2-9-17}
\begin{enumerate}
\item $\sum_{j = 0}^n \binom{n}{j} = 2^n$
\item $\sum_{j = 0}^n \binom{n}{j} (-1)^j = \begin{cases}0 & n \neq 0 \\ 1 & n = 0\end{cases}$
\end{enumerate}
\paragraph{Beweis:}
\label{sec-2-9-17-1}
Setze in Binomische Formel $x = 1, y = 1$ beziehungsweise $y = -1 \hfill\square$
\subsubsection{Lemma 1.51}
\label{sec-2-9-18}
Sei $m\in R$ nach oben (beziehungsweise nach unten) beschränkt \\
    Dann gilt
\begin{enumerate}
\item $s = \sup M \iff \Forall \eps > 0 \exists x\in M: s - \eps < x (\geq s)$
\item $l=\inf M \iff \Forall \eps > 0 \exists x\in M: (l \leq) x < l + \eps$
\end{enumerate}
\paragraph{Beweis}
\label{sec-2-9-18-1}
Wir beweisen 1. \\
     $s\neq \sup M\iff s$ ist nicht die kleinste obere Schranke von $m$ $\iff$ es gibt eine kleinere obere Schranke $s' = s - \eps$ von $M$ $\iff$
nicht $\Forall \eps > 0\exists x\in M: x > s - \eps \hfill \square$
\subsubsection{Lemma 1.52}
\label{sec-2-9-19}
$\mathbb{N}$ ist unbeschränkt in $\mathbb{R}$
\paragraph{Beweis}
\label{sec-2-9-19-1}
sonst $\exists x = \sup \mathbb{N}$ (nach Vollständigkeits Axiom), $x$ kleinste obere Schranke $\xRightarrow{\text{[[Lemma 1.51]]}} \eps= \frac{1}{2} \exists m_o \in \mathbb{N}:x - \frac{1}{2} < m_0 \implies m_0 + 1 \in \mathbb{N},m_0 + 1 > x + \frac{1}{2} > x$
$\implies x$ ist nicht die obere Schranke von $\mathbb{N}\hfill\square$
\subsubsection{Lemma 1.53 (Bernoullische Ungleichung)}
\label{sec-2-9-20}
\label{Lemma-1.53}
\[\Forall x\in \left[-1,\infty),n\in\mathbb{N}_0: (1 + x)^n \geq 1 + n x\]
\paragraph{Beweis}
\label{sec-2-9-20-1}
Beweis durch Induktion:
\begin{itemize}
\item \textbf{IA}: $n = 0$ klar
\item \textbf{IS}:
\begin{align}
n\to n + 1: (1 + x)^{n + 1} &= (1 + x)^n(1 + x) \\
&\geq (1 + n x) (1 + x) = 1 + nx^2 + (n + 1) x \\
&\geq 1 + (n + 1) x~\text{da $x^2 \geq 0$} \tag*{$\square$}
\end{align}
\end{itemize}
\subsubsection{Folgerung 1.54}
\label{sec-2-9-21}
\begin{enumerate}
\item Sei $y\in(1,\infty)$. Dann gilt $\Forall c > 0 \exists n_0 \in \mathbb{N},\Forall n\geq n_0 y^n \in (c,\infty)$ ("Konvergenz" von $y^n$ gegen 0)
\item Sei $y \in (-1,1)$. Dann gilt $\Forall \eps > 0\exists n_0 \in \mathbb{N}\Forall n\geq n_0:y^n \in I_\eps (0)$ ("Konvergenz" $y^n$ gegen 0)
\end{enumerate}
\paragraph{Beweis}
\label{sec-2-9-21-1}
\begin{enumerate}
\item \label{1541} Für $x = y - 1 > 0$ gilt dann nach \ref{Lemma-1.53} \[\underbrace{(1 + x)^n}_y \geq 1 + n x \implies y^n > n x\]
        Nach \ref{sec-2-9-19} existiert für $c > 0$ ein $n_0 \in \mathbb{N}$ mit $n_0 > \frac{c}{x} \implies$
        \[\Forall n\geq n_0: y^n > n x \geq n_0 x \geq \frac{c}{x} x = c \implies \Forall n\geq n_0: y^n\in(c,\infty)\]
\item Für  $x = \frac{1}{\abs{y}} > 1 \xRightarrow{\text{nach [[1541]] mit } c = \frac{1}{\eps}}$
        \[\Forall \eps > 0\exists n_0\in\mathbb{N}\Forall n\geq n_0: x^n > \frac{1}{\eps}\]
        \[\implies \frac{1}{\abs{y^n}} > \frac{1}{\eps} \implies \abs{y^n} < \eps \hfill \square\]
\end{enumerate}
\subsubsection{Satz 1.55 (Existenz der m-ten Wurzel)}
\label{sec-2-9-22}
\[\Forall m\in\mathbb{N}, a\in\left[a,\infty)~\text{gilt}~\exists ! x\in\left[0,\infty): x^m = a\]
\paragraph{Beweis (Skizze 1, 2)}
\label{sec-2-9-22-1}
Wir geben ein Iterationsverfahren
\[p_3 (x) = m\]
\[a_3 x^3 + a_2 x^2 + a_1 x + a_0, a_3 > 0\]
Ohne Beschränkung der Allgemeinheit $a > 0,m\geq 2$, $x$ muss die Gleichung $x^m -a = 0$ lösen, das heißt Nullstelle der Funktion $f:\left[0,\infty) \to \mathbb{R},x\mapsto x^m - a$ suchen.
Diese approximieren wir nach dem \textbf{Newton Verfahren} \\
     $x_0$ sodass $x_0^m - a \geq 0$
\[x_n - x_{n + 1} = \frac{f(x_n)}{f'(x_n)} \impliedby \frac{f(x_n)}{x_n - x_{n + 1}} = f'(x_n)\]
\[x_{n+1} := \underbrace{x_n - \frac{f(x_n)}{f'(x_n)}}_{F(x_n)} = x_n - \frac{x_n^m - a}{m x_n^{m - 1}}\]
\[= x_n(1 - \frac{1}{m}(1 - \frac{a}{x_n^m}))\]
Hoffnung: $x_n \to x^*$

\texttt{Skizze 3}

Sei $x_0^m > a$. Wir zeigen
\begin{enumerate}
\item \label{1.55.1} $x_n > 0$
\item \label{1.55.2} $x_n^m \geq a$
\item $x_{n + 1} \leq x_n$
\end{enumerate}
\textbf{Beweis:}
\begin{enumerate}
\item Induktion
\item Induktion
\begin{itemize}
\item $n = 0, x_0^m \geq \implies x_0 > 0$, da $a > 0,x_0\geq 0$
\item $n\to n + 1$ \[x_n > 0, x_n^m\geq a \implies x_{n + 1} = x_n(1 - \frac{1}{m}(1 - \frac{a}{x_n^m})) \geq 0\]
          weil \[x_{n + 1}^n = \underbrace{x_n^m}_{\geq 0} (1 - \frac{1}{m}(1 - \frac{a}{x_n^m}))^m \underbrace{\geq}_{\text{Bernoulli}} x_n^m(1 - \frac{1}{m}(1 - \frac{a}{x_n^m}))  = 0\]
          $\implies x_{n + 1} > 0$, da $a > 0$
\end{itemize}
\item Nach \ref{1.55.2}: \[x_n^m \geq a \implies 0 \leq 1 - \frac{1}{m}(1 - \frac{1}{x_n^m}) \leq 1\]
        Nach \ref{1.55.1}: \[x_m > 0 \implies x_{n + 1} = x_n(1 - \frac{1}{m}(1-\frac{a}{x_n^m})) < x_n\]
        Wegen \ref{1.55.1} ist $M = \{x_n:n\in\mathbb{N}_0\}$ nach unten beschränkt $\implies$
        \[x:= \inf M~\text{existiert}\]
        Wir wollen zeigen, dass $x^m = a$. Es gilt \[x \leq x_{n + 1} = (1 - \frac{1}{m})x_n + \frac{1}{m}\frac{a}{x^{m -1}_n}\]
        \[\leq (1 - \frac{1}{m})x_n + \frac{a}{m}\sup \{\frac{1}{x_n^{m - 1}\mid x\in\mathbb{N}_0}\} \]
\item \label{1.55.4} Es gilt nach nach \ref{1.56.2}
\[a\leq \inf \{x_n^m \mid n\in \mathbb{N}_0\} = (\inf \{x_n \mid n\in \mathbb{N}_0\})^m = x^m\]
und damit $x > 0$ \\
        Ferner gilt
\[y = \sup \{\frac{1}{x_n^{m - 1}} \mid n\in\mathbb{N}_0\} = \inf \{x_n^{m - 1}\mid x \in \mathbb{N}_0\}^{-1}\]
mit \ref{sec-2-9-23} \[= (\frac{1}{\inf\{x_n \mid n\in\mathbb{N}_0\}})^{m - 1} = \frac{1}{x^{m -1}} \implies a y\leq \frac{a}{x^{m - 1}}\]
\item \label{1.55.5} Von oben wissen wir, dass $x \leq a y$
        \[\implies x\leq a y  \leq \frac{a}{x^{m -1}} \implies x^m \leq a\]
\end{enumerate}

Aus \ref{1.55.4} und \ref{1.55.5} folgt $x^m = a\hfill\square$

\subsubsection{Lemma 1.56}
\label{sec-2-9-23}
\begin{enumerate}
\item Seien für $n\in \mathbb{N}_0:y_n > 0$ und $\inf \{x_n\mid x\in\mathbb{N}_0\} > 0$ \\
       Dann gilt \[\sup \{\frac{1}{y_n} \mid n\in \mathbb{N}_0\} = \frac{1}{\inf \{y_n \mid n\in\mathbb{N}_0\}}\]
\item \label{1.56.2} Seien für $n\in\mathbb{N}_0,y_n > 0, k\in\mathbb{N}_0$. Dann gilt:
\[\inf \{y_n^k \mid n\in\mathbb{N}_0\} = (\inf \{y_n\mid n\in\mathbb{N}_0\})^k\]
\end{enumerate}
(ohne Beweis)
\section{Komplexe Zahlen}
\label{sec-3}
\textbf{Motivation:} $x^2 + 1 = 0$ nicht lösbar in $\mathbb{R}$ \\
  Wir betrachten die Menge der Paare $\{x,y\} = \mathbb{R}\times\mathbb{R}$ auf denen die Addition und Multiplikation wie folgt definiert ist:
\begin{itemize}
\item \label{KA} (KA) $\{x_1,y_1\} + \{x_2,y_2\} = \{x_1 + x_2, y_2 + y_2\}$
\item \label{KM} (KM) $\{x_1,y_1\} \cdot \{x_2,y_2\} = \{x_1 x_2 - y_1 y_2, x_1 y_2 + x_2 y_1\}$
\end{itemize}
\subsection{Komplexer Zahlenkörper}
\label{sec-3-1}
\begin{enumerate}
\item Die Menge der Paare $z = \{x,y\} \in \mathbb{R}\times\mathbb{R}$ mit Addition \ref{KA} und Multiplikation \ref{KM} bildet den Körper $\mathbb{C}$ der \textbf{komplexen Zahlen} mit den neutralen Elementen $\{0,0\}$ und $\{1,0\}$
\item Die Gleichung $z^2 + \{1,0\} = \{0,0\}$ hat in $\mathbb{C}$ zwei Lösungen, welche mit $\I:= \{0,\pm 1\}$ bezeichnet werden
\item Der Körper $\mathbb{R}$ ist mit der Abbildung $x\in\mathbb{R}:x\mapsto\{x,0\}\in\mathbb{C}$ isomorph zu einem Unterkörper von $\mathbb{C}$
\end{enumerate}
\subsubsection{Beweis}
\label{sec-3-1-1}
\begin{enumerate}
\item Die Gültigkeit des Kommutativitäts-, Assoziativs-, und Distributivitätsgesetzes verifiziert man durch Nachrechnen. \\
       Neutrale Elemente: Wir lösen die Gleichung $a + z = \{0,0\}$ für beliebige gegebene $a\in\mathbb{C},a=\{a_1,a_2\}$
       \[\implies z = \{-a_1, -a_2\}\]
       \[a\cdot z = \{1,0\}\]
       \[z = \frac{1}{a}:=\{\frac{a_1}{a_1^2 + a_2^2},-\frac{a_2}{a_1^2 + a_2^2}\},~\text{weil}~a\cdot\frac{1}{a}\]
       \[\text{weil}~a\frac{1}{a}=\{a_1\frac{a_1}{a_1^2 + a_2^2} + \frac{a_2^2}{a_1^2 + a_2^2},\frac{a_1 a_2}{a_1^2 + a_2^2} - \frac{a_2 a_1}{a_1^2 + a_2^2}\}\]
\item $i:= \{0,1\}$ hat die Eigenschaft
\[1 + \I^2 = \{1,0\} + \{0^2 - 1^2, 0\} = \{0,0\} \implies 1+\I^2 = 0\]
Ähnlich $1 + (-\I)^2 = 0$
\item Die Zuordnung $x\in\mathbb{R}:x\mapsto\{x,0\}\in\mathbb{C}$ bildet $\mathbb{R}$ bijektiv auf eine Untermenge von $\mathbb{C}$ ab, welche bezüglich der komplexen Addition und Multiplikation wieder ein Körper ist $\hfill\square$
\end{enumerate}
\subsection{Notation}
\label{sec-3-2}
$z = \{x,y\} =: x + \I y,~x,y\in\mathbb{R}$
\begin{itemize}
\item $x$ ist Realteil $x = \Re{z}$
\item $y$ ist Imaginärteil $x = \Im{z}$
\end{itemize}
\[z_1 + z_2 = (x_1 + \I y_1) + (x_2 + \I y_2) = \underbrace{x_1 + x_2}_{\Re (z_1 + z_2)} + \I\underbrace{(y_1 + y_2)}_{\Im (z_1 + z_2)}\]
\[z_1 z_2 = (x_1 + \I y_1)(x_1 + \I y_2) = x_1 x_2 + \I y_1 x_2 + \I y_2 x_1 + (\I y_1)(\I y_2) = \underbrace{x_1 x_2 - y_1 y_2}_{\Re (z_1 z_2)} + \I\underbrace{(x_1y_2 + y_1 x_2)}_{\Im (z_1,z_2)}\]
\subsection{{\bfseries\sffamily TODO} Graphische Darstellung}
\label{sec-3-3}
\subsection{Bemerkung}
\label{sec-3-4}
Die reellen Zahlen  sind durch $\Im z = 0$ charakterisiert.
\[z_1 = z_2 \implies x_1 + \I y_i = x_2 + \I y_2 \iff x_1 = x_2,y_1 = y_2\]
\subsection{Korollar 1.59}
\label{sec-3-5}
Jede quadratische Gleichung
\[z^2+p z + q = 0,~p,q\in\mathbb{R}\]
besitzt in $\mathbb{C}$ genau zwei Lösungen
\[z_{1,2} = \begin{cases} -\frac{1}{2}\pm \frac{1}{2}\sqrt{p^2 - 4q} & p^2 \geq 4q \\ -\frac{1}{2}\pm \I\frac{1}{2}\sqrt{\abs{p^2 - 4q}} & p^2 - 4q < 0 \end{cases} \]
\subsection{Fundamentalsatz der Algebra}
\label{sec-3-6}
Jede algebraische Gleichung der Form \[z^n + \sum_{i = 0}^{n - 1} a_i z^i = 0\]
hat in $\mathbb{C}$ mindestens eine Lösung. Beweis $\rightarrow$ Funktionstheorie
\subsection{Betrag}
\label{sec-3-7}
Für komplexe Zahlen lässt sich ein Absolutbetrag definieren
\[r = \abs{z} = \sqrt{x^2 + y^2}\]
Damit:
\begin{align}
x &=& r\cos{\alpha}
y &=& r\sin{\alpha}
z &=& x + \I y = r(\cos{\alpha} + \I\sin{\alpha})
\end{align}
\subsection{Konjugation}
\label{sec-3-8}
Zu einem $z = x + \I y\in\mathbb{C}$ definieren wir eine konjugierte komplexe Zahl \[\bar z = x - \I y \in\mathbb{C}\]
Dann gilt \[\abs{z}^2 = x^2 + y^2 = z\bar z\]
Aus der Definition:
\begin{itemize}
\item $\overline{z_1 + z_2} = \overline{z_1} + \overline{z_2}$
\item $\overline{z_1 * z_2} = \overline{z_1} * \overline{z_2}$
\item $x = \frac{z + \bar z}{2}$
\item $y = \frac{z - \bar z}{2\I}$
\end{itemize}
\section{Folgen}
\label{sec-4}
Eine Folge von reellen Zahlen wird gegeben durch eine Abbildung \[\mathbb{N}_0 \to \mathbb{R},n\mapsto x_n\]
Wir bezeichnen die Folge auch mit $(x_n)_{n\in\mathbb{N}_0}$

Topologische Struktur auf Mengen.
\begin{itemize}
\item Abstände in $\mathbb{R}^1$ Betrag $\abs{x - y}$ $\xrightarrow{\text{Verallgemeinerung}}$ Norm / Metrik
\item Umgebung in $\mathbb{R}^1$ $\eps$-Intervall $\xrightarrow{\text{Verallgemeinerung}}$ Kugel Umgebung
\end{itemize}

Wir betrachten Folgen $\mathbb{N}\to\mathbb{R}, n\mapsto a_n$ (oder $\mathbb{C}$)
\subsection{Definition 2.1 Konvergenz}
\label{sec-4-1}
Wir sagen, dass die Folge $(a_n)_{n\in\mathbb{N}}$ in $\mathbb{K}$ ($\mathbb{R}$ oder $\mathbb{C}$) gegen den Grenzwert (oder Limes) $a\in\mathbb{K}$ konvergiert
\[a_n \xrightarrow{n\to\infty} a~(a=\lim_{n\to\infty} a_n)\]
wenn für beliebiges $\eps > 0$ von einem $n_\eps \in\mathbb{N}$ an gilt
\[\abs{a_n - a} < \eps,n\geq n_\eps\]
\[\iff \Forall\eps > 0\exists n_\eps \in\mathbb{N}:\Forall n\geq n_\eps a_n \in I_\eps(a)\]
\subsection{Folgerung 2.2}
\label{sec-4-2}
Sei $(a_n)_{n\in\mathbb{N}}$ eine monoton wachsende beziehungsweise fallende Folge reeller Zahlen $M=\{a_n\mid n\in\mathbb{N}\}$ und sei nach oben beziehungsweise unten beschränkt. Dann gilt \[a_n\to\sup M, a_n\to\inf M\]
Beweis $\to$ Übungen
\subsection{Definition 2.3 Cauchy Folgen}
\label{sec-4-3}
Eine Folge $(a_n)_{n\in\mathbb{N}}$ heißt Cauchy-Folge wenn:
\[\Forall\eps > 0 \exists n_\eps \in\mathbb{N}\Forall n,m \geq n_\eps: \abs{a_n - a_m} < \eps \]
(Cauchy Kriterium)
\subsection{Definition 2.4 Teilfolge}
\label{sec-4-4}
Eine Teilfolge einer gegebenen Folge $(a_n)_{n\in\mathbb{N}}$ ist eine Auswahl $(a_{n_k})_{k\in\mathbb{N}}$, wobei $a_{n_k}$ auch die Glieder von $(a_n)_{n\in\mathbb{N}}$ sind
\begin{ex}[Beispiel 2.5]
\[a_n = \frac{1}{m}\] ist eine Cauchy-Folge. Für ein $\eps > 0$ wählen wir $n_\eps$ so dass $n_\eps > \frac{1}{\eps}$. Für beliebiges $n\geq m > N$
\[\abs{a_m - a_n} = \abs{\frac{1}{m} - \frac{1}{n}} = \frac{n - m}{m n} \leq \frac{n}{m n} = \frac{1}{m} < \frac{1}{n_\eps} < \eps\hfill\square\]
\end{ex}
\begin{thm}[Jede Cauchy-Folge ist beschränkt]
\end{thm}
\begin{proof}
Sei $(a_n)_{n\in\mathbb{N}}$ eine Cauchy-Folge. Angenommen, die Folge ist nicht beschränkt. Dann gibt es eine Teilfolge $(a_{n_k})_{k\in\mathbb{N}}$ mit
\[\abs{a_{n_k}}\xrightarrow[k\to\infty]{} \infty\]
Aus dieser Teilfolge kann man eine weitere Teilfolge
\[(a_{n_{k_l}})_{l\in\mathbb{N}}\]
extrahieren
\[\abs{a_{n_{k_{i + 1}}}} > 2 \abs{a_{n_{k_l}}}\quad l\in\mathbb{N}\]
Dann gilt
\[\abs{a_{n_{k_{i + 1}}} - a_{n_{k_l}}} \geq \abs{a_{n_{k_{i + 1}}}} - \abs{a_{n_{k_l}}} > \abs{a_{n_{k_l}}}\xrightarrow[k\to\infty]{} \infty\]
im Widerspruch zur Cauchy-Folgen Eigenschaft.
\end{proof}
\begin{thm}[Jede konvergente Folge ist Cauchy-Folge]
\end{thm}
\begin{proof}
\begin{align*}
a_n \xrightarrow[k\to\infty]{} a &\implies \Forall \eps > 0 \exists n_\eps \in\mathbb{N} \Forall n\geq n_\eps: \abs{a - a_n} < \frac{\eps}{2} \\
&\implies \Forall n,m\in n_\eps: \abs{a_n - a_m} \leq \abs{a_n - a} + \abs{a - a_m} <\frac{\eps}{2} + \frac{\eps}{2} \tag*{\qedhere}
\end{align*}
\end{proof}
\begin{lemma}
\label{lemma:one_limit}
Sei $(a_n)_{n\in\mathbb{N}}$ eine Folge in $\mathbb{K}$ ($\mathbb{R}$ oder $\mathbb{C}$) welche gegen $a\in\mathbb{K}$ und $\tilde a \in \mathbb{K}$ konvergiert. Dann ist $a = \tilde a$.
\end{lemma}
\begin{proof}
Beweis durch Widerspruch. \\
Falls $\abs{a - \tilde a} > 0$, dann
\[\exists n_\eps \in\mathbb{N}\Forall n\geq n_\eps \eps = \abs{a - \tilde a}, \abs{a_n - a} < \frac{\eps}{2}\]
und ein $m_\eps$, sodass
\[\abs{a_n - \tilde a < \frac{\eps}{2}} \Forall n\geq m_\eps\]
Dann für $n \geq \max \{n_\eps, m_\eps\}$:
\[\abs{a - \tilde a} \leq \abs{a - a_n} + \abs{a_n - \tilde a} < \eps\text{\lightning}\]
\textbf{Widerspruch} $\implies a = \tilde a$
\end{proof}
\begin{remark}
Die Mengen Abständen heißen *vollständig*, wenn jede Cauchy-Folge in $M$ konvergiert
\end{remark}
\begin{defn}[Häufungwert, Häufungspunkt]
Ein $a\in\mathbb{K}$ heißt Häufungswert einer Folge $(a_n)_{n\in\mathbb{N}}$ in $\mathbb{K}$, wenn es zu beliebigen $\eps > 0$ unendlich viele Folgenelemente $a_n$ gibt mit $\abs{a - a_n} < \eps$

Ein $a\in\mathbb{K}$ heißt Häufungspunkt einer Teilmenge $M$ von $\mathbb{K}$, wenn $\Forall \eps > 0$ existieren unendlich viele $x\in M$, sodass $\abs{a - x} < \eps$
\end{defn}
\begin{ex} \mbox{}
\begin{enumerate}
\item $a_n = (-1)^n, n\in\mathbb{N}$
\begin{itemize}
   \item divergente Folge
   \item besitzt 2 Häufungswerte $a^{(1)} = 1, a^{(2)} = -1$
\end{itemize}
\item Wir nehmen $a_n \xrightarrow[n\to\infty]{} a, b_n \xrightarrow[n\to\infty]{} b$ und definieren eine neue Folge $c_n$ sodass
   \begin{align*}
   c_{2n} &:= b_n,n\in\mathbb{N} \\
   c_{2n + 1} &:= a_n,n\in\mathbb{N}
   \end{align*}
   $(c_n)_{n\in\mathbb{N}}$ hat 2 Häufungswerte $a$ und $b$
\end{enumerate}
\end{ex}
\begin{remark}
Nach \ref{lemma:one_limit} hat die konvergente Folge 1 Häufungswert
\end{remark}
\begin{lemma}[2.11]
\label{lemma:cauchy_haufungswert}
Sei $(a_n)_{n_\in\mathbb{N}}$ eine Cauchy-Folge in $\mathbb{K}$ und $a$ ein Häufungswert von $(a_n)_{n\in\mathbb{N}}$, dann konvergiert $a_n \xrightarrow[n\to\infty]{} a$
\end{lemma}
\begin{proof}
Sei $\eps > 0$ beliebig vorgegeben. Wir wählen $n_\eps \in\mathbb{N}$ sodass
\[\abs{a_n - a_m} < \frac{\eps}{2}\Forall n,m > n_\eps\hfill\text{ (aus Cauchy-Folge)}\]
und $m_\eps > n_\eps$ mit
\[\abs{a - a_{m_\eps}} < \frac{\eps}{2}\hfill\text{ (Häufungswert)}\]
Dann folgt
\[\Forall n > m_\eps: \abs{a - a_n} \leq \abs{a - a_{m_\eps}} + \abs{a_{m_\eps} - a_n} < \eps \implies a_n \xrightarrow[n\to\infty]{} a \qedhere\]
\end{proof}
\begin{thm}
\label{thm:finite}
$A$ abgeschlossen $\iff$ ($a$ Häufungspunkt von $A \implies a\in A$)
$A$ abgeschlossen in M $\iff$ $M\setminus A =: CA$ offen
\end{thm}
\begin{proof}
$(\impliedby)$: \\
Sei jeder Häufungspunkt von $A$ in $A$
$x\in CA (=\mathbb{R}\setminus A) \implies x$ kein Häufungspunkt von $A, x\not\in A$
\[\implies \eps: I_\eps (x) \cap A = \emptyset \implies \exists \eps > 0: I_\eps \subseteq CA\]
$\implies CA$ offen $\implies A$ abgeschlossen \\
$(\implies)$: \\
Sei $A$ abgeschlossen, also $CA$ offen, ist Häufungspunkt $x\not\in A$ das heißt $x \in CA$, so gilt
\[\exists \eps > 0: I_\eps \subseteq CA \implies I_\eps (x) \cap A = \emptyset \text{lightning}\]
\textbf{Widerspruch} zur Definition von Häufungspunkt $\implies$ jeder Häufungspunkt von $A$ ist in $A$
\end{proof}
\begin{lemma}[2.14]
\label{lemma:monoton_in_r}
Jede Folge $(a_n)_{n\in\mathbb{N}} \in \mathbb{R}$ besitzt eine monotone Teilfolge
\end{lemma}
\begin{proof}
Sei $B=\{n\in\mathbb{N}\mid \Forall k\geq n, a_n \geq a_k\}$
\begin{itemize}
\item Fall 1: $B$ unendlich. Wir zählen $B\subseteq \mathbb{N}$ monoton wachsend \\
\begin{align*}
n_0 &= \min{B} \\
n_{k+1} &= \min\{n\in B, n > n_k\}
\end{align*}
Dann ist die Teilfolge $(a_{n_k})_{k\in\mathbb{N}}$ von $(a_n)_{n\in\mathbb{N}}$ monoton fallend
\item Fall 2: $B$ ist endlich oder leer
\begin{align*}
\implies &\exists n_0 \in\mathbb{N}: \Forall n\geq n_0: n\not\in B \\
\shortintertext{das heißt}
\exists k\leq n: a_n < a_k
\end{align*}
Damit können wir definieren
\[n_{k + 1} = \min\{k \geq n_k: a_{n_k} < a_k\}\]
und die Folge $(a_{n_k})_{k\in\mathbb{N}}$ ist monoton wachsend
\end{itemize}
\end{proof}
\begin{ex}
\begin{enumerate}
\item $a_n = (-1)^n (1 + \frac{1}{n + 1}),\quad B=\{2 n \mid n \in\mathbb{N}\}$ monoton fallend
\item $a_n = (-1)^n n, (a_{2k})_{k\in\mathbb{N}}$ ist monotone Teilfolge
\end{enumerate}
\end{ex}
\begin{thm}[Satz von Bolzano Weierstrass]
Sei $A\subseteq \mathbb{R}$ ( gilt in $\mathbb{R}^n$!) Folgende Aussagen sind äquivalent:
\begin{enumerate}
\item \label{bolzano:1} $A$ ist beschränkt abgeschlossen
\item \label{bolzano:2} Jede Folge $(a_n)_{n\in\mathbb{N}}$ aus $A$ hat einen Häufungswert in $A$
\item \label{bolzano:3} Jede Folge $(a_n)_{n\in\mathbb{N}}$ aus $A$ besitzt eine in $A$ konvergente Teilfolge $(a_{n_k})_{k\in\mathbb{N}}$
\end{enumerate}
\end{thm}
\begin{proof}
Wir zeigen \ref{bolzano:3} $\implies$ \ref{bolzano:2} $\implies$ \ref{bolzano:1} $\implies$ \ref{bolzano:3} \\
   \ref{bolzano:3} $\implies$ \ref{bolzano:2}: \\
   Sei $(a_{n_k})_{k\in\mathbb{N}}$ konvergente Teilfolge von $(a_n)_{n\in\mathbb{N}}$ und $a = \lim_{k\to\infty} a_{n_k}$
$a$ ist auch der Häufungswert der Folge $(a_n)_{n\in\mathbb{N}}$ \\
   \ref{bolzano:2} $\implies$ \ref{bolzano:1}: \\
\begin{enumerate}
\item Beschränktheit:
Angenommen dies ist falsch. Dann
\[\exists (a_n)_{n\in\mathbb{N}} \in A: \abs{a_n - a} \geq n \Forall n\in\mathbb{N}~(a\in A)\]
Nach Voraussetzungen hat jede diese Folge einen Häufungspunkt $x\in A$ und es gilt
\[\abs{x - a} \geq \abs{a_n - a} - \abs{a_n - x} \geq n - \abs{x - a_n}\]
Dabei gilt $\abs{x - a_n} < 1$ für unendlich viele $n\in\mathbb{N}$ (aus Häufungswert)
\[\implies \abs{x - a} \geq n - 1\]
Für unendlich viele $n\in\mathbb{N}$ \lightning
\item Abgeschlossenheit: Wir nutzen Satz \ref{thm:finite}
Zu zeigen: wenn $a$ Häufungspunkt von $A \implies a\in A$
Für
\[I_{\frac{1}{n}}(a) = \{x\in\mathbb{R} \mid \abs{x - a} < \frac{1}{n}\}\]
gilt
\[I_{\frac{1}{n}}(a) \cap A \neq \emptyset \implies \exists a_n\in A: \abs{a_n - a} < \frac{1}{n}\]
Die Folge $(a_{n_k})_{k\in\mathbb{N}} \to a$, da $\frac{1}{n}\to 0$
Nach Voraussetzung hat $(a_n)_{n\in\mathbb{N}}$ einen Häufungswert $\tilde a \in A$. Wir zeigen $a = \tilde a$
Sei $\eps > 0$ beliebig.
\begin{align*}
&\exists n_\eps\in\mathbb{N}: \abs{a - a_n} <\frac{\eps}{2}\Forall n\geq n_\eps \tag{\text{Aus }$a_n \to a$} \\
&\exists m_\eps\geq n_\eps: \abs{\tilde a - a_{m_\eps}} <\frac{\eps}{2} \tag{\text{Aus Häufungswert}} \\
&\implies \abs{a - \tilde a} \leq \abs{a - a_{m_\eps}} + \abs{a_{m_\eps}} < \eps \\
&\implies \abs{a - \tilde a} = 0 \\
&\implies \tilde a = a \in A
\end{align*}
\end{enumerate}
\ref{bolzano:1} $\implies$ \ref{bolzano:3}: \\
   Sei nun $(a_n)_{n\in\mathbb{N}}$ eine Folge in $A$, $(a_{n_k})_{k\in\mathbb{N}}$ eine monotone Teilfolge (nach \ref{lemma:monoton_in_r}), $(a_{n_k})$ ist beschränkt, da $A$ beschränkt ist $\implies (a_{n_k})$ ist konvergent (\ref{sec-4-2}) \\
   Wir müssen zeigen, dass
\[a = \lim_{n\to\infty} a_{n_k} \in A\]
\begin{align*}
\intertext{Angenommen $a\not\in A \implies a \in \mathcal{C} A, \mathcal{C} A$ ist offen}
\implies \exists I_\eps (a) \subseteq \mathcal{C}A \implies I_\eps (a) \cap A = \emptyset \\
\shortintertext{Nun ist aber mit geeigneten $n_\eps \in\mathbb{N}$}
\Forall n\geq n_\eps: a_{n_k} \in I_\eps (a): a_{n_k} \in A \implies a_{n_k} \in I_\eps (a) \cap A~\lightning \tag*{\qedhere}
\end{align*}
\end{proof}
\begin{remark}
\begin{itemize}
\item Erweiterung zu $\mathbb{R}^n$ möglich
\item Ein Raum heißt folgenkompakt, wenn jede beschränkte Folge eine konvergente Teilfolge hat
\begin{itemize}
\item Nach B-W Satz ist $\mathbb{R} (\mathbb{R}^n)$ folgenkompakt
\end{itemize}
\item In $\mathbb{R}$ alle Cauchy-Folgen konvergieren
\begin{itemize}
\item Cauchy Folge in $\mathbb{R} \implies$ beschränkt und Wertemenge ist abgeschlossen $\xRightarrow{B-W Satz} (a_n)_{n\in\mathbb{N}}$ hat einen Häufungswert in $A \xRightarrow{\ref{lemma:cauchy_haufungswert}}$ konvergiert gegen $a\in A$
\end{itemize}
\end{itemize}
\end{remark}
\subsection{Rechenregeln für Grenzwerte von Folgen}
\label{sec-4-5}
\begin{thm}
Seien $(a_n)_{n\in\mathbb{N}}, (b_n)_{n\in\mathbb{N}}$ konvergente Folgen in $\mathbb{K} (\mathbb{R}~\text{oder}~\mathbb{C})$
\[b_0 \neq 0 \Forall n\in\mathbb{N}, \lim_{n\to\infty} b_n \neq 0\]
Dann gilt:
\begin{enumerate}
\item $\displaystyle \lim_{n\to\infty} (a_n + b_n) = \lim_{n\to\infty} a_n + \lim_{n\to\infty} b_n$
\item $\displaystyle \lim_{n\to\infty} (a_n b_n) = \lim_{n\to\infty} a_n \lim_{n\to\infty} b_n$
\item $\displaystyle \lim_{n\to\infty} (\frac{a_n}{b_n}) = \frac{\lim_{n\to\infty} a_n}{\lim_{n\to\infty} b_n}$
\end{enumerate}
\end{thm}

\begin{thm}[2.15]
Seien $(a_n)_{n\in\mathbb{N}}, (b_n)_{n\in\mathbb{N}}$ konvergente Folgen in $\mathbb{R}$. Dann gilt
\begin{enumerate}
\item $\displaystyle a_n \leq b_n \Forall n\in\mathbb{N} \implies \lim_{n\to\infty} a_n \leq \lim_{n\to\infty} b_n$
\item $\displaystyle \abs{a_n} \leq b_n \Forall n\in\mathbb{N} \implies \abs*{\lim_{n\to\infty a_n}} \leq \lim b_n$
\end{enumerate}
\end{thm}
\begin{proof}
\begin{enumerate}
\item Sei $\eps > 0$ vorgegeben
\begin{align*}
\exists n_\eps:\Forall n\geq n_\eps: b_n \leq \lim_{k\to\infty} b_n + \frac{\eps}{2} \\
\shortintertext{und}
\lim_{k\to\infty} a_k \leq a_n + \frac{\eps}{2} \\
&\implies \lim_{k\to\infty} a_k \leq a_n + \frac{\eps}{2} \leq b_n + \frac{\eps}{2} \leq \lim_{k\to\infty} b_k + \eps \Forall \eps > 0 \\
&\implies \lim_{\k\to\infty} a_k \leq \lim_{k\to\infty} b_k
\end{align*}
\item Wir wählen $a_n = \abs{a_n}$ und müssen noch zeigen
\[\lim_{n\to\infty} \abs{a_n} = \abs{\lim_{n\to\infty}} a_n\tag{Übung}\]
\end{enumerate}
\end{proof}
\subsection{Geometrische Folge}
\label{sec-4-6}
Die geometrische Folge ist definiert durch
\[a_n = c q^n\]

\begin{lemma}[2.16]
$\Forall q\in\mathbb{R},\abs{q} < 1$ konvergiert die geometrische Folge $a_n = c q^n$ gegen Null.
\end{lemma}
\begin{proof}
Sei $\eps > 0$ gegeben. Nach Annahme ist $\abs{q} < 1 \implies \abs{q}^{-1} > 1$, somit $\abs{q}^{-1} = 1 + x$ für ein $x > 0$. \\
\begin{align*}
\intertext{Zu zeigen: $\abs{c q^n - 0} < \eps$ für genug große $n$, das heißt}
c(\frac{1}{1+x})^n < \eps \iff \frac{c}{\eps} < (1 + x)^n \\
\intertext{Das Archimedisches Axiom garantiert die Existenz von $n_0\in\mathbb{N}$:}
n_0 > \frac{c}{x\eps} - \frac{1}{x} = \frac{c - \eps}{x\eps} \\
\Forall n\geq n_0: \frac{c}{\eps} = (\frac{c}{x\eps} - \frac{1}{x}x + 1 < n_0 x + 1 \leq n x + 1)
\intertext{daraus folgt aus der Bernoulli Ungleichung}
\frac{c}{\eps} < (1 + x)^n \implies c q^n \to 0 \tag*{\qedhere}
\end{align*}
\end{proof}
\begin{conc}[2.17]
Die geometrische Reihe
\[S_n = 1 + q + q^2 + \ldots + q^n = \sum_{i = 0}^n q^i\]
konvergiert für $\abs{q} < 1$ und $\lim_{n\to\infty} S_n = \frac{1}{1 - q}$
\end{conc}
\begin{proof}
\begin{gather*}
\intertext{zu Beweisen mit Induktion}
\string(1 - q\string)\string(1 + q + q^2 + \ldots + q^n\string) = 1 + q^{n + 1} \\
\implies S_n - \frac{1}{1 - q} = \frac{1 - q^{n + 1} - 1}{1 - q} = -\frac{q^{n + 1}}{1 - q} \\
\abs{S_n - \frac{1}{1 - q}} = c\abs{q}^n < \eps \Forall n\geq n_\eps \\
\intertext{$c = \abs*{\frac{1}{1 - q}}$}
s_n \to \frac{1}{1 -q} \tag*{\qedhere}
\end{gather*}
\end{proof}
\begin{ex}[2.18]
\mbox{}
\begin{enumerate}
\item $\displaystyle \lim_{n\to\infty} \frac{10^n}{n!} \leq \lim_{n\to\infty}cq^n$ mit $\abs{q} < 1$
\item $a_n = \sqrt{n}(\sqrt{n + 1} - \sqrt{n}) = \sqrt{n} \frac{n + 1 - 1}{\sqrt{n + 1} + \sqrt{n}} = \frac{\sqrt{n}}{\sqrt{n + 1}} + \sqrt{n} = \frac{1}{\sqrt{1 + 1_n} + 1} \xrightarrow{n \to \infty} \frac{1}{2}$
\item $a_n = \sqrt[m]{x}$, $x$ gegeben, $\xrightarrow{n\to\infty} 1$ \hfill Übungen
\item $a_n = \sqrt[n]{m}\xrightarrow{n\to\infty} 1$
\item $a_n = \sum_{i = 0}^{n} \frac{1}{i!}$
\begin{itemize}
\item $(a_n)_{n\in\mathbb{N}}$ ist monoton wachsend
\item beschränkt: $a_n < 3 \Forall n\in\mathbb{N}$
\item $\implies (a_n)_{n\in\mathbb{N}}$ konvergiert, Limes ist sogenannten Zahl $\euler$
\end{itemize}
\item $(a_n)_{n\in\mathbb{N}}$ rekursiv definiert: $a_0 = 0, a_1 = 1, a_n = a_{n - 1} + a_{n - 2}$ Fibonacci Folge
\end{enumerate}
\end{ex}
\subsection{Umgebung}
\label{sec-4-7}
\begin{defn}[2.19]
$A\subseteq \mathbb{K}$ heißt Umgebung von $a\in\mathbb{K} \iff \Exists \eps > 0 I_\eps (a) \subseteq A$
\end{defn}
\begin{conc}[2.20]
Aus der Definition folgt
\begin{enumerate}
\item Sei $U_i, i\in I$ Umgebung von $a$, so ist $\bigcup\limits_{i\in I}  U_i$ Umgebung von $a$
\item Sind $U_1,\ldots, U_n$ Umgebung von $a$, so ist auch $U_1 \cap \ldots U_n$ Umgebung von $a$
\item $\Forall$ Umgebung von $a: \Exists$ Umgebung von $a$, sodass $\Forall y\in V, U$ Umgebung von $y$ ist
\end{enumerate}
\end{conc}
\begin{proof}
\begin{enumerate}
\item Für irgendein \[i_0 \in I \Exists \eps > 0: I_\eps (a) \subseteq U_{i_0} \subseteq \bigcup_{i \in I} U_i\]
\item Es gilt nach Voraussetzung $\eps_1, \ldots \eps_n > 0$ mit $I_{\eps_i}(a) \subseteq U_i$ für $i = 1,\ldots,n$. Folglich gilt für $\eps:= \min \{\eps_1, \ldots, \eps_n\} > 0,~I_\eps(a) \subseteq U_i (\Forall i = 1,\ldots,n) \implies I_\eps (a) \subseteq U_1 \cap \ldots U_n$
\item Nach Voraussetzung gibt es für eine Umgebung $U$ von $a$ ein $\eps > 0$ mit $I_\eps(a) \subseteq U$ \\
      $V:= I_{\frac{\eps}{2}}(a) \subseteq U$ ist ebenfalls Umgebung von $a$ und $\Forall y\in V$ gilt
\[I_{\frac{\eps}{2}} \subseteq I_\eps(x) \subseteq U,~\text{denn}~ \underbrace{\abs{y - z}}_{z\in I_{\frac{\eps}{2}}} < \frac{\eps}{2} \implies \abs{x - z} \leq \abs{x - y} + \abs{x - z} < \eps\]
\end{enumerate}
\end{proof}
\begin{defn}[2.21]
\mbox{}
\begin{enumerate}
\item $A\subseteq \mathbb{K}$ ist offen $\iff \Forall a\in A$ ist $A$ die Umgebung von $a$ \\
      (in $\mathbb{R}\Forall a\in A \Exists \eps > 0 I_\eps (a) \subseteq A$)
Für Intervalle $(a,b)$ haben wir schon gezeigt, dass sie offen sind
\item $A\subseteq \mathbb{K}$ heißt abgeschlossen $\iff C_\mathbb{K} A$ offen
\item Abschließung von $A$: \[\bar A := \{a \in \mathbb{K} \mid a \in A \vee a ~\text{Häufungspunkt von}~A\}\]
\item Rand von $A$: \[\partial A:= \{a\in\mathbb{K}\mid \Forall ~\text{Umgebung $U$ von $a$}: A \cap U \neq \emptyset \wedge CA \cap U \neq \emptyset\}\]
\end{enumerate}
\end{defn}
\begin{ex}[2.22]
\begin{align*}
A&= (a,b\right] \\
\bar A &= [a,b] \\
\partial A &= \{ a,b \} \\
\Forall \eps > 0 I_\eps(a) \cap (a,b\right] \neq \emptyset \\
I_\eps (a) \cap \mathbb{R} \setminus (a,b\right] \neq \emptyset
\end{align*}
Sei $A = \mathbb{Q}$, dann $\bar A =\mathbb{\mathbb{R}}$, $\partial A =\mathbb{R}$ denn in jedem $\eps$-Intervall um eine rationale Zahl gibt es sowohl rationale als auch irrationale Zahlen
\end{ex}
\begin{remark}
\mbox{}
\begin{itemize}
\item Die Grenzwerte und Häufungswerte kann man auch in ganz \[\mathbb{R} \cup \{\infty\} \cup \{\infty\} =: \hat{\mathbb{R}}\]
     mit einer neuen Definition von Abstand: \[(x,y) := \abs{\xi(x) - \abs{\xi(y)}}\]
     \[\xi(x) := \begin{cases} \frac{\abs{x}}{1 + \abs{x}} & x\in\mathbb{R} \\ \pm 1 & x = \pm \infty \end{cases}\]
\item $\hat{\mathbb{R}}$ ist folgenkompakt
\item Algebraische Operationen in $\hat{\mathbb{R}}$
\begin{align*}
x + \infty &:= \infty + x := \infty \Forall x \in \mathbb{R} \cup \{\infty\} \\
x - \infty &:= -\infty + x := -\infty \Forall x \in \mathbb{R} \cup \{-\infty\} \\
x\cdot\infty &:= \infty \cdot x := \begin{cases} \infty & \Forall x\in\hat{\mathbb{R}}, x > 0 \\ -\infty & \Forall x\in\hat{\mathbb{R}},x < 0\end{cases} \\
\frac{1}{\infty} = \frac{1}{-\infty} =: 0 \\
\intertext{Sinnlos wäre:}
\infty - \infty, 0\cdot \infty, 0\cdot (-\infty), \frac{\infty}{\infty}, \ldots
\end{align*}
\item Damit könne wir die Rechenregeln auch für Folgen in $\hat{\mathbb{R}}$ formulieren
\item In $\hat{\mathbb{R}}$ hat jede Folge einen Häufungswert
\end{itemize}
\end{remark}
\begin{defn}[2.23]
\label{defn:223}
Sei $(a_n)_{n\in\mathbb{N}}$ ein Folge von reellen Zahlen, $\emptyset \neq H\subseteq \hat{\mathbb{R}}$ die Menge der Häufungswerte von $(a_n)$ in $\hat{\mathbb{R}}$. \\
   Dann sei:
\begin{align*}
\overline{\lim} a_n := \lim_{n\to\infty} \inf a_n := \inf H \tag{Limes inferior} \\
\underline{\lim} a_n := \lim_{n\to\infty} \sup a_n := \inf H \tag{Limes superior}
\end{align*}
\end{defn}
\begin{remark}
\mbox{}
\begin{enumerate}
\item Definition \ref{defn:223} kann man auch für $\mathbb{R}$ formulieren
\item \[a = \lim_{n\to\infty} \inf a_n \iff \Forall \eps \begin{cases} \string(1\string) \{n \mid \abs{a - a_n} < \eps\} ~\text{ist unendlich (weil $a$ Häufungswert ist)} \\ \string(2\string) \{n \mid a_n < a - \eps\} ~\text{ist endlich ($a$ ist kleinste Häufungswert)}\end{cases}\]
\end{enumerate}
\end{remark}
\begin{ex}[2.24]
\begin{align*}
a_n = n + (-1)^n n \\
a_{2n + 1} = 0 \Forall n \implies 0 ~\text{ist Häufungswert} \\
a_{2n} = 4n \to \infty \implies \infty~\text{ist Häufungswert}
\intertext{also gilt}
\lim_{n\to\infty} \inf a_n = 0 \\
\lim_{n\to\infty} \sup a_n = \infty \\
\end{align*}
\end{ex}
\begin{remark}
\mbox{}
\begin{itemize}
\item $a_n \to a$ in $\displaystyle \hat{\mathbb{R}} \iff \lim_{n\to\infty} \inf a_n = a = \lim_{n\to\infty}\sup a_n$
\item $\displaystyle \lim_{n\to\infty} \inf a_n + \lim_{n\to\infty}\inf b_n \leq \lim_{n\to\infty} \inf (a_n + b_n)$
\item $\displaystyle \lim_{n\to\infty} \inf a_n \cdot \lim_{n\to\infty}\inf b_n \leq \lim_{n\to\infty} \inf (a_n \cdot b_n)$ für $a_n,b_n > 0$
\item $\displaystyle \lim_{n\to\infty} \sup a_n + \lim_{n\to\infty}\sup b_n \geq \lim_{n\to\infty} (a_n + b_n)$ (zum Beispiel betrachte $a_n = n^2, b_n = \frac{1}{n}$)
\end{itemize}
\end{remark}
\section{Reihen (Unendliche Summen)}
\label{sec-5}
\begin{defn}[2.19]
Eine Reihe \[\sum_{k = 1}^\infty a_k\] (\textbf{unendliche Summe}) konvergiert, wenn die Folge ihrer \textbf{Partialsummen} konvergiert
\[s_n = \sum_{k = 1}^n\xrightarrow{n\to\infty} S_\infty < \infty\]
\end{defn}
\begin{ex}
\mbox{}
\begin{enumerate}
\item $\displaystyle \sum_{k = 1}^n k = \frac{n(n + 1)}{n}\xrightarrow{n\to\infty} \infty$
\item $\displaystyle S_n = \sum_{k = 1}^n (-1)^k = \begin{cases} -1 & n~\text{ungerade} \\ 0 & n~\text{gerade}\end{cases}$ S$_{\text{n}}$ ($= -1, 0, -1, 0,\ldots$) konvergiert nicht
\item $\displaystyle S_n = \sum_{j = 0}^n z^j = \frac{1 - z^{n + 1}}{1 - z}$ Für $\abs{z} < 1$ konvergiert $\displaystyle S_n \to \frac{1}{1 - z} \implies \sum_{j = 0}^\infty z^j = \frac{1}{1 - z}$
\item Harmonische Reihe: Seien $\displaystyle S_n = \sum_{k = 1}^n \frac{1}{k}$, Behauptung $\displaystyle \lim_{n\to\infty} S_n = \infty$, also divergent
\end{enumerate}
\end{ex}
\begin{proof}[Beweis von 4.]
\begin{align*}
S_{2^{n + 1}} = \sum_{k = 1}^{2^{n + 1}} \frac{1}{k} = 1 + \frac{1}{2} + \sum_{j = 1}^n \sum_{k = 2^j + 1}^{2^{j + 1}} \frac{1}{k} \geq 1 + \frac{1}{2} + \sum_{j = 1}^{n}\underbrace{\sum_{k = 2^j +1}^{2^{j + 1}}}_{\mathclap{2^j~\text{Summanden}}} \frac{1}{2^{j + 1}} \\
= 1+ \frac{1}{2} + \sum_{j = 1}^n 2^{j} \frac{1}{2^{j + 1}} = 1 + \frac{1}{2} + \sum_{j = 1}^{n} \frac{1}{2} = 1 + \frac{1}{2} + \frac{1}{2}n \xrightarrow{n\to\infty} \infty
\end{align*}
\end{proof}
\begin{thm}
Seien $\displaystyle \sum_{k = 0}^\infty a_k , \sum_{k = 0}^\infty b_k$ konvergente Reihen, $\alpha \in \mathbb{R}$, dann sind auch die Reihen
\[\sum_{k = 0}^\infty (a_k + b_k), \sum_{k = 0}^\infty \alpha a_k\]
konvergent und es gilt
\[\sum_{k = 0}^\infty (a_k + b_k) = \sum_{k = 0}^\infty a_k + \sum_{k = 0}^\infty b_k, \sum_{k = 0}^\infty \alpha a_k = \alpha \sum_{k = 0}^\infty a_k\]
\end{thm}
\begin{proof}
Aus den Rechenregeln für konvergente Folgen
\end{proof}
\subsection{Konvergenzkriterien}
\label{sec-5-1}
Cauchy Kriterium für Partialsummen besagt, dass eine Reihe genau dann konvergent ist, wenn
\[\Forall \eps > 0 \Exists n_\eps \in \mathbb{N}: \Forall n > m \geq n_\eps: \abs{s_n - s_m} = \abs{\sum_{k = m + 1}^n a_k} < \eps\]

\begin{lemma}[2.28 Reihenkonvergenz]
Eine Reihe $\displaystyle \sum_{k = 1}^\infty a_k$ kann nur dann konvergent sein, wenn ihre Partialsummen beschränkt sind und ihre Glieder eine Nullfolge bilden
\end{lemma}
\begin{proof}
Sei $\displaystyle s_\infty = \sum_{k = 1}^\infty a_k = \lim_{n \to \infty} s_n$. Dann gilt
\[\lim_{n\to\infty} a_n = \lim_{n\to\infty} (s_n - s_{n - 1}) = \lim_{n\to\infty} s_n - \lim_{n\to\infty} s_{n - 1} = s_\infty - s_\infty = 0\]
Die Beschränktheit der Partialsummen folgt notwendig aus der Beschränktheit konvergenter Folgen.
\end{proof}
\begin{thm}[2.29]
Sei $(a_k)_{k\in\mathbb{N}}$ eine Nullfolge. Dann $\displaystyle \sum_{k = 1}^\infty (a_k - a_{k + 1}) = a_1$
\end{thm}
\begin{proof}
\[s_n = \sum_{k = 1}^n (a_k - a_{k + 1}) = \sum_{k = 1}^n a_k - \sum_{k = 2}^{n + 1} a_k = a_1 - a_{n + 1} \implies \abs{s_n - a_1} = \abs{a_{n + 1}} \xrightarrow{n\to\infty} 0\]
\end{proof}
\begin{ex}[2.30]
\[\sum_{k = 1}^\infty \frac{1}{k(k + 1)} = \sum_{k = 1}^\infty (\underbrace{\frac{1}{k}}_{\mathclap{a_k}} - \underbrace{\frac{1}{k + 1}}_{\mathclap{a_{n + 1}}}) = a_1 = \frac{1}{2}\]
\end{ex}
\begin{defn}[2.31]
Eine Reihe $\displaystyle s_\infty = \sum_{k = 1}^\infty a_k$ in $\mathbb{R}$ heißt alternierend, wenn ihre Elemente alternierende Vorzeichen haben, das heißt $a_n \cdot a_{n + 1} \leq 0$
\end{defn}
\begin{thm}[2.32]
\begin{enumerate}
\item Eine alternierende Reihe $\displaystyle s_\infty = \sum_{k = 1}^\infty a_k$ ist konvergent, wenn die Absolutbeträge ihrer Glieder eine monoton fallende Nullfolge bilden
\item Für die Reihenreste gilt dabei die Abschätzung
\[\abs{\sum_{k = m}^\infty a_k} \leq \abs{a_m}\]
\end{enumerate}
\end{thm}
\begin{proof}
\begin{enumerate}
\item Sei ohne Beschränkung der Allgemeinheit $a_1 > 0$. Dann ist $a_{2n - 1} + a_{2 n} \geq 0, a_{2n} + a_{2n + 1} \geq 0$
      Und folglich
\begin{align*}
s_{2n + 1} = a_1 + a_2 + a_3 + \ldots + a_{2 n} + a_{2 n + 1} \leq s_{2n - 1} \leq \ldots \leq s_3 \leq s_1 \\
s_{2n} = (a_1) + (a_2 + a_4) + \ldots + (\underbrace{a_{2n - 1} + a_{2n}}_{\geq 0}) \geq s_{2n - 2} \geq \ldots \geq s_2 \\
\intertext{Ferner gilt}
s_{2n + 1} - s_{2 n} = a_{2n + 1} \geq 0
\intertext{und somit}
s_2 \leq \ldots \leq s_{2n} \leq s_{2n + 1} \leq \ldots \leq s_1
\intertext{$(S_{2n})$ monoton wachsend, $s_{2n + 1}$ monoton fallend, beide beschränkt}
\implies s_{2n} \xrightarrow{n\to\infty} s_\ast, \implies s_{2n + 1} \xrightarrow{n\to\infty} s^\ast \\
s_{sn} \leq s_\ast \leq s^\ast \leq s_{2n + 1} \\
\intertext{da $(a_n)$ Nullfolge}
\abs{s_{2n + 1} - s_{2 n}} = \abs{a_{2n + 1}} \to 0 \\
s_\ast = s^\ast = s_\infty
\end{align*}
\item Aus 1. folgt $m = 2n + 1$
\[0 \leq s_\infty - s_{2n} = \sum_{k = 2n + 1}^\infty a_k = s_\infty - s_{2n + 1} + a_{2n + 1} \leq a_{2n + 1}\]
und sonst
\[\abs{\sum_{k = 2n + 1}^\infty a_k} \leq \abs{a_{2n + 1}}\]
Analog im Fall $m = 2n$
\end{enumerate}
\end{proof}
\begin{ex}[2.33]
\begin{enumerate}
\item $\displaystyle s_\infty = \sum_{k = 1}^\infty \frac{(-1)^{k - 1}}{k} = 1 - \frac{1}{2} + \frac{1}{3} - \ldots$ konvergiert nach dem Leibniz Kriterium
\[\abs{\frac{(-1)^{k - 1}}{k}} = \frac{1}{k} \to 0~\text{monoton}\]
\item Die Leibniz Reihe $\displaystyle s_\infty =  \sum_{k = 0}^\infty \frac{(-1)^k}{2k + 1} = 1 - \frac{1}{3} + \frac{1}{5} - \frac{1}{7} + \ldots$ konvergiert nach Leibniz Kriterium
\end{enumerate}
\end{ex}
\begin{remark}[Monotonie ist wichtig]
\[\sum_{k = 1}^\infty a_k~\text{mit}~ a_{2k} := -\frac{1}{2^k}, a_{2k - 1}:= \frac{1}{k}\]
ist divergent:
\begin{itemize}
\item $(1 - 1) + (1 - 1) + (1 - 1) + \ldots = 0$, aber
\item $1 + (-1 + 1) + (-1 + 1) + \ldots = 1$
\end{itemize}
\end{remark}
\begin{defn}[2.34]
$\displaystyle \sum_{k = 1}^\infty a_k$ heißt absolut konvergent, genau dann wenn $\displaystyle \sum_{k = 1}^\infty \abs{a_k}$ konvergent ist
\end{defn}
\begin{thm}[2.35]
Sei $\displaystyle \sum_{k = 1}^\infty \abs{a_k}$ konvergent in $\mathbb{R}$. Dann ist $\displaystyle \sum_{k = 1}^\infty a_k$ konvergent
\end{thm}
\begin{proof}
Mit Cauchy Kriterium:
\[\abs{\sum_{k = m}^n a_k} \leq \sum_{k = m}^n \abs{a_k} < \eps\]
aus der absoluten Konvergenz
\end{proof}
\begin{thm}[2.36 Umordnungssatz]
Sei $\displaystyle \sum_{k = 1}^\infty a_k$ eine absolut konvergente Reihe in $\mathbb{R}$. Dann gilt
für jede bijektive Abbildung $\tau: \mathbb{N} \to \mathbb{N}$
\[\sum_{k = 1}^\infty a_{\tau(k)} = \sum_{k = 1}^\infty a_k\]
\end{thm}
\begin{proof}
Ranacher für spezifische Umordnung
\end{proof}
\begin{ex}[2.37]
$\displaystyle \sum_{k = 1}^\infty \frac{(-1)^{k - 1}}{k}$ konvergent (aber nicht absolut) \\
   Behauptung: $\Exists$ Umordnung $\tau$, sodass $\displaystyle \sum_{k = 1}^\infty \frac{(-1)^{\tau(k) - 1}}{\tau(k)}$ divergiert
Beachte
\[\frac{1}{2^j + 1} + \frac{1}{2^j + 3} + \ldots + \frac{2\cdot 2^j - 1} \leq 2^{j - 1} \frac{1}{2^{j + 1}} = \frac{1}{4}\]
$\implies$ Die Umordnung
\[1 - \frac{1}{2} + \frac{1}{3} - \frac{1}{4} + (\frac{1}{5} + \frac{1}{7}) - \frac{1}{6} + \underbrace{(\frac{1}{9} + \frac{1}{11} + \frac{1}{13} + \frac{1}{15})}_{\geq \frac{1}{4} - \frac{1}{8} = \frac{1}{8}} - \frac{1}{8} + \ldots + \underbrace{(\frac{1}{2^j + 1} + \frac{1}{2^j + 3} + \ldots + \frac{1}{2^{j + 1} - 1})}_{> \frac{1}{4} - \frac{1}{8} = \frac{1}{8}} - \frac{1}{2^k + 2}\]
konvergiert nicht
\end{ex}
\begin{thm}[2.38 Cauchyprodukt für Reihen]
Seien $\displaystyle \sum_{k = 1}^\infty a_k, \sum_{k = 1}^\infty b_k$ absolut konvergente Reihen (in $\mathbb{R}$ oder $\mathbb{C}$). Sei $\displaystyle c_m = \sum_{k = 1}^m a_k b_{m-k}$. Dann konvergiert
\[\sum_{m = 1}^\infty = (\sum_{k = 1}^\infty a_k)(\sum_{k = 1}^\infty b_k)\]
(ohne Beweis)
\end{thm}
\begin{thm}[2.39 Vergleichskriterium]
Gegeben seien zwei Reihen $\displaystyle s_\infty = \sum_{k = 1}^\infty a_k, \tilde s_\infty = \sum_{k = 1}^\infty \tilde a_k$
\begin{enumerate}
\item Gilt für fast alle $k \in \mathbb{N}$ mit einer Konstante $\alpha > 0\quad \abs{a_k} \leq \alpha \tilde a_k$ \\
      (für fast alle $n\in\mathbb{N} :=$ Für alle $n\in\mathbb{N}$ außer endlich viele) \\
      so ist $\tilde s_\infty$ eine \textbf{Majorante} von $s_\infty$ und aus der absoluten Konvergenz von $\tilde s_\infty$ folgt auch die von $s_\infty$, absolute Divergenz von $s_\infty$ impliziert die absolute Divergenz von $\tilde s_\infty$
\end{enumerate}
\end{thm}
\begin{proof}
ohne Beschränkung der Allgemeinheit nehmen wir an, dass die Voraussetzungen $\Forall k \in \mathbb{N}$ gelten
\begin{enumerate}
\item Ist $\tilde s_\infty$ konvergent
\[\implies \sum_{k = 1}^{n}\abs{a_k} \leq \alpha \sum_{k = 1}^{n} \abs{\tilde a_k} \leq \alpha \sum_{k = 1}^{\infty} \tilde a_k, \Forall n\in\mathbb{N}\]
$\implies S_n$ sind beschränkt, $S_\infty$ absolut konvergent
Umgekehrt folgt aus Divergenz von $\tilde S_\infty$ auch $\sum_{k = 1}^{\infty} \abs{a_k} \to \infty \implies \tilde S_\infty$ auch Divergent
\item Aus Voraussetzung
\begin{align*}
\abs{\frac{a_{k + 1}}{\tilde a_{k + 1}}} \leq \abs{\frac{a_{k + 1}}{a_k}} \abs{\frac{a_k}{\tilde a_{k + 1}}} \leq \abs{\frac{\tilde a_{k + 1}}{\tilde a_k}} \abs{\frac{a_k}{\tilde a_{k + 1}}} = \abs{\frac{a_k}{\tilde a_k}} \leq \ldots \leq \abs{\frac{a_1}{\tilde a_1}} =: \alpha
\end{align*}
$\implies \abs{a_{k + 1}} \leq \alpha \abs{a_k}$. Aus 1. folgt die Aussage
\end{enumerate}
\end{proof}
\begin{korollar}[2.34 Wurzelkriterium]
Eine Reihe $\displaystyle \sum_{k = 1}^{\infty}a_k$ konvergiert absolut, wenn es ein $g\in (0,1)$ gibt, mit dem für f.a. (fast alle) $k\in \mathbb{n}$ gilt $\sqrt[k]{\abs{a_k}} \leq q \leq 1$,
beziehungsweise $\lim_{k \to \infty} \sup \sqrt{\abs{a_k}} < 1$ \\
   Wenn für unendlich viele $k \in \mathbb{N}$ gilt $\sqrt[k]{\abs{a_k}} > 1$, beziehungsweise $\abs{a_k} > 1$, so ist die Reihe absolut divergent.
\end{korollar}
\begin{proof}
Nach Voraussetzung $\abs{a_k} \leq q^k$, das heißt die konvergierende geometrische Reihe $\tilde s_\infty$ mit $q\in(0,1)$ ist Majorante für $s_\infty$
\end{proof}
\begin{korollar}[2.41 Quotientenkriterium]
Eine Reihe $\displaystyle \sum_{k = 0}^{\infty} a_k$ konvergiert absolut, wenn es ein $q\in (0,1)$ gibt mit dem für f.a. $k\in\mathbb{N}$ gilt
\[\abs{\frac{a_{k + 1}}{a_k}} \leq q < 1, ~\text{bzw.}~ \lim_{k\to\infty} \sup \abs{\frac{a_{k + 1}}{a_k}} < 1\]
Wenn für fast alle $k\in\mathbb{N}$ gilt $\displaystyle \abs{\frac{a_{k + 1}}{a_k}} \geq 1$, so ist die Reihe absolut divergent
\end{korollar}
\begin{proof}
Vergleich mit
\[\tilde s_\infty \sum_{k = 1}^{\infty}q^k\]
\end{proof}
\begin{ex}[2.42]
\mbox{}
\begin{enumerate}
\item $\displaystyle s_\infty \sum_{k = 1}^{\infty} \frac{z^k}{k!}, z\in\mathbb{C}$ \\
      Quotientenkriterium: \[\abs{\frac{a_{k + 1}}{a_k}} = \abs{\frac{z^{k + 1}}{(k + 1)!} \frac{k!}{z^k}} = \abs{\frac{z}{k + 1}}\]
      Sei $k \geq 2 \abs{z} \implies \abs{\frac{z}{k + 1}} \leq \frac{1}{2} \implies s_\infty$ absolut konvergent.
\item $\displaystyle \sum_{k = 1}^{\infty} \frac{k!}{k^k}$
      \[\abs{\frac{(k + 1)!}{(k + 1)^{k + 1}} \frac{k^k}{k!}} = \abs{\frac{k}{k + 1}}^k = \frac{1}{(1 + \frac{1}{k})^k} \leq \frac{1}{1 + k \frac{1}{k}} = \frac{1}{2}\]
      $\implies s_\infty$ absolut konvergent
\end{enumerate}
\end{ex}
\begin{remark}
\begin{enumerate}
\item Falls $q = 1 \implies$ die Kriterien geben keine Entscheidung, zum Beispiel:
\begin{align*}
\sum_{k = 1}^{\infty} \frac{1}{k} \vee \sum_{k = 1}^{\infty} \frac{1}{k^2} \\
\abs{\frac{a_{k + 1}}{a_k}} = \abs{\frac{k}{k + 1}} \to 1 \\
\abs{\frac{a_{k + 1}}{a_k}} = \frac{k^2}{(k + 1)^2} \to 1 \\
\end{align*}
\item Für die Divergenz ist es wichtig, dass $\Exists n_0 \Forall n\geq n_0 a_n > 0$, Wir nehmen
\[a_n = \begin{cases} \frac{1}{n^2} & n = 2^k \\ 2(2^{-k})^2 & n - 1 = 2^k \\ 0\end{cases}\]
$\sum a_n$ konvergiert, aber $\lim_{a_n \neq 0} \frac{a_{n + 1}}{a_n} = 2$
\end{enumerate}
\end{remark}
\begin{lemma}[2.43 Cauchy Verdichtungssatz]
Eine Reihe $\displaystyle s_\infty = \sum_{k = 1}^{\infty} a_k$, mit $a_k \in \mathbb{R}_+$, die monoton fallende Nullfolge bilden hat dasselbe Konvergenzverhalten wie die verdichtete Reihe
\[\sum_{k = 0}^{\infty} 2^k a_{2^k} = a_1 + 2 a_2 + 4 a_4 + 8 a_8 + \ldots\]
\end{lemma}
\begin{proof}
Wir setzen $\displaystyle s_n := \sum_{k = 1}^{n} a_k, \tilde s_n := \sum_{k = 0}^{n} 2^k a_{2^k}$ \\
   Für $n < 2^{k + 1}$
\[S_n = a_1 + (a_2 + a_3) + \ldots + (a_{2^k} + \ldots + a_{k^{k + 1} - 1}) \leq a_1 + 2 a_2 + 4 a_4 + \ldots + 2^k a_{2^k} = \tilde s_n\]
$\implies$ Konvergenz von $\tilde s_k$ impliziert Konvergenz von $S_n$

Falls die verdichtete Reihe divergent ist, so folgt aus der für $n \geq 2^{k + 1}$ gültigen Beziehung
\begin{align*}
s_n &\geq a_1 + a_2 + (a_3 + a_4) + (a_5 + \ldots + a_8) + \ldots + (a_{2^k + 1} + \ldots + a_{2^{k + 1}}) \\
&\geq a_1 + a_2 + 2 a_4 + 4 a_8 + \ldots + 2^k a_{2^{k + 1}} \geq \frac{1}{2} \tilde S_{k + 1}
\end{align*}
auch die Divergenz von $S_n$
\end{proof}
\subsection{Potenzreihe}
\label{sec-5-2}
\[S_\infty = \sum_{k = 0}^{\infty} c_k (x - x_0)^k\]
mit den Koeffizienten $c_k \in \mathbb{K}$, Zentrum $x_0 \in \mathbb{K}$ und Argument $x\in \mathbb{K}$
\begin{itemize}
\item Die geometrische Reihe ist ein Spezialfall der allgemeinen Potenzreihe
\item Unendlicher Dezimalbruch
\[0, d_1, d_2, d_3, \ldots = \sum_{k = 1}^{\infty} d_k 10^{-k}, d_k \in \{0,1,\ldots, 9\}\]
\end{itemize}

\begin{thm}[2.44 Potenzreihen]
Eine Potenzreihe $\displaystyle \sum_{k = 0}^{\infty} c_k (x - x_0)^k$ konvergiert absolut $\Forall x\in\mathbb{K}$ mit der Eigenschaft
\[\abs{x - x_0} < \rho := \frac{1}{\lim_{k \to \infty} \sup \sqrt[k]{\abs{c_k}}}\]
Für $\abs{x - x_0} >  \rho$ ist sie divergent
\end{thm}
\begin{proof}
Für $x\neq x_0$ gilt
\[\lim_{k\to\infty} \sup \sqrt[k]{\abs{c_k \abs{x - x_0}^k}} = \abs{x - x_0} \lim_{k\to\infty} \sup \sqrt[k]{\abs{c_k}} = \frac{\abs{x - x_0}}{\rho} = \begin{cases} < 1 & \abs{x - x_0} < \rho \\ > 1 & \abs{x - x_0} > \rho \end{cases} \]
\end{proof}
\begin{remark}
Falls $\rho = \infty$, konvergiert die Reihe $\Forall x\in \mathbb{K}$ \\
   Falls $\rho = 0$, konvergiert die Reihe für kein $x\neq x_0$
\begin{itemize}
\item Die Konvergenzgrenze $\rho$ ist die größt mögliche und wird \textbf{Konvergenzradius} der Reihe bezeichnet
\item Für $\lim\sup \sqrt[k]{\abs{c_k}} = \infty$ konvergiert die Reihe für kein $x \neq x_0$ und wir setzen $\rho 0$
\item Falls $\lim\sup \sqrt[k]{\abs{c_k}} = 0 \implies \rho = \infty$
\end{itemize}
\end{remark}
\subsection{Exponentialreihe}
\label{sec-5-3}
\[\exp(x) := \sum_{k = 0}^{\infty} \frac{x^k}{k!}\] ist eine Potenzreihe.
Ihr Konvergenzradius
\[\rho = \frac{1}{\lim_{n\to\infty}\sup \sqrt[n]{\abs{a_n}}} = \frac{1}{\lim_{n\to\infty} \sqrt[n]{\frac{1}{n!}}} = \lim_{n\to\infty} \sqrt[n]{n!} = \infty\]

\begin{thm}[2.45]
Der Wert der $\exp$ Reihe für $x = 1$ ist die Eulersche Zahl $e$
\[\exp(1) = \sum_{k = 0}^{\infty} \frac{1}{k!} = \lim_{n\to\infty} (1 + \frac{1}{n})^n =: e\]
Diese ist irrational
\end{thm}
\begin{proof}
In Übung 6.2 gezeigt \[e = \lim_{n\to\infty} (1 + \frac{1}{n})^n\]
Angenommen $e = \frac{p}{q}, p,q\in\mathbb{N}, q > 1$. Betrachte Abschätzung, für die Restgliederdarstellung von $\euler$:
\begin{align*}
s_{n + m} - s_n &= (1 + \frac{1}{1!} + \ldots + \frac{1}{(m + n)!}) - (1 + \frac{1}{1!} + \ldots + \frac{1}{n!}) \\
&= \frac{1}{(n + 1)!} + \ldots + \frac{1}{(m + n)!} \\
&= \frac{1}{(n + 1)!}(1 + \frac{1}{n + 1} + \ldots + \frac{1}{(n + 1)^{m - 1}}) = \frac{1}{(n + 1)!} \sum_{k = 0}^{m - 1} \frac{1}{(n + 1)^k} \\
\intertext{für $x = \frac{1}{(n + 1)}$ erhält man}
&= \frac{1}{(n + 1)!} \frac{1 - x^m}{1 - x} \\
&\leq \frac{1}{(n + 1)!} \frac{1}{1 - x} = \frac{1}{(n + 1)!} \frac{n + 1}{n}
\end{align*}
Da dies für alle $m \in \mathbb{N}$, folgt
\[0 < e - s_n \leq \frac{1}{n! n} \implies 0 < e n! -s_n n! \leq \frac{1}{n}\]
\end{proof}
\section{Stetige Abbildungen}
\label{sec-6}
\subsection{Grenzwert einer Funktion, Stetigkeit}
\label{sec-6-1}
Wir betrachten die Funktion
\[f:\mathbb{R}\setminus\{0\} \to \mathbb{R}, x\mapsto \frac{e^x - }{x} \]
und wollen diese auf ganz $\mathbb{R}$ fortsetzen, das heißt Wir suchen ein $\tilde f: \mathbb{R} \to \mathbb{R}$
mit $\tilde f\mid \mathbb{R}\setminus \{0\} = f$ und einen Wert $\tilde f(0) \in \mathbb{R}$

Allgemeiner überprüft man für Funktionen $f: D \subseteq \mathbb{K} \to \mathbb{K}$ die
Fortsetzbarkeit auf den Abschluss $\bar D \subseteq \mathbb{K}$, wobei
\begin{align*}
\bar D &= \{x \in\mathbb{K} \mid x\in D \vee~\text{oder $x$ ist HP von D}\} \\
&= \{x \in \mathbb{K} \mid \Exists (x_n)_{n\in\mathbb{N}} \subseteq D \wedge x = \lim_{n \to 0} x_n\}
\end{align*}
(analog zur Plenarübung)

\begin{defn}[3.1]
Eine Funktion $f: D\subseteq \mathbb{K} \to \mathbb{K}$ hat im Punkt $x_0 \in \bar D$ einen
Grenzwert $a \in\mathbb{K}$, wenn alle Folgen $(x_n)_{n\int\mathbb{N}} \subseteq D$ gilt:
\[x_n \to x_0 (n \to \infty) \implies f(x_n) \to a (n\to\infty)\]
Wir schreiben kurz: $\lim_{x\to x_0} f(x) = a$
\end{defn}
\begin{remark}
\begin{itemize}
\item Falls der Grenzwert existiert, ist er eindeutig.
\item Ist $T\subseteq D \subseteq \mathbb{R}, T\neq \emptyset, f: D\to\mathbb{R}, x_\in \bar T$, dann
verstehen wir unter
\[\lim_{\substack{x\to x_0 \\ x\in T}} f(x)\]
den Grenzwert $\lim_{x\to x_0} f\mid T$, falls er existiert.
\item Spezialfälle:
\begin{align*}
T_{>} := \{x\in D\mid x > x_0\}: f(x_0^{+}) := \lim_{\substack{x\to x_0 \\ x\in T_{>}}} f(x) = \lim_{x\to x_0^{+}} f(x) \tag{rechtsseitiger Grenzwert} \\
T_{<} := \{x\in D\mid x < x_0\}: f(x_0^{-}) := \lim_{\substack{x\to x_0 \\ x\in T_{<}}} f(x) = \lim_{x\to x_0^{-}} f(x) \tag{linksseitiger Grenzwert}
\end{align*}
\item Existiert $\lim_{x\to x_0} f(x), x_0 \in \bar T \subseteq \bar D$, dann gilt
\[\lim_{x\to x_0 \\ x\in T} f(x) = \lim_{x\to x_0}  f(x)\]
\item Es gelten die üblichen Rechenregeln für Grenzwerte $(x,\cdot, :)$
\end{itemize}
\end{remark}
\begin{ex}[3.2]
\begin{enumerate}
\item $f:\mathbb{R}\setminus\{0\}, x\mapsto \frac{x}{\abs{x}}$
      \[\lim_{x\to 0^+} f(x) = 1 \wedge \lim_{x \to 0^-} f(x) = -1\]
      Also existiert $\lim_{x\to 0} f(x)$ nicht
\item $f:\mathbb{R}\setminus\{0\}, x\mapsto \frac{e^x - 1}{x}$ \\
      Es gilt $\lim_{x\to 0} f(x) = 1$, denn für $\abs{x} \leq 1, x\neq 0$ gilt
\[\abs{f(x) - 1} = \abs{\frac{e^x - 1 - x}{x}} = \abs{\sum_{k = 2}^{\infty} \frac{x^{k - 1}}{k!}} \leq \abs{x} \sum_{k = 2}^{\infty} \frac{\abs{x}^{k - 2}}{k!} \leq \abs{x} \sum_{k = 2}^{\infty} \frac{1}{k!} = \abs{x} \underbrace{(e - 2)}_{> 0}\]
Für Nullfolgen $(x_n)_{n\in\mathbb{N}} \subseteq [-1, 1]\setminus \{0\}$ folgt $\lim_{n \to \infty} f(x_n) = 1$
Das heißt $f$ besitzt eine Fortsetzung
\[\tilde f: \mathbb{R} \to \mathbb{R}, x\mapsto \begin{cases} \frac{e^x - 1}{x} & x \neq 0 \\ 1 & x = 0\end{cases}\]
\end{enumerate}
\end{ex}
\begin{defn}[3.3 Asymptotisches Verhalten]
Sei $\emptyset \neq D \subseteq \mathbb{R}$ nach oben (nach unten) unbeschränkt.
Die Funktion $f: D\to \mathbb{R}$ hat für $x \to + \infty (x \to -\infty)$ einen Grenzwert $a\in \mathbb{R}$, wenn gilt:
\[\Forall \eps > 0 \Exists y\in \mathbb{R}: \abs{f(x) - a} < \eps \Forall x\in D, x > y (x < y)\]
Schreibweise: $\lim_{x \to \infty}f(x) = a$, oder $\lim_{x\to -\infty} f(x) = a$

Sei $x_0 \in \bar D$. Die Funktion $f$ divergiert bestimmt gegen $+\infty (-\infty) :\iff \Forall K\in\mathbb{R}_{+} \Exists \delta > 0: f(x) > K (f(x) < -K) \Forall x\in I_\delta (x_0) \cap (D\setminus\{x_0\})$ \\
   Schreibweise: $f(x) \to +\infty (f(x) \to -\infty)$ für $x\to x_0$
\end{defn}
\begin{ex}[3.4]
\begin{enumerate}
\item $f:\mathbb{R}\setminus\{1\}, x\mapsto \frac{1}{x - 1}$
      \[\lim_{x\to\infty} \frac{1}{x} = 0 = \lim_{x\to -\infty} \frac{1}{x}\]
      wir schreiben kurz $\lim_{\abs{x} \to \infty} \frac{1}{x} = 0$
\item $\Forall k\in\mathbb{N}$ gilt
\[\lim_{x\to\infty} \frac{x^k}{e^x} = 0 = \lim_{x\to -\infty} x^k e^x,~\text{denn}~ e^x = \exp(x) \geq \frac{x^{k + 1}}{(k + 1)!}, x\geq 0 \]
\[\implies \frac{x^k}{e^x} \leq \frac{(k + 1)!}{x} \to 0 (x\to \infty)\]
\[x^k e^x = \frac{(-1)^k \abs{x}^k}{e^{\abs{x}}}, x < 0\]
\end{enumerate}
\end{ex}
\begin{defn}[3.5]
Eine Funktion $f:D \subseteq \mathbb{K} \to \mathbb{K}$ heißt stetig im Punkt $x_0 \in D$, wenn gilt:
Für alle Folgen $x_n \to x_0 (n\to\infty) \implies f(x_n) \to f(x_0) (n \to \infty)$
Andernfalls heißt sie unstetig in $x_0 \in D$.
$f$ heißt stetig (auf ganz $D$), wenn sie in jedem $x_0 \in D$ stetig ist. (\emph{insert Symbolbild hier}) \\
\end{defn}
\begin{lemma}[3.5]
\begin{enumerate}
\item Ist $f: D\to \mathbb{K}$ stetig, dann ist auch $f\mid T$ stetig, $T\subseteq D$
\item Ist $f:D\to \mathbb{K}$ stetig, so auch $\Re(f):D\to\mathbb{R}$, $\Im(f): D\to \mathbb{R}$, $\abs{f}: D\to\mathbb{R}_+$ stetig (auf ganz $D$)
\item Sind $f,g:D\to\mathbb{K}$ stetig, so auch $f + g, f\cdot g: D\to\mathbb{K}$
\item Ist $f:D\to f(D)\subseteq \mathbb{K},g:f(D) \to \mathbb{K}$ stetig in $x_0$, beziehungsweise in $f(x_0) =: y_0$ so auch $f \circ f: D\to\mathbb{K}$ stetig in $x_0 \in D:$
\end{enumerate}
\end{lemma}
\begin{proof}
\begin{enumerate}
\item Siehe Bemerkung zu Grenzwerte
\item Für $z = a + \I b$ gilt $\abs{\abs{a} - \abs{b}} \leq \abs{a - b}$ sowie $\abs{z}^2 = a^2 + b^2 \geq a^2 \geq b^2$
\item Siehe Bemerkung zu Grenzwerte
\item Sei $(x_n)_{n\in\mathbb{N}} \subseteq D$ mit $\lim_{n \to\infty} x_n = x_0$, dann folgt aus Stetigkeit von $f: \lim_{n\to\infty} f(x_n) = f(x_0)$ \\
      $(g\circ f)(x_n) = g(f(x_n)) \rightarrow g(f(x_0)) = (g\circ f)(x_0) (n\to\infty)$
\end{enumerate}
\end{proof}
\begin{lemma}[3.7 $\eps/\delta$ Kriterium]
Eine Funktion $f: D \to \mathbb{K}$ ist in $x_0 \in D$ genau dann stetig, wenn es zu jedem $\eps > 0$ ein $\delta = \delta(\eps) > 0$ gibt, sodass Für alle $x \in D$ gilt:
\[\abs{x - x_0} < \delta \implies \abs{f(x) - f(x_0)} < \eps\]
\end{lemma}
\begin{proof}
$(\impliedby)$: Gilt das $\eps / \delta$ Kriterium, so ist $f$ auch in $x_0$ offensichtlich stetig \\
   $(\implies)$: Sei also $f$ stetig in $x_0$. Angenommen, dass $\eps / \delta$-Kriterium gälte nicht,
das heißt es gibt ein $\eps > 0$, sodass $\Forall \delta > 0$ ein $x\in D$ mit $\abs{x - x_0} < \delta$ und $\abs{f(x) - f(x_0)} \geq \eps$ gibt. Widerspruch zu
\[\lim_{x \to x_0} f(x) = f(x_0)\]
\end{proof}
\begin{korollar}[3.8]
Sei $f: D \to \mathbb{K}$ stetig in $x_0 \in D$ mit $f(x_0) \neq 0$. Dann gibt es ein $\delta > 0$ mit $f(x) \neq 0$.
Dann gibt es ein $\delta > 0$ mit $f(x) \neq 0 \Forall x\in I_\sigma (x_0) \cap D$. Insbesondere ist $\frac{1}{f}: D\to\mathbb{K}$ stetig in $x_0 \in D$
\end{korollar}
\begin{proof}
Setze $\eps := \abs{f(x_0)} > 0$. Dann gibt es ein $\delta > 0$, sodass $\Forall x\in D$ mit $\abs{x - x_0} < \delta$ folgt $\abs{f(x) - f(x_0)} < \eps$ (aus Stetigkeit von $f$),
das heißt für $x \in I_\sigma (x_0) \cap D$ gilt
\[\abs{f(x)} \geq \abs{f(x_0)} - \abs{f(x) - f(x_0)} > \eps - \eps = 0\]
Insbesondere sind Folgen $x_n \to x_0$ wohldefiniert und die Aussage resultiert aus den Rechenregeln für Folgen
\end{proof}
\begin{ex}[3.9]
\mbox{}
\begin{enumerate}
\item $f: \mathbb{R} \to \mathbb{R}, f(x) = x$ ist stetig auf $\mathbb{R}$
\item Konstante Funktionen $f(x) = c \Forall x\in \mathbb{R}$ sind stetig auf $\mathbb{R}$
\item Seien $a_0, \ldots, a_n \in \mathbb{R}, a_n\neq 0$, Dann heißt
\[p: \mathbb{R} \to \mathbb{R}, x\mapsto \sum_{k = 0}^{n} a_k x^k\]
\textbf{Polynom} vom Grad $n\in\mathbb{N}_0$ und ist stetig (wegen 1. und 2. und Lemma 3.6)
\item Seien $p,q$ Polynome, dann heißt \[f: \{x\in\mathbb{R} \mid q(x) \neq 0\} \to \mathbb{R}, x\mapsto \frac{p(x)}{q(x)}\]
      \textbf{rationale Funktion} und ist stetig nach 3. und Korollar 3.8
\item $g:\mathbb{R} \to \mathbb{R}, x\mapsto \sqrt{1 + 3x^2}$ ist stetig nach 3., Lemma 3.6 und Übung 5.1
\item $\exp: \mathbb{R} \to \mathbb{R} \setminus \{0\}, x\mapsto e^x$ ist stetig auf $\mathbb{R}$, denn für $x \neq x_0$ ist
\[e^x = e^{x_0} e^{x - x_0} = e^{x_0} (1 + \underbrace{(x - x_0)}_{\to 0} \frac{e^{x - x_0} - 1}{\underbrace{(x - x_0)}_{1}})\]
(nach Beispiel 3.2)
\item $\displaystyle f(x) = \begin{cases} 1 & x \in\mathbb{Q} \\ 0 & x \in \mathbb{R} \setminus \mathbb{Q} \end{cases}$
\end{enumerate}
\end{ex}
\begin{defn}[3.10 Gleichmäßige Stetigkeit]
Eine Abbildung $f: D \to \mathbb{K}$ heißt \textbf{gleichmäßig stetig} auf $D$, wenn $\Forall \eps > 0 \Exists \delta = \delta(\eps) < 0: \Forall x,y \in D: \abs{x - y} < \delta \implies \abs{f(x) - f(y)} < \eps$
\end{defn}
\begin{remark}
Gleichmäßige Stetigkeit heißt, dass die $\delta$ gleichmäßig für alle Punkte $x\in D$ gewählt werden kann.
\end{remark}
\begin{ex}[3.11]
\[f: \mathbb{R} \setminus \{0\} \to \mathbb{R}, x \mapsto \frac{1}{x}\]
\begin{enumerate}
\item $f$ ist gleichmäßig stetig auf $A = \mathbb{R} \setminus (-a, a), a > 0$
\item $f$ ist \textbf{nicht} gleichmäßig stetig auf $\mathbb{R} \setminus \{0\}$
\end{enumerate}
\end{ex}
\begin{proof}
\[\abs{f(x) - f(y)} = \abs{\frac{1}{x} - \frac{1}{y}} = \frac{1}{\abs{xy}} \abs{x - y}\]
also $\abs{f(x) - f(y)} < \eps \iff \abs{x - y} < \abs{xy} \eps$
\begin{enumerate}
\item Für $x,y \in \mathbb{R} \setminus (-a, a)$ gilt $\abs{xy} \geq a^2$, also $\abs{x - y} < \eps a^2 := \delta \implies \abs{x - y} < \eps \abs{xy}$.
Daher $\Forall \eps > 0 \Forall x,y\in A: \abs{x - y} < \delta:= \eps a^2 \implies \abs{f(x) - f(y)} < \eps$
\item Dagegen können wir $\Forall \delta > 0, x, y \in \mathbb{R} \setminus \{0\}$ finden wir
$\abs{x - y} < \delta$, aber $\abs{f(x) - f(y)} \geq 1 \iff \abs{x - y} \geq \abs{xy}$ \\
      Sei $\delta > 0$. Wähle $n\in \mathbb{N}$, sodass $\frac{\delta}{n} < 1$. Nun gilt für
\begin{align*}
\abs{x - y} &= \frac{\delta}{2n} \\
\abs{xy} &< (\abs{x - y} + \abs{x}) \abs{x} \\
\intertext{für $\abs{x} < \frac{\delta}{2n}$}
&= (\frac{\delta}{2n} + \abs{x}) \abs{x} < \frac{\delta^2}{2n^2} \\
&= \frac{\delta}{n} \abs{x - y} \leq \abs{x - y}, ~\text{da}~ \frac{\delta}{n} \leq 1
\end{align*}
\end{enumerate}
\end{proof}
\begin{defn}[3.12 Lipschitz Stetigkeit]
Eine Funktion $f:D\to\mathbb{K}$ heißt Lipschitz stetig (kurz L-stetig) auf $D$, wenn
$\Exists L > 0$ (so genannte Lipschitz Konstante), sodass
\[f(x) - f(y) \leq L \abs{x - y} \Forall x,y \in D\]
\end{defn}
\begin{remark}
Menge von stetigen Funktionen $\supset$ Menge von gleichmäßig stetigen Funktionen $\supset$
Menge von Lipschitz-stetigen Funktionen
\end{remark}
\begin{defn}[3.13 Satz von der gleichmäßigen Stetigkeit, Satz von Heine für folgenkompakte metrische Räume]
Eine auf einer beschränkten, abgeschlossenen (das heißt kompakten) Teilmenge $D \subseteq \mathbb{K}$
stetige Funktion ist gleichmäßig stetig.
\end{defn}
\begin{proof}
Angenommen $f$ ist nicht gleichmäßig stetig. Dann gibt es ein $\eps > 0$, sodass $\Forall n\in \mathbb{N}$ Punkte $x_n, y_n \in D$ existieren mit
$\abs{x_n - y_n} < \frac{1}{n}, \abs{f(x_n) - f(y_n)} \geq \eps$ \\
   Nach dem Satz von Bolzano-Weierstraß besitzt die beschränkte Folge $(x_n)_{n\in\mathbb{N}}$ eine
konvergente Teilfolge $x_{n_k} \to x \in D$. Wegen $\abs{x_n - y_n}  < \frac{1}{n}$ ist auch $\lim_{k\to\infty} y_{n_k} = y = x$
Aus der Stetigkeit von $f$ folgt, dass
\[\abs{f(x_{n_k}) - f(y_{n_k})} \to \abs{f(x) - f(y)} = 0 \lightning\]
\end{proof}
\begin{remark}
\mbox{}
\begin{enumerate}
\item Wichtigkeit von Annahmen
\begin{itemize}
\item Abgeschlossenheit: $f(x) = x^{-1}$ für $x\in [-A, A] \setminus \{0\}$ Stetig, aber nicht gleichmäßig Stetig
\item Beschränktheit: $f(x) =x^2$ für $x\in\mathbb{R}$ ist stetig, aber nicht gleichmäßig stetig auf $\mathbb{R}$ \\
        für $x = m$ und $y = x + \frac{1}{n}$ gilt
\[\abs{x - y} \to 0, ~\text{aber}~ \abs{f(x) - f(y)} = \abs{x^2 - y^2} = \abs{(x - y)(x + y)} = 2 + \frac{1}{n} \to 2\]
\end{itemize}
\item Lipschitz-Stetigkeit von $f(x) = x^2$ \\
      \[\abs{f(x) - f(y)} = \abs{(x - y)(x + y)} \leq L \abs{x y}\]
wenn $D$ beschränkt $D = [-A, A] \implies \abs{x + y} \leq 2A \implies L = 2A \implies$ Lipschitz-Stetigkeit,
aber wenn $D = R \implies$ gibt keine $L < \infty$
\item Lipschitz-Stetigkeit impliziert gleichmäßige Stetigkeit, aber nicht umgekehrt. Zum Beispiel: $f(x) = \sqrt{x}, x \in [0, A]$ ist gleichmäßig stetig nach Satz 3.13, aber nicht Lipschitz-stetig in $0$.
\begin{align*}
\abs{\sqrt{x} - \sqrt{y}} \leq L \abs{x - y} \\
\abs{\frac{y - x}{\sqrt{x} - \sqrt{y}}} > n \abs{x - y}
\intertext{$\implies \not\Exists L > 0$}
\end{align*}
\end{enumerate}
\end{remark}
\begin{remark}
Stetigkeit kann interpretiert werden als "lokale Approximation" durch Konstanten, das heißt Funktion $f$ nach der Stelle $x_0$ durch eine Konstante $f(x_0)$ approximiert werden
kann und die Fehler der Approximation $\abs{f(x) - f(x_0)} < \eps$
\end{remark}
\subsection{Eigenschaften stetiger Funktionen}
\label{sec-6-2}
\begin{thm}[3.14 Satz von Beschränktheit]
Eine auf einer beschränkten, abgeschlossenen Teilmenge $D\subset\mathbb{K}$ stetige Funktion $f:D\to\mathbb{K}$ ist beschränkt,
$\Exists K > 0: \sup_{x\in D} \abs{f(x)} \leq K$
\end{thm}
\begin{proof}
Angenommen das eine stetige $f(x)$ nicht beschränkt auf $D$ ist.
Dann gibt zu jedem $n\in \mathbb{N}$ ein $x_n \in D$ mit $\abs{f(x_n)} >n$ \\
   Die Folge $(x_n)_{n\in\mathbb{N}}$ ist beschränkt (da $D$ beschränkt). Nach dem B.-W. Satz $\Exists x_{m_k} \to x \in D$ (weil $D$ abgeschlossen ist).
Aus der Stetigkeit von $f$
\[\abs{f(n_k)} \xrightarrow{x\to\infty} \abs{f(x)} < \infty \lightning\]
Widerspruch zur Annahme $f(x_m) \to \infty$
\end{proof}
\begin{thm}[3.15 Satz von Extremum]
Eine auf einer beschränkten, abgeschlossenen Teilmenge $D\subseteq\mathbb{K}$ stetige reellwertigen Funktion $f:D\to\mathbb{K}$ besitzt
dort ein Maximum und ein Minimum, das heißt:
\[\Exists x_{min}, x_{max} \in D: \sup_{x\in D} f(x) = f(x_{max}) \wedge \inf_{x\in D} f(x) = f(x_{min})\]
\end{thm}
\begin{proof}
\[\Exists K < \infty: K = \sup_{x\in D} < \infty\]
$\Exists$ eine Folge $(x_n)_{n\in\mathbb{N}} \in D: f(x_n) \xrightarrow{n\to\infty} K$.
Die Folge $(x_n)_{n\in\mathbb{N}}$ ist beschränkt und in $D$ abgeschlossen
\[\implies \Exists (x_{n_k})_{k\in\mathbb{K}} \in D: x_{n_k} \to x\in D\]
Aus $f(x_{n_k}) \xrightarrow{k\to\infty} f(x) \implies f(x) = K$ \\
   Analog für untere Grenze.
\end{proof}
\begin{defn}[3.16 Zwischenwertsatz]
Sei $f:[a,b] \to \mathbb{R}$ eine reelle stetige Funktion. Dann
gibt es zu jeder $y\in [f(a), f(b)]$ ein $x\in[a,b]$ mit $f(c) = y$
\end{defn}
\begin{proof}
Betrachte die (nicht leere, beschränkte) Menge
\[A = \{x\in [a,b] \mid f(x) \leq y\}\]
Entweder ist dann $\sup A = b$ (und dann $c = b$) oder es gibt per Definition ein $x\in [a,b]$ mit $x > c \implies x \not\in A \implies f(x) > y$
In beiden Fällen folgt $f(c) \leq y$
\begin{itemize}
\item Falls $c = b \implies y = f(c) = f(b) \implies f(c) \geq y$
\item Falls $c < b \implies$ Aus Stetigkeit von $f$, eine monoton fallende Folge von Punkten aus $A$ existiert, welche gegen $\sup A$ konvergiert
\end{itemize}
Aus Stetigkeit und Definition von $A$ folgt $f(c) \leq y$. Beide zusammen genommen ergibt $f(c) = y$
\end{proof}
\begin{remark}
Die Eigenschaften von stetigen Funktionen lassen sich zusammen formulieren:
Für eine auf einem abgeschlossenen, beschränkten Intervall definierte stetige Funktion ist der Bildbereich wieder ein abgeschlossenes Intervall
\end{remark}
\begin{lemma}[3.17 Treppenapproximation]
Jede auf einem beschränkten, abgeschlossenen Intervall $[a, b]$ definierte $f:[a,b] \to \mathbb{R}$ lässt
sich beliebig gut durch Treppenfunktion einschließen. das heißt
\[\Forall \eps > 0 \Exists ~\text{Treppenfunktion}~ \bar \phi_\eps, \ubar \phi_\eps\]
ohne Beschränkung der Allgemeinheit zu selben endlichen Zerlegung von $[a, b]$
mit den Eigenschaften $\Forall x\in [a,b]$
\begin{itemize}
\item $\ubar \phi_\eps \leq f(x) \leq \bar \phi_\eps(x)$
\item $\abs{\ubar\phi_\eps(x) - \bar\phi_\eps(x)} < \eps$
\end{itemize}
Zerlegung: ist mit Teilpunkten $a\leq x_k \leq b, k = 0,\ldots, N < \infty$ (endliche Zerlegung) $(a = x_0 \leq x_1 \leq \ldots \leq x_N = b)$ \\
   Treppenfunktion ist konstant auf Intervalle $\left[x_1, x_{i + 1}), 0 \leq 1 \leq N - 1$
\end{lemma}
\begin{proof}
Aus dem Satz von gleichmäßiger Stetigkeit ist $f$ auf $[a,b]$ gleichmäßig Stetig
\[\implies \Forall \eps > 0 \Exists \delta_\eps > 0: \Forall x\in[a,b], \abs{x - y} < \delta_\eps \implies \abs{f(x) - f(y)} < \frac{\eps}{2}\]
Sei $n\in\mathbb{N}$ so groß, dass $\frac{a - b}{n} < \delta\eps$. Mit den Teilpunkten
\[x_k .= a+ k \frac{b - a}{n}, k = 0, \ldots, n\]
erhalten wir eine äquidistante Zerlegung von $[a, b]$
\[a = x_0 < x_1 < \ldots < x_n = b, \abs{x_k - x_{k - a}} < \delta_\eps\]
Dann definieren wir
\[\bar \phi_\eps(x) := \sup \{f(x) \mid x_{k - 1} \leq x < x_k\}\]
\[\ubar \phi_\eps(x) := \inf \{f(x) \mid x_{k - 1} \leq x < x_k \}\]
Nach Konstruktion gemäß $\ubar \phi_\eps (x) \leq f(x) \leq \bar \phi_\eps(x) \Forall x\in [a,b]$ \\
   Nach dem Satz von Extremum $\Forall [x_1, \ldots, x_k] \Exists \bar \xi_k, \ubar\xi_k$ sodass
\begin{align*}
f(\bar\xi_k) = \sup \{f(x) \mid x_{k - 1} \leq x \leq x_k\}
f(\ubar\xi_k) = \inf \{f(x) \mid x_{k - 1} \leq x \leq x_k\}
\end{align*}
Nach Wahlfreiheit von $\delta_\eps$ gilt
\[\abs{\ubar\phi_\eps (x) - \bar\phi_\eps(x)} = \abs{f(\ubar\xi_k) - f(\bar\xi_k)} \leq \abs{f(\ubar\xi_k) - f(x)} + \abs{f(x) - f(\bar\xi_k)} \underarrow[<]{aus gleichmäßiger Stetigkeit} \frac{1}{2} \eps + \frac{1}{2} \eps = \eps\]
\end{proof}
\subsection{Konvergenz von Funktionen}
\label{sec-6-3}
\begin{defn}[3.18]
Seien $f_n: D\to\mathbb{R}, n \in\mathbb{N}$ Funktionen mit einem gemeinsamen Definitionsbereich $D\subseteq \mathbb{R}$.
Wir nennen die folge $(f_n)_{n\in\mathbb{N}}$ \textbf{punktweise} konvergenz gegener eine Funktion $f:D\to\mathbb{R}$,
wenn für jedes $x\in D$ gilt $f_n(x) \xrightarrow{n\to\infty} f(x)$
\end{defn}
\begin{ex}[3.19]
\mbox{}
\begin{enumerate}
\item \[f_n(x) = \sum_{k = 0}^{n} \frac{x^k}{k!} \xrightarrow{n\to\infty} \sum_{k = 0}^{\infty} \frac{x^k}{k!} = e^x\]
      Hier ist $f_n(x)$ stetig und $f(x)$ stetig.
\item $f_n(x) = 1 - x^n, x\in[0,1] \subseteq \mathbb{R}$
      \[\underarrow[f_n(x)]{stetig} \xrightarrow{n\to\infty} \underarrow[f(x)]{nicht stetig} := \begin{cases} 1 & 0 \leq x \leq 1 \\ 0 & x = 1 \end{cases}\]
\end{enumerate}
\end{ex}
\begin{defn}[3.19 Gleichmäßige Konvergenz]
Eine Folge von Funktionen $f_n:D\to\mathbb{R}, n\in\mathbb{N}$ heißt \textbf{gleichmäßig konvergent} gegen eine Funktion $f:D\to\mathbb{R}$, wenn
\[\Forall \eps > 0 \Exists n_\eps \in \mathbb{N}: n\geq n_\eps \implies \abs{f_n(x) - f(x)} < \eps \Forall x\in D\]
\end{defn}
\begin{thm}[3.20 Satz von der gleichmäßigen Konvergenz]
Konvergiert eine Folge stetiger Funkitonen $f_n: D\to\mathbb{R}, n\in\mathbb{N}$ gleichmäßig gegen $f:D\to\mathbb{R}$,
so ist auch die Grenzfunktion $f$ stetig.
\end{thm}
\begin{proof}
Seien $x_0 \in D$ und $\eps > 0$ gegeben. Zu zeigen:
\[\Exists \delta_\eps > 0: \Forall x \in D \abs{x - x_0} < \delta_eps \implies \abs{f(x) - f(x_0)} < \eps\]
Wegen der gleichmäßigen Konvergenz von $(f_n)_{n\in\mathbb{N}}$:
\[\Exists x\in \mathbb{N} \Forall x\in D: \abs{f_n(x) - f(x)} < \frac{1}{3} \eps\]
Aus Stetigkeit von $f_n$:
\[\Exists \delta_\eps > 0 Forall x\in D: \abs{x - x_0} < \delta_\eps \implies \abs{f_n(x) - f_n(x_0)} < \frac{1}{3} \eps\]
\[\implies \Forall x\in D \abs{f(x) - f(x_0)} \leq \underbrace{\abs{f(x) - f_n(x)}}_{< \frac{1}{3}} + \underbrace{\abs{f_n(x) - f_n(x_0)}}_{< \frac{1}{3}} + \underbrace{\abs{f_n(x_0) - f(x_0)}}_{< \frac{1}{3}} < \eps\]
das heißt $f$ ist stetig.
\end{proof}
\subsection{Reellwertige stetige Funktionen}
\label{sec-6-4}
\begin{defn}[3.21]
\[C(\mathbb{K}) := \{f:\mathbb{K} \to \mathbb{R} \mid f~\text{ist stetig auf}~\mathbb{K}\}\]
ist der Raum der stetigen reellwertigen Funktionen auf $\mathbb{K}$
\end{defn}
\begin{remark}
Seien $f,g \in C(\mathbb{K}), \lambda \in \mathbb{R}$. Dann ist auch $f + g, f \cdot g, \lambda f$ wieder eine Funktion aus $C(\mathbb{K})$.
$C(\mathbb{K})$ bildet dann einen Ring.
\end{remark}
\begin{defn}[3.22]
Seien $f,g: \mathbb{K} \to \mathbb{R}$.
\begin{align*}
\max_{x\in\mathbb{K}} (f,g)(x) &:= \max_{x\in\mathbb{K}}(f(x), g(x))
\min_{x\in\mathbb{K}} (f,g)(x) &:= \min_{x\in\mathbb{K}}(f(x), g(x))
\end{align*}
\end{defn}
\begin{thm}[3.23]
$\max(f,g)$ und $\min(f,g)$ sind in $C(\mathbb{K})$ für $f,g\in C(\mathbb{K})$
\end{thm}
\begin{proof}
Es genügt, dass mit $f$ auch $\abs{f}$ (als Komposition stetige Abbildung) stetig ist, denn
\begin{align*}
\max(f,g) &= \frac{1}{2}(f + g) + \frac{1}{2}\abs{f - g} \\
\min(f,g) &= -\max(-f, -g) \tag*{\qedhere}
\end{align*}
\end{proof}
Wir betrachten jetzt $C(\underbrace{[a,b]}_{\mathbb{K}})$ und definieren
\[\norm{f}_\infty := \max_{x\in [a,b]} \abs{f(x)}\]
\begin{defn}[3.24]
Sei $\mathbb{K}$ ein Körper (mit dem Betrag $\abs{\,}$), Sei $V$ ein Vektorraum über $\mathbb{K}$.
\[\norm{\,}: V \to\mathbb{R}\]
heißt eine \textbf{Norm} auch $V \iff$:
\begin{itemize}
\item (N1) $\Forall x\in V: \norm{x} \geq 0 \wedge (\norm{x} = 0 \iff x = 0)$
\item (N2) $\Forall x\in V: \alpha\in\mathbb{K} \norm{\alpha x} = \abs{\alpha}\norm{x}$
\item (N3) $\Forall x,y\in V: \norm{x + y} \leq \norm{x} + \norm{y}$
\end{itemize}
$(V,\norm{\,})$ heißt normierter Vektorraum. \\
   $C([a,b])$ ist ein Vektorraum. Die Normeigenschaften von $\norm{\,}_\infty$ als Abbildung von $C([a,b])$ nach $\left[0,\infty)$
folgt direkt aus den Eigenschaften des Absolutbetrags
\begin{align*}
\norm{f}_\infty &\implies f(x) = 0 \Forall x\in [a,b] \tag{Definitheit} \\
\norm{\alpha f} &= \abs{\alpha} \norm{f}_\infty, \alpha \in \mathbb{R} \tag{Homogenität} \\
\norm{f + g}_\infty &\leq \norm{f}_\infty + \norm(g)_\infty \tag{Dreiecksungleichung}
\end{align*}
Wir definieren sogenannte Normkonvergenz
\[f_n \xrightarrow{n\to\infty} f~\text{in Norm} \iff \norm{f - f_n}_\infty \xrightarrow{n\to\infty} 0\]
Für $\norm{\,}_\infty$ Konvergenz in Norm ist die gleichmäßige Konvergenz.
\end{defn}
\begin{lemma}[3.25]
Für eine Funktionenfolge $(f_n)_{n\in\mathbb{N}} \in C([a,b])$ ist die gleichmäßige Konvergenz gegen eine Grenzfunktion. $f:[a,b] \to \mathbb{R}$
gleichbedeutend mit $\norm{f_n - f}_\infty \xrightarrow{n\to\infty} 0$
\end{lemma}
\begin{proof}
aus Definition.
\end{proof}
\begin{defn}[3.26 Cauchy Folge von Funktionen]
Eine Folge $(f_n)_{n\in\mathbb{N}} \in C([a,b])$ heißt Cauchy-Folge, wenn
\[\Forall \eps \Exists n_\eps \in \mathbb{N}: n,m \geq n_\eps \implies \norm{f_n - f_m}_\infty < \eps\]
\end{defn}
\begin{lemma}[3.27]
Eine Folge $(f_n)_{n\in\mathbb{N}} \in C([a,b])$ welche gegen eine Grenzfunktion $f\in C([a,b])$ konvergiert ist Cauchy-Folge.
\end{lemma}
\begin{proof}
analog wie Beweis für Zahlenfolgen
\end{proof}
\begin{thm}[3.28 Satz von der Vollständigkeit]
$(C([a,b],\norm{\,}_\infty))$ ist vollständig bezüglich der gleichmäßigen Konvergenz, das heißt jede Cauchy-Folge $(f_n)_{n\in\mathbb{N}} \in C([a,b])$ besitzt ein Limes $f \in C([a,b])$
\end{thm}
\begin{proof}
Sei $(f_n)_{n\in\mathbb{N}} \in C([a,b])$ eine Cauchy-Folge. Dann ist für jedes feste $x\in [a,b]~(f_n(x))_{n\in\mathbb{N}}$ eine Cauchy-Folge von Zahlen und besitzt einen
(eindeutig bestimmten) Limes $f(x) \in \mathbb{R}$. \\
   Wir wollen zeigen, dass diese Konvergenz gleichmäßig ist. Angenommen $f_n \to f$ nicht gleichmäßig \\
   $\implies \Exists \eps > 0$ und $\Forall n\in\mathbb{N}$ einen Punkt $x_n\in [a,b]$ sodass $\abs{f_n(x_n) - f(x_n)} > \eps$.
Die Punktfolge $(x_n)_{n\in\mathbb{N}}$ besitzt eine konvergente Teilfolge (nach Bolzano-Weierstrass Satz, $[a,b]$ beschränkt und abgeschlossen).
Wegen der Cauchy-Folgen Eigenschaft
\[\Exists n_\eps\in\mathbb{N}: m\geq n_\eps \implies \norm{f_{n_\eps} - f_n}_\infty < \frac{1}{2} \eps\]
Wegen der Konvergenz $f_m(x_{n_\eps}) \xrightarrow{n\to\infty} f(x_{n_\eps})$:
\[\Exists m_\eps \geq n_\eps: \abs{f_{m_\eps}(x_{n_\eps}) - f(x_{n_\eps})} < \frac{1}{2} \eps\]
\[\implies \abs{f_{n_\eps} - f(x_{n_\eps})} \leq \abs{f_{n_\eps}(x_{n_\eps}) - f_{m_\eps}(x_{n_\eps})} + \abs{f_{m_\eps}(x_{n_\eps}) - f(x_{n_\eps})} < \eps\]
$\implies f_n\to f$ gleichmäßig und im Wiederspruch zur Annahme. $\implies f\in C([a,b])$ (aus Satz 3.20)
\end{proof}
\begin{remark}
Vollständige normierte Räume werden Banach Räume genannt.
$C([a,b])$ ist also ein Banach Raum.
\end{remark}
\begin{thm}[3.29 Satz von Arzela-Ascoli]
Sei $(f_n)_{n\in\mathbb{N}}$ eine Folge von Funktionen in $C([a,b])$ welche \textbf{gleichmäßig beschränkt} und \textbf{gleichmäßig stetig} sind.
das heißt
\begin{enumerate}
\item $\displaystyle\sup_{n\in\mathbb{N}} \norm{f_n}_\infty < \infty$
\item $\displaystyle\Forall \eps > 0 \Exists \delta_\eps > 0 \Forall n\in\mathbb{N}: \max_{\substack{x,y\in [a,b] \\ \abs{x - a} \leq \delta_\eps}} \abs{f_n(x) - f_n(y)} < \eps$
\end{enumerate}
Dann existiert eine Teilfolge $(f_{n_k})_{k\in\mathbb{N}}$ welche gegen ein $f\in C([a,b])$ konvergiert, das heißt
\[\norm{f_{n_k} - f}_\infty \xrightarrow{k\to\infty} 0\]
Annahmen: $f_n \in C([a,b])$,
\begin{itemize}
\item gleichmäßig beschränkt: $\sup_{n\in\mathbb{N}} \norm{f_n}_\infty < \infty$
\item gleichmäßig stetig:
\[\Forall \eps \Exists \delta_\eps > 0 \Forall n\in\mathbb{N \m} \max_{\substack{x,y\in[a,b] \\ \abs{x - y} \leq \delta_\eps}} \abs{f_n(x) - f_n(y)} < \eps\]
\end{itemize}
Aussage: $\Exists$ eine Teilfolge $(f_{n_k})_{k\in\mathbb{N}}$, sodass $f_{n_k} \xrightarrow{k\to\infty} f\in C([a,b])$
\end{thm}
\begin{proof}
Sei $(r_k)_{k\in\mathbb{N}}$ eine Folge der rationalen Punkte in $[a,b]$. Für jedes $r_k$, nach Vorraussetzung $\sup_{n\in\mathbb{N}} \abs{f_n(r_k)} < \infty$
\begin{equation*}
\begin{pmatrix}
&\RED {f_{n_1^{(1)}}},& f_{n_2^{(1)}},& \ldots,& f_{n_k^{(1)}} & \text{konvergiert in $r_1$} \\
&f_{n_1^{(2)}},& \RED{f_{n_2^{(2)}}},& \ldots,& f_{n_k^{(2)}} & \text{konvergiert auch in $r_2$} \\
&f_{n_1^{(3)}},& f_{n_2^{(3)}},& \ldots,& f_{n_k^{(3)}} & \text{konvergiert auch in $r_2$} \\
&f_{n_1^{(k)}},& f_{n_2^{(k)}},& \ldots,& \RED{f_{n_k^{(k)}}} & \text{konvergiert auch in $r_k$} \\
\end{pmatrix}
\end{equation*}
\RED{Diagonalfolge} \\
   Nach sukzessiver Anwendung des Bolzano-Weierstrass Satz bekommen wir eine Folge von Teilfolgen. Die Folgen $(f_{n_j^{(k)}}(r_k))_{j\in\mathbb{N}}$ sind konvergent,
$(n_j^{(k + 1)})_{j\in\mathbb{N}}$ ist Teilfolge von $(n_j^{(k)})_{j\in\mathbb{N}}$. $(f_{n_j^{(k)}}(r_l))_{j\in\mathbb{N}}$ ist konvergent für $l = 1,\ldots, k$.
\end{proof}
% Emacs 25.1.1 (Org mode 8.2.10)
\end{document}