% Created 2016-12-18 So 16:33
\documentclass[a4paper]{scrartcl}
\usepackage[utf8]{inputenc}
\usepackage[T1]{fontenc}
\usepackage{fixltx2e}
\usepackage{graphicx}
\usepackage{longtable}
\usepackage{float}
\usepackage{wrapfig}
\usepackage{rotating}
\usepackage[normalem]{ulem}
\usepackage{amsmath}
\usepackage{textcomp}
\usepackage{marvosym}
\usepackage{wasysym}
\usepackage{amssymb}
\usepackage{hyperref}
\tolerance=1000
\usepackage{siunitx}
\usepackage{fontspec}
\sisetup{load-configurations = abbrevations}
\newcommand{\estimates}{\overset{\scriptscriptstyle\wedge}{=}}
\usepackage{mathtools}
\DeclarePairedDelimiter\abs{\lvert}{\rvert}%
\DeclarePairedDelimiter\norm{\lVert}{\rVert}%
\DeclareMathOperator{\Exists}{\exists}
\DeclareMathOperator{\Forall}{\forall}
\author{Robin Heinemann}
\date{\today}
\title{Mathematischer Vorkurs}
\hypersetup{
  pdfkeywords={},
  pdfsubject={},
  pdfcreator={Emacs 25.1.1 (Org mode 8.2.10)}}
\begin{document}

\maketitle
\tableofcontents


\section{Messwert und Maßeinheit}
\label{sec-1}
Zu jeder phys. Größe gehören \uline{Messwert} und \uline{Maßeinheit}, d.h. Zahlewert $\cdot$ Einheit

\subsection{Beispiel}
\label{sec-1-1}
Geschw. $v = \si{\kilo\meter\per\second}$

\subsection{Bezeichungen}
\label{sec-1-2}
\begin{center}
\begin{tabular}{ll}
Abkürzung & Bedeutung\\
\hline
t & time\\
m & mass\\
v & velocity\\
a & acceleration\\
F & Force\\
E & Energy\\
T & Temperature\\
p & momentum\\
I & electric current\\
V & potential\\
\end{tabular}
\end{center}

Wenn das lateinische  Alphabet nicht ausreicht: griechische Buchstaben
\[\alpha, \beta, \gamma, \delta, \Delta, \Gamma, \epsilon, \zeta, \eta, \Theta, \kappa, \lambda, \mu, \nu, \Xi, \pi, \rho, \sigma, \tau, \phi, \chi, \psi, \omega, \Omega\]

\subsection{Maßeinheiten}
\label{sec-1-3}
Maßeinheiten werden über Maßstäbe definiert.

\subsubsection{Bespiel:}
\label{sec-1-3-1}
\SI{1}{\meter} = Strecke, die das Licht in $\frac{1}{299792458}\si{\second}$ zurücklegt.

\subsubsection{SI-Einheiten}
\label{sec-1-3-2}
Internationaler Standart (außer die bösen Amerikaner :D)

\begin{center}
\begin{tabular}{lll}
Größe & Einheit & Symbol\\
\hline
Länge & Meter & \si{\meter}\\
Zeit & Sekunden & \si{\second}\\
Masse & Kilogramm & \si{\kilogram}\\
elektrischer Strom & Ampere & \si{\ampere}\\
Temperatur & Kelvin & \si{\kelvin}\\
Lichstärke & Candela & \si{\candela}\\
ebener Winkel & Radiant & \si{\radian}\\
Raumwinkel & Steradiant & \si{\steradian}\\
Stoffmenge & Mol & \si{\mol}\\
\end{tabular}
\end{center}

\paragraph{Radiant}
\label{sec-1-3-2-1}
Kreisumfang $U = 2\pi r$
Bogenmaß $b = \phi r$

Umrechung in Winkelgrad
\[\SI{2\pi}{\radian} \estimates \SI{360}{\degree}\]

\[\frac{Winkel in Radiant}{2\pi} = \frac{Winkel in Grad}{360}\]

\paragraph{Steradiant}
\label{sec-1-3-2-2}
\[\Omega = \frac{A}{r^2}\]

\paragraph{Abgeleitete Einheiten}
\label{sec-1-3-2-3}
\begin{center}
\begin{tabular}{llll}
Gröpe & Einheit & Symbol & Equivalent\\
\hline
Frequenz & Hertz & \si{\hertz} & \si{1\per\second}\\
Kraft & Newton & \si{\newton} & \si{\kilogram\meter\per\square\second}\\
Energie & Joule & \si{\joule} & \si{\newton\meter}\\
Leistung & Watt & \si{\watt} & \si{\joule\per\second}\\
Druck & Pascal & \si{\pascal} & \si{\newton\per\square\meter}\\
elektrischer Ladung & Coulomb & \si{\coulomb} & \si{\ampere\second}\\
elektrisches Potenzal & Volt & \si{\volt} & \si{\joule\per\coulomb}\\
elektrischer Wiederstand & Ohm & \si{\ohm} & \si{\volt\per\ampere}\\
Kapazität & Farad & \si{\farad} & \si{\coulomb\per\newton}\\
magn. Fluss & Weber & \si{\weber} & \si{\volt\per\second}\\
\end{tabular}
\end{center}

\paragraph{Prefix / Größenordungen}
\label{sec-1-3-2-4}
\begin{center}
\begin{tabular}{lrl}
Prefix & $\log$\{10\} & Abkürzung\\
\hline
Dezi & -1 & d\\
Zenti & -2 & c\\
Milli & -3 & m\\
Mikro & -6 & $\mu$\\
Nano & -9 & n\\
Piko & -12 & p\\
Femto & -15 & f\\
Atto & -18 & a\\
Zepta & -21 & z\\
Yokto & -24 & y\\
Deka & 1 & D\\
Hekto & 2 & h\\
Kilo & 3 & k\\
Mega & 6 & \si{\mega}\\
Giga & 9 & G\\
Tera & 12 & T\\
Peta & 15 & P\\
Exa & 18 & E\\
Zetta & 21 & Z\\
Yotta & 24 & Y\\
\end{tabular}
\end{center}
\subsection{Natürliches Einheitensystem der Teilchenphysik}
\label{sec-1-4}
\subsubsection{Grundlage}
\label{sec-1-4-1}
\[\SI{2.9979e8}{\meter\per\second}\]
\[\si{\planckbar} = \frac{h}{2\pi} = \SI{6.5822e-22}{\MeV\second}\]
betrachte $\frac{\si{\planckbar} c}{\si{\MeV\meter}}=\num{197.33e-15}$
\subsubsection{natürliches Einheitensystem}
\label{sec-1-4-2}
$h = c = 1$
In diesem Fall ist \$\si{1\per\mega\electronvolt} = \SI{197.44}{\femto\meter}
In diesem Einheitensystem ist die Einheit von $[Energie] = [Masse] = [L\ddot{a}nge]^-1 = [Zeit]^-1$

\subsection{Endliche Messgenauigkeit}
\label{sec-1-5}
z.B. Plancksches Wirkungsquantum
\[\si{\planckbar} = \SI{1.05457168(18)e-34}{\joule\second}\]
Das bedeutet, dass der Wert von $\si{\planckbar}$ mit einer Wahrscheinlichkeit von $\SI{68}{\percent}$ zwischen den beiden Schranken liegt \[\SI{1.05457150e-34}{\joule\second} \leq \si{\planckbar} \leq \SI{1.05457186e-34}{\joule\second}\]

\section{Zeichen und Zahlen}
\label{sec-2}
\subsection{Symbole}
\label{sec-2-1}
\begin{center}
\begin{tabular}{ll}
Zeichen & Bedeutung\\
\hline
$+$ & plus\\
$\cdot$ & mal\\
$=$ & gleich\\
$<$ & ist kleiner als\\
$>$ & ist größer als\\
$\angle$ & Windel zwischen\\
$-$ & minus\\
$/$ & geteilt\\
$\neq$ & ungleich\\
$\leq$ & kleiner gleich\\
$\geq$ & größer gleich\\
$\simeq$ & ungefähr gleich\\
$\pm$ & plus oder minus\\
$\perp$ & steht senkrecht auf\\
$\equiv$ & ist identisch gleich\\
$\ll$ & ist klein gegen\\
$\gg$ & ist groß gegen\\
$\infty$ & größer als jede Zahl\\
$\to \infty$ & eine Größe wächst über alle Grenzen $\backslash$ Limes\\
$\sum$ & Summe\\
$\in$ & Element von\\
$\subseteq$ & ist Untermenge von oder gleich\\
$\cup$ & Vereiningungsmenge\\
$\exists$ & es existiert ein\\
$\implies$ & daraus folgt, ist hinreichende Bedingung für\\
$\impliedby$ & gilt wenn, ist notwendige Bedingung für\\
$\exists!$ & es existiert genau ein\\
$\notin$ & kein Element von\\
$:=$ & ist definiert durch\\
$\emptyset$ & Nullmenge\\
$\Forall$ & für alle\\
\end{tabular}
\end{center}

\subsubsection{Summenzeichn}
\label{sec-2-1-1}
\paragraph{Beispiel}
\label{sec-2-1-1-1}
\begin{enumerate}
\item \[\sum_{n=1}^3a_n=a_1 + a_2 + a_3\]
\item Summe der ersten $m$ natürlichen Zahlen
\end{enumerate}
\[\sum_{n=1}^{m}n = 1 + 2 + \ldots + \left(m -1\right) + m = \frac{m (m + 1)}{2}\]
\begin{enumerate}
\item Summe der ersten $m$ Quadrate der natürlichen Zahlen
\end{enumerate}
\[\sum_{n=1}^m n^2 = 1 + 4 + \ldots + \left(m-1\right)^2 + m^2 = \frac{m(m+1)(2m+1)}{6}\]
\begin{enumerate}
\item Summe der ersten $m$ Potenzen einer Zahl ($q \neq 1$)
\end{enumerate}
\[\sum_{n=0}^m q^n = 1+q+\dots+q^{m-1}+q^m = \frac{1 - q^{m + 1}}{1-q}\]
sog. \emph{geometrische Summe}
\begin{itemize}
\item Beweis
\[s_m = 1 + \ldots + q^m\]
\[q s_m = q + \ldots + q^{m+1}\]
\[s_m - q s_m = s_m\left(1-q\right) = 1-q^{m+1}\]
\end{itemize}

\paragraph{Rechenregeln}
\label{sec-2-1-1-2}
\begin{enumerate}
\item \[\sum_{k=m}^n a_k = \sum_{j=m}^n a_j\]
\item \[c\sum_{k=m}^n a_k = \sum_{k=m}^n c a_k\]
\item \[\sum_{k=m}^n a_k \pm \sum_{j=m}{n} b_k = \sum_{k=m}^n \left(a_k \pm b_k\right)\]
\item \[\sum_{k=m}^n a_k + \sum_{k=n+1}^p a_k = \sum_{k=m}^{p} a_k\]
\item \[\sum_{k=m}^n a_k = \sum_{k=m+p}^{n+p} a_{k-p} = \sum_{k=m-p}^{n-p} a_{k+p}\]
\item \[\left(\sum_{i=1}^n a_i\right)(\sum_{j=1}^m b_j) = \sum_{i=1}^n \sum_{j=1}^m a_i b_j = \sum_{j=1}^m \sum_{i=1}^n a_i b_j\]
                   falls $n=m$ \[\sum_{i,j=1}^n a_i b_j\]
\end{enumerate}

\subsubsection{Produktzeichen}
\label{sec-2-1-2}
\paragraph{Beispiel}
\label{sec-2-1-2-1}
\[\prod_{n=1}^3 a_n = a_1 a_2 a_3\]

\subsubsection{Fakultätszeichen}
\label{sec-2-1-3}
\[m! = 1 \cdot 2 \cdot \ldots \cdot \left(m-1\right) \cdot m = \prod_{n=1}^m n\]
        \[0! = 1\]

\subsection{Zahlen}
\label{sec-2-2}
Erinnerung
natürliche Zahlen $\mathbb{N}={1,2,3,\ldots}$
ganze Zahlen $\mathbb{Z}=\mathbb{N} \cup {0} \cup {-a\mid a \in \mathbb{N}}$
rationale Zahlen $\mathbb{Q}=\mathbb{Z}\cup {\frac{b}{a} \mid a \in \mathbb{Z} \setminus \{0\} \and b \in \mathbb{Z}}$
reelle Zahlen $\mathbb{R} = \mathbb{Q} \cup \text{unendliche Dezimalbrüche}$
Die reellen Zahlen lassen sich umkehrbar eindeutig auf die Zahlengerade abbilden, dh.h jedem Punkt entspricht genau eine reelle Zahl und umgekehrt
\subsubsection{Rechengesetze für reelle Zahlen}
\label{sec-2-2-1}
\paragraph{Addition}
\label{sec-2-2-1-1}
\begin{itemize}
\item Assoziativität $(a+b) + c = a + (b + c)$
\item Kommutativität $a + b = b + a$
\item neutrales Element $a + 0 = a$
\item Existenz des Negatives $a + x = b$ hat immer genau eine Lösung: $x = b - a$ für $0 - a$ schreibe wir $-a$
\end{itemize}
\paragraph{Multiplikation:}
\label{sec-2-2-1-2}
\begin{itemize}
\item Assoziativität $(a \cdot b) \cdot c = a \cdot (b \cdot c)$
\item Kommutativität $a \cdot b = b \cdot a$
\item neutrales Element $a \cdot 1 = a$
\item Inverses $a \cdot x = b$ hat für jedes $a \neq a$ genau eine Lösung $x = \frac{b}{a} \text{für} \frac{1}{a}$ schreiben wir $a^-1$
\item Distributivgesetz $a \cdot (b + c) = a\cdot b + a\cdot c$
\end{itemize}
\paragraph{Ordung der reellen Zahlen}
\label{sec-2-2-1-3}
Die kleiner-Beziehung $a<b$, oder auch $b > a$ hat folgende Eigenschaften:
\begin{itemize}
\item Trichotomie: Es gilt immer genau eine Beziehung
$a < b$, $a = b$ $a > b$
\item Transitivität: Aus $a < b$ und $b < c$ folgt $a < c$
\end{itemize}
\paragraph{Beispiele, Folgerungen}
\label{sec-2-2-1-4}
\subparagraph{Rechenregeln für Potenzen}
\label{sec-2-2-1-4-1}
$b^n := b\cdot b \cdot \ldots \cdot b$ $n\in \mathbb{N}$ Faktoren
\[b^0 := 1\]
\[b^-n = \frac{1}{b^n}\]
\[b^n \cdot b^m = b^{n+m}\]
\[(b^n)^m = b^{n\cdot m}\]
\[(a\cdot b)^n = a^n \cdot b^n\]
\paragraph{Betrag einer reellen Zahl}
\label{sec-2-2-1-5}
\[\abs{a} := \begin{cases} a & a \leq 0 \\ -a & a > 0 \end{cases} \]
\subparagraph{Eigenschaften}
\label{sec-2-2-1-5-1}
\[\abs{a} \geq 0 \Forall a\in\mathbb{R}\]
\[\abs{a} = 0\] nur für $a = 0$
\[\abs{a + b} \leq \abs{a} + \abs{b}\] Dreieckungleichung
\subsubsection{Satz des Pythagoras}
\label{sec-2-2-2}
\[a^2 + b^2 = c^2\]
\subsubsection{binomische Formeln:}
\label{sec-2-2-3}
\[(a\pm b)^2 = a^2 \pm 2 a b + b^2\]
\[(a+b)(a-b) = a^2 - b^2\]
Allgemein:
\[(a \pm b)^n = \sum_{k=0}^n{\frac{n!}{k!(n-k)!}a^{n-k}(\pm)^k}\] (Klammer) Binominial koeffizienten
\[\binom{n}{k} := \frac{n!}{k!(n-k)!}a^{n-k}\]

\subsubsection{Pascalsches Dreieck}
\label{sec-2-2-4}
\begin{center}
$n = 0$ 1 \\
$n = 1$ 1 1 \\
$n = 2$ 1 2 1 \\
$n = 3$ 1 3 3 1 \\
$n = 4$ 1 4 6 4 1 \\
$n = 5$ 1 5 10 10 5 1 \\
\end{center}

\subsubsection{Beweisprinzip der Vollständingen Induktion}
\label{sec-2-2-5}
\paragraph{Beispiel}
\label{sec-2-2-5-1}
Für alle $n \in \mathbb{N}$ soll die Summe der ersten $n$ Quadratzahlen beiesen werden
\[A(n) := \sum_{k=1}^n{k^1} = 1^2 + 2^2 + \ldots + n^2 = \frac{1}{6}n(n+1)(2n+1)\]
\begin{enumerate}
\item Induktionsanfang $A(1) = 1$ $\checkmark$
\item Induktonsschritt Falls $A(k)$ richtig ist, wird gezeigt, dass auch $A(k+1)$ richtig ist
\[A(k+1) = \underbrace{1^2 + 2^2 + \ldots + k^2}_{A(n)} + (k+1)^2 = \frac{1}{6}k(k+1)(2k+1)+(k+1)^2\]
\[=\frac{1}{6}(k+1)(k(2k+1)+6(k+1))\]
\[=\frac{1}{6}(k+1)(k+2)(2k+3)\]
\[=\frac{1}{6}(k+1)(k+2)(2(k+1)+1)\]
\end{enumerate}

\subsubsection{Quadratische Ergänzung}
\label{sec-2-2-6}
\[x^2 + a x + b = 0\]
\[x_{1,2}=-\frac{a}{2}\pm \sqrt{\frac{a^2}{4}-b}\]

\section{Folgen und Reihen}
\label{sec-3}
\subsection{Folge}
\label{sec-3-1}
\subsubsection{Definition}
\label{sec-3-1-1}
Vorschrift, die jeder natürlichen Zahl $n$ eine reelle Zahl $a_n$ zuweist.
\[(a_n)_{n\in \mathbb{N}}\]
\subsubsection{Beispiele}
\label{sec-3-1-2}
\begin{itemize}
\item die natürlichen Zahlen selbst \[n_{n\in \mathbb{N}} = (1, 2, 3, \ldots)\]
\item alternierende Folge \[((-1)^{n+1})_{n\in \mathbb{N}} = (1, -1, 1, -1, \ldots)\]
\item harmonische Folge \[(\frac{1}{n})_{n\in \mathbb{N}} = (1, \frac{1}{2}, \frac{1}{3}, \ldots)\]
\item inverse Fakultäten \[(\frac{1}{n!})_{n\in \mathbb{N}}= (1, \frac{1}{2}, \frac{1}{6}, \ldots)\]
\item Folge echter Brüche \[(\frac{n}{n + 1})_{n\in \mathbb{N}} = (\frac{1}{2}, \frac{2}{3}, \frac{3}{4}, \ldots)\]
\item geometrische Folge \[(q^n)_{n\in \mathbb{N}} = (q, q^2,q^3, \ldots)\]
charakteristische Eigenschaft der geometrischen Folge $\frac{a_{n+1}}{a_n} = q$ q heißt Quotient der Folge
allgemeines Bildungsgesetz $a_n = a_1 q^{n-1}$
\item Folge der Ungeraden Zahlen (arithmetische Folge) \[(1+(n-1)*2)_{n\in \mathbb{N}} = (1, 3, 5, 7, \ldots)\]
$a_{n+1} - a_n = d$ $d$ heißt Differenz der Folge
allgemeines Bildungsgesetz $a_n = a_1 + (n - 1) d$
\item "zusammengesetzte Folgen" (hier Exponentialfolge) \[((1 + \frac{1}{n})^n)_{n\in \mathbb{N}} = (2, \frac{3}{2}^2, \frac{4}{3}^2, \ldots)\]
\end{itemize}
\subsubsection{Frage}
\label{sec-3-1-3}
Kann man etwas über das Verhalten von $(a_n)_{n\in \mathbb{N}}$ für $n \to \infty$ aussagen, ohne tatsächlich "die Reise ins Unendliche" anzutreten"
\subsubsection{Beschränktheit}
\label{sec-3-1-4}
Eine Folge heißt \uline{nach oben beschänkt}, wenn es eine obere Schranke B für die Flieder der Folge gibt: $a_n \leq B$, d.h. $\exists B: a_n \leq B \Forall n \in \mathbb{N}$
Nach unten beschränkt: $\exists A: A \geq a_n \Forall n\in\mathbb{N}$
\subsubsection{Monotonie}
\label{sec-3-1-5}
\begin{itemize}
\item Eine Folge heißt \uline{monoton steigend}, wenn aufeinanderfolgende Glieder mit wachsender Nummer immer größer werden: $a_n \leq a_{n+1} \Forall n\in\mathbb{N}$
\item \uline{streng monoton steigend} $a_n < a_{n+1} \Forall n\in\mathbb{N}$
\item \uline{monoton fallend} $a_n \geq a_{n+1} \Forall n\in\mathbb{N}$
\item \uline{streng monoton fallend} $a_n > a_{n+1} \Forall n\in\mathbb{N}$
\end{itemize}
\subsubsection{Konvergenz}
\label{sec-3-1-6}
Eine Folge $(a_n)_{n\in\mathbb{N}}$ \uline{konvergiert} gegen a oder hat den \uline{Grenzwert} a, wenn es zu jedem $\epsilon > 0$ ein $N(\epsilon)\in\mathbb{N}$ gibt mit $\abs{a-a_n} < \epsilon \Forall n > N(\epsilon)$
Wir schreiben $\lim_{n\to\infty}a_n = a$
\paragraph{Beispiel}
\label{sec-3-1-6-1}
\begin{itemize}
\item $\lim_{n\to\infty}\frac{1}{n} = 0$
\item $\lim_{n\to\infty}(1-\frac{1}{\sqrt{n}}) = 1$
\end{itemize}
\paragraph{Grenzwertfreie Konvergenzkriterien}
\label{sec-3-1-6-2}
\begin{itemize}
\item jede monoton wachsend, nach oben beschränkte Folge ist konvergent, entsprechend ist jede monoton fallende, nach unten beschränkte Folge konvergent
\item Cauchy-Kriterium: Eine Folge (a$_{\text{n}}$)$_{\text{n}\in\mathbb{N}}$ konvergiert genau dann, wenn es zu jedem $\epsilon > 0$ ein  $N(\epsilon)\in\mathbb{N}$ gibt, so dass \[\abs{a_n - a_m} < \epsilon\Forall n,m > N(\epsilon)\]
\end{itemize}
\subparagraph{Für harmonische Folge $(\frac{1}{n})_{n\in\mathbb{N}}$}
\label{sec-3-1-6-2-1}
\[\abs{a_n - a_m} = \abs{\frac{1}{n} - \frac{1}{m}} = \abs{\frac{m-n}{m n}} < \abs{\frac{m}{m n}} = \frac{1}{n} < \epsilon \text{für} n > N(\epsilon) = \frac{1}{\epsilon}\]
\subsection{Reihen (unendliche Reihen)}
\label{sec-3-2}
Sei $(a_n)_{n\in\mathbb{N}}$ eine Folge reeller Zahlen, Die Folge \[s_n := \sum_{k=1}^n a_k, n\in\mathbb{N}\] der Partialsumme heißt (unendliche) Reihe und wird oft mit $\sum_{k=1}^\infty a_k$ bezeichnet
Konvergiert die Folge (s$_{\text{n}}$)$_{\text{n}\in\mathbb{N}}$, so wird ihr Grenzwert ebenfalls mit $\sum_{k=1}^\infty a_k$ bezeichnet
\subsubsection{Bemerkung}
\label{sec-3-2-1}
Ergebnisse für Folgen gelten auch für Reihen
\subsubsection{Rechenregeln für konvergente Reihen}
\label{sec-3-2-2}
Seien $\sum_{k=1}^\infty a_k$ und $\sum_{k=1}^\infty b_k$ zwei konvergente Reihen und $\lambda\in\mathbb{R}$, dann sind auch die Reihen \[\sum_{k=1}^\infty a_k + b_k, \sum_{k=1}^\infty a_k - b_k, \sum_{k=1}^\infty \lambda a_k\] konvergent und es gilt
\[\sum_{k=1}^\infty(a_k \pm b_k) = \sum_{k=1}^\infty a_k \pm \sum_{k=1}^\infty b_k\]
\[\sum_{k=1}^\infty \lambda a_k = \lambda \sum_{k=1}^\infty a_k\]
\paragraph{Bemerkung:}
\label{sec-3-2-2-1}
Für das Produkt zweier unendlicher Reihen gilt i.A. keine so einfache Formel
\subsubsection{Beispiel}
\label{sec-3-2-3}
geometrische Reihe \[\sum_{n=0}^\infty q^n = \lim_{m\to\infty}(\sum_{n=0}^m q^n) = \lim_{m\to\infty}\frac{1-q^{m+1}}{1-q} = \frac{1}{1-q} \text{für} q < 1, q\neq 0\]
\subsubsection{Absolute Konvergenz}
\label{sec-3-2-4}
Eine Reihe \[\sum_{k=1}^\infty a_k\] heißt absolut konvergent, wenn die Reihe \[\sum_{k=1}^\infty\abs{a_k}\] konvergiert. Absolut konvergente Reihen können ohne Änderung der Grenzwertes umgeordnet werden, d.h. jede ihrer Umordungen konvergiert wieder und zwar immer gegen den gleichen Grenzwert.
\section{{\bfseries\sffamily TODO} what was done after this? (Funktionen? (only?))}
\label{sec-4}
\section{Funktionen}
\label{sec-5}
\subsection{Normal-Hyperbel}
\label{sec-5-1}
\[y=\frac{1}{x}\quad D_f=\mathbb{R}\setminus\{0\}\quad W_f=\mathbb{R}\setminus\{0\}\]
\subsubsection{Physik-Beispiel}
\label{sec-5-1-1}
\begin{itemize}
\item Boyle-Mariettsches Gesetz
\item Druck $p$ eines idealen Gases in einem Volumen $V$ bei konstanter Temperatur und Gasmenge: $p = \frac{\text{cons}}{V}$
\end{itemize}
\subsection{kubische Parabel}
\label{sec-5-2}
\[y=a x^3\]
\subsubsection{Physik-Beispiel}
\label{sec-5-2-1}
\[V=\frac{4}{3}\pi r^3\]
\subsubsection{Verallgemeinerung}
\label{sec-5-2-2}
\[y=a x^n\quad n\in\mathbb{N}\]
\subsection{$y=a x^{-2}$}
\label{sec-5-3}
\subsubsection{Physik-Beispiel}
\label{sec-5-3-1}
Coulomb Gesetz der Elektrostatik \[F=\frac{1}{4\pi\epsilon}\frac{q_1 q_2}{r^2}\]
\subsection{Symmetrieeigenschaften der Potenzfunktionen}
\label{sec-5-4}
\[y=f(x)=x^n\]
\begin{itemize}
\item gerade n: f ist symmetrisch, d.h. $f(-x) = f(x)$
\item ungerade n: f ist antisymmetrisch, d.h. $f(-x) = -f(x)$
\end{itemize}
\subsection{Potenzfunktionen als "Bausteine" in susammengesetzten Funktionen}
\label{sec-5-5}
Polynom m-ten Grades \[y=P_m(x) = a_0 + a_1 x + \ldots + a_m x^m = \sum_{k=0}^m a_k x^k\]
\subsection{Rationale Funktionen}
\label{sec-5-6}
\[y=\frac{P_m(x)}{Q_n(x)}\quad D_f = \{x\in\mathbb{R}\mid Q_n(x)\neq 0\}\]
$P_m(x)$ Polynom m-ten Grades, $Q_n(x)$ n-ten Grades
\subsubsection{Beispiel}
\label{sec-5-6-1}
\[f(x) = \frac{1}{x^2 + 1}\]
"Lorentz-Verteilung beschreibt die Linienbreite einer Spektrallinie"
\subsection{Trigonometrische Funktionen}
\label{sec-5-7}
\[\sin{\alpha} = \frac{a}{c} = \cos{\beta}\]
\[\cos{\alpha} = \frac{b}{c} = \sin{\beta}\]
\[\tan{\alpha} = \frac{a}{b}=\frac{\sin{\alpha}}{\cos{\alpha}} = \cot{\beta} = \frac{1}{\cot{\alpha}}\]
\[\cot{\alpha} = \frac{b}{a}=\frac{\cos{\alpha}}{\sin{\alpha}} = \tan{\beta} = \frac{1}{\tan{\alpha}}\]
\[\cos{\alpha}^2 + \sin{\alpha}^2 = 1\]

\begin{center}
\begin{tabular}{llll}
$\alpha$ & $\sin{\alpha}$ & $\cos{\alpha}$ & $\tan{\alpha}$\\
\hline
$0$ & $0$ & $1$ & $0$\\
$\SI{30}{\degree}$ & $\frac{1}{2}$ & $\frac{\sqrt{3}}{2}$ & $\frac{1}{\sqrt{3}}$\\
$\SI{45}{\degree}$ & $\frac{\sqrt{2}}{2}$ & $\frac{\sqrt{2}}{2}$ & $1$\\
$\SI{60}{\degree}$ & $\frac{\sqrt{3}}{2}$ & $\frac{1}{2}$ & $\sqrt{3}$\\
$\SI{90}{\degree}$ & $1$ & $0$ & $\to\infty$\\
\end{tabular}
\end{center}
\subsubsection{{\bfseries\sffamily TODO} Table Formula?}
\label{sec-5-7-1}
\subsubsection{{\bfseries\sffamily TODO} Veranschaulichung am Einheitskreis}
\label{sec-5-7-2}
$\sin{\alpha} = y$
Periodische Erweiterung auf $\alpha < 0,~\alpha>\frac{\pi}{2}$ \\
        Periodische Funktion: \[\sin{x + 2\pi} = \sin{x}\quad\text{Periode: }2\pi\]
\[\cos{x + 2\pi} = \cos{x}\quad\text{Periode: }2\pi\]
\paragraph{Beispiel}
\label{sec-5-7-2-1}
\[\sin{x + \pi} = -\sin{x}\]
\[\cos{x + \pi} = -\cos{x}\]
\[\cos{x} = \sin{\frac{\pi}{2}-x}\]
\paragraph{{\bfseries\sffamily TODO} Graphik}
\label{sec-5-7-2-2}
\subsubsection{Tangens/Cotangens}
\label{sec-5-7-3}
\[\tan{x} = \frac{\sin{x}}{\cos{x}}\]
\paragraph{{\bfseries\sffamily TODO} Graphik}
\label{sec-5-7-3-1}
\subsubsection{Additionstheoreme}
\label{sec-5-7-4}
\[\sin{\alpha\pm\beta} = \sin{\alpha}\cos{\beta}\pm\cos{\alpha}\sin{\beta}\]
\[\cos{\alpha\pm\beta} = \cos{\alpha}\cos{\beta}\pm\sin{\alpha}\sin{\beta}\]
\[\sin{2\alpha} = 2\sin{\alpha}\cos{\alpha}\]
\[\cos{2\alpha} = \cos{\alpha}^2 - \sin{\alpha}^2=1 - 2\sin{\alpha}^2 = 2\cos{\alpha}^2 - 1\]
\subsection{Exponentialfunktionen}
\label{sec-5-8}
\[y=f(x)=b^x\quad b>0,~x\in\mathbb{R}\]
\subsubsection{Rechenregeln}
\label{sec-5-8-1}
\[b^x b^y = b^{x+y}\quad \left(b^x\right)^y = b^{xy}\]
natürliche Exponentialfunktion mit Zahl $e$ als Basis
\[y=f(x)=e^x=\sum_{k=0}^\infty \frac{x^k}{k!}\]
\subsubsection{Beispiel radioaktiver Zerfall}
\label{sec-5-8-2}
\[N(t) = N(0)e^\frac{-t}{\tau}\]
\subsection{Cosinus hyperbolicus}
\label{sec-5-9}
\[y=\cosh{x}:=\frac{1}{2}\left(e^x + e^{-x}\right)\]
\subsection{Sinus hyperbolicus}
\label{sec-5-10}
\[y=\sinh{x}:=\frac{1}{2}\left(e^x - e^{-x}\right)\]
Es gilt:
\[\cosh^2{x} - \sinh^2{x}=1\]
\subsection{Tangens hyperbolicus}
\label{sec-5-11}
\[y=\tanh{x}:=\frac{\sinh{x}}{\cosh{x}}=\frac{e^x - e^{-x}}{e^x + e^{-x}}\]
\subsection{Cotangens hyperbolicus}
\label{sec-5-12}
\[y=\coth{x}:=\frac{1}{\tanh{x}}=\frac{e^x + e^{-x}}{e^x - e^{-x}}\]
\subsection{Wurzelfunktion}
\label{sec-5-13}
Umkehrfunktion der Potenzfunktionen \[y=f(x)=x^n\quad n\in\mathbb{Z}\]
Wurzelfunktion: \[y=f(x)=\sqrt[n]{x} = x^\frac{1}{n}\]
n gerade: vor der Umkehrung ist die Einschränkung des Definitionsbereiches auf $x\geq 0$ notwendig
\subsubsection{Beispiel}
\label{sec-5-13-1}
\[y=f(x)=x^2 + 1\quad x\geq 0\]
Umkehrfunktion: \[y=\sqrt{x-1}\]
\section{Funktionen mit Ecken und Sprüngen}
\label{sec-6}
\subsection{Betragsfunktion}
\label{sec-6-1}
\[y=\abs{x}:=\begin{cases}x& x \geq 0\\ -x& x < 0\end{cases}\]
\subsection{Heaviside-Stufenfunktion}
\label{sec-6-2}
\[y=\Theta(x):=\begin{cases}1&x>0\\0&x<0\\\frac{1}{2}&x=0\end{cases}\]
\subsubsection{{\bfseries\sffamily TODO} Graphik}
\label{sec-6-2-1}
\subsubsection{Beispiel}
\label{sec-6-2-2}
\[y=\Theta(x)\Theta(-x+a)\]
    \textbf{TODO} Graphik
\subsection{"symmetrischer Kasten" der Breite $2a$ und der Höhe $\frac{1}{2a}$ (Dirak Delta Funktion)}
\label{sec-6-3}
\[\Theta_a (x):=\frac{\Theta(x+a)\Theta(-x+a)}{2a}\]
\[\lim_{a\to 0}\Theta_a=\text{"(Dirak) $\delta$-Funktion"}\]
\subsubsection{{\bfseries\sffamily TODO} Graphik}
\label{sec-6-3-1}
\section{Verkettung von Funktionen}
\label{sec-7}
Seinen \[f:D_f \to \mathbb{R}\] \[g:D_g\to\mathbb{R}\] mit $w_g \subseteq D_f$, dann ist die Funktion $f\circ g: D_g\to\mathbb{R}$ definiert durch \[(f\circ g)(x):=f(g(x))\quad\Forall x\in D_g\]
\subsection{Beispiel}
\label{sec-7-1}
\[z = g(x) = 1+x^2\quad W_g: z\geq 1\]
\[y=f(z)=\frac{1}{z}\quad D_f=\mathbb{R}\setminus\{0\}\]
also $W_g\subset D_f$, sodass \[(f\circ g)(x)=f(g(x)) = \frac{1}{g(x)}= \frac{1}{1+x^2}\]
\subsection{Spiegelsymmetrie (Siegelung an der y-Achse, d.h. $x\to -x$)}
\label{sec-7-2}
Eine Funktion $f(x)$ heißt
\begin{itemize}
\item gerade(symmetrisch) wenn $f(-x) = f(x)$
\item ungerade (antisymmetrisch) wenn $f(-x) = -f(x)$
\end{itemize}
\subsubsection{Beispiel}
\label{sec-7-2-1}
\paragraph{gerade Funktionen}
\label{sec-7-2-1-1}
\begin{itemize}
\item $f(x) = x^{2n}\quad n\in\mathbb{N}$
\item $f(x) = \cos{x}$
\item $f(x) = \abs{x}$
\end{itemize}
\paragraph{ungerade Funktionen}
\label{sec-7-2-1-2}
\begin{itemize}
\item $f(x) = x^{2n + 1}$
\item $f(x)=\frac{1}{x}$
\item $f(x)=\sin(x)$
\end{itemize}
\paragraph{keins von beidem}
\label{sec-7-2-1-3}
\begin{itemize}
\item $f(x) = s x + c$
\end{itemize}
\subsubsection{Zerlegung}
\label{sec-7-2-2}

\textbf{Jede Funktion lässt sich in einen geraden und ungeraden Anteil zerlegen}
\begin{itemize}
\item gerader Anteil: \[f_+(x)=\frac{1}{2}\left(f(x) + f(-x)\right)=f_+(-x)\]
\item ungerader Anteil: \[f_-(x)=\frac{1}{2}\left(f(x)-f(-x)\right)=-f_-(-x)\]
\item check: \[f_+(x) + f_-(x)=f(x)\quad\checkmark\]
\end{itemize}
\section{Eigenschaften von Funktionen}
\label{sec-8}
\subsection{Beschränktheit}
\label{sec-8-1}
$f$ heißt nach oben beschränkt im Intervall $[a,b]$, wenn es eine obere Schranke gibt, d.h. \[\exists B\in\mathbb{R}: f(x)\leq B\Forall x\in [a,b]\]
analog: nach unten beschränkt \[\exists A\in\mathbb{R}: f(x)\geq A\Forall x\in [a,b]\]
\subsubsection{Beispiel}
\label{sec-8-1-1}
$f(x) = x^2$ durch $A=0$ nach unten beschränkt\\
        $f(x) = \Theta(x)$ $B=1$, $A=0$
\subsection{Monotonie}
\label{sec-8-2}
Eine Funktion $f:D_f\to\mathbb{R}$ heißt monoton steigend im Intervall $[a,b] \subseteq D_f$, wenn aus $x_1,x_2\in [a,b]$ mit $x_1<x_2$ stets folgt $f(x_1) \leq f(x_2)$
Gilt sogar $f(x_1) < f(x_2)$ so heißt $f$ streng monoton steigend im Intervall $[a,b]$
Analog heißt $f$ monoton (streng monoton) fallend, wenn stets folgt $f(x_1) \geq f(x_2)$ ($f(x_1) > f(x_2)$)
\subsubsection{Beispiel}
\label{sec-8-2-1}
$f(x) = x^3\quad$ streng monoton steigend
\section{Umkehrfunktionen}
\label{sec-9}
Sei $f : D_f\to W_f$ eineindeutig(bijektiv), dann kann man die Gleichung $y=f(x)$ eindeutig nach $x$ auflösen \[x=f^{-1}(y):=g(y)\quad\quad D_g = W_f,\quad W_g = D_f\] \[f^{-1}=g:W_f\to D_f\]
Die ursprüngliche Abbildung $y=f(x)$ und die Umkehrabbildung $x=f^{-1}(y)=g(y)$ heben sich in ihrer Wirkung auf \[f^{-1}(f(x))= x\]
\subsection{Graph der Umkehrfunktion}
\label{sec-9-1}
\begin{enumerate}
\item Gegebenfalls Einschränktung von $D_f$, sodass eine bijektive Funktion vorliegt
\item Auflösen der Gleichung $y=f(x)\implies x=f^-1(y)$
\item Umbennenung der Variablen: die unabhängige Variable $y$ wird wieder $x$ genannt, die abhängige wieder $y$: \$y=f$^{\text{-1}}$(x)
\end{enumerate}
\subsubsection{Beispiel $y=x^2$}
\label{sec-9-1-1}
\begin{enumerate}
\item Einschränktung $D_f$ auf $x\geq 0$
\item $y=x^2, x\geq 0 \iff x = \sqrt{y}$
\item Umbenennung: $y=\sqrt{x} = x^\frac{1}{2}$
\end{enumerate}
\subsubsection{Graphisch}
\label{sec-9-1-2}
Spiegelung an $y=x$
\section{what after this?}
\label{sec-10}
\section{Integral und Differenzialrechnung}
\label{sec-11}
\[\int_a^b f(x)\mathrm{d}x=F(b) - F(a)\]
Haupsatz:
\[F'(x) = \frac{\mathrm{d}F(x)}{\mathrm{d}x} = f(x)\]
\begin{center}
\begin{tabular}{lll}
$F(x)=\int f(x)\mathrm{d}x$ & f(x) & Bemerkungen\\
\hline
const & 0 & \\
$x^r$ & $r x^{r-1}$ & $r\in\mathbb{R}$\\
$\frac{x^{r+1}}{r+1}$ & $x^r$ & $-1 \neq  r\in\mathbb{R}$\\
\end{tabular}
\end{center}
\subsection{Die Kunst des Integrierens}
\label{sec-11-1}
\[\int_1^e \frac{1}{x}\mathrm{d}x = \ln{x}\mid_1^e = \ln{e} - \ln{1} = 1\]
\[\int_0^{\frac{\pi}{2}} cos(t)\mathrm{d}t=\sin{t}\mid_0^{\frac{\pi}{2}} = \sin{\frac{\pi}{2}} - \sin{0} = 1\]
\[\int_a^b\frac{1}{1+x^2}\mathrm{d}x = \arctan{x}\mid_a^b\]
\subsection{Ableiten über Umkehrfunktion}
\label{sec-11-2}
\[\frac{\mathrm{d}f^-1(x)}{\mathrm{d}x}=\frac{1}{f'(f^-1(x))}\]
\subsection{Integrationsregeln}
\label{sec-11-3}
\subsubsection{Lineare Zerlegung}
\label{sec-11-3-1}
\[\int_{a_1}^{a_2} c f(x) + b g(x)\mathrm{d}x = c\int_{a_1}^{a_2}f(x)\mathrm{d}x + b\int_{a_1}^{a_2}g(x)\mathrm{d}x\]
\paragraph{Beispiel}
\label{sec-11-3-1-1}
\[F=\int_0^1 \sqrt{x} - x^2\mathrm{d}x = \int_0^1 \sqrt{x}\mathrm{d}x - \int_0^1 x^2\mathrm{d}x = \frac{2}{3}x^\frac{3}{2}\mid_0^1 - \frac{1}{3}x^3\mid_0^1 = \frac{1}{3}\]
\[\int_0^1 (1-x^2)^2\mathrm{d}x = \int_0^1 1-2x^2 + x^4\mathrm{d}x = \int_0^1 1\mathrm{d}x - 2\int_0^1 x^2\mathrm{d}x + \int_0^1 x^4\mathrm{d}x = \frac{8}{15}\]
\subsubsection{Substitutionsregel}
\label{sec-11-3-2}
\[\int_a^b f(g(x))g'(x)\mathrm{d}x=\int_{g(a)}^{g(b)}f(y)\mathrm{d}y\]
merke: $\frac{\mathrm{g(x)}}{\mathrm{d}x} \mathrm{d}x = g'(x)\mathrm{d}x = \mathrm{d}y$
\[y=g(x),\quad\frac{\mathrm{d}y}{\mathrm{d}x}=g'(x),\quad\mathrm{d}y = g'(x)\mathrm{d}x\]
\paragraph{Beweis}
\label{sec-11-3-2-1}
$F$ sei die Stammfunktion zu $f$, $F' = f$
\[(F(g(t)))' = F'(g(t))g'(t) = f(g(t))g'(t)\]
\[\int_a^b f(g(t))g'(t)\mathrm{d}t = F(g(t))\mid_a^b=F(g(b)) - F(g(a)) = F(x)\mid_{g(a)}^{g(b)} = \int_{g(a)}^{g(b)}f(y)\mathrm{d}y\]
\paragraph{Beispiel}
\label{sec-11-3-2-2}
\begin{itemize}
\item \[\int_1^5\sqrt{2x+1}\mathrm{d}x = \int_1^9\sqrt{y}\frac{1}{2}\mathrm{d}y=\frac{26}{3}\]
           \[y=2x-1\quad y'=g'(x) =\frac{\mathrm{d}y}{\mathrm{d}x} = g'(x) = 2 \implies \mathrm{d}y = 2\mathrm{d}x \implies \frac{1}{2}\mathrm{d}y = \mathrm{d}x\]
\item \[\int_0^b t e^{-\alpha t^2}\mathrm{d}t = -\frac{1}{2\alpha}\int_0^{-\alpha b^2} e^y\mathrm{d}y = -\frac{1}{2\alpha}(e^{-\alpha b^2} - 1)\]
           \[y=g(t)=-\alpha ^2 \implies \frac{\mathrm{d}y}{\mathrm{d}t}=-2\alpha t \implies \mathrm{d}y=-2\alpha t \mathrm{d}t \implies \mathrm{d}t = -\frac{1}{2\alpha t}\mathrm{d}y\]
\item \[\int_0^T \cos{\omega t}\mathrm{d}t = \frac{1}{\omega}\int_0^{\omega T}\mathrm{d}y\]
\item \[\int_a^b \frac{g'(x)}{g(x)}\mathrm{d}x = \int_{g(a)}^{g(b)}\frac{1}{y}\mathrm{d}y=\ln{\abs{y}}\mid_{g(a)}^{g(b)}\]
\item \[\int \frac{\mathrm{d}x}{ax\pm b} = \frac{1}{a}\ln{\abs{ax\pm b}} + c\]
\item \[\int_a^b g^n(x)g'(x)\mathrm{d}x = \int_{g(a)}^{g(b)} y^n\mathrm{d}y\]
\end{itemize}
\subsubsection{Partielle Integration}
\label{sec-11-3-3}
\[\int_a^b f'(x)g(x)\mathrm{d}x = f(x)g(x)\mid_a^b - \int_a^b f(x)g'(x)\mathrm{d}x\]
\paragraph{Beweis}
\label{sec-11-3-3-1}
\[F(x)=f(x)g(x)\implies F'(x) = f'(x)g(x) + f(x)g'(x)\]
\[\int_a^b F'(x)\mathrm{d}x = \int_a^b f'(x)g(x)\mathrm{d}x + \int_a^b f(x)g'(x)\mathrm{d}x\]
\[f(x)g(x)\mid_a^b = \int_a^b f'(x)g(x)\mathrm{d}x + \int_a^b f(x)g'(x)\mathrm{d}x\]
\[f(x)g(x)\mid_a^b - \int_a^b f(x)g'(x)\mathrm{d}x = \int_a^b f'(x)g(x)\mathrm{d}x\]
\paragraph{Beispiel}
\label{sec-11-3-3-2}
\begin{itemize}
\item \[\int_a^b x\ln{x}\mathrm{d}x = \frac{1}{2}x^2\ln(x)\mid_a^b - \int_a^b \frac{1}{2}x^2 \frac{1}{x}\mathrm{d}x = \frac{1}{2}x^2\ln(x)\mid_a^b - \frac{1}{2}\int_a^b x\mathrm{d}x\]
\item \[\int 1\ln{x}\mathrm{d}x = x\ln{x} - \int x\frac{1}{x}\mathrm{d}x = x\ln{x} - \int 1\mathrm{d}x = x\ln{x}-x+c=x(\ln{x}-1)+c\]
\item \[\int x\sin{x}\mathrm{d}x=-x\cos{x} + \int cos{x}\mathrm{d}x = -x\cos{x}+\sin{x}\]
\end{itemize}
\paragraph{Kreisfläche}
\label{sec-11-3-3-3}
\[y=f(x)=\sqrt{1-x^2}\]
\[\int_a^b\sqrt{1-x^2}\mathrm{d}x = \int_{\arcsin{a}}^{arcsin{b}}\sqrt{1-\sin^2{t}}\cos{t}\mathrm{t}d = \int_{\arcsin{a}}^{arcsin{b}} \cos{t}\cos{t}\mathrm{d}t = \frac{1}{2}(\arcsin{b} + b\sqrt{1-b^2} - \arcsin{a} - a\sqrt{1-a^2}) \text{ mit } a=-1,b=1\quad\implies \frac{1}{2}(\frac{\pi}{2} + \frac{\pi}{2}) = \frac{\pi}{2}\]
\[x=\sin{t} \implies t = \arcsin{x},\quad \frac{\mathrm{x}}{\mathrm{d}t} = \cos{t},\quad \mathrm{d}x = \cos{t}\mathrm{d}t\]
\[\int \cos{t}\cos{t} = \sin{t}\cos{t} + \int sin^2{t}\mathrm{d}t = \sin{t}\cos{t} + \int 1 - cos^2{t}\mathrm{d}t = \frac{\sin{t}\cos{t} + t}{2}\]
\subparagraph{In Polarkoordinaten}
\label{sec-11-3-3-3-1}
\[y=\sin{t}\]
\[x=\cos{t}\]
\[\mathrm{d}x = \sin{t}\mathrm{d}t\]
\[\mathrm{d}A = y\mathrm{d}x = \sin^2{t}\mathrm{d}t\]
\[A = \int_0^\pi \sin^2{t} = \frac{\pi}{2}\]
\subparagraph{Zerlegung}
\label{sec-11-3-3-3-2}
\[\mathrm{d}A = 2\pi r \mathrm{d}r\]
\[\int \mathrm{d}A = \int_0^R 2\pi r\mathrm{d}r=2\pi\frac{1}{2}r^2\mid_0^R = \pi R^2\]
\subsubsection{Weitere Integrationstricks}
\label{sec-11-3-4}
\paragraph{Partialbruchzerlegung}
\label{sec-11-3-4-1}
$\implies$ Integration rationaler Funktionen
\[\int_a^b\frac{\mathrm{d}}{1-x^2} \text{ mit } \{-1,1\}\not\in [a,b] \]
\[1-x^2 = (1-x)(1+x)\]
\[\frac{1}{1-x^2} = \frac{\alpha}{1-x} + \frac{\beta}{1+x} = \frac{\alpha(1+x)+\beta{1-x}}{(1-x)(1+x)} = \frac{\alpha + \beta + x(\alpha - \beta)}{1-x^2} \implies \alpha = \beta \frac{1}{2}\]
\[\int_a^b \frac{\mathrm{d}x}{1-x^2} = \frac{1}{2}(\int_a^b\frac{1}{1+x} + \int_a^b\frac{1}{1+x})\]
\subsection{Uneigentliche Integrale}
\label{sec-11-4}
\subsubsection{Unendliches Integralintervall}
\label{sec-11-4-1}
\paragraph{Definition}
\label{sec-11-4-1-1}
Sei $f:[a,\infty)\rightarrow\mathbb{R}$ eine Funktion, die über jedem Intervall $[a,R),~a<R<\infty$ (Riemann-)integrierbar ist. Falls der Grenzwert $\lim_{R\to\infty}\int_a^R f(x)\mathrm{d}x$ existiert setzt man \[\int_a^\infty f(x)\mathrm{d}x=\lim_{R\to\infty}\int_a^R f(x)\mathrm{d}x\]
\paragraph{Beispiel}
\label{sec-11-4-1-2}
\[\int_1^\infty\frac{\mathrm{d}x}{x^s}=\begin{cases}\frac{1}{s-1}&s>1\\ \infty & s\leq 1\end{cases}\]
\subsection{Cauchy Hauptwert}
\label{sec-11-5}
\[P\int_{-\infty}^\infty f(x)\mathrm{d}x := \lim_{c\to\infty}\int_{-c}^c f(x)\mathrm{d}x\]
P := "principal Value"
\[\int_{-\infty}^\infty x^{2n-1}\mathrm{d}x = \lim_{a\to\infty}\int_{-a}^c x^{2n-1}\mathrm{d}x + \lim_{b\to\infty}\int_c^b x^{2n-1}\mathrm{d}x = \infty\]
\[P\int_{-\infty}^\infty x^{2n-1}\mathrm{d}x = \lim_{c\to\infty}\int_{-c}^c x^{2n-1}\mathrm{d}x = \lim_{c\to\infty}(\frac{1}{2\pi}(\underbrace{c^{2n}-(-c)^{2n}}_{=0})) = 0\]
\subsubsection{Unbeschränkter Integrand}
\label{sec-11-5-1}
Situation: Integrand wird an einer Stelle $x_0 \in [a,b]$ unbeschränkt
\paragraph{Definition}
\label{sec-11-5-1-1}
Sei $f:(a,b] \to \mathbb{R}$ eine Fnunkion, die über jedem Teilintervall $[a+\eta, b],~0<\eta<b-a$ (Riemann-)integrierbar ist.
Falls der Grenzwert $\lim_{\eta\to 0}\int_{a+\eta}^b f(x)\mathrm{d}x$ existiert, heipßt das Integral $\int_a^b f(x)\mathrm{d}x$ konvergent
\[\int_a^b f(x)\mathrm{d}x = \lim_{\eta\to 0}\int_{a+\eta}^b f(x)\mathrm{d}x\]
\paragraph{Beispiel}
\label{sec-11-5-1-2}
\[\int_0^b \frac{1}{x^{1-\epsilon}}\mathrm{d}x = \lim_{\eta\to 0} \int_\eta^b \frac{1}{x^{1-\epsilon}}\mathrm{d}x = \lim_{\eta\to 0} \frac{1}{\epsilon}(b^\epsilon - \eta^\epsilon) = \frac{1}{\eta}b^\epsilon\]
\paragraph{Principal value}
\label{sec-11-5-1-3}
\[P\int_a^b f(x)\mathrm{d}x = \lim_{\eta\to 0} \int_a^{x_0 - \eta} f(x)\mathrm{d}x + \int_{x_0+\eta}^b f(x)\mathrm{d}x\]
\subsection{Integralfunktionen}
\label{sec-11-6}
\[\ln{x} = \int_1^x \frac{\mathrm{d}x}{x}\]
\[\arctan{x} = \int_0^y \frac{\mathrm{d}x}{1+x^2}\]
\[erf(x) = \frac{2}{\sqrt{\pi}}\int_0^y e^{-x^2}\mathrm{d}x\]
Elliptisches Integral
\subsection{Gamma-Funktion}
\label{sec-11-7}
\subsubsection{Definition}
\label{sec-11-7-1}
\[\Gamma(x):=\int_0^\infty t^{x-1}e^{-t}\mathrm{d}t\]
Satz: Es gilt $\Gamma(1) = 1,~\Gamma(m+1) = m! \Forall n\in\mathbb{N},~x\Gamma(x) = \Gamma(x+1)$
\[\Gamma(1)=\int_0^\infty e^{-t}\mathrm{d}t=-e^{-t}\mid_0^\infty = 1\]
\[\Gamma(x+1) = \int_\epsilon^R t^x e^{-t}\mathrm{d}t = \underbrace{t^x e^{-t}\mid_\epsilon^R}_{R\to\infty t} + x\int_\epsilon^R t^{x-1}e^{-t}\mathrm{d}t\]
\[f(t) = -e^{-t} \impliedby f'(t)=e^{-t}\]
\[g(t) = t^x \implies xt^{t-1} = g'(t)\]
\section{Vektoren}
\label{sec-12}
\subsection{$\mathbb{R}^3$}
\label{sec-12-1}
\subsubsection{Orthonormal}
\label{sec-12-1-1}
Länge eins, senkrecht aufeinander und sie bilden eine Basis, also jeder Vektor hat genau eine Darstellung: \[\vec{a} = a_1 \vec{e_1} + a_2 \vec{e_2} + a_3 \vec{e_3} = \sum_{k=1}^3 a_k \vec{e_k}a = \underbrace{a_k e_k}_{\text{Einsteinsche Summenkonvention}}\]
\subsection{Skalarprodukt und Kronecker-Symbol}
\label{sec-12-2}
\subsubsection{Motivation: mechanische Arbeit}
\label{sec-12-2-1}
\subsubsection{Definition}
\label{sec-12-2-2}
\[<\vec{a},\vec{b}> = \vec{a}\cdot\vec{b} := \abs{\vec{a}}\abs{\vec{b}}\cos{\angle (\vec{a},\vec{b})}\]
\subsubsection{Spezialfälle}
\label{sec-12-2-3}
\[\vec{a}\|\vec{b}\implies \vec{a}\cdot\vec{b}=\abs{\vec{a}}\abs{\vec{b}}\]
$\vec{a}$ und $\vec{b}$ antiparallel:
\[\vec{a}\cdot\vec{b}=-\abs{\vec{a}}\abs{\vec{b}}\]
\[\vec{a}\bot\vec{b}\implies\vec{a}\cdot\vec{b}=0\]
\subsubsection{Betrag:}
\label{sec-12-2-4}
\[<\vec{a},\vec{b}>=\abs{\vec{a}}^2=a^2\]
\subsubsection{Eigenschaften}
\label{sec-12-2-5}
\begin{itemize}
\item Kommutativgesetz
\[<\vec{a},\vec{b}>=<\vec{b},\vec{a}>\]
\item Homogenität
\[<\lambda\vec{a},\vec{b}>=\lambda<\vec{a},\vec{b}>=<\vec{a},\lambda\vec{b}>\]
\item Distributivgesetz
\[<\vec{a}+\vec{b},\vec{c}>=<\vec{a},\vec{c}>+<\vec{b},\vec{c}>\]
\[<\vec{a},\vec{b}+\vec{c}>=<\vec{a},\vec{b}>+<\vec{a},\vec{c}>\]
\item \[<\vec{a},\vec{a}>\geq 0 \quad <\vec{a},\vec{a}>=0\iff\vec{a}=0\]
\end{itemize}
\subsubsection{Orthonormalbasis der kartesischen Koordinatensystem}
\label{sec-12-2-6}
Basisvecktoren $\vec{e_k}, k=1,2,4$
Orthogonalität $<\vec{e_k},\vec{e_l}> = 0\quad l\neq k$
Für $k=l:~<\vec{e_k},\vec{e_k}>=\cos(0)=1$
Orthonormalität
\subsubsection{Kronecker Symbol}
\label{sec-12-2-7}
\[\delta_{kl}:=\begin{cases}1&k=l\\0&k\neq l\end{cases}\]
Entspricht Komponenten der Einheitsmatrix
Symmetrie gegen Vertauschung der Indizes \$$\delta$$_{\text{kl}}$=$\delta$\{lk\}
Spur: $\delta_{kk} = \underbrace{\sum_{k=1}^3 \delta_{kk}=3}_{\text{Einsteinsche Summenkonvention}}$
\subsubsection{Komponentendarstellung des Skalarprodukts}
\label{sec-12-2-8}
\[\vec{a}=\sum_{k=1}^3 a_k \vec{e_k}=\underbrace{a_k \vec{e_k}}_{\text{Einsteinsche Summenkonvention}}\]
\[\vec{b}=\sum_{k=1}^3 b_k \vec{e_k}=\underbrace{b_k \vec{e_k}}_{\text{Einsteinsche Summenkonvention}}\]
\[<\vec{a},\vec{b}>=(\sum_{k=1}^3 a_k\vec{e_k})\cdot (\sum_{k=1}^3 b_k\vec{e_k}) = \sum_{k,l=1}^3 a_k b_k \underbrace{<\vec{e_k},\vec{e_l}>}_{=\delta{kl}} = \sum_{k=1}^3 a_k b_k\]
\section{Matrizen}
\label{sec-13}
\subsection{Determinante}
\label{sec-13-1}
$\det A = \sum_{\sigma \in S_n} \left(\operatorname{sgn}(\sigma) \prod_{i=1}^n a_{i, \sigma(i)}\right)$
Summe über alle Permutationen von $S_n$, Vorzeichen der Permutation ist positiv, wenn eine gerade Anzahl an Vertauschungen notwendig ist, und entsprechend negativ bei einer ungeraden Anzahl.
\subsection{Homogenes Gleichungssystem}
\label{sec-13-2}
\[A\vec{x}=0\quad \begin{pmatrix}
   a_{11} & a_{12} & a_{13} \\
   a_{21} & a_{22} & a_{23} \\
   a_{31} & a_{32} & a_{33} \\
   \end{pmatrix}\begin{pmatrix}
   x_1\\
   x_2\\
   x_3\\
   \end{pmatrix}\] 
\[
   x_1 \underbrace{ \begin{matrix} a_{11} \\ a_{21} \\ a_{31} \end{matrix}}_{\vec{a_1}} +
   x_2 \underbrace{ \begin{matrix} a_{12} \\ a_{22} \\ a_{32} \end{matrix}}_{\vec{a_2}} +
   x_3 \underbrace{ \begin{matrix} a_{13} \\ a_{23} \\ a_{33} \end{matrix}}_{\vec{a_3}}
   = \begin{matrix} 0 \\ 0 \\ 0\end{matrix}
   \]
sind $\vec{a_1},\vec{a_2}, \vec{a_3}$ linear unabhängig, dann gibt es nur die Lösung $x_1=x_2=x_3=0$
Nichttriviale Lösung nur wenn $\vec{a_1},\vec{a_2}, \vec{a_3}$ linear abhängig $\implies \lambda,\mu\in\mathbb{R}$, sodass z.B. $\vec{a_1} = \lambda\vec{a_2} + \mu\vec{a_3}$
Wenn $\vec{a_1},\vec{a_2}, \vec{a_3}$ linear unabhängig, dann $\det A = 0$.
\subsection{Levi Civita Symbol}
\label{sec-13-3}
\begin{equation}
\varepsilon_{ijk \dots} =
\begin{cases}
+1, & \mbox{falls }(i,j,k,\dots) \mbox{ eine gerade Permutation von } (1,2,3,\dots) \mbox{ ist,} \\
-1, & \mbox{falls }(i,j,k,\dots) \mbox{ eine ungerade Permutation von } (1,2,3,\dots) \mbox{ ist,} \\
0,  & \mbox{wenn mindestens zwei Indizes gleich sind.}
\end{cases}
\end{equation}
\begin{equation}
\varepsilon_{i_1\dots i_n} =
\prod_{1\le p<q\le n} \frac{i_p-i_q}{p-q}
\end{equation}
\begin{equation}
\varepsilon_{k,l,m}=\delta_{k1}(\delta_{l2}\delta_{m3} - \delta_{l3}\delta_{m2}) + \delta_{k2}(\delta_{l3}\delta_{m1} - \delta_{l1}\delta_{m3}) + \delta_{k3}(\delta_{l1}\delta_{m2} - \delta_{l2}\delta_{m1})
\end{equation}
\subsection{Vektorprodukt / Kreuzprodukt}
\label{sec-13-4}
\begin{equation}
\vec{a}\times\vec{b}
=
\begin{pmatrix}a_1 \\ a_2 \\ a_3\end{pmatrix}
\times
\begin{pmatrix}b_1 \\ b_2 \\ b_3 \end{pmatrix}
=
\begin{pmatrix}
a_2b_3 - a_3b_2 \\
a_3b_1 - a_1b_3 \\
a_1b_2 - a_2b_1
\end{pmatrix}
\end{equation}
  \begin{align}
  \vec a \times \vec b &=\det \begin{pmatrix}\vec e_1 & a_1 & b_1 \\ \vec e_2 & a_2 & b_2 \\ \vec e_3 & a_3 & b_3\end{pmatrix}\\
  &= \vec e_1 \begin{vmatrix} a_2 & b_2 \\ a_3 & b_3 \end{vmatrix}
- \vec e_2 \begin{vmatrix} a_1 & b_1 \\ a_3 & b_3 \end{vmatrix}
+ \vec e_3 \begin{vmatrix} a_1 & b_1 \\ a_2 & b_2 \end{vmatrix} \\
&= (a_2 \,b_3 - a_3 \, b_2) \, \vec e_1 + (a_3 \, b_1 - a_1 \, b_3) \, \vec e_2 + (a_1 \, b_2 - \, a_2 \, b_1) \, \vec e_3 \,,
\end{align}
\[\vec{a}\times\vec{b} = \sum_{i,j,k=1}^3 \varepsilon_{ijk} a_i b_j \vec e_k = \varepsilon_{ijk}a_i b_j \vec{e_k}\]
\subsection{Spatprodukt}
\label{sec-13-5}
\[\abs{( \vec{a} \times \vec{b} ) \vec{c}} = \text{Volumen einees Spats}\]
\[(\vec{a}\vec{b}\vec{c})=(\vec{a}\times\vec{b})\vec{c}=(\vec{c}\times\vec{a})\vec{b}=(\vec{b}\times\vec{c})\vec{a}=-(\vec{b}\times\vec{a})\vec{c}\]
\subsection{Geschachteltes Vektorprodukt}
\label{sec-13-6}
\[\vec{a}(\vec{b}\times\vec{v})=(\vec{a}\vec{c})\vec{b}-(\vec{a}\vec{b})\vec{c}=\vec{b}(\vec{a}\vec{c})-\vec{c}(\vec{a}\vec{b})\]
\subsubsection{Beweis}
\label{sec-13-6-1}
\[\vec{a}=(\vec{b}\times\vec{c})=\vec{a}\times(\varepsilon_{ijk}b_i c_j \vec{e_k})=\varepsilon_{pqm}a_p\varepsilon_{ijk}b_i c_j \vec{e_m}\]
\section{misc}
\label{sec-14}
\begin{itemize}
\item mathe für physiker vs. analysis
\item klausuren gebündelt
\item auslandssemester
\end{itemize}
% Emacs 25.1.1 (Org mode 8.2.10)
\end{document}
